\renewcommand{\theequation}{\theenumi}
\begin{enumerate}[label=\arabic*.,ref=\thesubsection.\theenumi]
\numberwithin{equation}{enumi}
\item Find the equation of the plane $P$
containing the vectors 
%
\begin{align}
\label{eq:least_vecs}
\vec{x}_1 = \myvec{1 \\ 1 \\ 1}
,
\vec{x}_2 = \myvec{0 \\ 1 \\ 2}
\end{align}
%
\item Show that the vector 
\begin{align}
\vec{y} = \myvec{6 \\ 0 \\ 0}
\end{align}
lies outside $P$.
\item Find the point $\vec{w} \in P$ closest to $\vec{y}$.
\item Show that
\begin{align}
\norm{\vec{y}-\vec{X}\vec{w}}^2 &= \norm{\vec{y}}^2 - \vec{w}^T\vec{X}^T\vec{y} 
\\
& \quad - \vec{y}^TA\vec{w}+\vec{w}^T\vec{X}^T\vec{X}\vec{w}
\end{align}
%
\item Assuming $2\times 2$ matrices and $2 \times 1$ vectors, show that
\begin{align}
\frac{\partial}{\partial\vec{w}}\vec{w}^T\vec{X}^T\vec{y} = \frac{\partial}{\partial\vec{w}}\vec{y}^T\vec{X}\vec{w} = 
\vec{y}^T\vec{X}
\end{align}
\item Show that
\begin{align}
\frac{\partial}{\partial\vec{w}}\vec{w}^T\vec{X}^T\vec{X}\vec{w} = 2\vec{w}^T\brak{\vec{X}^T\vec{X}}
\end{align}
\item Show that 
\begin{align}
\hat{\vec{w}} &= \min_{\vec{w}}\norm{\vec{y}-\vec{X}\vec{w}}^2
\\
 &= \brak{\vec{X}^T\vec{X}}^{-1}\vec{X}^T \vec{y}
\label{eq:least_sol}
\end{align}
\item Let 
\begin{align}
\vec{X} = \myvec{\vec{x}_1 & \vec{x}_2}.
\end{align}
from \eqref{eq:least_vecs}.
Verify \eqref{eq:least_sol}.
\item
Run the following Python code and comment on the output for different values of $\mbf{x}$
\begin{lstlisting}
 
codes/matrix/Prob1_4.py
\end{lstlisting}
%	\begin{verbatim}
%%Code written by GVV Sharma March 30, 2016
%%Released under GNU GPL.  Free to use for anything.
%
%%This program compares the norm defined for the least-squares solution
%%for the correct solution vs other data points.
%%You will find that the metric is the smallest for the correct value.
%
%clear;
%close;
%
%A = [1 0; 1 1; 1 2]; %The input matrix
%b = [6;0;0]; %The output vector
%
%P = inv(A'*A)*A';%pseudoinverse
%
%x_ls = P*b; %The least squares solution
%
%x = [5;-5]; %Any random input
%
%exact_ls_metric = norm(b-A*x_ls)^2 %The metric for actual soltuion
%random_ls_metric = norm(b-A*x)^2 %metric for a random value of x
%	
%	\end{verbatim}


%
%\item
%	Type the following code in Python and observe the output.
%\begin{verbatim}
%%Code written by GVV Sharma March 31, 2016
%%Released under GNU GPL.  Free to use for anything.
%
%
%%This program plots the least squares metric for a range of
%%vectors x in the mesh with vertices (-10,-10),(-10,10),(10,-10)
%%%and (10,10)
%
%%The result is a 3-D mesh.  The theoretical minimum is (5,-3)
%%Values obtained through the following program are close to the 
%%theoreticl solution
%
%
%clear;
%close;
%
%A = [1 0; 1 1; 1 2]; %The input matrix
%b = [6;0;0]; %The output vector
%
%
%x1 = linspace(-10,10,50); %generating points in x-axis
%x2 = linspace(-10,10,50);  %generating points in y-axis
%
%[xx, yy] = meshgrid(x1,x2);
%
%ffun = @(x,y) norm(b-A*[x;y])^2;
%
%f = arrayfun(ffun,xx,yy);
%
%mesh(xx,yy,f)
%
%[M I] = min(f(:)); %vectorize the 50 x 50 matrix f, find min
%%M = min value , I is the index of the f_min
%
%[I_r I_c] = ind2sub(size(f),I); %Get the row, col index of f_min
%
%
%%The least square solution
%xx(I_r,I_c) 
%yy(I_r,I_c)
%%The minimum value of metric
%M
%
%\end{verbatim}
%
%
\item
	Compare the results obtained by typing the following code with the results in the previous problem.
\begin{lstlisting}
 
codes/matrix/Prob1_6.py
\end{lstlisting}
%\begin{verbatim}
%Code written by GVV Sharma March 31, 2016
%Released under GNU GPL.  Free to use for anything.


%This program finds the theoretical least squares solution using 
%SVD 

%clear;
%close;
%
%A = [1 0; 1 1; 1 2]; %The input matrix
%b = [6;0;0]; %The output vector
%
%
%[U S V] = svd(A); % Computing the SVD of A
%
%temp_S = 1./diag(S); %inverting the diagonal values of S
%
%Splus = [diag(temp_S) zeros(2,1)]; %inverse transpose of S
%
%Aplus = V*Splus*U'; %The Moore-Penrose pseudo-inverse
%
%Aplus*b %least squares solution.
%
%\end{verbatim}

%
\item
	Type the following code in Python and run.  Comment.

%
\begin{lstlisting}
 
codes/matrix/Prob1_7.py
\end{lstlisting}
\end{enumerate}

