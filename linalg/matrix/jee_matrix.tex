\documentclass[journal,12pt,twocolumn]{IEEEtran}
\usepackage{setspace}
\usepackage{gensymb}
\usepackage{caption}
%\usepackage{multirow}
%\usepackage{multicolumn}
%\usepackage{subcaption}
%\doublespacing
\singlespacing
\usepackage{csvsimple}
\usepackage{amsmath}
\usepackage{multicol}
%\usepackage{enumerate}
\usepackage{amssymb}
%\usepackage{graphicx}
\usepackage{newfloat}
%\usepackage{syntax}
\usepackage{listings}
\usepackage{iithtlc}
\usepackage{color}
\usepackage{tikz}
\usetikzlibrary{shapes,arrows}



%\usepackage{graphicx}
%\usepackage{amssymb}
%\usepackage{relsize}
%\usepackage[cmex10]{amsmath}
%\usepackage{mathtools}
%\usepackage{amsthm}
%\interdisplaylinepenalty=2500
%\savesymbol{iint}
%\usepackage{txfonts}
%\restoresymbol{TXF}{iint}
%\usepackage{wasysym}
\usepackage{amsthm}
\usepackage{mathrsfs}
\usepackage{txfonts}
\usepackage{stfloats}
\usepackage{cite}
\usepackage{cases}
\usepackage{mathtools}
\usepackage{caption}
\usepackage{enumerate}	
\usepackage{enumitem}
\usepackage{amsmath}
%\usepackage{xtab}
\usepackage{longtable}
\usepackage{multirow}
%\usepackage{algorithm}
%\usepackage{algpseudocode}
\usepackage{enumitem}
\usepackage{mathtools}
\usepackage{hyperref}
%\usepackage[framemethod=tikz]{mdframed}
\usepackage{listings}
    %\usepackage[latin1]{inputenc}                                 %%
    \usepackage{color}                                            %%
    \usepackage{array}                                            %%
    \usepackage{longtable}                                        %%
    \usepackage{calc}                                             %%
    \usepackage{multirow}                                         %%
    \usepackage{hhline}                                           %%
    \usepackage{ifthen}                                           %%
  %optionally (for landscape tables embedded in another document): %%
    \usepackage{lscape}     


\usepackage{url}
\def\UrlBreaks{\do\/\do-}


%\usepackage{stmaryrd}


%\usepackage{wasysym}
%\newcounter{MYtempeqncnt}
\DeclareMathOperator*{\Res}{Res}
%\renewcommand{\baselinestretch}{2}
\renewcommand\thesection{\arabic{section}}
\renewcommand\thesubsection{\thesection.\arabic{subsection}}
\renewcommand\thesubsubsection{\thesubsection.\arabic{subsubsection}}

\renewcommand\thesectiondis{\arabic{section}}
\renewcommand\thesubsectiondis{\thesectiondis.\arabic{subsection}}
\renewcommand\thesubsubsectiondis{\thesubsectiondis.\arabic{subsubsection}}

% correct bad hyphenation here
\hyphenation{op-tical net-works semi-conduc-tor}

%\lstset{
%language=C,
%frame=single, 
%breaklines=true
%}

%\lstset{
	%%basicstyle=\small\ttfamily\bfseries,
	%%numberstyle=\small\ttfamily,
	%language=Octave,
	%backgroundcolor=\color{white},
	%%frame=single,
	%%keywordstyle=\bfseries,
	%%breaklines=true,
	%%showstringspaces=false,
	%%xleftmargin=-10mm,
	%%aboveskip=-1mm,
	%%belowskip=0mm
%}

%\surroundwithmdframed[width=\columnwidth]{lstlisting}
\def\inputGnumericTable{}                                 %%
\lstset{
%language=C,
frame=single, 
breaklines=true,
columns=fullflexible
}
 

\begin{document}
%
\tikzstyle{block} = [rectangle, draw,
    text width=3em, text centered, minimum height=3em]
\tikzstyle{sum} = [draw, circle, node distance=3cm]
\tikzstyle{input} = [coordinate]
\tikzstyle{output} = [coordinate]
\tikzstyle{pinstyle} = [pin edge={to-,thin,black}]

\theoremstyle{definition}
\newtheorem{theorem}{Theorem}[section]
\newtheorem{problem}{Problem}
\newtheorem{proposition}{Proposition}[section]
\newtheorem{lemma}{Lemma}[section]
\newtheorem{corollary}[theorem]{Corollary}
\newtheorem{example}{Example}[section]
\newtheorem{definition}{Definition}[section]
%\newtheorem{algorithm}{Algorithm}[section]
%\newtheorem{cor}{Corollary}
\newcommand{\BEQA}{\begin{eqnarray}}
\newcommand{\EEQA}{\end{eqnarray}}
\newcommand{\define}{\stackrel{\triangle}{=}}

\bibliographystyle{IEEEtran}
%\bibliographystyle{ieeetr}

\providecommand{\nCr}[2]{\,^{#1}C_{#2}} % nCr
\providecommand{\nPr}[2]{\,^{#1}P_{#2}} % nPr
\providecommand{\mbf}{\mathbf}
\providecommand{\pr}[1]{\ensuremath{\Pr\left(#1\right)}}
\providecommand{\qfunc}[1]{\ensuremath{Q\left(#1\right)}}
\providecommand{\sbrak}[1]{\ensuremath{{}\left[#1\right]}}
\providecommand{\lsbrak}[1]{\ensuremath{{}\left[#1\right.}}
\providecommand{\rsbrak}[1]{\ensuremath{{}\left.#1\right]}}
\providecommand{\brak}[1]{\ensuremath{\left(#1\right)}}
\providecommand{\lbrak}[1]{\ensuremath{\left(#1\right.}}
\providecommand{\rbrak}[1]{\ensuremath{\left.#1\right)}}
\providecommand{\cbrak}[1]{\ensuremath{\left\{#1\right\}}}
\providecommand{\lcbrak}[1]{\ensuremath{\left\{#1\right.}}
\providecommand{\rcbrak}[1]{\ensuremath{\left.#1\right\}}}
\theoremstyle{remark}
\newtheorem{rem}{Remark}
\newcommand{\sgn}{\mathop{\mathrm{sgn}}}
\providecommand{\abs}[1]{\left\vert#1\right\vert}
\providecommand{\res}[1]{\Res\displaylimits_{#1}} 
\providecommand{\norm}[1]{\left\Vert#1\right\Vert}
\providecommand{\mtx}[1]{\mathbf{#1}}
\providecommand{\mean}[1]{E\left[ #1 \right]}
\providecommand{\fourier}{\overset{\mathcal{F}}{ \rightleftharpoons}}
%\providecommand{\hilbert}{\overset{\mathcal{H}}{ \rightleftharpoons}}
\providecommand{\system}{\overset{\mathcal{H}}{ \longleftrightarrow}}
	%\newcommand{\solution}[2]{\textbf{Solution:}{#1}}
\newcommand{\solution}{\noindent \textbf{Solution: }}
\newcommand{\myvec}[1]{\ensuremath{\begin{pmatrix}#1\end{pmatrix}}}
\providecommand{\dec}[2]{\ensuremath{\overset{#1}{\underset{#2}{\gtrless}}}}
\DeclarePairedDelimiter{\ceil}{\lceil}{\rceil}
%\numberwithin{equation}{subsection}
%\numberwithin{equation}{section}
%\numberwithin{problem}{subsection}
%\numberwithin{definition}{subsection}
\makeatletter
\@addtoreset{figure}{section}
\makeatother

\let\StandardTheFigure\thefigure
%\renewcommand{\thefigure}{\theproblem.\arabic{figure}}
\renewcommand{\thefigure}{\thesection}


%\numberwithin{figure}{subsection}

%\numberwithin{equation}{subsection}
%\numberwithin{equation}{section}
%\numberwithin{equation}{problem}
%\numberwithin{problem}{subsection}
\numberwithin{problem}{section}
%%\numberwithin{definition}{subsection}
%\makeatletter
%\@addtoreset{figure}{problem}
%\makeatother
\makeatletter
\@addtoreset{table}{section}
\makeatother

\let\StandardTheFigure\thefigure
\let\StandardTheTable\thetable
\let\vec\mathbf
%%\renewcommand{\thefigure}{\theproblem.\arabic{figure}}
%\renewcommand{\thefigure}{\theproblem}

%%\numberwithin{figure}{section}

%%\numberwithin{figure}{subsection}



\def\putbox#1#2#3{\makebox[0in][l]{\makebox[#1][l]{}\raisebox{\baselineskip}[0in][0in]{\raisebox{#2}[0in][0in]{#3}}}}
     \def\rightbox#1{\makebox[0in][r]{#1}}
     \def\centbox#1{\makebox[0in]{#1}}
     \def\topbox#1{\raisebox{-\baselineskip}[0in][0in]{#1}}
     \def\midbox#1{\raisebox{-0.5\baselineskip}[0in][0in]{#1}}

\vspace{3cm}

\title{ 
	\logo{
JEE Problems in Matrices
	}
}

%\author{ G V V Sharma$^{*}$% <-this % stops a space
%	\thanks{*The author is with the Department
%		of Electrical Engineering, Indian Institute of Technology, Hyderabad
%		502285 India e-mail:  gadepall@iith.ac.in. All content in this manual is released under GNU GPL.  Free and open source.}
	
%}	

\maketitle

%\tableofcontents

\bigskip

\renewcommand{\thefigure}{\theenumi}
\renewcommand{\thetable}{\theenumi}


\begin{abstract}
	A  collection of problems from JEE papers related to matrices are available in this document.  Verify your soluions using  Python.
\end{abstract}
\begin{enumerate}[label=\arabic*.]
\item For any two $3 \times 3$ matrices $A$ and $B$, let $A+B = 2B^T$ and $3A+2B=I_3$.  Which of the following 
is true?
\begin{enumerate}
\item $5A+10B=2I_3$.
\item $10A+5B=3I_3$.
\item $2A+B=3I_3$.
\item $3A+6B=2I_3$.
\end{enumerate}
\item Let $\vec{P}$ be the point on the parabola
\begin{equation}
\label{eq:parab_circ_parab}
\vec{x}^T\myvec{0 & 0 \\ 0 & 1}\vec{x} -\myvec{8 & 0 }\vec{x} 
 = 0
\end{equation}
which is at a minimum distance from the centre $\vec{C}$ of the circle
\begin{equation}
\label{eq:parab_circ_circ}
\vec{x}^T\vec{x} +\myvec{0 & 12 }\vec{x} 
 = 1 
\end{equation} 
Find the equation of the circle passing through $\vec{C}$ and having its centre at $\vec{P}$. 
\item Let $\vec{P}$ 
be a point on the parabola
\begin{equation}
\vec{x}^T\myvec{1 & 0 \\ 0 & 0}\vec{x} +\myvec{0 & 4 }\vec{x} 
 = 0
\end{equation}
%%
Given that the distance of $\vec{P}$ from the centre of the circle
\begin{equation}
\vec{x}^T\vec{x} +\myvec{6 \\ 0 }\vec{x} + 8 = 0
\end{equation}
%
is minimum.  Find the equation of the tangent to the parabola at $\vec{P}$.
\item Find the eccentricity of the hyperbola whose length of the latus rectum is equal to 8 and the length of 
its conjugate axis is equal to half the distance between its foci. 
\item $\vec{P}$ and $\vec{Q}$ are two distinct points on the parabola
\begin{equation}
\vec{x}^T\myvec{0 & 0 \\ 0 & 1}\vec{x} -\myvec{4 & 0 }\vec{x} 
 = 0
\end{equation}
with parameters $t$ and $t_1$ respectively.  If the normal at $\vec{P} $ passes through $\vec{Q}$, then find 
the minimum value of $t_1^2$ using a descent algorithm.
\item A tangent at a point on the ellipse 
\begin{equation}
\vec{x}^TV\vec{x} =51
\end{equation}
%
where
\begin{equation}
V = \myvec{3 & 0 \\ 0 & 27}
\end{equation}
%
meets the coordinate axes at  $\vec{A}$  and  $\vec{B}$.  If   $\vec{O}$  be the origin, find the minimum area 
of $\triangle OAB$.
\item Find the shortest distance between the line 
\begin{equation}
\myvec{1 & -1 }\vec{x}  =0
\end{equation}
%
and the curve
\begin{equation}
\vec{x}^T\myvec{0 & 0 \\ 0 & 1}\vec{x} - \myvec{1 & 0}\vec{x} + 2 =0
\end{equation}
\item Let $S$ be the set of all complex numbers $z$ satisfying $\abs{z-2+\j} \ge \sqrt{5}$. If the complex number $z_0$ is such that $\frac{1}{\abs{z_0-1}}$ is the maximum of the set $\cbrak{\frac{1}{\abs{z-1}}: z \in S}$, find the principal argument of 
\begin{align}
\frac{4-z_0-\bar{z}_0}{z_0-\bar{z}_0+2\j}
\end{align}
\item Let $\omega \ne 1$ be a cube root of unity. Find the minimum of the set
\begin{align}
\abs{a+b\omega+c\omega^2}, 
\end{align}
%
where $a,b,c$ are distinct nonzero integers.
\item Let 
\begin{align}
\label{eq:mat_def}
\vec{M} = \myvec{\sin^4\theta & -1-\sin^2\theta \\ 1+ \cos^2\theta & \cos^4\theta} = \alpha\vec{I} + \beta \vec{M}^{-1}
\end{align}
where $\alpha, \beta$ are real functions of $\theta$ and $\vec{I}$ is the identity matrix. If 
\begin{align}
\alpha^{*} &= \min_{\theta}\alpha\brak{\theta}
\\
\beta^{*} &= \min_{\theta}\beta\brak{\theta}, 
\end{align}
find $\alpha^{*} + \beta^{*}$.
\item Find the minimum value of 
\begin{align}
\cos \brak{P+Q}
\cos \brak{Q+R}
\cos \brak{R+P}
\end{align}
%
in $\triangle PQR$.
\item Find the minimum value of $\alpha$ for which 
\begin{align}
4\alpha x^2 + \frac{1}{x} \ge 1, x > 0.
\end{align}
\item Let 
\begin{align}
S = S_1\cap S_2\cap S_3,
\end{align}
%
where
\begin{align}
S_1 &= \cbrak{z\in C: \abs{z}< 4}
\\
S_2 &= \cbrak{z\in C: \Im\sbrak{\frac{z-1+\j\sqrt{3}}{1-\j\sqrt{3}}}},
\\
S_3 &= \cbrak{z\in C: \Re\brak{z} > 0}
\end{align}
Find
\begin{align}
\min_{z\in S}\abs{1-3\j-z}
\end{align}
\item A line 
\begin{align}
L: \myvec{m -1}\vec{x} = -3
\end{align}
passes through
\begin{align}
\vec{E} = \myvec{0\\3}
\end{align}
and 
\begin{align}
\vec{x}\myvec{0 & 0 \\ 0 & 1}\vec{x} - \myvec{16 & 0}\vec{x} = 0, 0 \le \myvec{0 & 1}\vec{x} \le 6
\end{align}
%
at the point $\vec{F}$.  Find $m$ such that the area of $\triangle EFG$ is maximum.
\item If $\abs{z-3-2j} \le 2$, find
\begin{align}
\min_{z}\abs{2z-6+5\j}
\end{align}
\item Find 
\begin{align}
\max_{z}\abs{Arg\brak{\frac{1}{1-z}}}
\\
s.t\quad \abs{z} = 1, z \ne 1.
\end{align}
\item Find the maximum value of the function 
\begin{align}
f(x) = 2x^3-15x^2+36x-48
\end{align}
on the set 
\begin{align}
A = \cbrak{x : x^2+20 \le 9x}
\end{align}
\item Find the minimum distance of a point on the curve 
\begin{align}
\vec{x}^T\myvec{1 & 0 \\ 0 & 1}\vec{x} + \myvec{0 -1}\vec{x}=4
\end{align}
%
from the origin.
\item Find 
\begin{align}
\min_{z}\abs{z+\frac{1}{2}}
\\
s.t\quad \abs{z} \ge 2
\end{align}
\item Find  the minimum value of
\begin{align}
\tan A + \tan B
\\
s.t \quad A+B=6,
\\
A, B \ge 0
\end{align}
\item Show that 
\begin{multline}
\sin A_1+\sin A_2 + \dots + \sin A_n 
\\
\le n \sin \brak{\frac{A_1+A_2+\dots + A_n}{n}}
\\
0 < A_i < \pi, i = 1, 2, \dots , n, n \ge 1
\end{multline}
\item Let $\vec{F}_1, \vec{F}_2$ be the foci of the standard ellipse with parameters $a$ and $b$.  If $\vec{P}$ be any point on the ellipse, find the maximum area of $\triangle PF_1F_2$.
\item A circle $\norm{\vec{x}}=1$ intersects the $X-$ axis at $\vec{P}$ and $\vec{Q}$.  Another circle with centre $\vec{Q}$ intersects this circle above the $X-$ axis at $\vec{R}$ and the line segent $PQ$ at $\vec{S}$.  Find the maximum area of $\triangle QSR$.
\item Let $\vec{M}$ be a fixed point in the first quadrant.   A line through $\vec{M}$ intersects the positive axes at $\vec{P}, \vec{Q}$ respectively. If $\vec{O}$ be the origin, find the minimum area of $\triangle OPQ$.
\end{enumerate}
\end{document}
