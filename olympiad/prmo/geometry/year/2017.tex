\item In a rectangle ABCD, E is the midpoint of AB; F is a point on AC such that BF is perpendicular to AC; and FE perpendicular to BD. Suppose $BC = \sqrt{3}$, find AB?

\item Suppose in the plane 10 pairwise non-parallel lines intersect one another. What is the maximum possible number of polygons that can be formed?

\item Let P be an interior point of a triangle ABC whose sidelengths are 26, 65, 78. The line through P parallel to BC meets the AB in K and AC in L. The line through P parallel to CA meets BC in M and BA in N. The line through P parallel to AB meets CA in S and CB in T. If KL, MN, ST are of equal lengths, find this common length?

\item Let ABCD be a rectangle and let E and F be points on CD and BC respectively such that are(ADE) = 16, area(CEF) = 9 and area(ABF) = 25. What is the area of triangle AEF?

\item Let AB and CD be two parallel chords in a circle with radius 5 such that the centre O lies between these chords. Suppose AB = 6, CD = 8. Suppose further that the area of the part of the circle lying between the chords AB and CD is 
$m\pi + n/k$, where m, n, k are positive integers with $gcd(m, n, k) = 1$. What is the value of $m + n + k$?

\item Let $\Omega_1$ be a circle with centre O and let AB be a diameter of $\Omega_1$. Let $P$ be a point on the segment OB different from O. Suppose another circle $\Omega_2$ with centre P lies in the interior of $\Omega_1$. Tangents are drawn form A and B to the circle $\Omega_2$ intersectiong $\Omega_1$ again at $A_1$ and $B_1$ respectively such that $A_1$ and $B_1$ are on the opposite sides of AB. Given that $A_1B = 5$, $AB_1 = 15$ and $OP = 10$, find the radius of $\Omega_1$.

\item Consider the areas of the 4 triangles obtained by drawing the diagonals AC and BD of a trapezium ABCD. The product of these areas,taken two at time, are computed. If among the 6 products so obtained, 2 products are 1296 and 576, determine the square root of the maximum possible area of the trapezium to the nearest integer.