\renewcommand{\theequation}{\theenumi}
\begin{enumerate}[label=\arabic*.,ref=\thesubsection.\theenumi]
\numberwithin{equation}{enumi}
	%
\item Sum of the angles of a quadrilateral is 360$\degree$. 
\\
\solution Draw the diagonal and use the fact that sum of the angles of a triangle is 180$\degree$.
\item  A diagonal of a parallelogram divides it into two congruent triangles. 
\\
\solution The alternate angles for the parallel sides are equal.  The diagonal is common.  Use ASA congruence.
%
\item  In a parallelogram, 
\begin{enumerate}
\item opposite sides are equal 
\item  opposite angles are equal
\item  diagonals bisect each other
\end{enumerate}
%
\solution Since the diagonal divides the parallelogram into two congruent triangles, all the above results follow.
%
\item  A quadrilateral is a parallelogram, if 
%
\begin{enumerate}
\item opposite sides are equal or 
\item  opposite angles are equal or 
\item  diagonals bisect each other or 
\item a pair of opposite sides is equal and parallel
\end{enumerate}
%
\solution All the above lead to a quadrilateral that has two parallel sides, by showing that the alternate angles are equal.
%
%
\item A rectangle is a parallelogram with one angle that is 90$\degree$.  Show that all angles of the rectangle are 90$\degree$.
%
\\
\solution Draw a diagonal.  Since the diagonal divides the rectangle into two congruent triangles, the angle opposite to the right angle is also 90$\degree$. Using congruence, it can be shown that the other two angles are equal.  Now use the fact that the sum of the angles of a quadrilateral is 360$\degree$.
%
\item  Diagonals of a rectangle bisect each other and are equal and vice-versa. 
%
\\
\solution Use Baudhayana's theorem for equality of diagonals.
%
\item  Diagonals of a rhombus bisect each other at right angles and vice-versa. 
%
\\
\solution The median of an isoceles triangle is also its perpendicular bisector.
%
\item  Diagonals of a square bisect each other at right angles and are equal, and vice-versa. 
%
\\
\solution A square has the properties of a rectangle as well as a rhombus.
%
%
\item  The quadrilateral formed by joining the mid-points of the sides of a quadrilateral, in order, is a parallelogram.
%
\\
\solution Draw one diagonal and use Problem \eqref{prob:quad_similar}.  Repeat for the other diagonal to show that the sides are parallel.
%
\item Two parallel lines l and m are intersected by a transversal p. Show that the quadrilateral formed by the bisectors of interior angles is a rectangle.
%
\item Show that the bisectors of angles of a parallelogram form a rectangle.
%
\item A quadrilateral is a parallelogram if a pair of opposite sides is equal and parallel.
%
\item $ABCD$ is a parallelogram in which $P$ and $Q$ are mid-points of opposite sides $AB$ and $CD$. If $AQ$ intersects $DP$ at $S$ and $BQ$ intersects $CP$ at $R$, show that: 
%
\begin{enumerate}
\item  $APCQ$ is a parallelogram. 
\item $DPBQ$ is a parallelogram. 
\item $PSQR$ is a parallelogram.
\end{enumerate}
%
\item $l, m$ and $n$ are three parallel lines intersected by transversals $p$ and $q$ such that $l, m$ and $n$ cut off equal intercepts $AB$ and $BC$ on $p$ . Show that $l, m$ and $n$ cut off equal intercepts $DE$ and $EF$ on $q$ also.
%
\item Parallelograms on the same base (or equal bases) and between the same parallels are equal in area.
\item Area of a parallelogram is the product of its base and the corresponding altitude. 
\item Parallelograms on the same base (or equal bases) and having equal areas lie between the same parallels.
\item If a parallelogram and a triangle are on the same base and between the same parallels, then area of the triangle is half the area of the parallelogram.
\item In parallelogram $ABCD$, two points $P$ and $Q$ are taken on diagonal $BD$ such that $DP = BQ$. show that \begin{enumerate}
 \item  $\triangle  APD  \cong   \triangle  CQB$ 
\item $AP = CQ$ \item  $\triangle  AQB  \cong   \triangle  CPD$ 
\item $AQ = CP$ 
\item $APCQ$ is a parallelogram
\end{enumerate}
\item $ABCD$ is a parallelogram and $AP$ and $CQ$ are perpendiculars from vertices $A$ and $C$ on diagonal $BD$ . Show that 
\begin{enumerate} 
\item  $\triangle  APB  \cong   \triangle  CQD $ 
\item $AP = CQ$
\end{enumerate}
%
\item In  $\triangle  ABC$ and  $\triangle  DEF, AB = DE, AB  \parallel  DE, BC = EF$ and $BC  \parallel  EF$. Vertices $A, B$ and $C$ are joined to vertices $D, E$ and $F$ respectively. Show that 
\begin{enumerate}
\item quadrilateral $ABED$ is a parallelogram 
\item quadrilateral $BEFC$ is a parallelogram 
\item $AD  \parallel  CF$ and $AD = CF$ 
\item quadrilateral $ACFD$ is a parallelogram 
\item $AC$ = $DF$ 
\item  $\triangle  ABC  \cong   \triangle  DEF$.
%
\end{enumerate}

\item $ABCD$ is a trapezium in which $AB$  $\parallel$  $CD$ and $AD = BC$. Show that 
\begin{enumerate} 
\item$\angle A$ =  $\angle B$  
\item  $\angle C  =  \angle D$  \item  $\triangle  ABC  \cong   \triangle  BAD$ 
\item diagonal $AC$ = diagonal $BD$ 
\end{enumerate}
%
\item $ABCD$ is a quadrilateral in which $P, Q, R$ and $S$ are mid-points of the sides $AB, BC, CD$ and $DA$ $AC$ is a diagonal. Show that 
\begin{enumerate} 
\item $SR$  $\parallel$  $AC$ and $SR =\frac{1}{ 2}AC$
\item $PQ = SR$ 
\item  $PQRS$  is a parallelogram.
\end{enumerate}
%
\item $ABCD$ is a rhombus and  $P, Q, R$ and $S$  are the mid-points of the sides  $AB, BC, CD$ and $DA$ respectively. Show that the quadrilateral  $PQRS$  is a rectangle.
\item $ABCD$ is a rectangle and  $P, Q, R$ and $S$  are mid-points of the sides  $AB, BC, CD$ and $DA$ respectively. Show that the quadrilateral  $PQRS$  is a rhombus.
\item $ABCD$ is a trapezium in which $AB  \parallel  DC, BD$ is a diagonal and $E$ is the mid-point of $AD$. A line is drawn through $E$ $\parallel$  $AB$ intersecting $BC$ at $F$. Show that $F$ is the mid-point of $BC$.
\item In a parallelogram $ABCD$, $E$ and $F$ are the mid-points of sides $AB$ and $CD$ respectively . Show that the line segments $AF$ and $EC$ trisect the diagonal $BD$.
\item Show that the line segments joining the mid-points of the opposite sides of a quadrilateral bisect each other.
\item $ABCD$ is a parallelogram in which $P$ and $Q$ are mid-points of opposite sides $AB$ and $CD$. If $AQ$ intersects $DP$ at $S$ and $BQ$ intersects $CP$ at $R$, show that: 
%
\begin{enumerate}
\item  $APCQ$ is a parallelogram. 
\item $DPBQ$ is a parallelogram. 
\item $PSQR$ is a parallelogram.
\end{enumerate}
%
\item $l, m$ and $n$ are three parallel lines intersected by transversals $p$ and $q$ such that $l, m$ and $n$ cut off equal intercepts $AB$ and $BC$ on $p$ . Show that $l, m$ and $n$ cut off equal intercepts $DE$ and $EF$ on $q$ also.
%
\item Diagonal $AC$ of a parallelogram $ABCD$ bisects $\angle A$ . show that 
\begin{enumerate}
\item it bisects  $\angle C$  also, 
\item $ABCD$ is a rhombus.
\end{enumerate}
%
\item $ABCD$ is a rhombus. Show that diagonal $AC$ bisects $\angle A$ as well as  $\angle C$  and diagonal $BD$ bisects  $\angle B$  as well as  $\angle D$ .
\item $ABCD$ is a rectangle in which diagonal $AC$ bisects $\angle A$ as well as  $\angle C$ . Show that 
\begin{enumerate}
\item $ABCD$ is a square 
\item diagonal $BD$ bisects  $\angle B$  as well as  $\angle D$ .
%
\end{enumerate}

\item If $E,F,G$ and $H$ are respectively the mid-points of the sides of a parallelogram $ABCD$, show that
\begin{align}
ar \brak{EFGH} =
\frac{1}{ 2}
ar \brak{ABCD} .
\end{align}
%
\item $P$ and $Q$ are any two points lying on the sides $DC$ and $AD$ respectively of a parallelogram $ABCD$. Show that $ar (APB) = ar (BQC)$.
%
\item P is a point in the interior of a parallelogram $ABCD$. Show that
\begin{enumerate}
\item $ar (APB) + ar (PCD) = \frac{1}{ 2}ar (ABCD)$
\item $ar (APD) + ar (PBC) = ar (APB) + ar (PCD)$
\end{enumerate}
%
\item $PQRS$ and $ABRS$ are parallelograms and $X$ is any point on side $BR$. show that 
\begin{enumerate} 
\item $ar (PQRS) = ar (ABRS)$
\item $ar (AX S) = \frac{1}{ 2} ar (PQRS)$
\end{enumerate}
%
\item A farmer was having a field in the form of a parallelogram $PQRS$. She took any point $A$ on $RS$ and joined it to points $P$ and $Q$. In how many parts the fields is divided? What are the shapes of these parts? The farmer wants to sow wheat and pulses in equal portions of the field separately. How should she do it?
%
\item $ABCD$ is a quadrilateral and $BE  \parallel  AC$ and also $BE$ meets $DC$ produced at $E$. Show that area of $ \triangle  ADE$ is equal to the area of the quadrilateral $ABCD$.
%
\item $E$ is any point on median $AD$ of a  $\triangle  ABC$. Show that $ar (ABE) = ar (ACE)$.
\item  In a $\triangle ABC, E$ is the mid-point of median $AD$. Show that $ar (BED) = \frac{1}{ 4}ar(ABC)$ .
\item  Show that the diagonals of a parallelogram divide it into four triangles of equal area.
\item   $ABC$ and $ABD$ are two triangles on the same base $AB$. If line- segment $CD$ is bisected by $AB$ at $O$, show that $ar(ABC) = ar (ABD)$.
%
\item $D$, $E$ and $F$ are respectively the mid-points of the sides $BC, CA$ and $AB$ of a $ \triangle  ABC$. show that 
\begin{enumerate}
\item $BDEF$ is a parallelogram. 
\item $ar (BDEF) =
\frac{1}{ 2}
ar (ABC)$
\end{enumerate}
%
\item   Diagonals $AC$ and $BD$ of quadrilateral $ABCD$ intersect at $O$ such that $OB = OD$. If $AB = CD$, then show that 
\begin{enumerate}
\item $ar (DOC) = ar (AOB)$
 \item $ar (DCB) = ar (ACB)$
\item $ar (DEF) =
\frac{1}{ 4}
ar (ABC)$ 
\end{enumerate}
\item $D$ and $E$ are points on sides $AB$ and $AC$ respectively of $ \triangle  ABC$ such that $ar (DBC) = ar (EBC)$. Prove that $DE  \parallel  BC$.
\item $XY$ is a line parallel to side $BC$ of a $\triangle ABC$. If $BE  \parallel  AC$ and $CF  \parallel  AB$ meet $XY$ at $E$ and $F$ respectively, show that
$ar (ABE) = ar (ACF)$.
\item The side $AB$ of a parallelogram $ABCD$ is produced to any point $P$. A line through $A$ and parallel to $CP$ meets $CB$ produced at $Q$ and then parallelogram $PBQR$ is completed. Show that $ar ($ABCD$) = ar (PBQR)$. \item Diagonals $AC$ and $BD$ of a trapezium $ABCD$ with $AB  \parallel  DC$ intersect each other at $O$. Prove that $ar (AOD) = ar (BOC)$.
\item  $ABCDE$ is a pentagon. A line through $B$ parallel to $AC$ meets $DC$ produced at $F$. Show that 
\begin{enumerate}
\item $ar (ACB) = ar (ACF)$
 \item $ar (AEDF) = ar (ABCDE)$
. 
\end{enumerate}
\item A villager Itwaari has a plot of land of the shape of a quadrilateral. The Gram Panchayat of the village decided to take over some portion of his plot from one of the corners to construct a Health Centre. Itwaari agrees to the above proposal with the condition that he should be given equal amount of land in lieu of his land adjoining his plot so as to form a triangular plot. Explain how this proposal will be implemented.
\item $ABCD$ is a trapezium with $AB  \parallel  DC$. A line parallel to $AC$ intersects $AB$ at $X$ and $BC$ at $Y$. Prove that $ar (ADX) = ar (ACY)$.
\item  $AP  \parallel  BQ  \parallel  CR$. Prove that $ar (AQC) = ar (PBR)$.
\item Diagonals $AC$ and $BD$ of a quadrilateral $ABCD$ intersect at $O$ in such a way that $ar (AOD) = ar (BOC)$. Prove that $ABCD$ is a trapezium.
\item  $AB \parallel DC \parallel RP$.  $ar (DRC) = ar (DPC)$ and $ar (BDP) = ar (ARC)$. Show that both the quadrilaterals $ABCD$ and $DCPR$ are trapeziums.

\item Parallelogram $ABCD$ and rectangle $ABEF$ are on the same base $AB$ and have equal areas. Show that the perimeter of the parallelogram is greater than that of the rectangle.
\item  In $\triangle ABC$,  $D$ and $E$ are two points on $BC$ such that $BD = DE = EC$. Show that $ar (ABD) = ar (ADE) = ar (AEC)$.
\item $ABCD, DCFE$ and $ABFE$ are parallelograms. Show that ar$ (ADE) = ar (BCF)$.
\item  $ABCD$ is a parallelogram and $BC$ is produced to a point $Q$ such that $AD = CQ$. If $AQ$ intersect $DC$ at $P$, show that $ar (BPC) = ar (DPQ)$.
$ABC$ and $BDE$ are two equilateral triangles such that $D$ is the mid-point of $BC$. If $AE$ intersects $BC$ at$ F$, show that 
\begin{enumerate}
\item $ar (BDE) = \frac{1}{ 4} ar (ABC)$
\item $ar (BDE) = \frac{1}{ 2} ar (BAE)$
\item $ar (ABC) = 2 ar (BEC)$
 \item $ar (BFE) = ar (AFD)$ 
\item $ar (BFE) = 2 ar (FED)$
\item $ar (FED) =
\frac{1}{ 8}
ar (AFC)$
\end{enumerate}
\item Diagonals $AC$ and $BD$ of a quadrilateral $ABCD$ intersect each other at $P$. Show that $ar (APB)  \times  ar (CPD) = ar (APD)  \times  ar (BPC)$.
\item  $P$ and $Q$ are respectively the mid-points of sides AB and BC of a $\triangle ABC$ and $R$ is the mid-point of $AP$, show that 
\begin{enumerate}
\item $ar (PRQ) = \frac{1 }{2}ar (ARC) $
\item $ar (PBQ) = ar (ARC)$
\item $ar (RQC) =
\frac{3}{ 8}
ar (ABC)$
\end{enumerate}
%
\item $ABC$ is a right triangle right angled at $A$. $BCED$, $ACFG$ and $ABMN$ are
squares on the sides $BC, CA$ and $AB$ respectively. Line segment $AX \perp  DE$ meets $BC$ at $Y$. Show that 
\begin{enumerate}
\item $ \triangle  MBC \cong  \triangle  ABD$
\item $ar (BYXD) = ar (ABMN)$ \item $ar (CYXE) = 2 ar (FCB)$
\item $ar (BYXD) = 2 ar (MBC)$ 
\item $ \triangle  FCB \cong  \triangle  ACE$
\item $ar (CYXE) = ar (ACFG)$
\item  $ar (BCED) = ar (ABMN) + ar (ACFG)$
\end{enumerate}
\item $L$ is a point on the diagonal $AC$ of quadrilateral $ABCD$.  If LM || CB and LN || CD, prove that $\frac{AM}{AB}=\frac{ AN}{  AD}$

\end{enumerate}
