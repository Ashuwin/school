\renewcommand{\theequation}{\theenumi}
\begin{enumerate}[label=\arabic*.,ref=\thesubsection.\theenumi]
\numberwithin{equation}{enumi}
\item Find the coordinates of a point $\vec{A}$, where $AB$ is the diameter of a circle whose centre is \myvec{2,-3} and $\vec{B} = \myvec{1\\4}$.
\item Find the centre $O$f a circle passing through the points \myvec{6\\-6}, \myvec{3\\-7} and  \myvec{3\\3}.
\item Sketch the circles with 
\begin{enumerate}
\item centre \myvec{0\\2} and radius 2
\item centre \myvec{-2\\32} and radius 4
\item centre $\myvec{\frac{1}{2}\\ \frac{1}{4}}$ and radius $\frac{1}{12}$.
\item centre \myvec{1\\1} and radius $\sqrt{2}$.
\item centre \myvec{-a\\-b} and radius $\sqrt{a^2-b^2}$.
\end{enumerate}
\item 
\item Sketch the circles with equation
\begin{enumerate}
\item $\norm{\vec{x}-\myvec{5\\-3}}^2 = 36$
\item $\vec{x}^T\vec{x}-\myvec{4\\8}\vec{x} -45= 0$
\item $\vec{x}^T\vec{x}-\myvec{8\\-10}\vec{x} -12= 0$
\item $2\vec{x}^T\vec{x}-\myvec{1\\0}\vec{x} = 0$
\end{enumerate}
%
\item Find the equation of the circle passing through the points \myvec{4\\1} and \myvec{6\\5} and whose centre is on the line $\myvec{4 & 1}\vec{x} = 16$.
\item Find the equation of the circle passing through the points \myvec{2\\3} and \myvec{–1\\1} and whose centre is on the line $\myvec{1 & -3}\vec{x} = 11$.
\item Find the equation of the circle with radius 5 whose centre lies on x-axis and passes through the point \myvec{2\\3}.
\item Find the equation of the circle passing through \myvec{0\\0} and making intercepts a and b on the coordinate axes.
\item Find the equation of a circle with centre \myvec{2\\2} and passes through the point \myvec{4\\5}. 
\item  Does the point \myvec{–2.5\\ 3.5} lie inside, outside or on the circle $\vec{x}^T\vec{x} = 25$?
\item Find the locus of all the unit vectors in the xy-plane.
%
\item $ABCD$ is a cyclic quadrilateral in which $AC$ and $BD$ are its diagonals. If $\angle DBC = 55\degree$ and $\angle BAC = 45\degree$, find $\angle BCD$
%
\item Two circles of radii 5 cm and 3 cm intersect at two points and the distance between their centres is 4 cm. Find the length of the common chord.
%
%
\item  A,B and C are three points on a circle with centre $O$ such that $\angle BOC = 30\degree $ and $ \angle AOB = 60\degree$. If D is a point on the circle other than the arc ABC, find $\angle ADC$.
%
\item $ \angle PQR = 100\degree$, where $P, Q$ and R are
points on a circle with centre $O$. Find $\angle OPR$.
\item $A, B, C, D$ are points on a circle such that $ \angle ABC = 69\degree, \angle ACB = 31\degree$, find
$\angle BDC$.
\item $A, B, C$ and $D$ are four points on a
circle. $AC$ and $BD$ intersect at a point $E$ such
that $\angle BEC = 130\degree$ and $\angle ECD = 20\degree$. Find $\angle BAC$.
\item $ABCD$ is a cyclic quadrilateral whose diagonals intersect at a point $E$. If $\angle DBC = 70\degree,
\angle BAC$ is $30\degree$, find $\angle BCD$. Further, if $AB = BC$, find $\angle ECD$.
%
\item Two chords $AB$ and $CD$ of lengths 5 cm and 11 cm respectively of a circle are parallel
to each other and are on opposite sides of its centre. If the distance between $AB$ and
$CD$ is 6 cm, find the radius of the circle.
\item The lengths of two parallel chords of a circle are 6 cm and 8 cm. If the smaller chord is
at distance 4 cm from the centre, what is the distance of the other chord from the
centre?
%
\item A tangent $PQ$ at a point $P$ of a circle of radius 5 cm meets a line through the centre $O$ at a point $Q$ so that $OQ =$ 12 cm. Find the length of $PQ$.
%
\item $PQ$ is a chord of length 8 cm of a circle of radius 5 cm. The tangents at $P$ and $Q$ intersect at a point $T$. Find the length $TP$.
%
\item From a point $Q$, the length of the tangent to a circle is 24 cm and the distance of $Q$ from the centre is 25 cm. Find the radius of the circle is 
\item  If $TP$ and $TQ$ are the two tangents to a circle with centre $O$ so that  $\angle  POQ = 110 \degree $, then find  $\angle  PTQ$
\item  If tangents $PA$ and $PB$ from a point $P$ to a circle with centre $O$ are inclined to each other at angle of 80 $\degree$ , then find  $\angle  POA $
%
\item The length of a tangent from a point $A$ at distance 5 cm from the centre of the circle is 4 cm. Find the radius of the circle.
\item  Two concentric circles are of radii 5 cm and 3 cm. Find the length of the chord of the larger circle which touches the smaller circle.
%
\item A $\triangle ABC$ is drawn to circumscribe a circle of radius 4 cm such that the segments $BD$ and $DC$ into which $BC$ is divided by the point of contact $D$ are of lengths 8 cm and 6 cm respectively. Find the sides $AB$ and $AC$.
%
\item The cost of fencing a circular field at the rate of \rupee 24 per metre is \rupee 5280. The field is to be ploughed at the rate of \rupee 0.50 per $m^2$.  Find the cost of ploughing the field.	
\item The radii of two circles are 19 cm and 9 cm respectively. Find the radius of the circle which has circumference equal to the sum of the circumferences of the two circles.
\item The radii of two circles are 8 cm and 6 cm respectively. Find the radius of the circle having area equal to the sum of the areas of the two circles.
\item A circular  archery target is marked with its five scoring regions from the centre outwards as Gold, Red, Blue, Black and White. The diameter of the region representing Gold score is 21 cm and each of the other bands is 10.5 cm wide. Find the area of each of the five scoring regions.
\item The wheels of a car are of diameter 80 cm each. How many complete revolutions does each wheel make in 10 minutes when the car is travelling at a speed of 66 km per hour?
%
 \item Find the area of the sector of a circle with radius 4 cm and of angle 30 $\degree$ . Also, find the area of the corresponding major sector.
\item Find the area of the segment $AYB$, if radius of the circle is 21 cm and
 $\angle  AOB = 120 \degree$ .
%
\item Find the area of a sector of a circle with radius 6 cm if angle of the sector is 60 $\degree$ . 
\item Find the area of a quadrant of a circle whose circumference is 22 cm. 3. The length of the minute hand of a clock is 14 cm. Find the area swept by the minute hand in 5 minutes.
\item A chord of a circle of radius 10 cm subtends a right angle at the centre. Find the area of the corresponding : 
\begin{enumerate}
\item minor segment 
\item major sector.
\end{enumerate}

\item In a circle of radius 21 cm, an arc subtends an angle of 60 $\degree$  at the centre. Find: 
\begin{enumerate}
\item the length of the arc 
\item area of the sector formed by the arc 
\item area of the segment formed by the corresponding chord
\end{enumerate}
\item A chord of a circle of radius 15 cm subtends an angle of 60 $\degree$  at the centre. Find the areas of the corresponding minor and major segments of the circle. 
\item A chord of a circle of radius 12 cm subtends an angle of 120 $\degree$  at the centre. Find the area of the corresponding segment of the circle. 
\item A horse is tied to a peg at one corner of a square shaped grass field of side 15 m by means of a 5 m long rope. Find 
\begin{enumerate}
\item the area of that part of the field in which the horse can graze.
\item the increase in the grazing area if the rope were 10 m long instead of 5 m.
\end{enumerate}
\item A brooch is made with silver wire in the form of a circle with diameter 35 mm. The wire is also used in making 5 diameters which divide the circle into 10 equal sectors. Find : 
\begin{enumerate}
\item the total length of the silver wire required. 
\item the area of each sector of the brooch
\end{enumerate}
\item An umbrella has 8 ribs which are equally spaced. Assuming umbrella to be a flat circle of radius 45 cm, find the area between the two consecutive ribs of the umbrella.
\item A car has two wipers which do not overlap. Each wiper has a blade of length 25 cm sweeping through an angle of 115 $\degree$ . Find the total area cleaned at each sweep of the blades.
\item  To warn ships for underwater rocks, a lighthouse spreads a red coloured light over a sector of angle 80 $\degree$  to a distance of 16.5 km. Find the area of the sea over which the ships are warned.
\item  A round table cover has six equal designs. If the radius of the cover is 28 cm, find the cost of making the designs at the rate of \rupee 0.35 per $cm^2$
. 
%
\item Two circular flower beds are located on opposite sides of a square lawn $ABCD$ of side 56 m. If the centre $O$f each circular flower bed is the point of intersection O of the diagonals of the square lawn, find the sum of the areas of the lawn and the flower beds.
%
\item Four circles are inscribed  inside a square $ABCD$ of side 14 cm such that each one touches exernally two adjacent sides of the square and two  other circles.  Find the region between the circles and the square.
\item  $ABCD$ is a square of side 10 cm and semicircles are drawn with each side of the square as diameter. Find the area enclosed by the circular arcs.
%
\item P is a point on the semi-circle formed with diameter $QR$. Find the area between the semi-circle and $\triangle PQR$ if $PQ$ = 24 cm, PR = 7 cm and O is the centre $O$f the circle.
\item $AC$ and $BD$ are two arcs on concentric circles with radii 14 cm and 7 cm respectively, such that $\angle AOC = 40\degree$.  Find the area of the region $ABDC$.
%
\item Find the area between a square $ABCD$ of side 14cm and the semi circles $APD$ and $BPC$.
\item Find the area of the  region enclosed by  a circular arc of radius 6 cm drawn with vertex $O$ of an equilateral triangle OAB of side 12 cm as centre.
%
\item From each corner of a square of side 4 cm a quadrant of a circle of radius 1 cm is cut and also a circle of diameter 2 cm is cut. Find the area of the remaining portion of the square.\item In a circular table cover of radius 32 cm, a design is formed leaving an equilateral $\triangle ABC$ in the middle. Find the area of the design.
%
\item $ABCD$ is a square of side 14 cm. With centres A, B, C and D, four circles are drawn such that each circle touches externally two of the remaining three circles. Find the area within the square that lies outside the circles.
\item The left and right ends of a racing track are semicircular.
The distance between the two inner parallel line segments is 60 m and they are each 106 m long. If the track is 10 m wide, find : 
\begin{enumerate}
\item the distance around the track along its inner edge 
\item the area of the track.
\end{enumerate}
\item $AB$ and $CD$ are two diameters of a circle (with centre $O$) perpendicular to each other and OD is the diameter of a  smaller circle inside. If $OA$ = 7 cm, find the area of the smaller circle.
\item The area of an equilateral $\triangle ABC$ is 17320.5 $cm^2$
. With each vertex of the triangle as centre, a circle is drawn with radius equal to half the length of the side of the triangle. Find the area of region within the triangle but outside the circles. 
\item On a square handkerchief, nine circular designs are inscribed touching each other, each of radius 7 cm. Find the area of the remaining portion of the handkerchief.
\item $OACB$ is a quadrant of a circle with centre $O$ and radius 3.5 cm. $D$ is a point on $OA$.  If OD = 2 cm, find the area of the
\begin{enumerate}
\item quadrant $OACB$,
 \item the region between the quadrant and $\triangle OBD$.
\end{enumerate}
\item A square $OABC$ is inscribed in a quadrant $OPBQ$. If $OA$ = 20 cm, find the area between the square and the quadrant.
\item $AB$ and $CD$ are respectively arcs of two concentric circles of radii 21 cm and 7 cm and centre $O$.  If  $\angle  AOB = 30 \degree$ , find the area of the region $ABCD$.
\item ABC is a quadrant of a circle of radius 14 cm and a semicircle is drawn with $BC$ as diameter. Find the area of the crescent formed.
\item Find the area common between the two quadrants of circles of radius 8 cm each if the centres of the circles lie on opposite sides of a square.
\item Find the area of the sector of a circle with radius 4 cm and of angle 30$\degree$. Also, find the area of the corresponding major sector.

\end{enumerate}
