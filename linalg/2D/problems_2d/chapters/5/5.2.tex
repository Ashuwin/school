In the following problems, the equation of a parabola is assumed to be $y^2=4ax$ 
and capital letters refer to Fig. \ref{fig:parabola} unless the contrary is stated.
\begin{enumerate}[1.]
\item Prove that as $P$ moves along the curve $GP^2$ varies as $SG$.
\item Prove that, if $PP_1$ is a double ordinate and $PX$ meets the curve in $Q$,
then $P_1Q$ passes through $S$.
\item Prove that, if $PSP_1$ is a focal chord and $AP$, $AP_1$ meet the latus rectum in $Q$, $Q_1$,
the $SQ$, $SQ_1$ are equal to the ordinates of $P_1$ and $P$.
\item Prove that, if the tangents at $P$, $Q$ intersect in $T$, then
\begin{align*}
ST^2=SP.SQ.
\end{align*}
\item Prove that, if the tangent at the end $Q$ of a focal chord $PSQ$ meets the latus rectum
in $R$, then $PGR$ is a right angle.
\item Tangents at $P$, $Q$, $R$ on a parabola form a triangle $UVW$.  Show that the centroids of
the triangles $PQR$ and $UVW$ lie on the same diameter.
\item Prove that, if the difference of the ordinates of two points on a
parabola is constant, then the locus of the point of intersection of the tangents at
these points is an equal parabola.
\item Prove that, if two tangents intercept a fixed length on the tangent at the vertex, the locus
of their intersections is an equal parabola.
\item The chord of contact of tangents from any point $Q$ meets the tangent
at the vertex in $R$.  Prove that the tangent of the angle which $AQ$ makes with the axis
is $\frac{2a}{AR}$.
\item The parameters, $t$, $t_1$ of two points on a parabola are
connected by the relation $t=k^2t_1$, prove that the tangents at the points intersect on the
curve
\begin{align*}
y^2=\brak{k+\frac{1}{k}}^2ax.
\end{align*}
\item Show that the length of the normal chord at the point of parameter $t$ is
\begin{align*}
\frac{4a}{t^2}\brak{1+t^2}^{\frac{3}{2}}
\end{align*}
\item Prove that the locus of intersection of tangents at the ends of a normal
chord is
\begin{align*}
\brak{x+2a}y^2+4a^2=0.
\end{align*}
\item Prove that the locus of the point of intersection of perpendicular normals
is the parabola $y^2=a\brak{x-3a}$.
\item Prove that if the tangents at two points on the parabola intersect in
the point $\brak{x_1,y_1}$, the corresponding normals intersect in the
point 
\begin{align*}
\brak{2a-x_1+\frac{y_1^2}{a},-\frac{x_1y_1}{a}}.
\end{align*}
\item Show that, if the tangent at $P$ meets the latus rectum in $K$, then $SK$ is a mean
proportional between the segments of the focal chord through $P$.
\item Show that, if the tangents from $Q$ to the parabola form with the tangent at the vertex,
a triangle of constant area $c^2$, then the locus
of $Q$ is the curve
\begin{align*}
x^2\brak{y^2-4ax} = 4c^4.
\end{align*}
\item Show that the normals at the ends of each of a series of parallel chords of a parabola intersect
on a fixed straight line, itself a normal to the parabola.
\item $P$, $Q$ are points on the parabola subtending a constant angle
$\alpha$ at the vertex.  Show that the locus of the intersection of the tangents
at $P$, $Q$ is the curve 
\begin{align*}
4\brak{y^2-4ax} = \brak{x+4a}^2\tan^2\alpha
\end{align*}
\item Prove that the exterior angle between two tangents to a parabola is equal to the
angle which either of them subtends at the focus.
\item Two perpendicular focal chords of a parabola meet the directrix in $T$ and $T_1$.
Show that the tangents to the parabola which are parallel to these chords intersect in the middle point of $TT_1$.
\item Prove that, if the tangents at the points $Q$, $R$ intersect at $P$, then 
\begin{align*}
PQ^2:PR^2=SQ:SR
\end{align*}
\item The tangents at any two points $P$, $Q$ meet at $T$ and the normals meet at $N$.  Prove that the projection
of $TN$ on the axis is equal to the sum of the distances of $P$ and $Q$ from the directrix.
\item Prove that the circumscribing circle of the triangle formed by three tangents to a parabola
passes through the focus.
\item $PQ$ is a chord of a parabola normal at $P$; the circle on $PQ$ as diameter cuts the parabola again in $R$.  Prove that the projection
of $QR$ on the axis is twice the latus rectum.
\item Prove that the distance between a tangent and the parallel normal is $a\text{cosec}\theta\sec^2\theta$, where $\theta$ is the
angle which either makes with the axis.
\item Prove that, if the normals at $P$ and $Q$ intersect on the curve, then $PQ$ cuts the axis in
a fixed point.
\item Prove that, if the normals at $P$ and $Q$ meet at the point $R$ $\brak{x_1,y_1}$ on the parabola, and the tangents at $P$ and $Q$
meet at $T$, then
\begin{align*}
TP.TQ=\frac{1}{2}\brak{x_1-8a}\sqrt{y_1^2+4a^2}.
\end{align*}
\item Show that, in the last problem, as $R$ moves along the parabola, the middle point of $PQ$
always lies on the parabola
\begin{align*}
y^2=2a\brak{x+2a}.
\end{align*}
\item Prove that the area of the triangle formed by the tangents at the points $t_1,t_2$, and their chord of 
contact is
\begin{align*}
\frac{1}{2}a^2\brak{t_1-t_2}^2
\end{align*}
\item Prove that the area of the triangle formed by three points $t_1$, $t_2$, $t_3$ on the parabola is
\begin{align*}
a^2\brak{t_2-t_3}\brak{t_3-t_1}\brak{t_1-t_2}
\end{align*}
and that this is double the area of the triangle formed by the tangents at these points.
\item Prove that, if a line through any point $P \brak{x_1,y_1}$ making an angle $\theta$ with the axis meets the parabola
at $Q$ and $R$, then
\begin{align*}
PQ.PR = \brak{y_1^2-4ax_1}\text{cosec}^2\theta.
\end{align*}
\item Two chords $QR$, $Q_1R_1$ of a parabola meet at $O$, and the diameters bisecting them meet the curve at $P$ and $P_1$.  Prove that
\begin{align*}
QO.OR:Q_1O.OR_1=SP:SP_1
\end{align*}
\item Show that, if $P$ is on the parabola, the length of the chord through $P$ that makes an angle $\theta$ with the axis is
\begin{align*}
4a\sin\brak{\alpha-\theta}\text{cosec}^2\theta\text{cosec}\alpha
\end{align*}
where $\alpha$ is the inclination of the tangent at $P$ to the axis.
\item Show that the locus of the middle point of a chord which passes through the fixed point $\brak{h,k}$ is the parabola
\begin{align*}
y^2-ky = 2a\brak{x-h}
\end{align*}
\item A tangent to the parabola $y^2+4bx=0$ meets the parabola $y^2=4ax$ at $P$, $Q$.  Prove that the locus of the middle point of $PQ$
is
\begin{align*}
y^2\brak{2a+b}=4a^2x.
\end{align*}
\item Prove that the polar of the focus of a parabola is the directrix.
\item Prove that, if a chord of the parabola subtends a right angle at the vertex, the locus of its pole is $x+4a=0$.
\item Show that, if parabolas $y^2=4ax$ are drawn corresponding to different values of $a$, the feet of the perpendiculars
from a fixed point on its polar lines all lie on a circle passing through the point.
\item Prove that, if from a point $Q \brak{x_1,y_1}$ a perpendicular be drawn to the polar
of $Q$ with regard to the parabola cutting it in $R$ and the axis in $G$, then
\begin{align*}
SG=SR=x_1+a
\end{align*}
\item Prove that, if the diameter through a point $P$ of a parabola meets any chord in $O$ and the tangents at the
ends of the chord in $T, T_1$, then
\begin{align*}
PO^2=PT.PT_1
\end{align*}
\item $QQ_1$ is a chord of a parabola and $TOR$ is a diameter which meets the tangent at $Q$ in $T$, the curve in $O$ and $QQ_1$ in $R$.  Prove
that 
\begin{align*}
TO:OR=QR:RQ_1
\end{align*}
\item Prove that if the normals at three points $P$, $Q$, $R$ on a parabola concur, then the points $P$, $Q$, $R$ and the vertex of
the parabola are concyclic.
\item Prove that in general, two members of the family of parabolas $y^2=4a\brak{x+a}$, where $a$ is the parameter specifying members
of the family, pass through any assigned point of the plane, and that these two parabolas cuut orthogonally at $P$.
\end{enumerate}