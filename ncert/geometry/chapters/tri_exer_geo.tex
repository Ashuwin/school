\renewcommand{\theequation}{\theenumi}
\begin{enumerate}[label=\arabic*.,ref=\thesubsection.\theenumi]
\numberwithin{equation}{enumi}
%\chapter{The Optimum Receiver}
%\item Angles opposite to equal sides of a triangle are equal. 
%\label{prob:tri_ang_side_eq}
%\\
%\solution Using the sine formula in \eqref{eq:tri_sin_form},%
%\begin{align}
%\frac{\sin A}{a} = \frac{\sin B}{b}
%\end{align}
%%
%Thus, if $A=B$, $\sin A = \sin B \implies a =b$.
\item  Sides opposite to equal angles of a triangle are equal. 
%\\
%\solution Use \eqref{eq:tri_sin_form} and the argument in Problem \ref{prob:tri_ang_side_eq}
%
\item  Each angle of an equilateral triangle is of 60$\degree$. 
%\\
%\solution In an equilateral $\triangle$, 
%%
%\begin{align}
%A=B=C.&
%\\
%\because A+B+C = 180\degree, 3A = 180\degree&
%\\
%\implies A = 60\degree&
%\end{align}
%


%\subsection{Problem}
\item Triangles on the same base (or equal bases) and between the same parallels are equal in area.
\item Triangles on the same base (or equal bases) and having equal areas lie between the same parallels.
\item In $\triangle ABC$, the bisector $AD$ of $\angle  A$ is perpendicular to side $BC$. Show that $AB = AC$ and $\triangle ABC$ is isosceles.
\item $E$ and $F$ are respectively the mid-points of equal sides $AB$ and AC of $\triangle ABC$. Show that $BF = CE$. 
\item In an isosceles $\triangle ABC$ with $AB$ = AC, D and E are points on $BC$ such that $BE = CD$. Show that $AD = AE$. 
%
\item $AB$ is a line-segment. $P$ and $Q$ are points on opposite sides of $AB$ such that each of them is equidistant from the points $A$ and $B$. Show that the line $PQ $ is the perpendicular bisector of $AB$.
%
\item $P$ is a point equidistant from two lines $l$ and $m$ intersecting at point $A$.  Show that the line  $AP$  bisects the angle between them.
%
\item $D$ is a point on side $BC$ of $\triangle  ABC$ such that $AD = AC$. Show that $AB > AD$

%
\item $AB$ is a line segment and line $l$ is its perpendicular bisector. If a point $P$ lies on $l$, show that $P$ is equidistant from $A$ and $B$.
\item Line-segment $AB$ is parallel to another line-segment $CD$. $O$ is the mid-point of $AD$. Show that 
\begin{enumerate}
\item  $\triangle AOB \cong \triangle DOC$ 
\item  $O$ is also the mid-point of $BC$.
\end{enumerate}
%
\item In quadrilateral $ACBD, AC = AD$ and $AB$ bisects $\angle  A$. Show that $\triangle  ABC \cong \triangle  ABD$. What can you say about $BC$ and $BD$?
%
\item $ABCD$ is a quadrilateral in which $AD = BC$ and $\angle  DAB = \angle  CBA$ . Prove that
\begin{enumerate}
\item  $\triangle  ABD \cong  \triangle  BAC $
\item $ BD = AC $
\item  $\angle  ABD = \angle  BAC$.
\end{enumerate}
%
\item $l$ and $m$ are two parallel lines intersected by another pair of parallel lines p and q 
to form the quadrilateral $ABCD$. Show that $\triangle  ABC \cong  \triangle  CDA$.
%
\item Line $l$ is the bisector of $ \angle  A$ and $B$ is any point on $l$. $BP$ and $BQ$ are perpendiculars from $B$ to the arms of $\angle  A$ (see Fig. 7.20). Show that: 
\begin{enumerate}
\item  $\triangle  APB \cong  \triangle  AQB$ 
\item  $BP = BQ$ or $B$ is equidistant from the arms of $\angle  A$.
\end{enumerate}
%
\item $ABCE$ is a quadrilateral and $D$ is a point on $BC$ such that, $AC = AE, AB = AD$ and $\angle  BAD = \angle  EAC$. Show that $BC = DE$.
%
\item In right triangle $ABC$, right angled at $C, M$ is the mid-point of hypotenuse $AB$. $C$ is joined to $M$ and produced to a point $D$ such that $DM = CM$. Point $D$ is joined to point $B$.
Show that: 
\begin{enumerate}
\item $ \triangle  AMC \cong  \triangle  BMD $
\item $\angle  DBC$ is a right angle. 
\item $\triangle  DBC \cong  \triangle  ACB$
\item $ CM = \frac{1}{ 2} AB$
\end{enumerate}
%
\item In an isosceles $\triangle ABC$, with $AB = AC$, the bisectors of $\angle B$ and $\angle C$ intersect each other at $O$. Join $A$ to $O$. Show that :
\begin{enumerate} 
\item $OB = OC$ 
\item $AO$ bisects $\angle A$
\end{enumerate}
\item In $\triangle ABC$, $AD$ is the perpendicular bisector of $BC$. Show that $\triangle ABC$ is an isosceles triangle in which $AB = AC$.
\item $ABC$ is an isosceles triangle in which altitudes $BE$ and $CF$ are drawn to equal sides $AC$ and $AB$ respectively . Show that these altitudes are equal.
%
\item $ABC$ is a triangle in which altitudes $BE$ and $CF$ to sides $AC$ and $AB$ are equal. Show that
%
\begin{enumerate} 
\item $\triangle  ABE \cong  \triangle  ACF $
\item  $AB = AC$, i.e., $ABC$ is an isosceles triangle.
\end{enumerate}
%
\item $ABC$ and $DBC$ are two isosceles triangles on the same base $BC$. Show that $\angle ABD = \angle ACD$.
%
\item  $\triangle  ABC$ and $\triangle  DBC$ are two isosceles triangles on the same base $BC$ and vertices $A$ and $D$ are on the same side of $BC$. If $AD$ is extended to intersect $BC$ at $P$, show that
\begin{enumerate}
\item $\triangle  ABD \cong  \triangle  ACD $
\item $\triangle  ABP \cong  \triangle  ACP $
\item $AP$ bisects $\angle  A$ as well as $\angle  D$. 
\item $AP$ is the perpendicular bisector of $BC$.
\end{enumerate}
\item $AD$ is an altitude of an isosceles $\triangle ABC$ in which $AB = AC$. Show that 
\begin{enumerate}
\item $AD$ bisects $BC$
\item $AD$ bisects $\angle  A$. 
\end{enumerate}

\item  Two sides $AB$ and $BC$ and median $AM$ of one triangle $ABC$ are respectively equal to sides $PQ$ and $QR$ and median $PN$ of $\triangle  PQR$. Show that: 
\begin{enumerate}
\item $\triangle  ABM \cong  \triangle  PQN $
\item $\triangle  ABC \cong  \triangle  PQR$
\end{enumerate}
\item  $BE$ and $CF$ are two equal altitudes of a triangle $ABC$. Using RHS congruence rule, prove that the triangle $ABC$ is isosceles.
\item  $ABC$ is an isosceles triangle with $AB = AC$. Draw $AP \perp BC$ to show that $\angle  B = \angle  C$.
%
\item $\triangle ABC$ is an isosceles triangle in which $AB = AC$. Side $BA$ is produced to $D$ such that $AD = AB$. Show that $\angle BCD$ is a right angle.
%
\item $ABC$ is a right angled triangle in which $\angle A$ = 90$\degree$ and $AB = AC$. Find $\angle B$ and $\angle C$.
%
\item Show that in a right angled triangle, the hypotenuse is the longest side.
\item Sides AB and AC of $\triangle  ABC$ are extended to points P and Q respectively. Also, $\angle  PBC < \angle  QCB$. Show that $AC > AB$.

\item Line segments $AD$ and $BC$ intersect at $O$ and form $\triangle OAB$ and $\triangle ODC$. $\angle  B < \angle  A$ and $\angle  C < \angle  D$. Show that $AD < BC$.

\item $AB$ and $CD$ are respectively the smallest and longest sides of a quadrilateral $ABCD$. Show that $\angle  A > \angle  C$ and $\angle  B > \angle  D$.
%
\item In $\triangle PQR,  PR > PQ$ and $PS$ bisects $\angle  QPR$. Prove that $\angle  PSR > \angle  PSQ$.
%
\item Show that of all line segments drawn from a given point not on it, the perpendicular line segment is the shortest.
%
\item $ABCD$ is a trapezium with $AB || DC$. $E and F$ are points on non-parallel sides $AD$ and $BC$ respectively such that $EF$ is parallel to $AB$
. Show that
$\frac{AE}{ED}=\frac{ BF}{  FC}$ .
\item \frac{PS}{ SQ} =
=
\item $ST$ is a line joining two points on $PQ$ and $PR$ in $\triangle PQR$.  If $\frac{PT}{ TR}$ and $∠ PRQ$, prove that $PQR$ is an isosceles triangle.
\item $D$ is a point on $AB$ and $E, F$ are points on $BC$ such that $DE || AC$ and $DF || AE$. Prove that $\frac{BF} {FE} =\frac{BE}  {EC}$
%
\item $O$ is a point in the interior of $\triangle PQR$. $D$ is a point on $OP$.  If $DE || OQ$ and $DF || OR$. Show that $EF || QR$.
\item $O$ is a point in the interior of $\triangle PQR$.  $A, B and C$ are points on $OP, OQ$ and $OR$ respectively such that $AB || PQ$ and $AC || PR$. Show that $BC || QR$.
%
\item $ABCD$ is a trapezium in which $AB || DC$ and its diagonals intersect each other at the point O. Show
that
$\frac{AO}{ BO}=\frac{CO}{  DO}$
%
\item The diagonals of a quadrilateral $ABCD$ intersect each other at the point O such that $\frac{AO}{ BO}=\frac{CO}{  DO}$.
 Show that $ABCD$ is a trapezium.
 \item $CM$ and $RN$ are respectively the medians of $∆ ABC$ and $∆ PQR$. If $∆ ABC ~ ∆ PQR$, prove that :
\begin{enumerate}
(i) $∆ AMC ~ ∆ PNR$ 
(ii) \frac{CM}{RN{=\frac{ AB}{  PQ}
\item $∆ CMB ~ ∆ RNQ$
%
\end{enumerate}
\item Diagonals $AC$ and $BD$ of a trapezium $ABCD$ with $AB || DC$ intersect each other at the point O. Using a similarity criterion for two triangles, show that
$\frac{OA}{OC} =  \frac{OB}  {OD}$
%
\item QR QT QS PR
= that ∆ PQS ~ ∆ TQR.
5. S and T are points on sides PR and QR of ∆ PQR such that ∠ P = ∠ RTS. Show that ∆ RPQ ~ ∆ RTS.
6. In Fig. 6.37, if ∆ ABE ≅ ∆ ACD, show that ∆ ADE ~ ∆ ABC.
7. In Fig. 6.38, altitudes AD and CE of ∆ ABC intersect each other at the point P. Show that:
(i) ∆ AEP ~ ∆ CDP (ii) ∆ ABD ~ ∆ CBE (iii) ∆ AEP ~ ∆ ADB (iv) ∆ PDC ~ ∆ BEC
8. E is a point on the side AD produced of a parallelogram ABCD and BE intersects CD at F. Show that ∆ ABE ~ ∆ CFB.
9. In Fig. 6.39, ABC and AMP are two right triangles, right angled at B and M respectively. Prove that: (i) ∆ ABC ~ ∆ AMP
(ii)
CA BC PA MP
=
10. CD and GH are respectively the bisectors of ∠ ACB and ∠ EGF such that D and H lie on sides AB and FE of ∆ ABC and ∆ EFG respectively. If ∆ ABC ~ ∆ FEG, show that:
(i)
CD AC GH FG
=
(ii) ∆ DCB ~ ∆ HGE (iii) ∆ DCA ~ ∆ HGF
Fig. 6.39 Fig. 6.38 Fig. 6.36 and ∠ 1 = ∠ 2. Show
In Fig. 6.40, E is a point on side CB produced of an isosceles triangle ABC with AB = AC. If AD ⊥ BC and EF ⊥ AC, prove that ∆ ABD ~ ∆ ECF.
12. Sides AB and BC and median AD of a triangle ABC are respectively proportional to sides PQ and QR and median PM of ∆ PQR (see Fig. 6.41). Show that ∆ ABC ~ ∆ PQR.
13. D is a point on the side BC of a triangle ABC such that ∠ ADC = ∠ BAC. Show that CA2
= CB.CD.
14. Sides AB and AC and median AD of a triangle ABC are respectively proportional to sides PQ and PR and median PM of another triangle PQR. Show that ∆ ABC ~ ∆ PQR.
Fig. 6.40 Fig. 6.41
15. A vertical pole of length 6 m casts a shadow 4 m long on the ground and at the same time a tower casts a shadow 28 m long. Find the height of the tower.
16. If AD and PM are medians of triangles ABC and PQR, respectively where ∆ ABC ~ ∆ PQR, prove that
AB AD PQ PM
= ⋅
the line segment XY is parallel to side AC of ∆ ABC and it divides the triangle into two parts of equal
areas. Find the ratio
AX AB
⋅Let ∆ ABC ~ ∆ DEF and their areas be, respectively, 64 cm2 15.4 cm, find BC.
and 121 cm2 . If EF =
2. Diagonals of a trapezium ABCD with AB || DC intersect each other at the point O. If AB = 2 CD, find the ratio of the areas of triangles AOB and COD.
In Fig. 6.44, ABC and DBC are two triangles on the same base BC. If AD intersects BC at O, show that
ar (ABC) AO ar (DBC) DO
= ⋅
4. If the areas of two similar triangles are equal, prove that they are congruent.
Fig. 6.44
5. D, E and F are respectively the mid-points of sides AB, BC and CA of ∆ ABC. Find the ratio of the areas of ∆ DEF and ∆ ABC.
6. Prove that the ratio of the areas of two similar triangles is equal to the square of the ratio of their corresponding medians.
7. Prove that the area of an equilateral triangle described on one side of a square is equal to half the area of the equilateral triangle described on one of its diagonals.
ABC and BDE are two equilateral triangles such that D is the mid-point of BC. Ratio of the areas of triangles ABC and BDE is
Sides of two similar triangles are in the ratio 4 : 9. Areas of these triangles are in the ratio
Fig. 6.48, ∠ ACB = 90° and CD ⊥ AB. Prove that Solution :
BC BD AD
A ladder is placed against a wall such that its foot is at a distance of 2.5 m from the wall and its top reaches a window 6 m above the ground. Find the length of the ladder.
In Fig. 6.50, if AD ⊥ BC, prove that AB2
+ CD2 = BD2 + AC2 .
\item BL and CM are medians of a triangle ABC right angled at A. Prove that 4 (BL2
) = 5 BC2 .
\item O is any point inside a rectangle ABCD (see Fig. 6.52). Prove that
Sides of triangles are given below. Determine which of them are right triangles. In case of a right triangle, write the length of its hypotenuse. (i) 7 cm, 24 cm, 25 cm (ii) 3 cm, 8 cm, 6 cm (iii) 50 cm, 80 cm, 100 cm (iv) 13 cm, 12 cm, 5 cm
2. PQR is a triangle right angled at P and M is a point on QR such that PM ⊥ QR. Show that PM2
= QM . MR.
3. In Fig. 6.53, ABD is a triangle right angled at A and AC ⊥ BD. Show that
(i) AB2 (ii) AC2 (iii) AD2
= BC . BD = BC . DC = BD . CD
4. ABC is an isosceles triangle right angled at C. Prove that AB2 5. ABC is an isosceles triangle with AC = BC. If AB2
= 2 AC2 triangle.
6. ABC is an equilateral triangle of side 2a. Find each of its altitudes. 7. Prove that the sum of the squares of the sides of a rhombus is equal to the sum of the squares of its diagonals.
Fig. 6.53 = 2AC2
. , prove that ABC is a right
ABC is an equilateral triangle of side 2a. Find each of its altitudes. 7. Prove that the sum of the squares of the sides of a rhombus is equal to the sum of the squares of its diagonals.
Fig. 6.54, O is a point in the interior of a triangle ABC, OD ⊥ BC, OE ⊥ AC and OF ⊥ AB. Show that
(i) OA2 (ii) AF2
+ OB2 + BD2
+ OC2 + CE2
– OD2 = AE2
– OE2 + CD2
– OF2 + BF2
= AF2 .
9. A ladder 10 m long reaches a window 8 m above the ground. Find the distance of the foot of the ladder from base of the wall.
10. A guy wire attached to a vertical pole of height 18 m is 24 m long and has a stake attached to the other end. How far from the base of the pole should the stake be driven so that the wire will be taut?
1
1 2
Fig. 6.54
11. An aeroplane leaves an airport and flies due north at a speed of 1000 km per hour. At the same time, another aeroplane leaves the same airport and flies due west at a speed of 1200 km per hour. How far apart will be the two planes after
hours?
12. Two poles of heights 6 m and 11 m stand on a plane ground. If the distance between the feet of the poles is 12 m, find the distance between their tops.
13. D and E are points on the sides CA and CB respectively of a triangle ABC right angled at C. Prove that AE2
+ BD2 = AB2 + DE2 .
14. The perpendicular from A on side BC of a ∆ ABC intersects BC at D such that DB = 3 CD (see Fig. 6.55). Prove that 2 AB2
= 2 AC2 + BC2 .
15. In an equilateral triangle ABC, D is a point on side BC such that BD = 9 AD2
= 7 AB2 .
16. In an equilateral triangle, prove that three times the square of one side is equal to four times the square of one of its altitudes.
17. Tick the correct answer and justify : In ∆ ABC, AB = 6 3 cm, AC = 12 cm and BC = 6 cm. The angle B is : (A) 120°
(C) 90°
(B) 60° (D) 45°
1 3
BC. Prove that Fig. 6.55 + BD2
, PS is the bisector of ∠ QPR of ∆ PQR. Prove that
QS PQ SR PR
= ⋅
, D is a point on hypotenuse AC of ∆ ABC, such that BD ⊥ AC, DM ⊥ BC and DN ⊥ AB. Prove that :
(i) DM2 = DN . MC = AB2 + BC2 + 2 BC . BD. (ii) DN2 = DM . AN
 (ii) DN2 = DM . AN
3. In Fig. 6.58, ABC is a triangle in which ∠ ABC > 90° and AD ⊥ CB produced. Prove that AC2
= AB2 + BC2 + 2 BC . BD.
ABC is a triangle in which ∠ ABC < 90° and AD ⊥ BC. Prove that AC2
= AB2 + BC2 – 2 BC . BD.
AD is a median of a triangle ABC and AM ⊥ BC. Prove that :
(i) AC2 = AD2 + BC . DM +
     
BC 2
2
AB2 = AD2 – BC . DM +
     
BC 2
2 (iii) AC2 + AB2 = 2 AD2 +
1 2
BC2
Prove that the sum of the squares of the diagonals of parallelogram is equal to the sum of the squares of its sides.
7. In Fig. 6.61, two chords AB and CD intersect each other at the point P. Prove that : (i) ∆ APC ~ ∆ DPB
(ii) AP . PB = CP . DP
two chords AB and CD of a circle intersect each other at the point P (when produced) outside the circle. Prove that (i) ∆ PAC ~ ∆ PDB
(ii) PA . PB = PC . PD
9. In Fig. 6.63, D is a point on side BC of ∆ ABC such that
BD AB CD AC
bisector of ∠ BAC.
10. Nazima is fly fishing in a stream. The tip of her fishing rod is 1.8 m above the surface of the water and the fly at the end of the string rests on the water 3.6 m away and 2.4 m from a point directly under the tip of the rod. Assuming that her string (from the tip of her rod to the fly) is taut, how much string does she have out (see Fig. 6.64)? If she pulls in the string at the rate of 5 cm per second, what will be the horizontal distance of the fly from her after 12 seconds?
Fig. 6.63 = ⋅ Prove that AD is the
\end{enumerate}


