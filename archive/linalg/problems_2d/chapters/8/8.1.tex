\renewcommand{\theequation}{\theenumi}
\begin{enumerate}[label=\arabic*.,ref=\thesubsection.\theenumi]
\numberwithin{equation}{enumi}
\item Find the polar equations of the curves whose cartesian equations are:
\begin{enumerate}
\item 
$
x^2+y^2+2gx+2hy+c = 0
$
\item 
$
x^2+y^2 = ax
$
\item 
$
x^2-y^2=c^2
$
\item 
$
xy = \frac{c^2}{2}
$
\item 
$
x^3-8xy^2 = a^3.
$

\end{enumerate}
\item Find the Cartesian equations corresponding to the following
equations in polar coordinates:
\begin{enumerate}
\item 
$
r\sin\theta + a = 0
$
\item 
$
r = a\sin\theta + b\cos\theta
$
\item 
$
r^2\cos2\theta = a^2
$
\item 
$
\frac{1}{r} = 1 + e\cos\theta
$
\item 
$
r^3\sin3\theta = a^3
$
\item 
$
r^2=a^2+b^2\cos2\theta
$
\end{enumerate}
\item Using the polar equation of a conic with focus as pole, show that
the semi-latus rectum is a harmonic mean between the segments of
a focal chord.
\item Draw the curve $r=a\cos2\theta$, noting carefully when $r$
changes sign and showing that the curve forms a double figure of eight.
\item Draw the curve
\begin{align*}
r = a\brak{2+\cos\theta}
\end{align*}
\item Draw the curve $r = a\brak{1+2\cos\theta}$, showing that it consists of two loops one within
the other.
\item Deduce from the polar equations of the curves the pedal equations for
the circle, parabola with focus as pole, ellipse with focus as pole and
ellipse with centre as pole.
\item Find the pedal equation of:
\begin{enumerate}
\item The rectangular hyperbola
$
xy = c^2
$
\item the lemniscate 
$
r^2=a^2\cos2\theta
$
\item the curve 
$
r^n = a^n\cos n\theta
$
\item the spiral
$
r\theta = a.
$
\end{enumerate}
\item Prove that, for the curve $r\theta = a$, if a perpendicular through the pole be drawn
to the radius vector at any point, the length intercepted on it by the tangent
is constant.  
\item Prove that, for the curve $r = a \cos\brak{\theta-\alpha}$, $\phi = \frac{\pi}{2}+\theta-\alpha$.
\item Show that, for the curve $r = a\brak{1+\cos\theta}$, the locus of the foot of the perpendicular from the pole to the tangent is 
$r = 2a\cos^3\frac{\theta}{3}$.
\item Prove that, for the curve $r = a\sec^3\frac{\theta}{3}$, the locus of the foot of the perpendicular from the pole to the tangent
is a parabola.
\item Trace the curve $r^2=a^2\cos\theta$.
\item Prove that the pedal equation of the curve $r\cos n\theta = a$ is
\begin{align*}
\frac{1}{p^2} = \frac{1-n^2}{r^2}+\frac{n^2}{a^2}
\end{align*}
\end{enumerate}
