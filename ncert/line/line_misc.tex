\renewcommand{\theequation}{\theenumi}
\begin{enumerate}[label=\arabic*.,ref=\thesubsection.\theenumi]
\numberwithin{equation}{enumi}

\item Aftab tells his daughter, ``Seven years ago, I was seven times as old as you were then. Also, three years from now, I shall be three times as old as you will be." (Isn’t this interesting?) Represent this situation algebraically and graphically.
\item  The coach of a cricket team buys 3 bats and 6 balls for \rupee 3900. Later, she buys another bat and 3 more balls of the same kind for \rupee 1300. Represent this situation algebraically and geometrically.
\item  The cost of 2 kg of apples and 1kg of grapes on a day was found to be \rupee 160. After a month, the cost of 4 kg of apples and 2 kg of grapes is \rupee 300. Represent the situation algebraically and geometrically.
\item In $\triangle ABC$, Show that the centroid 
\begin{align}
\vec{O} = \frac{\vec{A}+\vec{B}+\vec{C}}{3}
\end{align}
%
\item (Cauchy-Schwarz Inequality:) Show that 
%
\begin{align}
\abs{\vec{a}^T\vec{b}} \le \norm{\vec{a}}\norm{\vec{b}}
\end{align}
%
%
\item (Triangle Inequality:) Show that 
%
\begin{align}
\norm{\vec{a}+\vec{b}} \le \norm{\vec{a}}+\norm{\vec{b}}
\end{align}
%
%
\item The base of an equilateral triangle with side $2a$ lies along the y-axis such that the mid-point of the base is at the origin. Find vertices of the triangle.
\item Find the distance between $\vec{P}= \myvec{x_1\\ y_1}$ and $\vec{Q} =\myvec{x_2\\ y_2}$ when
\begin{enumerate}
\item PQ is parallel to the y-axis.
\item PQ is parallel to the x-axis.
\end{enumerate}
\item If three points \myvec{h\\ 0}, \myvec{a\\ b} and \myvec{0\\ k} lie on a line, show that
$\frac{a}{h}+\frac{b}{k}= 1$.
\item $\vec{P}=\myvec{a\\ b}$ is the mid-point of a line segment between axes. Show that equation of the line is
\begin{align}
\myvec{\frac{1}{a} & \frac{1}{b}}\vec{x} = 2
\end{align}
\item  Point $\vec{R}= \myvec{h\\ k}$ divides a line segment between the axes in the ratio 1: 2. Find equation of the line.
\item Show that two lines 
\begin{align}
\myvec{a_1 & b_1}\vec{x} +c_1&= 0
\\
\myvec{a_2 & b_2}\vec{x} +c_2&= 0
\end{align}
are 
\begin{enumerate}
\item parallel if $\frac{a_1}{b_1}=\frac{a_2}{b_2}$ and
\item perpendicular if $a_1a_2-b_1b_2 = 0$.
\end{enumerate}
%
\item Find the distance between the parallel lines
%
\begin{align}
l\myvec{1 & 1}\vec{x} = -p
\\
l\myvec{1 & 1}\vec{x} = r
\end{align}
%
\item Find th equation of the line through the point $\vec{x}_1$ and parallel to the line
%
\begin{align}
\myvec{A & B}\vec{x} = -C
\end{align}
%
\item If $p$ and $q$ are the lengths of perpendiculars from the origin to the lines 
%
\begin{align}
\myvec{\cos\theta & \sin\theta}\vec{x} &= k\cos2\theta
\\
\myvec{\sec\theta & \cosec\theta}\vec{x} &= k
\end{align}
%
respectively, prove that $p^2+4q^2=k^2$.
\item If $p$ is the length of the perpendicular from the origin to the line whose intercepts on the axes are $a$ and $b$, then show that 
%
\begin{align}
\frac{1}{p^2} = \frac{1}{a^2}+\frac{1}{b^2}.
\end{align}
%
\item Show that the area of the triangle formed by the lines
%
\begin{align}
\myvec{-m_1 & 1}\vec{x} = c_1
\\
\myvec{-m_2 & 1}\vec{x} = c_2
\\
\myvec{1 & 0}\vec{x} = 0
\end{align}
%
is $\frac{\brak{c_1-c_2}^2}{2\abs{m_1-m_2}}$.
\item Find the values of $k$ for which the line 
%
\begin{align}
\myvec{k-3 & -\brak{4-k^2}}\vec{x} +k^2-7k+6= 0
\end{align}
%
is
\begin{enumerate}
\item parallel to the x-axis
\item parallel to the y-axis
\item passing through the origin.
\end{enumerate}
%
\item Find the perpendicular distance from the origin to the line joining the points \myvec{\cos\theta\\\sin\theta} and \myvec{\cos\phi\\ \sin \phi}.
\item Find the area of the triangle formed by the lines
%
\begin{align}
\myvec{1 & -1}\vec{x} &= 0
\\
\myvec{1 & 1}\vec{x} &= 0
\\
\myvec{1 & 0}\vec{x} &= k
\end{align}
%
\item If three lines whose equations are 
%
\begin{align}
\myvec{-m_1 & 1}\vec{x} &= c_1
\\
\myvec{-m_2 & 1}\vec{x} &= c_2
\\
\myvec{-m_3 & 1}\vec{x} &= c_3
\end{align}
%
are concurrent, show that
%
\begin{align}
m_1\brak{c_2-c_3}+
m_2\brak{c_3-c_1}+
m_3\brak{c_1-c_2} = 0
\end{align}
%
\item Find the equation of the line passing through the origin and making an angle $\theta$ with the line %
\begin{align}
\myvec{-m & 1}\vec{x} &= c
\end{align}
%
\item Prove that the product of the lengths of the perpendiculars drawn from the points $\myvec{\sqrt{a^2-b^2}\\0}$ and $\myvec{\sqrt{a^2-b^2}\\0}$ to the line 
%
\begin{align}
\myvec{\frac{\cos\theta}{a} & \frac{\sin\theta}{b}}\vec{x} &= 1
\end{align}
%
is $b^2$.

\item If 
$
\myvec{l_1\\m_1\\n_1}
$
and
$
\myvec{l_2\\m_2\\n_2}
$
are the unit direction vectors of two mutually perpendicular lines, the shown that the unit direction vector of the line perpendicular to both of these is
$
\myvec{m_1n_2-m_2n_1\\n_1l_2-n_2l_1\\l_1m_2-l_2m_1}.
$
\item A line makes angles $\alpha, \beta, \gamma, \delta$ with the diagonals of a cube, prove that \begin{align}
\cos^2\alpha + \cos^2\beta + \cos^2\gamma +\cos^2\delta = \frac{4}{3}.
\end{align}
\item Show that the lines 
\begin{align}
\frac{x-a+d}{\alpha-\delta} = \frac{y-a}{\alpha} &= \frac{z-a-d}{\alpha+\delta}, 
\\
\frac{x-b+c}{\beta-\gamma} = \frac{y-b}{\beta} &= \frac{z-b-c}{\beta+\gamma} 
\end{align}
%
are coplanar.
\item Find $\vec{R}$ which divides the line joining the points 
\begin{align}
\vec{P} = 2\vec{a}+\vec{b}
\\
\vec{Q} = \vec{a}-\vec{b}
\end{align}
externally in the ratio $1:2$.
\item Find $\norm{\vec{a}}$ and $\norm{\vec{b}}$ if 
\begin{align}
\brak{\vec{a}+\vec{b}}^T\brak{\vec{a}-\vec{b}} &= 8
\\
\norm{\vec{a}}&=8\norm{\vec{b}}
\end{align}
\item Evaluate the product 
\begin{align}
\brak{3\vec{a}-5\vec{b}}^T\brak{2\vec{a}+7\vec{b}} 
\end{align}
\item Find $\norm{\vec{a}}$ and $\norm{\vec{b}}$, if
\begin{align}
\norm{\vec{a}} &= \norm{\vec{b}},
\\
\vec{a}^T\vec{b} = \frac{1}{2} 
\end{align}
and the angle between $\vec{a}$ and $\vec{b}$ is $60\degree$.
\item Show that 
\begin{align}
\brak{\norm{\vec{a}}\vec{b}+\norm{\vec{b}}\vec{a}}\perp \brak{\norm{\vec{a}}\vec{b}-\norm{\vec{b}}\vec{a}}
\end{align}
\item If $\vec{a}^T\vec{a}=0$ and  $\vec{a}\vec{b}=0$, what can be concluded about the vector $\vec{b}$?
\item If $\vec{a},\vec{b},\vec{c}$ are unit vectors such that 
\begin{align}
\vec{a}+\vec{b}+\vec{c} = 0,
\end{align}
find the value of 
\begin{align}
\vec{a}^T\vec{b}+\vec{b}^T\vec{c}+\vec{c}^T\vec{a}.
\end{align}
\item If $\vec{a} \ne \vec{0}, \lambda \ne 0$, then $\norm{\lambda \vec{a}} = 1$ if
\begin{enumerate}
\item $\lambda =1$
\item $\lambda = -1$
\item $\norm{\vec{a}}=\abs{\lambda}$
\item $\norm{\vec{a}}=\frac{1}{\abs{\lambda}}$
\end{enumerate}
\item If a unit vector $\vec{a}$ makes angles $\frac{\pi}{3}$ with the x-axis and $\frac{\pi}{4}$ with the y-axis and an acute angle $\theta$ with the z-axis, find $\theta$ and $\vec{a}$.
\item Show that 
\begin{align}
\brak{\vec{a}-\vec{b}}\times \brak{\vec{a}+\vec{b}} = 2\brak{\vec{a}\times\vec{b}}
\end{align}
\item If $\vec{a}^T\vec{b} = 0$ and $\vec{a}\times \vec{b}$ = 0, what can you conclude about $\vec{a}$ and $\vec{b}$?
\item Find $\vec{x}$ if  $\vec{a}$ is a unit vector such that
\begin{align}
\brak{\vec{x}-\vec{a}}^T\brak{\vec{x}+\vec{a}} = 12.
\end{align}
\item If $\norm{\vec{a}} = 3, \norm{\vec{b}} =\frac{\sqrt{2}}{3}$, then $\vec{a}\times \vec{b}$ is a unit vector if the angle between $\vec{a}$ and $\vec{b}$ is 
\begin{enumerate}[itemsep = 2pt]
\begin{multicols}{2}
\item $\frac{\pi}{6}$
\item $\frac{\pi}{4}$
\item $\frac{\pi}{3}$
\item $\frac{\pi}{2}$
\end{multicols}
\end{enumerate}
\item Prove that 
\begin{align}
\brak{\vec{a}+\vec{b}}^T\brak{\vec{a}+\vec{b}} &= \norm{\vec{a}}^2+\norm{\vec{b}}^2
\\
\iff \vec{a}&\perp\vec{b}.
\end{align}
\item If $\theta$ is the angle between two vectors $\vec{a}$ and $\vec{b}$, then $\vec{a}^T\vec{b} \ge $ only when 
\begin{enumerate}[itemsep = 2pt]
\begin{multicols}{2}
\item $0 < \theta < \frac{\pi}{2}$
\item $0 \le \theta \le \frac{\pi}{2}$
\item $0 < \theta < {\pi}$
\item $0 \le \theta \le {\pi}$
\end{multicols}
\end{enumerate}
\item Let $\vec{a}$ and $\vec{b}$ be two unit vectors and $\theta$ be the angle between them.  Then $\vec{a}+\vec{b}$ is a unit vector if 
\begin{enumerate}[itemsep = 2pt]
\begin{multicols}{2}
\item $\theta = \frac{\pi}{4}$
\item $\theta = \frac{\pi}{3}$
\item $\theta = \frac{\pi}{2}$
\item $\theta = \frac{2\pi}{3}$
\end{multicols}
\end{enumerate}
\item If $\theta$ is the angle between any two vectors $\vec{a}$ and $\vec{b}$, then 
$\norm{\vec{a}^T\vec{b}} = \norm{\vec{a} \times \vec{b}}$ when $\theta$ is equal to 
\begin{enumerate}[itemsep = 2pt]
\begin{multicols}{2}
\item 0
\item $\frac{\pi}{4}$
\item $\frac{\pi}{2}$
\item $\pi$.
\end{multicols}
\end{enumerate}
\item Find the angle between the lines whose direction vectors are $\myvec{a\\b\\c}$ and $\myvec{b-c\\c-a\\a-b}$.
\item Find the equation of a line parallel to the x-axis and passing through the origin.
\item Find the equation of a plane passing through \myvec{a\\b\\c} and parallel to the plane 
%
\begin{align}
\myvec{1 & 1 & 1}\bm{x}&=2
\end{align}
%
\item Prove that if a plane has the intercepts $a, b, c$ and is at a distance of $p$ units from the origin, then, 
\begin{align}
\frac{1}{a^2}+\frac{1}{b^2}+\frac{1}{c^2}=\frac{1}{p^2} 
\end{align}

%
\end{enumerate}
