\renewcommand{\theequation}{\theenumi}
\begin{enumerate}[label=\arabic*.,ref=\thesubsection.\theenumi]
\numberwithin{equation}{enumi}
\item Find the coordinates of the point which divides the join of 
\begin{align}
\myvec{-1\\7}, = \myvec{4\\-3}
%\vec{P} = \myvec{2\\-3}, \vec{Q} = \myvec{10\\y}
\end{align}
%
in the ratio $2:3$.
\item Find the coordinates of the points of trisection of the line segment joining \myvec{4\\-1} and \myvec{-2\\-3}.
\item Find the ratio in which the line segment joining the points \myvec{-3\\10} and \myvec{6\\-8} is divided by \myvec{-1\\6}.
\item Find the ratio in whcih the line segment joining $\vec{A}=\myvec{1\\-5}, \vec{B}=\myvec{-4\\5}$ is divided by the $x$-axis.  Also find the coordinates of the point of division.
\item If \myvec{1\\2}, \myvec{4\\y}, \myvec{x\\6} and \myvec{3\\5} are the vertices of a parallelogram taken in order, find $x$ and $y$.
\item If $\vec{A}=\myvec{-2\\-2}, \vec{B}=\myvec{2\\-4}$ respectively, find the coordinates of $\vec{P}$ such that $AP = \frac{3}{7}AB$ and $\vec{P}$ lies on the line segment $AB$.
\item Find the coordinates of the points which divide the line segment joining $\vec{A}=\myvec{-2\\2}, \vec{B}=\myvec{2\\8}$ into four equal parts.
\item Find the value of $k$ if the points $\vec{A}=\myvec{2\\3}, \vec{B}=\myvec{4\\k}$ and $\vec{C}=\myvec{6\\-3}$ are collinear.
\item  In each of the following, find the value of $k$ for which the points are collinear

\begin{enumerate}
\item \myvec{7\\-2},  \myvec{5\\1},  \myvec{3\\k} 
\item \myvec{8\\1},  \myvec{k\\-4},  \myvec{2\\-5} 
\end{enumerate}
\item Find a condition on $\vec{x}$  such that the points $\vec{x}, \myvec{1\\2}\myvec{7\\0}$ are collinear.
\item Show that the points 
$\vec{A}=\myvec{1\\2\\7}, \vec{B}=\myvec{2\\6\\3}$ and $ \vec{C}=\myvec{3\\10\\-1}$ are collinear.
\item Show that the points 
$\vec{A}=\myvec{1\\-2\\8}, \vec{B}=\myvec{5\\0\\-2}$ and $ \vec{C}=\myvec{11\\3\\7}$ are collinear, and find the ratio in which $\vec{B}$ divides $AC$.
\item Show that 
$\vec{A}=\myvec{1\\1\\1}, \vec{B}=\myvec{2\\5\\0}, \vec{C}=\myvec{3\\2\\-3}$  and $ \vec{D}=\myvec{1\\-6\\-1}$, are collinear.
\item Show that 
$
\vec{A}=\myvec{2\\3\\-4}, 
\vec{B}=\myvec{1\\-2\\3} \text{ and } 
\vec{C}=\myvec{3\\8\\-11}$  
are collinear.
\item Show that 
$
\vec{A}=\myvec{2\\3\\4}, 
\vec{B}=\myvec{-1\\-2\\1} \text{ and } 
\vec{C}=\myvec{5\\8\\7}$  
are collinear.
\end{enumerate}
%
