
\section{Signal Processing: Z Transform}
\renewcommand{\theequation}{\theenumi}
\begin{enumerate}[label=\arabic*.,ref=\thesubsection.\theenumi]
\numberwithin{equation}{enumi}

\item Let
\begin{align}
a(n) &= \frac{\alpha^n-\beta^n}{\alpha-\beta}u(n)
\\
b(n) &=a(n-1)+a(n+1) - \delta(n)
\end{align}
where $\alpha, \beta$ are the roots of the equation
\begin{equation}
z^2-z-1 = 0
\end{equation}
and
\begin{align}
u(n)
=
\begin{cases}
0, & n < 0
\\
1, & n \ge 0
\end{cases}
\\
\delta(n)
=
\begin{cases}
0, & n \ne 0
\\
1, & n = 0
\end{cases}
\end{align}
\item Verify your results through a C program.
%\begin{enumerate}[label=\theenumi.\arabic*
%,ref=\thesection.\theenumi]
\item Show that the $Z$ transform of $u(n)$
\begin{align}
U(z)&\triangleq \sum_{n=-\infty}^{\infty}u(n)z^{-n} 
\\
&= \frac{1}{1- z^{-1}}, \quad \abs{z} > 1
\end{align}
\item  Show that 
\begin{align}
A(z)&= \frac{z^{-1}}{1- z^{-1}-z^{-2}}
\end{align}
\item Let 
\begin{align}
y(n)=a(n)*u(n) \triangleq \sum_{k=-\infty}^{\infty}a(k) u(n-k)
\end{align}
%
Show that 
\begin{align}
y(n) = \sum_{k=0}^{n}a(k) 
\end{align}
%
\item Show that 
\begin{align}
Y(z) &= A(z)U(z)
\\
&= \frac{z^{-1}}{\brak{1- z^{-1}-z^{-2}}\brak{1- z^{-1}}}
\end{align}
\item Show that 
\begin{align}
w(n)&=\sbrak{a(n+2)-1}u(n-1)
\\
&=a(n+2)-u(n+1)+2\delta(n)
\end{align}
\item Is $W(z)=Y(z)$?
\item Verify if 
\begin{align}
\sum_{n=1}^{\infty}\frac{a(n)}{10^n} = \frac{10}{89}
\end{align}
\item Verify if 
\begin{align}
\sum_{n=1}^{\infty}\frac{b(n)}{10^n} = \frac{8}{89}
\end{align}
\section{Algebra: Modular Arithmetic}
Let $AP\brak{a;d}$ denote an A.P. with $d > 0$
\renewcommand{\theequation}{\theenumi}
\begin{enumerate}[label=\arabic*.,ref=\thesubsection.\theenumi]
\numberwithin{equation}{enumi}

\item Express $AP\brak{a;d}$ in modulo arithmetic.
\\
\solution 
\begin{align}
\label{eq:14_ap_mod_def}
A &\equiv a\pmod{d}
\end{align}
\item Express the intersection of $AP\brak{1;3}, AP\brak{2;5}$ and $AP\brak{3;7}$  using modulo arithmetic.
\\
\solution The desired AP can be expressed as
\begin{align}
\label{eq:14_ap_mod}
A &\equiv 1\pmod{3}
\\
& \equiv 2\pmod {5}
\\
&\equiv 3\pmod{7}
\end{align}
%
\item Two numbers are said to be coprime if their greatest common divisor (gcd) is 1. Verify if (3,5), (5,7) and (3,7) are pairwise coprime.
\item Does a solution for \eqref{eq:14_ap_mod} exist?
\\
\solution  The Chinese remainder theorem guarantees that the system in \eqref{eq:14_ap_mod} has a solution since 3,5,7 are pairwise coprime.
%\item Find the modular multiplicative inverse of 
%\begin{align}
%\label{eq:14_ap_modinv}
%A &\equiv \brak{7\times 11}\pmod{5}
%\end{align}
% 
\item Simplify 
\begin{align}
\label{eq:14_ap_mod_753}
\brak{7\times 5}\pmod{3}
\end{align}
%
\solution \eqref{eq:14_ap_mod_753} can be expressed as
%
\begin{align}
\label{eq:14_ap_mod_753_sol}
\brak{7\times 5}\pmod{3} &= 35\pmod{3} 
\nonumber \\
&=2\pmod{3}
\end{align}

\item Find $x$ in 
%
\begin{align}
\label{eq:14_ap_mod_753_prob}
2x =1\pmod{3}
\end{align}
%
%
\solution By inspection, for $x = 2$, 
\begin{align}
%\label{eq:14_ap_mod_753_sol}
2x = 2\times 2 = 4 = 3+1 =1\pmod{3}
\end{align}
%
Thus $x = 2$ is a solution of \eqref{eq:14_ap_mod_753_prob}. 

\item In general, $x$ in 
\begin{align}
\label{eq:14_ap_mod_inv_def}
ax =1\pmod{d}
\end{align}
%
is defined to be the modular multiplicative inverse of \eqref{eq:14_ap_mod_def}.
\item Show that the multiplicative inverse of 
\begin{align}
\label{eq:14_ap_mod_357_sol}
\brak{3\times 5}\pmod{7} &= y=1
\end{align}
%
\item Show that the multiplicative inverse of 
\begin{align}
\label{eq:14_ap_mod_375_sol}
\brak{3\times 7}\pmod{5} &= z=1
\end{align}
%
\item Find $a+d$.
\\
\solution 
\begin{multline}
\label{eq:14_ap_mod_ad_sol}
\brak{5\times 7 \times 1 \times x}+
\brak{3\times 5\times 3 \times y} 
\\
+
\brak{3\times 7\times 2 \times z} =  157
\end{multline}
%
\item Find $a$ and $d$.
\\
\solution
\begin{align}
\label{eq:14_ap_mod_d}
d &= LCM\brak{3,5,7} = 105
\\
A&=157 \pmod{105} 
\nonumber \\
&= 52\pmod{105} 
\nonumber \\
\implies a &= 52
\end{align}
\item Given the APs 
\begin{align}
\label{eq:14_ap_mod_gen}
a_1&\pmod{d_1}
\\
a_2 &\pmod {d_2}
\\
a_3 &\pmod {d_3},
\end{align}
%
such that 
\begin{align}
gcd(d_1,d_2)=gcd(d_2,d_3)=gcd(d_3,d_1)=1,
\end{align}
show that their intersection
\begin{align}
\label{eq:14_ap_mod_gen_int}
a\pmod{d}
\end{align}
%
is obtained through
\begin{multline}
\label{eq:14_ap_mod_ad_gen}
a+d =
\\
 \brak{d_1\times d_2 \times a_3 \times x}+
\brak{d_2\times d_3 \times a_1 \times y}
\\
+
\brak{d_3\times d_1 \times a_2 \times z}
\end{multline}
\begin{align}
\label{eq:14_ap_mod_d_gen}
d &= LCM\brak{d_1,d_2,d_3},
\end{align}
%
where $x, y, z$ are the modular multiplicative inverses given by
\begin{align}
\label{eq:14_ap_mod_mul_inv_gen}
x &= \sbrak{ \brak{d_1\times d_2}\pmod{d_3}}^{-1}
\\
y &= \sbrak{ \brak{d_2\times d_3}\pmod{d_1}}^{-1}
\\
z &= \sbrak{  \brak{d_3\times d_1}\pmod{d_2}}^{-1}
\end{align}
respectively.
%
\item Write  a C program to find $x,y$ and $z$.
%\item Find the vector $\myvec{n \\ k \\ r}$ with the smallest length such that $A_n \cap B_k \cap C_r $.
%\\
%\solution This is obtained by considering the following set of equations
%\begin{align}
% 3n-2= 5k-3
%\\
%5k-3 = 7r-4
%%\\
%%7r-4 = a + md
%\end{align}
%%
%which can be simplified to obtain the matrix equation 
%\begin{align}
%\myvec{3 & -5 & 0 \\ 0 & 5 & -7  }\myvec{n \\ k \\ r} = \myvec{-1 \\ -1 }
%%\myvec{3 & 5 & 0 \\ 0 & 5 & 7 \\ 7 &  0 & 0 }\myvec{n \\ k \\ r} = \myvec{-1 \\ -1 \\ a + md}
%\end{align}
%%
%yielding the augmented matrix 
%\begin{align}
%\myvec{3 & -5 & 0 & -1\\ 0 & 5 & -7 &-1 }\leftrightarrow \myvec{3 & 0 & -7 & -2\\ 0 & 5 & -7 &-1 } 
%\\
%\implies \myvec{3n \\ 5k \\ r} = \myvec{ -2 \\ -1\\ 0 } + r \myvec{ 7 \\ 7  \\ 1} 
%\\
%\implies \myvec{n \\ k \\ r} = \myvec{ \frac{7r-2}{3} \\ \frac{7r-1}{5}\\ r }
%%\myvec{3 & 5 & 0 \\ 0 & 5 & 7 \\ 7 &  0 & 0 }\myvec{n \\ k \\ r} = \myvec{-1 \\ -1 \\ a + md}
%\end{align}
%%
%For $r = 8, n = 18, k = 11$ are integers.  
%\item  Find $a+d$ such that $A_n \cap B_k \cap C_r \cap AP\brak{a:d}$.
%\\
%\solution Substituting $r = 8$ in \eqref{eq:14_r},
%\begin{align}
%a = 7r - 4 = 52
%\end{align}
%Also, 
%\begin{align}
%a+d &= a + 3i = a + 5j = a + 7m,
%\\
%\implies d &= 3i=5j=7m
%\end{align}
%%
%for some positive integers $i, j, m$. The smallest integer $d$ satisfying the above condition is $LCM\brak{3,5,7} = 105$.  Hence, $a + d = 52+105 = 157$.
\end{enumerate}

\end{enumerate}
\section{Discrete Fourier Transform}
\renewcommand{\theequation}{\theenumi}
\begin{enumerate}[label=\arabic*.,ref=\thesubsection.\theenumi]
\numberwithin{equation}{enumi}

\item Show that 
\begin{align}
\label{eq:qp2_3_cexp}
\sum_{k=0}^{n-1}e^{\j \frac{2 \pi k}{n}} 
= 
\begin{cases}
1 & n = 1,
\\
0 & n > 1
\end{cases}
\end{align}
\item Show that 
%
\begin{align}
\label{eq:qp2_3_cosr}
\sum_{k=0}^{n}\cos \brak{\frac{2k+r}{n+2}\pi} = -\cos\brak{\frac{r-2}{n+2}\pi} 
\end{align}
\solution From \eqref{eq:qp2_3_cexp},
\begin{align}
\label{eq:qp2_3_cosr_sol}
\sum_{k=0}^{n+1}e^{\j \frac{2k+r}{n+2}\pi} &=0
\nonumber \\
\implies \sum_{k=0}^{n}e^{\j \frac{2k+r}{n+2}\pi} + e^{\j \frac{2\brak{n+1}+r}{n+2}\pi} &=0
\nonumber \\
\implies \sum_{k=0}^{n}e^{\j \frac{2k+r}{n+2}\pi} &=  -e^{\j \frac{2\brak{n+2}+r-2}{n+2}\pi} 
\nonumber \\
&=-e^{\j \frac{r-2}{n+2}\pi} 
\end{align}
Taking the real part on both sides yields \eqref{eq:qp2_3_cosr}.
%
\item Show that 
\begin{align}
\label{eq:qp2_3_sum}
f(n) & = \frac{\sum_{k=0}^{n}\sin\brak{\frac{k+1}{n+2}\pi}
\sin\brak{\frac{k+2}{n+2}\pi} }{\sum_{k=0}^{n}\sin^2\brak{\frac{k+2}{n+2}\pi}}
%\nonumber 
\\
& = 
\frac{\brak{n+1}\cos \brak{\frac{\pi}{n+2}}}{n+\cos\brak{\frac{2\pi}{n+2}}}
\label{eq:qp2_3_sum_ans}
\end{align}
%\frac{n\cos \cbrak{\frac{\pi}{n+2}}-\sum_{k=0}^{n}\cos \cbrak{\frac{2k+3}{n+2}\pi}}{n-\sum_{k=0}^{n}\cos\cbrak{\frac{2k+4}{n+2}\pi}}
\solution 
Let
\begin{align}
\label{eq:qp2_3_theta}
\theta_n = \frac{\pi}{n+2}
\end{align}
%
\begin{multline}
\label{eq:qp2_3_cos}
\because \sin\cbrak{\brak{k+1}\theta_n}
\sin\cbrak{\brak{k+2}\theta_n} ,
\\
= \frac{1}{2}\sbrak{\cos \theta_n-\cos \cbrak{\brak{2k+3}\theta_n}}
\end{multline}
from \eqref{eq:qp2_3_sum} and \eqref{eq:qp2_3_cosr},
\begin{align}
\label{eq:qp2_3_sum_sol}
f(n) & = \frac{n\cos \theta_n-\sum_{k=0}^{n}\cos \cbrak{\brak{2k+3}\theta_n}}{n-\sum_{k=0}^{n}\cos\cbrak{\brak{2k+4}\theta_n}}
\nonumber \\
&= \frac{n\cos \brak{\frac{\pi}{n+2}}+\cos \brak{\frac{\pi}{n+2}}}{n+\cos\brak{\frac{2\pi}{n+2}}}
\end{align}
resulting in \eqref{eq:qp2_3_sum_ans}.  Verify if 
\item 
\begin{align}
f(4) = \frac{\sqrt{3}}{2}
\end{align}
\item 
\begin{align}
\lim_{n \to \infty}f(n) = \frac{1}{2}
\end{align}
\item 
\begin{align}
\sin\brak{7 \cos^{-1}f(5)} = 0
\end{align}
\item If 
\begin{align}
\alpha = 
\tan\brak{ \cos^{-1}f(6)} 
\end{align}
verify if 
\begin{align}
\alpha^2 +2\alpha -1 =  0 
\end{align}
\end{enumerate}
%
\section{Combinatorics}
\renewcommand{\theequation}{\theenumi}
\begin{enumerate}[label=\arabic*.,ref=\thesubsection.\theenumi]
\numberwithin{equation}{enumi}
\item Find 
\begin{align}
\label{eq:2019_qp2_comb_nat}
\sum_{k=0}^{n}k
\end{align}
\solution  \eqref{eq:2019_qp2_comb_nat} can be expressed as
\begin{align}
\label{eq:2019_qp2_comb_nat_sol}
\frac{n(n+1)}{2}
\end{align}
\item Find 
\begin{align}
\label{eq:2019_qp2_comb_k2}
\sum_{k=0}^{n}\nCr{n}{k}k^2
\end{align}
\solution 
\begin{align}
\label{eq:2019_qp2_comb}
\brak{1+x}^n&=\sum_{k=0}^{n}\nCr{n}{k}x^k
%\nonumber 
\\
\implies n\brak{1+x}^{n-1}&=\sum_{k=0}^{n}k\nCr{n}{k}x^{k-1}
\label{eq:2019_qp2_comb_k2_bin}
\end{align}
upon differentiation. Multiplying \eqref{eq:2019_qp2_comb_k2_bin} by $x$ and differentiating,
\begin{align}
\frac{d}{dx} \sbrak{nx\brak{1+x}^{n-1}}&=\sum_{k=0}^{n}k^2\nCr{n}{k}x^{k-1}
\end{align}
\begin{multline}
\implies n\brak{n-1}x\brak{1+x}^{n-2}+n\brak{1+x}^{n-1}  
\\
=\sum_{k=0}^{n}k^2\nCr{n}{k}x^{k-1}
\label{eq:2019_qp2_comb_k2_bin_diff}
\end{multline}
%
Substituting $x=1$ in \eqref{eq:2019_qp2_comb_k2_bin_diff},
\begin{align}
\sum_{k=0}^{n}\nCr{n}{k}k^2 &= n\brak{n-1}2^{n-2}+n2^{n-1}
\nonumber \\
&= n\brak{n+1}2^{n-2}
\label{eq:2019_qp2_comb_k2_bin_sol}
\end{align}
\item Find 
\begin{align}
\label{eq:2019_qp2_comb_k}
\sum_{k=0}^{n}\nCr{n}{k}k
\end{align}
\solution Substituting $x=1$ in \eqref{eq:2019_qp2_comb_k2_bin},
\begin{align}
\label{eq:2019_qp2_comb_k_sol} 
\sum_{k=0}^{n}\nCr{n}{k}k= n2^{n-1}
\end{align}
\item Find 
\begin{align}
\label{eq:2019_qp2_comb_3k}
\sum_{k=0}^{n}\nCr{n}{k}3^k
\end{align}
\solution Substituting $x=2$ in \eqref{eq:2019_qp2_comb},
\begin{align}
\label{eq:2019_qp2_comb_3k_sol}
\sum_{k=0}^{n}\nCr{n}{k}3^k = 4^n
\end{align}
\item If
\begin{align}
\label{eq:2019_qp2_comb_det}
\begin{vmatrix}
\frac{n(n+1)}{2} & n\brak{n+1}2^{n-2}
\\
n2^{n-1} & 4^n
\end{vmatrix} = 0
\end{align}
for some $n$, find 
\begin{align}
\label{eq:2019_qp2_comb_det_prob}
\sum_{k=0}^{n}\frac{\nCr{n}{k}}{k+1}
\end{align}
%
\solution \eqref{eq:2019_qp2_comb_det} can be expressed as
\begin{align}
\label{eq:2019_qp2_comb_det_sol}
n(n+1)2^{2n-3}
\begin{vmatrix}
1 & 1
\\
n & 4
\end{vmatrix} = 0
\\
\implies n = 4
\end{align}
%
Integrating \eqref{eq:2019_qp2_comb} from 0 to 1,

\begin{align}
\label{eq:2019_qp2_comb_int}
\frac{2^{n+1}}{n+1}&=\sum_{k=0}^{n}\frac{\nCr{n}{k}}{k+1}
\end{align}
%
Substituting $n=4$ in the above, 
\begin{align}
\label{eq:2019_qp2_comb_int_sol}
\sum_{k=0}^{n}\frac{\nCr{n}{k}}{k+1}= \frac{2^{5}-1}{5} = \frac{31}{5}
\end{align}
\end{enumerate}
%\section{Probability}
%Table \ref{table:2019_7} lists the number of red (R) and green (G) balls in bags $B_1, B_2$ and $B_3$. Also listed are the probabilities of each bag.
%\begin{table}[!h]
%\centering
%%\resizebox {0.5\columnwidth} {!} {
%%%%%%%%%%%%%%%%%%%%%%%%%%%%%%%%%%%%%%%%%%%%%%%%%%%%%%%%%%%%%%%%%%%%%%%%
%%                                                                  %%
%%  This is the header of a LaTeX2e file exported from Gnumeric.    %%
%%                                                                  %%
%%  This file can be compiled as it stands or included in another   %%
%%  LaTeX document. The table is based on the longtable package so  %%
%%  the longtable options (headers, footers...) can be set in the   %%
%%  preamble section below (see PRAMBLE).                           %%
%%                                                                  %%
%%  To include the file in another, the following two lines must be %%
%%  in the including file:                                          %%
%%        \def\inputGnumericTable{}                                 %%
%%  at the beginning of the file and:                               %%
%%        \input{name-of-this-file.tex}                             %%
%%  where the table is to be placed. Note also that the including   %%
%%  file must use the following packages for the table to be        %%
%%  rendered correctly:                                             %%
%%    \usepackage[latin1]{inputenc}                                 %%
%%    \usepackage{color}                                            %%
%%    \usepackage{array}                                            %%
%%    \usepackage{longtable}                                        %%
%%    \usepackage{calc}                                             %%
%%    \usepackage{multirow}                                         %%
%%    \usepackage{hhline}                                           %%
%%    \usepackage{ifthen}                                           %%
%%  optionally (for landscape tables embedded in another document): %%
%%    \usepackage{lscape}                                           %%
%%                                                                  %%
%%%%%%%%%%%%%%%%%%%%%%%%%%%%%%%%%%%%%%%%%%%%%%%%%%%%%%%%%%%%%%%%%%%%%%



%%  This section checks if we are begin input into another file or  %%
%%  the file will be compiled alone. First use a macro taken from   %%
%%  the TeXbook ex 7.7 (suggestion of Han-Wen Nienhuys).            %%
\def\ifundefined#1{\expandafter\ifx\csname#1\endcsname\relax}


%%  Check for the \def token for inputed files. If it is not        %%
%%  defined, the file will be processed as a standalone and the     %%
%%  preamble will be used.                                          %%
\ifundefined{inputGnumericTable}

%%  We must be able to close or not the document at the end.        %%
	\def\gnumericTableEnd{\end{document}}


%%%%%%%%%%%%%%%%%%%%%%%%%%%%%%%%%%%%%%%%%%%%%%%%%%%%%%%%%%%%%%%%%%%%%%
%%                                                                  %%
%%  This is the PREAMBLE. Change these values to get the right      %%
%%  paper size and other niceties.                                  %%
%%                                                                  %%
%%%%%%%%%%%%%%%%%%%%%%%%%%%%%%%%%%%%%%%%%%%%%%%%%%%%%%%%%%%%%%%%%%%%%%

	\documentclass[12pt%
			  %,landscape%
                    ]{report}
       \usepackage[latin1]{inputenc}
       \usepackage{fullpage}
       \usepackage{color}
       \usepackage{array}
       \usepackage{longtable}
       \usepackage{calc}
       \usepackage{multirow}
       \usepackage{hhline}
       \usepackage{ifthen}

	\begin{document}


%%  End of the preamble for the standalone. The next section is for %%
%%  documents which are included into other LaTeX2e files.          %%
\else

%%  We are not a stand alone document. For a regular table, we will %%
%%  have no preamble and only define the closing to mean nothing.   %%
    \def\gnumericTableEnd{}

%%  If we want landscape mode in an embedded document, comment out  %%
%%  the line above and uncomment the two below. The table will      %%
%%  begin on a new page and run in landscape mode.                  %%
%       \def\gnumericTableEnd{\end{landscape}}
%       \begin{landscape}


%%  End of the else clause for this file being \input.              %%
\fi

%%%%%%%%%%%%%%%%%%%%%%%%%%%%%%%%%%%%%%%%%%%%%%%%%%%%%%%%%%%%%%%%%%%%%%
%%                                                                  %%
%%  The rest is the gnumeric table, except for the closing          %%
%%  statement. Changes below will alter the table's appearance.     %%
%%                                                                  %%
%%%%%%%%%%%%%%%%%%%%%%%%%%%%%%%%%%%%%%%%%%%%%%%%%%%%%%%%%%%%%%%%%%%%%%

\providecommand{\gnumericmathit}[1]{#1} 
%%  Uncomment the next line if you would like your numbers to be in %%
%%  italics if they are italizised in the gnumeric table.           %%
%\renewcommand{\gnumericmathit}[1]{\mathit{#1}}
\providecommand{\gnumericPB}[1]%
{\let\gnumericTemp=\\#1\let\\=\gnumericTemp\hspace{0pt}}
 \ifundefined{gnumericTableWidthDefined}
        \newlength{\gnumericTableWidth}
        \newlength{\gnumericTableWidthComplete}
        \newlength{\gnumericMultiRowLength}
        \global\def\gnumericTableWidthDefined{}
 \fi
%% The following setting protects this code from babel shorthands.  %%
 \ifthenelse{\isundefined{\languageshorthands}}{}{\languageshorthands{english}}
%%  The default table format retains the relative column widths of  %%
%%  gnumeric. They can easily be changed to c, r or l. In that case %%
%%  you may want to comment out the next line and uncomment the one %%
%%  thereafter                                                      %%
\providecommand\gnumbox{\makebox[0pt]}
%%\providecommand\gnumbox[1][]{\makebox}

%% to adjust positions in multirow situations                       %%
\setlength{\bigstrutjot}{\jot}
\setlength{\extrarowheight}{\doublerulesep}

%%  The \setlongtables command keeps column widths the same across  %%
%%  pages. Simply comment out next line for varying column widths.  %%
\setlongtables

\setlength\gnumericTableWidth{%
	133pt+%
	53pt+%
	57pt+%
0pt}
\def\gumericNumCols{3}
\setlength\gnumericTableWidthComplete{\gnumericTableWidth+%
         \tabcolsep*\gumericNumCols*2+\arrayrulewidth*\gumericNumCols}
\ifthenelse{\lengthtest{\gnumericTableWidthComplete > \linewidth}}%
         {\def\gnumericScale{\ratio{\linewidth-%
                        \tabcolsep*\gumericNumCols*2-%
                        \arrayrulewidth*\gumericNumCols}%
{\gnumericTableWidth}}}%
{\def\gnumericScale{1}}

%%%%%%%%%%%%%%%%%%%%%%%%%%%%%%%%%%%%%%%%%%%%%%%%%%%%%%%%%%%%%%%%%%%%%%
%%                                                                  %%
%% The following are the widths of the various columns. We are      %%
%% defining them here because then they are easier to change.       %%
%% Depending on the cell formats we may use them more than once.    %%
%%                                                                  %%
%%%%%%%%%%%%%%%%%%%%%%%%%%%%%%%%%%%%%%%%%%%%%%%%%%%%%%%%%%%%%%%%%%%%%%

\ifthenelse{\isundefined{\gnumericColA}}{\newlength{\gnumericColA}}{}\settowidth{\gnumericColA}{\begin{tabular}{@{}p{100pt*\gnumericScale}@{}}x\end{tabular}}
\ifthenelse{\isundefined{\gnumericColB}}{\newlength{\gnumericColB}}{}\settowidth{\gnumericColB}{\begin{tabular}{@{}p{50pt*\gnumericScale}@{}}x\end{tabular}}
\ifthenelse{\isundefined{\gnumericColC}}{\newlength{\gnumericColC}}{}\settowidth{\gnumericColC}{\begin{tabular}{@{}p{70pt*\gnumericScale}@{}}x\end{tabular}}

%\begin{longtable}[c]{%
\begin{table}[!h]


\begin{tabular}[c]{
	b{\gnumericColA}%
	b{\gnumericColB}%
	b{\gnumericColC}%
	}

%%%%%%%%%%%%%%%%%%%%%%%%%%%%%%%%%%%%%%%%%%%%%%%%%%%%%%%%%%%%%%%%%%%%%%
%%  The longtable options. (Caption, headers... see Goosens, p.124) %%
%	\caption{The Table Caption.}             \\	%
% \hline	% Across the top of the table.
%%  The rest of these options are table rows which are placed on    %%
%%  the first, last or every page. Use \multicolumn if you want.    %%

%%  Header for the first page.                                      %%
%	\multicolumn{3}{c}{The First Header} \\ \hline 
%	\multicolumn{1}{c}{colTag}	%Column 1
%	&\multicolumn{1}{c}{colTag}	%Column 2
%	&\multicolumn{1}{c}{colTag}	\\ \hline %Last column
%	\endfirsthead

%%  The running header definition.                                  %%
%	\hline
%	\multicolumn{3}{l}{\ldots\small\slshape continued} \\ \hline
%	\multicolumn{1}{c}{colTag}	%Column 1
%	&\multicolumn{1}{c}{colTag}	%Column 2
%	&\multicolumn{1}{c}{colTag}	\\ \hline %Last column
%	\endhead

%%  The running footer definition.                                  %%
%	\hline
%	\multicolumn{3}{r}{\small\slshape continued\ldots} \\
%	\endfoot

%%  The ending footer definition.                                   %%
%	\multicolumn{3}{c}{That's all folks} \\ \hline 
%	\endlastfoot
%%%%%%%%%%%%%%%%%%%%%%%%%%%%%%%%%%%%%%%%%%%%%%%%%%%%%%%%%%%%%%%%%%%%%%

\hhline{|-|-|-}
	 \multicolumn{1}{|p{\gnumericColA}|}%
	{\gnumericPB{\centering}\textbf{Component}}
	&\multicolumn{1}{p{\gnumericColB}|}%
	{\gnumericPB{\raggedright}\textbf{Value}}
	&\multicolumn{1}{p{\gnumericColC}|}%
	{\gnumericPB{\centering}\textbf{Quantity}}
\\
\hhline{|---|}
	 \multicolumn{1}{|p{\gnumericColA}|}%
	{\gnumericPB{\centering}Breadboard}
	&\multicolumn{1}{p{\gnumericColB}|}%
	{\gnumericPB{\raggedright} }
	&\multicolumn{1}{p{\gnumericColC}|}%
	{\gnumericPB{\centering}1}
\\
\hhline{|---|}
	 \multicolumn{1}{|p{\gnumericColA}|}%
	{\gnumericPB{\centering}Resistor}
	&\multicolumn{1}{p{\gnumericColB}|}%
	{\gnumericPB{\raggedright} $\ge 220 \Omega$}
	&\multicolumn{1}{p{\gnumericColC}|}%
	{\gnumericPB{\centering}1}
\\
\hhline{|---|}
	 \multicolumn{1}{|p{\gnumericColA}|}%
	{\gnumericPB{\centering}Pi}
	&\multicolumn{1}{p{\gnumericColB}|}%
	{Model B, Rev 3}
	&\multicolumn{1}{p{\gnumericColC}|}%
	{\gnumericPB{\centering}1}
\\
\hhline{|---|}
	 \multicolumn{1}{|p{\gnumericColA}|}%
	{\gnumericPB{\centering}Seven Segment Display}
	&\multicolumn{1}{p{\gnumericColB}|}%
	{Common Anode}
	&\multicolumn{1}{p{\gnumericColC}|}%
	{\gnumericPB{\centering}1}
\\
\hhline{|---|}
	 \multicolumn{1}{|p{\gnumericColA}|}%
	{\gnumericPB{\centering}\gnumbox{Jumper Wires}}
	&\multicolumn{1}{p{\gnumericColB}|}%
	{Female-Male}
	&\multicolumn{1}{p{\gnumericColC}|}%
	{\gnumericPB{\centering}\gnumbox{20}}
\\
\hhline{|-|-|-|}
%\end{longtable}
\end{tabular}
\caption{}
\label{table:components}
\end{table}
\ifthenelse{\isundefined{\languageshorthands}}{}{\languageshorthands{\languagename}}
\gnumericTableEnd

%%%%%%%%%%%%%%%%%%%%%%%%%%%%%%%%%%%%%%%%%%%%%%%%%%%%%%%%%%%%%%%%%%%%%%%
%%                                                                  %%
%%  This is the header of a LaTeX2e file exported from Gnumeric.    %%
%%                                                                  %%
%%  This file can be compiled as it stands or included in another   %%
%%  LaTeX document. The table is based on the longtable package so  %%
%%  the longtable options (headers, footers...) can be set in the   %%
%%  preamble section below (see PRAMBLE).                           %%
%%                                                                  %%
%%  To include the file in another, the following two lines must be %%
%%  in the including file:                                          %%
%%        \def\inputGnumericTable{}                                 %%
%%  at the beginning of the file and:                               %%
%%        \input{name-of-this-file.tex}                             %%
%%  where the table is to be placed. Note also that the including   %%
%%  file must use the following packages for the table to be        %%
%%  rendered correctly:                                             %%
%%    \usepackage[latin1]{inputenc}                                 %%
%%    \usepackage{color}                                            %%
%%    \usepackage{array}                                            %%
%%    \usepackage{longtable}                                        %%
%%    \usepackage{calc}                                             %%
%%    \usepackage{multirow}                                         %%
%%    \usepackage{hhline}                                           %%
%%    \usepackage{ifthen}                                           %%
%%  optionally (for landscape tables embedded in another document): %%
%%    \usepackage{lscape}                                           %%
%%                                                                  %%
%%%%%%%%%%%%%%%%%%%%%%%%%%%%%%%%%%%%%%%%%%%%%%%%%%%%%%%%%%%%%%%%%%%%%%



%%  This section checks if we are begin input into another file or  %%
%%  the file will be compiled alone. First use a macro taken from   %%
%%  the TeXbook ex 7.7 (suggestion of Han-Wen Nienhuys).            %%
\def\ifundefined#1{\expandafter\ifx\csname#1\endcsname\relax}


%%  Check for the \def token for inputed files. If it is not        %%
%%  defined, the file will be processed as a standalone and the     %%
%%  preamble will be used.                                          %%
\ifundefined{inputGnumericTable}

%%  We must be able to close or not the document at the end.        %%
	\def\gnumericTableEnd{\end{document}}


%%%%%%%%%%%%%%%%%%%%%%%%%%%%%%%%%%%%%%%%%%%%%%%%%%%%%%%%%%%%%%%%%%%%%%
%%                                                                  %%
%%  This is the PREAMBLE. Change these values to get the right      %%
%%  paper size and other niceties.                                  %%
%%                                                                  %%
%%%%%%%%%%%%%%%%%%%%%%%%%%%%%%%%%%%%%%%%%%%%%%%%%%%%%%%%%%%%%%%%%%%%%%

	\documentclass[12pt%
			  %,landscape%
                    ]{report}
       \usepackage[latin1]{inputenc}
       \usepackage{fullpage}
       \usepackage{color}
       \usepackage{array}
       \usepackage{longtable}
       \usepackage{calc}
       \usepackage{multirow}
       \usepackage{hhline}
       \usepackage{ifthen}

	\begin{document}


%%  End of the preamble for the standalone. The next section is for %%
%%  documents which are included into other LaTeX2e files.          %%
\else

%%  We are not a stand alone document. For a regular table, we will %%
%%  have no preamble and only define the closing to mean nothing.   %%
    \def\gnumericTableEnd{}

%%  If we want landscape mode in an embedded document, comment out  %%
%%  the line above and uncomment the two below. The table will      %%
%%  begin on a new page and run in landscape mode.                  %%
%       \def\gnumericTableEnd{\end{landscape}}
%       \begin{landscape}


%%  End of the else clause for this file being \input.              %%
\fi

%%%%%%%%%%%%%%%%%%%%%%%%%%%%%%%%%%%%%%%%%%%%%%%%%%%%%%%%%%%%%%%%%%%%%%
%%                                                                  %%
%%  The rest is the gnumeric table, except for the closing          %%
%%  statement. Changes below will alter the table's appearance.     %%
%%                                                                  %%
%%%%%%%%%%%%%%%%%%%%%%%%%%%%%%%%%%%%%%%%%%%%%%%%%%%%%%%%%%%%%%%%%%%%%%

\providecommand{\gnumericmathit}[1]{#1} 
%%  Uncomment the next line if you would like your numbers to be in %%
%%  italics if they are italizised in the gnumeric table.           %%
%\renewcommand{\gnumericmathit}[1]{\mathit{#1}}
\providecommand{\gnumericPB}[1]%
{\let\gnumericTemp=\\#1\let\\=\gnumericTemp\hspace{0pt}}
 \ifundefined{gnumericTableWidthDefined}
        \newlength{\gnumericTableWidth}
        \newlength{\gnumericTableWidthComplete}
        \newlength{\gnumericMultiRowLength}
        \global\def\gnumericTableWidthDefined{}
 \fi
%% The following setting protects this code from babel shorthands.  %%
 \ifthenelse{\isundefined{\languageshorthands}}{}{\languageshorthands{english}}
%%  The default table format retains the relative column widths of  %%
%%  gnumeric. They can easily be changed to c, r or l. In that case %%
%%  you may want to comment out the next line and uncomment the one %%
%%  thereafter                                                      %%
\providecommand\gnumbox{\makebox[0pt]}
%%\providecommand\gnumbox[1][]{\makebox}

%% to adjust positions in multirow situations                       %%
\setlength{\bigstrutjot}{\jot}
\setlength{\extrarowheight}{\doublerulesep}

%%  The \setlongtables command keeps column widths the same across  %%
%%  pages. Simply comment out next line for varying column widths.  %%
\setlongtables

\setlength\gnumericTableWidth{%
	28pt+%
	12pt+%
	13pt+%
	75pt+%
0pt}
\def\gumericNumCols{4}
\setlength\gnumericTableWidthComplete{\gnumericTableWidth+%
         \tabcolsep*\gumericNumCols*2+\arrayrulewidth*\gumericNumCols}
\ifthenelse{\lengthtest{\gnumericTableWidthComplete > \linewidth}}%
         {\def\gnumericScale{\ratio{\linewidth-%
                        \tabcolsep*\gumericNumCols*2-%
                        \arrayrulewidth*\gumericNumCols}%
{\gnumericTableWidth}}}%
{\def\gnumericScale{1}}

%%%%%%%%%%%%%%%%%%%%%%%%%%%%%%%%%%%%%%%%%%%%%%%%%%%%%%%%%%%%%%%%%%%%%%
%%                                                                  %%
%% The following are the widths of the various columns. We are      %%
%% defining them here because then they are easier to change.       %%
%% Depending on the cell formats we may use them more than once.    %%
%%                                                                  %%
%%%%%%%%%%%%%%%%%%%%%%%%%%%%%%%%%%%%%%%%%%%%%%%%%%%%%%%%%%%%%%%%%%%%%%

\ifthenelse{\isundefined{\gnumericColA}}{\newlength{\gnumericColA}}{}\settowidth{\gnumericColA}{\begin{tabular}{@{}p{28pt*\gnumericScale}@{}}x\end{tabular}}
\ifthenelse{\isundefined{\gnumericColB}}{\newlength{\gnumericColB}}{}\settowidth{\gnumericColB}{\begin{tabular}{@{}p{12pt*\gnumericScale}@{}}x\end{tabular}}
\ifthenelse{\isundefined{\gnumericColC}}{\newlength{\gnumericColC}}{}\settowidth{\gnumericColC}{\begin{tabular}{@{}p{13pt*\gnumericScale}@{}}x\end{tabular}}
\ifthenelse{\isundefined{\gnumericColD}}{\newlength{\gnumericColD}}{}\settowidth{\gnumericColD}{\begin{tabular}{@{}p{75pt*\gnumericScale}@{}}x\end{tabular}}

\begin{tabular}[c]{%
	b{\gnumericColA}%
	b{\gnumericColB}%
	b{\gnumericColC}%
	b{\gnumericColD}%
	}

%%%%%%%%%%%%%%%%%%%%%%%%%%%%%%%%%%%%%%%%%%%%%%%%%%%%%%%%%%%%%%%%%%%%%%
%%  The longtable options. (Caption, headers... see Goosens, p.124) %%
%	\caption{The Table Caption.}             \\	%
% \hline	% Across the top of the table.
%%  The rest of these options are table rows which are placed on    %%
%%  the first, last or every page. Use \multicolumn if you want.    %%

%%  Header for the first page.                                      %%
%	\multicolumn{4}{c}{The First Header} \\ \hline 
%	\multicolumn{1}{c}{colTag}	%Column 1
%	&\multicolumn{1}{c}{colTag}	%Column 2
%	&\multicolumn{1}{c}{colTag}	%Column 3
%	&\multicolumn{1}{c}{colTag}	\\ \hline %Last column
%	\endfirsthead

%%  The running header definition.                                  %%
%	\hline
%	\multicolumn{4}{l}{\ldots\small\slshape continued} \\ \hline
%	\multicolumn{1}{c}{colTag}	%Column 1
%	&\multicolumn{1}{c}{colTag}	%Column 2
%	&\multicolumn{1}{c}{colTag}	%Column 3
%	&\multicolumn{1}{c}{colTag}	\\ \hline %Last column
%	\endhead

%%  The running footer definition.                                  %%
%	\hline
%	\multicolumn{4}{r}{\small\slshape continued\ldots} \\
%	\endfoot

%%  The ending footer definition.                                   %%
%	\multicolumn{4}{c}{That's all folks} \\ \hline 
%	\endlastfoot
%%%%%%%%%%%%%%%%%%%%%%%%%%%%%%%%%%%%%%%%%%%%%%%%%%%%%%%%%%%%%%%%%%%%%%

\hhline{|-|-|-|-}
	 \multicolumn{1}{|p{\gnumericColA}|}%
	{\gnumericPB{\centering}\gnumbox{\textbf{Bag}}}
	&\multicolumn{1}{p{\gnumericColB}|}%
	{\gnumericPB{\centering}\gnumbox{\textbf{R}}}
	&\multicolumn{1}{p{\gnumericColC}|}%
	{\gnumericPB{\centering}\gnumbox{\textbf{G}}}
	&\multicolumn{1}{p{\gnumericColD}|}%
	{\gnumericPB{\centering}\gnumbox{\textbf{Probability}}}
\\
\hhline{|----|}
	 \multicolumn{1}{|p{\gnumericColA}|}%
	{\gnumericPB{\centering}\gnumbox{$B_1$}}
	&\multicolumn{1}{p{\gnumericColB}|}%
	{\gnumericPB{\centering}\gnumbox{5}}
	&\multicolumn{1}{p{\gnumericColC}|}%
	{\gnumericPB{\centering}\gnumbox{5}}
	&\multicolumn{1}{p{\gnumericColD}|}%
	{\gnumericPB{\centering}\gnumbox{$\Pr(B_1)=\frac{3}{10}$}}
\\
\hhline{|----|}
	 \multicolumn{1}{|p{\gnumericColA}|}%
	{\gnumericPB{\centering}\gnumbox{$B_2$}}
	&\multicolumn{1}{p{\gnumericColB}|}%
	{\gnumericPB{\centering}\gnumbox{3}}
	&\multicolumn{1}{p{\gnumericColC}|}%
	{\gnumericPB{\centering}\gnumbox{5}}
	&\multicolumn{1}{p{\gnumericColD}|}%
	{\gnumericPB{\centering}\gnumbox{$\Pr(B_2)=\frac{3}{10}$}}
\\
\hhline{|----|}
	 \multicolumn{1}{|p{\gnumericColA}|}%
	{\gnumericPB{\centering}\gnumbox{$B_3$}}
	&\multicolumn{1}{p{\gnumericColB}|}%
	{\gnumericPB{\centering}\gnumbox{5}}
	&\multicolumn{1}{p{\gnumericColC}|}%
	{\gnumericPB{\centering}\gnumbox{3}}
	&\multicolumn{1}{p{\gnumericColD}|}%
	{\gnumericPB{\centering}\gnumbox{$\Pr(B_3)=\frac{4}{10}$}}
\\
\hhline{|-|-|-|-|}
\end{tabular}

\ifthenelse{\isundefined{\languageshorthands}}{}{\languageshorthands{\languagename}}
\gnumericTableEnd

%%}
%\caption{}
%\label{table:2019_7}
%\end{table}
%\renewcommand{\theequation}{\theenumi}
%\begin{enumerate}[label=\arabic*.,ref=\thesubsection.\theenumi]
%\numberwithin{equation}{enumi}
%
%\item Show that 
%\begin{align}
%\pr{G|B_3} = \frac{3}{8}
%\end{align}
%\item Show that 
%\begin{align}
%\pr{G} = \frac{39}{80} 
%\end{align}
%\solution 
%\begin{align}
%\because \pr{G|B_1} &= \frac{1}{2}, \pr{G|B_2} = \frac{5}{8}, \pr{G|B_3} = \frac{3}{8},
%\nonumber \\
%\pr{G} &= \sum_{i=1}^{3}\pr{G|B_i}\pr{B_i}
%\\
% &= \frac{1}{2}\times\frac{3}{10}+\frac{5}{8} \times\frac{3}{10}+\frac{3}{8}\times \frac{4}{10}
%\\
% &= \frac{39}{80}
%\end{align}
%\item Is
%\begin{align}
%\pr{B_3|G} = \frac{5}{13}\, ?
%\end{align}
%\solution 
%\begin{align}
%\pr{B_3|G} &= \frac{\pr{G|B_3}\pr{B_3}}{\pr{G}}
%\\
%&= \frac{\frac{3}{8}\times \frac{4}{10}}{\frac{39}{80}} = \frac{4}{13} \ne \frac{5}{13}
%\end{align}
%\item Is
%\begin{align}
%\pr{B_3 \cap G} = \frac{3}{10}\, ?
%\end{align}
%\solution 
%\begin{align}
%\pr{B_3 \cap G} &= \pr{G|B_3}\pr{B_3} \\
%\\
%=& \frac{3}{8} \times \frac{4}{10}=\frac{3}{20}\ne\frac{3}{10}
%\end{align}
%\end{enumerate}
%\section{Combinatorics}
%\begin{enumerate}[label=\thesection.\arabic*
%,ref=\thesection.\theenumi]
%\item Five persons $A, B, C, D $ and $E$ are seated in a circuar arrangment. Let $A, C, E$ be given the colour green. Find the number of ways to distribute blue and red to $B$ and $D$.
%\\
%\solution Both $B$  and $D$ can be given either blue or red.  The number of possible ways is
%\begin{align}
%\label{eq:qp2_10_ace_g}
%2\times 2 = 4
%\end{align}
%\item Repeat the above exercise with $B,D, A$ having the colour green.
%\item Find the total number of ways in which the colour green can be distributed to alternate persons so that persons seated in adjacent seats get different coloured hats.
%\\
%\solution The number of such ways is
%\begin{align}
%\label{eq:qp2_10_g}
%2\times 2 \times 2= 8
%\end{align}
%\item  If each person  is given a hat of one of 3 colours red, blue and green, then find the number of ways of distributing the hats such that the persons seated in adjacent seats get different coloured hats.
%\\
%\solution The number of ways is 
%\begin{align}
%\label{eq:qp2_10_sol}
%2\times 2 \times 2 \times 3= 24
%\end{align}
%\end{enumerate}
