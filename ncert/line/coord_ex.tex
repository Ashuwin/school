\renewcommand{\theequation}{\theenumi}
\begin{enumerate}[label=\arabic*.,ref=\thesubsection.\theenumi]
\numberwithin{equation}{enumi}
\item Find the distance between the following pairs of points
\begin{enumerate}
\item 
\begin{align}
\myvec{2\\3}, \myvec{4\\1}
\end{align}
\item 
\begin{align}
\myvec{-5\\7}, \myvec{-1\\3}
\end{align}
\item 
\begin{align}
\myvec{a\\b}, \myvec{-1\\b}
\end{align}
\end{enumerate}
\item Find the distance between the points 
\begin{align}
\myvec{0\\0}, \myvec{36\\15}
\end{align}
%
\item A town B is located 36km east and 15 km north of the town A.  How would you find the distance from town A to town B without actually measuring it?
\item Determine if the points 
\begin{align}
\myvec{1\\5}, \myvec{2\\3}, \myvec{-2\\-11}
\end{align}
%
are collinear.	
\item Check whether 
\begin{align}
\myvec{5\\-2}, \myvec{6\\4}, \myvec{7\\-2}
\end{align}
are the vertices of an isosceles triangle.
\item Name the type of quadrilateral formed, if any, by the following points, and give reasons for your answer.
\begin{enumerate}
\item 
\begin{align}
\myvec{-1\\-2}, \myvec{1\\0},
\myvec{-1\\2}, \myvec{-3\\0}
\end{align}
\item 
\begin{align}
\myvec{-3\\5}, \myvec{3\\1},
\myvec{0\\3}, \myvec{-1\\-4}
\end{align}
\item 
\begin{align}
\myvec{4\\5}, \myvec{7\\6},
\\
\myvec{4\\3}, \myvec{1\\2}
\end{align}
\end{enumerate}
\item Find the point on the $x$-axis which is equidistant from 
\begin{align}
\myvec{2\\-5}, \myvec{-2\\9},
\end{align}
\item Find the values of $y$ for which the distance between the points 
\begin{align}
\vec{P} = \myvec{2\\-3}, \vec{Q} = \myvec{10\\y}
\end{align}
is 10 units.
\item Find the values of $x, y, z$ such that 
\begin{align}
\myvec{x\\2\\z}= \myvec{2\\y\\1}
\end{align}
\item If
\begin{align}
\vec{a} = \myvec{1\\2}, \vec{b} = \myvec{2\\1},
\end{align}
verify if  
\begin{enumerate}
\item $\norm{\vec{a}}=\norm{\vec{b}}$

\item $\vec{a}=\vec{b}$
\end{enumerate}
\item Find a vector $\vec{x}$ in the direction of \myvec{1\\-2} such that $\norm{\vec{x}} = 7$.
\item Find a unit vector in the direction of $\vec{a}+\vec{b}$, where 
\begin{align}
\vec{a} = \myvec{2\\2\\-5}, \vec{b} = \myvec{2\\1\\3}.
\end{align}
\item Show that each of the given three vectors is a unit vector
\begin{align}
 \frac{1}{7}\myvec{2\\3\\6}, \frac{1}{7}\myvec{3\\-6\\2}, \frac{1}{7}\myvec{6\\2\\-3}.
\end{align}
Also,  show that they are mutually perpendicular to each other.
\item Find $\norm{\vec{a}}$ and $\norm{\vec{b}}$ if 
\begin{align}
\brak{\vec{a}+\vec{b}}^T\brak{\vec{a}-\vec{b}} &= 8
\\
\norm{\vec{a}}&=8\norm{\vec{b}}
\end{align}
\item Evaluate the product 
\begin{align}
\brak{3\vec{a}-5\vec{b}}^T\brak{2\vec{a}+7\vec{b}} 
\end{align}
\item Find $\norm{\vec{a}}$ and $\norm{\vec{b}}$, if
\begin{align}
\norm{\vec{a}} &= \norm{\vec{b}},
\\
\vec{a}^T\vec{b} = \frac{1}{2} 
\end{align}
and the angle between $\vec{a}$ and $\vec{b}$ is $60\degree$.
\item Find $\vec{x}$ if  $\vec{a}$ is a unit vector such that
\begin{align}
\brak{\vec{x}-\vec{a}}^T\brak{\vec{x}+\vec{a}} = 12.
\end{align}
\item For 
\begin{align}
\vec{a} = \myvec{2\\2\\3}, \vec{b} = \myvec{-1\\2\\1}, \vec{c} = \myvec{3\\1\\0},
\end{align}
$\brak{\vec{a}+\lambda\vec{b}}\perp\vec{c}$.  Find $\lambda$.
\item Show that 
\begin{align}
\brak{\norm{\vec{a}}\vec{b}+\norm{\vec{b}}\vec{a}}\perp \brak{\norm{\vec{a}}\vec{b}-\norm{\vec{b}}\vec{a}}
\end{align}
\item If $\vec{a}^T\vec{a}=0$ and  $\vec{a}\vec{b}=0$, what can be concluded about the vector $\vec{b}$?
\item If $\vec{a},\vec{b},\vec{c}$ are unit vectors such that 
\begin{align}
\vec{a}+\vec{b}+\vec{c} = 0,
\end{align}
find the value of 
\begin{align}
\vec{a}^T\vec{b}+\vec{b}^T\vec{c}+\vec{c}^T\vec{a}.
\end{align}
\item If $\vec{a} \ne \vec{0}, \lambda \ne 0$, then $\norm{\lambda \vec{a}} = 1$ if
\begin{enumerate}
\item $\lambda =1$
\item $\lambda = -1$
\item $\norm{\vec{a}}=\abs{\lambda}$
\item $\norm{\vec{a}}=\frac{1}{\abs{\lambda}}$
\end{enumerate}
\item Given
\begin{align}
\vec{a}=\myvec{2\\1\\3},
\vec{b}=\myvec{3\\5\\-2},
\end{align}
find $\norm{\vec{a} \times \vec{b}}$.
\item Find ${\vec{a} \times \vec{b}}$ if 
\begin{align}
\vec{a}=\myvec{1\\-7\\7},
\vec{b}=\myvec{3\\-2\\2}.
\end{align}
\item Find a unit vector perpendicular to each of the vectors 
$\vec{a}+\vec{b}$ and $\vec{a}-\vec{b}$, where 
\begin{align}
\vec{a}=\myvec{3\\2\\2},
\vec{b}=\myvec{1\\2\\-2}.
\end{align}
\item If a unit vector $\vec{a}$ makes angles $\frac{\pi}{3}$ with the x-axis and $\frac{\pi}{4}$ with the y-axis and an acute angle $\theta$ with the z-axis, find $\theta$ and $\vec{a}$.
\item Show that 
\begin{align}
\brak{\vec{a}-\vec{b}}\times \brak{\vec{a}+\vec{b}} = 2\brak{\vec{a}\times\vec{b}}
\end{align}
\item If $\vec{a}^T\vec{b} = 0$ and $\vec{a}\times \vec{b}$ = 0, what can you conclude about $\vec{a}$ and $\vec{b}$?
\item If $\norm{\vec{a}} = 3, \norm{\vec{b}} =\frac{\sqrt{2}}{3}$, then $\vec{a}\times \vec{b}$ is a unit vector if the angle between $\vec{a}$ and $\vec{b}$ is 
\begin{enumerate}[itemsep = 2pt]
\begin{multicols}{2}
\item $\frac{\pi}{6}$
\item $\frac{\pi}{4}$
\item $\frac{\pi}{3}$
\item $\frac{\pi}{2}$
\end{multicols}
\end{enumerate}
\item  If 
$
\vec{a}=\myvec{1\\1\\1},
\vec{b}=\myvec{2\\-1\\3},
\vec{c}=\myvec{1\\-2\\1},
$
find a unit vector parallel to the vector $2\vec{a}-\vec{b}+3\vec{c}$.
\item Find a vector of magnitude 5 units, and parallel to the resultant of the vectors 
$
\vec{a}=\myvec{2\\3\\-1},
\vec{b}=\myvec{1\\-2\\1},
$
\item Show that the unit direction vector inclined equally to the coordinate axes is $\myvec{\frac{1}{\sqrt{3}}\\\frac{1}{\sqrt{3}}\\ \frac{1}{\sqrt{3}}}$.
\item Let 
$
\vec{a}=\myvec{1\\4\\2},
\vec{b}=\myvec{3\\-2\\7} \text{ and }
\vec{c}=\myvec{2\\-1\\4}.
$
Find a vector $\vec{d}$ such that $\vec{d}\perp\vec{a},\vec{d}\perp\vec{b}$ and $\vec{d}^T\vec{c} = 15$.
\item The scalar product of \myvec{1\\1\\1} with a unit vector along the sum  of the vectors \myvec{2\\4\\-5} and \myvec{\lambda\\2\\3} is unity.  Find the value of $\lambda$.
\item Prove that 
\begin{align}
\brak{\vec{a}+\vec{b}}^T\brak{\vec{a}+\vec{b}} &= \norm{\vec{a}}^2+\norm{\vec{b}}^2
\\
\iff \vec{a}&\perp\vec{b}.
\end{align}
\item If $\theta$ is the angle between two vectors $\vec{a}$ and $\vec{b}$, then $\vec{a}^T\vec{b} \ge $ only when 
\begin{enumerate}[itemsep = 2pt]
\begin{multicols}{2}
\item $0 < \theta < \frac{\pi}{2}$
\item $0 \le \theta \le \frac{\pi}{2}$
\item $0 < \theta < {\pi}$
\item $0 \le \theta \le {\pi}$
\end{multicols}
\end{enumerate}
\item Let $\vec{a}$ and $\vec{b}$ be two unit vectors and $\theta$ be the angle between them.  Then $\vec{a}+\vec{b}$ is a unit vector if 
\begin{enumerate}[itemsep = 2pt]
\begin{multicols}{2}
\item $\theta = \frac{\pi}{4}$
\item $\theta = \frac{\pi}{3}$
\item $\theta = \frac{\pi}{2}$
\item $\theta = \frac{2\pi}{3}$
\end{multicols}
\end{enumerate}
\item The value of 
\begin{multline}
\myvec{1\\0\\0}^T\brak{\myvec{0\\1\\0}\times \myvec{0\\0\\1}}
+\myvec{0\\1\\0}^T\brak{\myvec{1\\0\\0}\times \myvec{0\\0\\1}}
\\
+\myvec{0\\0\\1}^T\brak{\myvec{1\\0\\0}\times \myvec{0\\1\\0}}
\end{multline}
%
is 
\begin{enumerate}[itemsep = 2pt]
\begin{multicols}{2}
\item 0
\item -1
\item 1
\item 3
\end{multicols}
\end{enumerate}
\item If $\theta$ is the angle between any two vectors $\vec{a}$ and $\vec{b}$, then 
$\norm{\vec{a}^T\vec{b}} = \norm{\vec{a} \times \vec{b}}$ when $\theta$ is equal to 
\begin{enumerate}[itemsep = 2pt]
\begin{multicols}{2}
\item 0
\item $\frac{\pi}{4}$
\item $\frac{\pi}{2}$
\item $\pi$.
\end{multicols}
\end{enumerate}
\item Let $\norm{\vec{a}} = 3, \norm{\vec{b}}= 4, \norm{\vec{c}} = 5$ such that each vector is perpendicular to the other two.  Find $\norm{\vec{a}+\vec{b}+\vec{c}}$.
\item Given 
\begin{align}
 \vec{a}+\vec{b}+\vec{c} = \vec{0}, 
\end{align}
evaluate 
\begin{align}
 \vec{a}^T\vec{b}+\vec{b}^T\vec{c}+\vec{c}^T\vec{a},
\end{align}
given that $\norm{ \vec{a}}=3, \norm{ \vec{b}}= 4$ and $\norm{ \vec{c}} = 2 $.
\item Let $\bm{\alpha} = \myvec{3\\-1\\0}, \bm{\beta} = \myvec{2\\1\\-3}$.  Find $\bm{\beta}_1, \bm{\beta}_2 $ such that $\bm{\beta}_1 \parallel  \bm{\alpha} $ and $\bm{\beta}_2 \perp \bm{\alpha} $.
\item Find a unit vector that makes an angle of $90\degree, 60\degree$ and $30\degree$ with the positive x, y and z axis respectively.
\item Find a unit vector in the direction of \myvec{2\\-1\\-2}.
\item Find a unit vector in the direction of the line passing through \myvec{-2\\4\\-5} and $1\\2\\3$.
\item Find a unit vector that makes an angle of $90\degree, 135\degree$ and $45\degree$ with the positive x, y and z axis respectively.
\item Show that the lines with direction vectors \myvec{12\\-3\\-4}, \myvec{4\\12\\3} and \myvec{3\\-4\\12} are mutually perpendicular.
\item Show that the line through the points \myvec{1\\-1\\2}, \myvec{3\\4\\-2} is parallel to the line through the points   \myvec{0\\3\\2}, \myvec{3\\5\\6}.
\item Show that the line through the points \myvec{4\\7\\8}, \myvec{2\\3\\4} is parallel to the line through the points   \myvec{-1\\-2\\1}, \myvec{1\\2\\5}.
\end{enumerate}
%
