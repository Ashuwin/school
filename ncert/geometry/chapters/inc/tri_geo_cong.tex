%%
\renewcommand{\theequation}{\theenumi}
\begin{enumerate}[label=\arabic*.,ref=\thesubsection.\theenumi]
\numberwithin{equation}{enumi}

\item In $\triangle$s $IDB$ and $IFB$ in  Fig. \ref{fig:tri_icentre},  $ID\perp BC, IF\perp AB, IB$ is a common side and $ID = IF$, i.e. both triangles are right angled, have the same hypotenuse and one equal leg.  This information was sufficient to show that $BD = BF$.  Similarly, it can be shown that all angles of both triangles are equal.  Such triangles are known as {\em congruent} triangles and denoted by $\triangle IDB \cong \triangle IEB$.
\item Show that $IA, IB, IC$ bisect the angles $A, B$ and $C$ respectively.  
\item Angle bisectors of $\triangle ABC$ meet at the incentre $\vec{I}$.

\item
To show that two triangles are congruent, it is sufficient to show that some corresponding angles and sides are equal.  

\item
RHS:	For two right angled triangles, if the hypotenuse and one of the sides are equal, show that the triangles are congruent.


\item
SSS:	Show that if the corresponding sides of three triangles are equal, the triangles are congruent.

\item
ASA:	Show that if two angles and any one side  are equal in corresponding triangles, the triangles are congruent.

\item
SAS:	Show that if two sides and the angle between them are equal in corresponding triangles, the triangles are congruent.%

\end{enumerate}

