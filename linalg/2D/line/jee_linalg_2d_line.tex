\documentclass[journal,12pt,twocolumn]{IEEEtran}
\usepackage{setspace}
\usepackage{gensymb}
\usepackage{caption}
%\usepackage{multirow}
%\usepackage{multicolumn}
%\usepackage{subcaption}
%\doublespacing
\singlespacing
\usepackage{csvsimple}
\usepackage{amsmath}
\usepackage{multicol}
%\usepackage{enumerate}
\usepackage{amssymb}
%\usepackage{graphicx}
\usepackage{newfloat}
%\usepackage{syntax}
\usepackage{listings}
\usepackage{iithtlc}
\usepackage{color}
\usepackage{tikz}
\usetikzlibrary{shapes,arrows}



%\usepackage{graphicx}
%\usepackage{amssymb}
%\usepackage{relsize}
%\usepackage[cmex10]{amsmath}
%\usepackage{mathtools}
%\usepackage{amsthm}
%\interdisplaylinepenalty=2500
%\savesymbol{iint}
%\usepackage{txfonts}
%\restoresymbol{TXF}{iint}
%\usepackage{wasysym}
\usepackage{amsthm}
\usepackage{mathrsfs}
\usepackage{txfonts}
\usepackage{stfloats}
\usepackage{cite}
\usepackage{cases}
\usepackage{mathtools}
\usepackage{caption}
\usepackage{enumerate}	
\usepackage{enumitem}
\usepackage{amsmath}
%\usepackage{xtab}
\usepackage{longtable}
\usepackage{multirow}
%\usepackage{algorithm}
%\usepackage{algpseudocode}
\usepackage{enumitem}
\usepackage{mathtools}
\usepackage{hyperref}
%\usepackage[framemethod=tikz]{mdframed}
\usepackage{listings}
    %\usepackage[latin1]{inputenc}                                 %%
    \usepackage{color}                                            %%
    \usepackage{array}                                            %%
    \usepackage{longtable}                                        %%
    \usepackage{calc}                                             %%
    \usepackage{multirow}                                         %%
    \usepackage{hhline}                                           %%
    \usepackage{ifthen}                                           %%
  %optionally (for landscape tables embedded in another document): %%
    \usepackage{lscape}     


\usepackage{url}
\def\UrlBreaks{\do\/\do-}


%\usepackage{stmaryrd}


%\usepackage{wasysym}
%\newcounter{MYtempeqncnt}
\DeclareMathOperator*{\Res}{Res}
%\renewcommand{\baselinestretch}{2}
\renewcommand\thesection{\arabic{section}}
\renewcommand\thesubsection{\thesection.\arabic{subsection}}
\renewcommand\thesubsubsection{\thesubsection.\arabic{subsubsection}}

\renewcommand\thesectiondis{\arabic{section}}
\renewcommand\thesubsectiondis{\thesectiondis.\arabic{subsection}}
\renewcommand\thesubsubsectiondis{\thesubsectiondis.\arabic{subsubsection}}

% correct bad hyphenation here
\hyphenation{op-tical net-works semi-conduc-tor}

%\lstset{
%language=C,
%frame=single, 
%breaklines=true
%}

%\lstset{
	%%basicstyle=\small\ttfamily\bfseries,
	%%numberstyle=\small\ttfamily,
	%language=Octave,
	%backgroundcolor=\color{white},
	%%frame=single,
	%%keywordstyle=\bfseries,
	%%breaklines=true,
	%%showstringspaces=false,
	%%xleftmargin=-10mm,
	%%aboveskip=-1mm,
	%%belowskip=0mm
%}

%\surroundwithmdframed[width=\columnwidth]{lstlisting}
\def\inputGnumericTable{}                                 %%
\lstset{
%language=C,
frame=single, 
breaklines=true,
columns=fullflexible
}
 

\begin{document}
%
\tikzstyle{block} = [rectangle, draw,
    text width=3em, text centered, minimum height=3em]
\tikzstyle{sum} = [draw, circle, node distance=3cm]
\tikzstyle{input} = [coordinate]
\tikzstyle{output} = [coordinate]
\tikzstyle{pinstyle} = [pin edge={to-,thin,black}]

\theoremstyle{definition}
\newtheorem{theorem}{Theorem}[section]
\newtheorem{problem}{Problem}
\newtheorem{proposition}{Proposition}[section]
\newtheorem{lemma}{Lemma}[section]
\newtheorem{corollary}[theorem]{Corollary}
\newtheorem{example}{Example}[section]
\newtheorem{definition}{Definition}[section]
%\newtheorem{algorithm}{Algorithm}[section]
%\newtheorem{cor}{Corollary}
\newcommand{\BEQA}{\begin{eqnarray}}
\newcommand{\EEQA}{\end{eqnarray}}
\newcommand{\define}{\stackrel{\triangle}{=}}

\bibliographystyle{IEEEtran}
%\bibliographystyle{ieeetr}

\providecommand{\nCr}[2]{\,^{#1}C_{#2}} % nCr
\providecommand{\nPr}[2]{\,^{#1}P_{#2}} % nPr
\providecommand{\mbf}{\mathbf}
\providecommand{\pr}[1]{\ensuremath{\Pr\left(#1\right)}}
\providecommand{\qfunc}[1]{\ensuremath{Q\left(#1\right)}}
\providecommand{\sbrak}[1]{\ensuremath{{}\left[#1\right]}}
\providecommand{\lsbrak}[1]{\ensuremath{{}\left[#1\right.}}
\providecommand{\rsbrak}[1]{\ensuremath{{}\left.#1\right]}}
\providecommand{\brak}[1]{\ensuremath{\left(#1\right)}}
\providecommand{\lbrak}[1]{\ensuremath{\left(#1\right.}}
\providecommand{\rbrak}[1]{\ensuremath{\left.#1\right)}}
\providecommand{\cbrak}[1]{\ensuremath{\left\{#1\right\}}}
\providecommand{\lcbrak}[1]{\ensuremath{\left\{#1\right.}}
\providecommand{\rcbrak}[1]{\ensuremath{\left.#1\right\}}}
\theoremstyle{remark}
\newtheorem{rem}{Remark}
\newcommand{\sgn}{\mathop{\mathrm{sgn}}}
\providecommand{\abs}[1]{\left\vert#1\right\vert}
\providecommand{\res}[1]{\Res\displaylimits_{#1}} 
\providecommand{\norm}[1]{\lVert#1\rVert}
\providecommand{\mtx}[1]{\mathbf{#1}}
\providecommand{\mean}[1]{E\left[ #1 \right]}
\providecommand{\fourier}{\overset{\mathcal{F}}{ \rightleftharpoons}}
%\providecommand{\hilbert}{\overset{\mathcal{H}}{ \rightleftharpoons}}
\providecommand{\system}{\overset{\mathcal{H}}{ \longleftrightarrow}}
	%\newcommand{\solution}[2]{\textbf{Solution:}{#1}}
\newcommand{\solution}{\noindent \textbf{Solution: }}
\newcommand{\myvec}[1]{\ensuremath{\begin{pmatrix}#1\end{pmatrix}}}
\providecommand{\dec}[2]{\ensuremath{\overset{#1}{\underset{#2}{\gtrless}}}}
\DeclarePairedDelimiter{\ceil}{\lceil}{\rceil}
%\numberwithin{equation}{subsection}
%\numberwithin{equation}{section}
%\numberwithin{problem}{subsection}
%\numberwithin{definition}{subsection}
\makeatletter
\@addtoreset{figure}{section}
\makeatother

\let\StandardTheFigure\thefigure
%\renewcommand{\thefigure}{\theproblem.\arabic{figure}}
\renewcommand{\thefigure}{\thesection}


%\numberwithin{figure}{subsection}

%\numberwithin{equation}{subsection}
%\numberwithin{equation}{section}
%\numberwithin{equation}{problem}
%\numberwithin{problem}{subsection}
\numberwithin{problem}{section}
%%\numberwithin{definition}{subsection}
%\makeatletter
%\@addtoreset{figure}{problem}
%\makeatother
\makeatletter
\@addtoreset{table}{section}
\makeatother

\let\StandardTheFigure\thefigure
\let\StandardTheTable\thetable
\let\vec\mathbf
%%\renewcommand{\thefigure}{\theproblem.\arabic{figure}}
%\renewcommand{\thefigure}{\theproblem}

%%\numberwithin{figure}{section}

%%\numberwithin{figure}{subsection}



\def\putbox#1#2#3{\makebox[0in][l]{\makebox[#1][l]{}\raisebox{\baselineskip}[0in][0in]{\raisebox{#2}[0in][0in]{#3}}}}
     \def\rightbox#1{\makebox[0in][r]{#1}}
     \def\centbox#1{\makebox[0in]{#1}}
     \def\topbox#1{\raisebox{-\baselineskip}[0in][0in]{#1}}
     \def\midbox#1{\raisebox{-0.5\baselineskip}[0in][0in]{#1}}

\vspace{3cm}

\title{ 
	\logo{
The Straight Line
	}
}

\author{ G V V Sharma$^{*}$% <-this % stops a space
	\thanks{*The author is with the Department
		of Electrical Engineering, Indian Institute of Technology, Hyderabad
		502285 India e-mail:  gadepall@iith.ac.in. All content in this manual is released under GNU GPL.  Free and open source.}
	
}	

\maketitle

%\tableofcontents

\bigskip

\renewcommand{\thefigure}{\theenumi}
\renewcommand{\thetable}{\theenumi}


\begin{abstract}
	Solved problems from JEE mains papers related to 2D lines in coordinate geometry are 
available in this document.  These problems are solved using linear algebra/matrix analysis.
\end{abstract}
\begin{enumerate}[label=\arabic*.]
\item A straight line through the origin   $\vec{O}$ meets the lines
\begin{align} 
\label{eq:line_1}
\myvec{4 & 3}\vec{x} &= 10
\\
\myvec{8 & 6}\vec{x} +5&= 0
\end{align} 
%
at $\vec{A}$ and $\vec{B}$ respectively.  Find the ratio in which  $\vec{O}$ divides $AB$.
\\
\solution Let 
\begin{align} 
\vec{n} =\myvec{4 & 3}
\end{align} 
%
Then \eqref{eq:line_1} can be expressed as
\begin{align} 
\label{eq:line_1_normal}
\vec{n}^T\vec{x} &= 10
\\
2\vec{n}^T\vec{x} &= -5
\end{align} 
%
and since $\vec{A}, \vec{B}$ satisfy \eqref{eq:line_1_normal} respectively,
\begin{align} 
\label{eq:line_1_normal_a}
\vec{n}^T\vec{A} &= 10
\\
2\vec{n}^T\vec{B} &= -5
\label{eq:line_1_normal_b}
\end{align} 
%
Let  $\vec{O}$ divide the segment $AB$ in the ratio $k:1$. Then
\begin{align} 
\label{eq:line_1_section}
\vec{O}=\frac{k\vec{B} +\vec{A} }{k+1}
\end{align} 
%
\begin{align} 
%\label{eq:line_1_section}
\because \vec{O}&= \vec{0},
\\
\vec{A} &=-k\vec{B}
\end{align} 
%
Substituting in \eqref{eq:line_1_normal_a}, and simplifying, 
\begin{align} 
\label{eq:line_1_normal_subs_a}
\vec{n}^T\vec{B} &= \frac{10}{-k}
\\
\vec{n}^T\vec{B} &= \frac{-5}{2}
%\label{eq:line_1_normal_b}
\end{align} 
resulting in 
\begin{align} 
\frac{10}{-k} = \frac{-5}{2} \implies k = 4
\end{align} 
\item The 
point 
\begin{equation} 
\vec{P}=\myvec{2\\ 1} 
\end{equation} 
is translated parallel to the line 
\begin{equation} 
\label{line_2}
L: \myvec{1 & -1}\vec{x} = 4 
\end{equation} 
% 
by $2\sqrt{3}$ units.  If the new point $\vec{Q}$ lies in the third 
quadrant, then find the equation of the line passing through $\vec{Q}$ and perpendicular to $L$. 
\\
\solution From \eqref{line_2}, the direction vector of $L$ is
\begin{equation} 
\label{line_2_m}
\vec{m} = \myvec{1 \\ 1} 
\end{equation} 
Thus, 
\begin{equation} 
\label{line_2_q}
\vec{Q}= \vec{P} + \lambda \vec{m}
\end{equation} 
However, 
\begin{align} 
PQ &= 2\sqrt{3}
\\
\implies\norm{\vec{P}- \vec{Q}} &= \abs{\lambda}\norm{\vec{m}} =  2\sqrt{3}
\\
\implies \lambda &= \pm \frac{2\sqrt{3}}{\norm{\vec{m}}} = \pm \sqrt{6}
\label{line_2_lam}
\end{align} 
%
\begin{align} 
\because \norm{\vec{m}} = \sqrt{\vec{m}^T\vec{m}}= \sqrt{2}
\end{align} 
%
from \eqref{line_2_m}.  Since $\vec{Q}$ lies in the third quadrant, from \eqref{line_2_q} and \eqref{line_2_lam},
\begin{align} 
\vec{Q} = \myvec{2\\ 1}  -  \sqrt{6}\myvec{1\\ 1} =  \myvec{2-\sqrt{6}\\ 1-\sqrt{6}}
\end{align} 
%
The equation of the desired line is then obtained as 
\begin{align} 
\label{line_2_final}
\vec{m}^T\brak{\vec{x}-\vec{Q}}&= 0
\\
 \myvec{1 & 1}\vec{x} &= 3 -\sqrt{6}
\end{align} 
\item A variable line drawn through the 
intersection of the lines 
\begin{align} 
\label{lines_3}
\myvec{4 & 3}\vec{x} &=12 
\\ 
\myvec{3 & 4}\vec{x} &=12 
\end{align} 
meets the coordinate axes at $\vec{A}$ and $\vec{B}$, then find the locus of the midpoint of $AB$. 
\\
\solution The intersection of the lines in \eqref{lines_3} is obtained through the matrix equation 
\begin{align}
\myvec{4 & 3 \\ 3 & 4}\vec{x}  &=\myvec{12\\12}
\end{align}
by forming the augmented matrix and row reduction as  
\begin{align}
\myvec{4 & 3 &12 \\ 3 & 4 &12} &\leftrightarrow \myvec{4 & 3 &12 \\ 0 & 7 &12} \leftrightarrow \myvec{28 & 0 & 48 \\ 0 & 7 &12}
\nonumber \\
\leftrightarrow \myvec{7 & 0 & 12 \\ 0 & 7 &12}&
\end{align}
resulting in 
\begin{align}
\vec{C}=\frac{1}{7}\myvec{12\\12}
\end{align}
%
Let the $\vec{R}$ be the mid point of $AB$. Then,
\begin{align}
\label{eq:lines_3_abr}
\vec{A} =2 \myvec{1 & 0\\0 & 0}\vec{R} 
\\
\vec{B} =2 \myvec{0 & 0\\0 & 1}\vec{R} 
\end{align}
%
Let the equation of $AB$ be 
%equation of a line passing through $\vec{C}$ be 
\begin{align}
\vec{n}^T\brak{\vec{x} - \vec{C}} = 0
\label{eq:lines_3_def}
\end{align}
Since $\vec{R}$ lies on $AB$, 
\begin{align}
\vec{n}^T\brak{\vec{R} - \vec{C}} = 0
\label{eq:lines_3_rc}
\end{align}
Also, 
\begin{align}
\vec{n}^T\brak{\vec{A} - \vec{B}} = 0
\label{eq:lines_3_ab}
\end{align}
%
Substituting from \eqref{eq:lines_3_abr} in \eqref{eq:lines_3_ab},
\begin{align}
\vec{n}^T\myvec{1 & 0 \\ 0 & -1}\vec{R} = 0
\label{eq:lines_3_r_sub}
\end{align}
%
From \eqref{eq:lines_3_rc} and \eqref{eq:lines_3_r_sub},
\begin{align}
\brak{\vec{R} - \vec{C}} = k\myvec{1 & 0 \\ 0 & -1}\vec{R}
\label{eq:lines_3_k}
\end{align}
%
for some constant $k$.
Multiplying both sides of \eqref{eq:lines_3_k} by 
\begin{align}
\vec{R}^T\myvec{0 & 1 \\ 1 & 0},
\end{align}
\begin{align}
\vec{R}^T\myvec{0 & 1 \\ 1 & 0}\brak{\vec{R} - \vec{C}} &= k\vec{R}^T\myvec{0 & 1 \\ 1 & 0}\myvec{1 & 0 \\ 0 & -1}\vec{R}
\nonumber \\
&= k\vec{R}^T\myvec{0 & -1 \\ 1 & 0}\vec{R} = 0
\label{eq:lines_3_mul}
\end{align}
\begin{align}
\because \vec{R}^T\myvec{0 & -1 \\ 1 & 0}\vec{R} = 0
\end{align}
which can be easily verified for any $\vec{R}$.
%
from \eqref{eq:lines_3_mul},
\begin{align}
\vec{R}^T\myvec{0 & 1 \\ 1 & 0}\brak{\vec{R} - \vec{C}} = 0
\nonumber \\
\implies \vec{R}^T\myvec{0 & 1 \\ 1 & 0}\vec{R} - \vec{R}^T\myvec{0 & 1 \\ 1 & 0}\vec{C} = 0
\nonumber \\
\implies \vec{R}^T\myvec{0 & 1 \\ 1 & 0}\vec{R} - \vec{C}^T\myvec{0 & 1 \\ 1 & 0}\vec{R} = 0
\end{align}
%
which is the desired locus.
\item Two sides of a rhombus are along the lines
\begin{align}
\myvec{1 & -1}\vec{x} + 1 &=0
\\
\myvec{7 & -1}\vec{x} -5 &=0.
\end{align}
%
If its diagonals intersect at 
\begin{equation}
\myvec{-1\\ -2},
\end{equation}
find its vertices.
\item Let $k$ be an integer such that the triangle with vertices
\begin{equation}
\myvec{k\\-3k},
\myvec{5\\k},
\myvec{-k\\2}
\end{equation}
has area 28.  Find the orthocentre of this triangle.
\item If an equlateral triangle, having centroid at the origin, has a side along the line
\begin{equation}
\myvec{1 & 1}\vec{x} = 2,
\end{equation}
then find the area of this triangle.
\item Find the locus of the point of intersection of the straight lines
\begin{align}
\myvec{t & -2 }\vec{x} -3t &= 0
\\
\myvec{1 & -2t }\vec{x} +3 &= 0
\end{align}
\item A square, of each side 2, lies above the $x$-axis and has one vertex at the origin.  If one of the sides 
passing through the origin makes an angle $30^{\degree}$ with the positive direction of the $x$-axis, then 
find the 
sum of the $x$-coordinates of the vertices of the square.
\item Find the locus of the point of intersection of the lines
\begin{align}
\myvec{\sqrt{2} & -1 }\vec{x} + 4 \sqrt{2}k &= 0
\\
\myvec{\sqrt{2}k & k }\vec{x} - 4 \sqrt{2} &= 0
\end{align}
\item The sides of a rhombus $ABC$ are parallel to the lines
\begin{align}
\myvec{1 & -1}\vec{x} + 2 &=0
\\
\myvec{7 & -1}\vec{x} + 3 &=0.
\end{align}
If the diagonals of the rhombus intersect at
\begin{align}
\vec{P} = \myvec{1 \\ 2}
\end{align}
and the vertex $\vec{A}$ (different) from the origin is on the $y$-axis, then find the ordinate of $A$.

\end{enumerate}
\end{document}
