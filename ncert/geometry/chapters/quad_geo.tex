\renewcommand{\theequation}{\theenumi}
\begin{enumerate}[label=\arabic*.,ref=\thesubsection.\theenumi]
\numberwithin{equation}{enumi}
	%
\item Sum of the angles of a quadrilateral is 360$\degree$. 
\\
\solution Draw the diagonal and use the fact that sum of the angles of a triangle is 180$\degree$.
\item  A diagonal of a parallelogram divides it into two congruent triangles. 
\\
\solution The alternate angles for the parallel sides are equal.  The diagonal is common.  Use ASA congruence.
%
\item  In a parallelogram, 
\begin{enumerate}
\item opposite sides are equal 
\item  opposite angles are equal
\item  diagonals bisect each other
\end{enumerate}
%
\solution Since the diagonal divides the parallelogram into two congruent triangles, all the above results follow.
%
\item  A quadrilateral is a parallelogram, if 
%
\begin{enumerate}
\item opposite sides are equal or 
\item  opposite angles are equal or 
\item  diagonals bisect each other or 
\item a pair of opposite sides is equal and parallel
\end{enumerate}
%
\solution All the above lead to a quadrilateral that has two parallel sides, by showing that the alternate angles are equal.
%
%
\item A rectangle is a parallelogram with one angle that is 90$\degree$.  Show that all angles of the rectangle are 90$\degree$.
%
\\
\solution Draw a diagonal.  Since the diagonal divides the rectangle into two congruent triangles, the angle opposite to the right angle is also 90$\degree$. Using congruence, it can be shown that the other two angles are equal.  Now use the fact that the sum of the angles of a quadrilateral is 360$\degree$.
%
\item  Diagonals of a rectangle bisect each other and are equal and vice-versa. 
%
\\
\solution Use Baudhayana's theorem for equality of diagonals.
%
\item  Diagonals of a rhombus bisect each other at right angles and vice-versa. 
%
\\
\solution The median of an isoceles triangle is also its perpendicular bisector.
%
\item  Diagonals of a square bisect each other at right angles and are equal, and vice-versa. 
%
\\
\solution A square has the properties of a rectangle as well as a rhombus.
%
%
\item  The quadrilateral formed by joining the mid-points of the sides of a quadrilateral, in order, is a parallelogram.
%
\\
\solution Draw one diagonal and use Problem \eqref{prob:quad_similar}.  Repeat for the other diagonal to show that the sides are parallel.
%
\item Two parallel lines l and m are intersected by a transversal p. Show that the quadrilateral formed by the bisectors of interior angles is a rectangle.
%
\item Show that the bisectors of angles of a parallelogram form a rectangle.
%
\item A quadrilateral is a parallelogram if a pair of opposite sides is equal and parallel.
%
\item $ABCD$ is a parallelogram in which $P$ and $Q$ are mid-points of opposite sides $AB$ and $CD$. If $AQ$ intersects $DP$ at $S$ and $BQ$ intersects $CP$ at $R$, show that: 
%
\begin{enumerate}
\item  $APCQ$ is a parallelogram. 
\item $DPBQ$ is a parallelogram. 
\item $PSQR$ is a parallelogram.
\end{enumerate}
%
\item $l, m$ and $n$ are three parallel lines intersected by transversals $p$ and $q$ such that $l, m$ and $n$ cut off equal intercepts $AB$ and $BC$ on $p$ . Show that $l, m$ and $n$ cut off equal intercepts $DE$ and $EF$ on $q$ also.
%
\item Parallelograms on the same base (or equal bases) and between the same parallels are equal in area.
\item Area of a parallelogram is the product of its base and the corresponding altitude. 
\item Parallelograms on the same base (or equal bases) and having equal areas lie between the same parallels.
\item If a parallelogram and a triangle are on the same base and between the same parallels, then area of the triangle is half the area of the parallelogram.

\end{enumerate}
