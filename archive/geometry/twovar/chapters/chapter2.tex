\subsection{Lines and Angles}
\begin{problem}
\label{prob:ch2_line_slope}
In Fig. \ref{fig:ch2_line}, the line passing through the points $A,B,C$ makes an angle $\theta$ with the $X$-axis.  Show that the equation of the line is
\begin{equation}
y = mx + c,
\label{eq:ch2_line_slope}
\end{equation}
where $\brak{x,y}$ is any point on the line and $m = \tan \theta$.  $m$ is also known as the slope of the line.
%
\end{problem}
%
\begin{figure}[!h]
\centering
\resizebox {\columnwidth} {!} {
\begin{tikzpicture}
  [
    scale=2,
    >=stealth,
    point/.style = {draw, circle,  fill = black, inner sep = 0.5pt},
    dot/.style   = {draw, circle,  fill = black, inner sep = .2pt},
%    every label/.append style={text=black, font=\scriptsize}
  ]
  \def\aa{3}
  \def\ab{4}    
  \def\ba{2}
  \def\bb{3}    
  \newcommand{\ranglesize}{0.1cm}  
  
  \coordinate [point, label={above right:$(a_1,a_2)$ $A$}] (A) at (\aa, \ab);
  \coordinate [point, label={above left:$(b_1,b_2)$ $B$}] (B) at (\ba, \bb);
  \coordinate [point, label={below left:$(0,0)$ $O$}] (O) at (0, 0);  
  \coordinate [point, label={below :$(a_1,0)$ $A_1$}] (A1) at (\aa, 0);  
  \coordinate [point, label={left:$(0,a_2)$ $A_2$}] (A2) at (0, \ab);  
 \coordinate [ label={below:$X$}] (X) at ($ (\aa,0) +(1,0) $);   
 \coordinate [ label={left :$Y$}] (Y) at ($ (0,\ab,0) +(0,1) $);    
 \coordinate [ point, label={left:$C$ $\brak{0,c}$}] (C) at (0,1);     
 \coordinate [ point, label={left:$Q$ }] (Q) at (-1,0);      
  \node (P) at ($(B)!0.5!(C)$) [point, label = {above left:$P$ $\brak{x,y}$}]{}; 
%  \coordinate [point, label={below: $X$}] (X) at ($(0, \ab)+(0,2)$);   
 % \coordinate [point, label={below right:$X$}] (X) at ($ (0,\aa) +(1,0) $);   
  \coordinate [point, label={below :$(b_1,0)$ $B_1$}] (B1) at (\ba, 0);  
  \node (P1) at ($(B1)!0.5!(O)$) [point, label = {below:$P_1$ $\brak{x,0}$}]{};     
  \node (R) at ($(P1)!0.5!(P)$) [point, label = {right:$R$}]{};       
%  \coordinate [point, label={left:$(0,b_2)$ $B_2$}] (B2) at (0, \bb);        
%  \coordinate [point, label={right:$(a_1,b_2)$ $P$}] (P) at (\aa, \bb); 
%  \coordinate [point, label={below:$\substack{(a_1-b_1,0) \\ P^{'}}$}] (PP) at (\aa-\ba, 0);         
%  \coordinate [point, label={right:$\substack{(a_1-b_1,a_2-b_2) \\ A^{\prime}}$ }] (AP) at (\aa-\ba, \ab-\bb);  
%%  \coordinate [point, label={right:$(a_1,b_2)$ $P$}] (P) at (0, \ab-\bb);           
  \path
     (A)    edge      (Q);
\draw[->]     (Q)   --	(X) ;    
\draw[->]     (O)    --	(Y)  ;        

  \draw (O) -- (0, \ranglesize) -- (\ranglesize, \ranglesize) --(\ranglesize, 0);     
  \draw [dashed] 
  (A)  -- (A1)
  (C)  -- (R)
  (P)  -- (P1)
  (B)  -- (B1)
%  (O)  -- (AP)
%  (PP)  -- (AP)  
  ;  
\path[clip] (Q) -- (X) -- (A);  
\fill[black, opacity=0.5, draw=black] 
(Q) circle (5mm);
\node at ($(Q)+(22.5:7mm)$) {$\theta$};
  
%  \coordinate [point, label={above:$A$ $(0,b)$}] (A) at (0, 1);  
%  \coordinate [point, label={below right:$C$ $(c,0)$}] (C) at (1,0);
%
%  \path
%     (A)    edge      (B)  
%	 (B)    edge      (C)
%	 (C)    edge  node[sloped, anchor=center, above, text width=2.0cm] { $\sqrt{b^2+c^2}$}     (A)	 
%	 ;

%  \draw (B) -- (0, \ranglesize) -- (\ranglesize, \ranglesize) --(\ranglesize, 0);


%  \draw  [line width=0.00mm ] (A) -- (C) -- (B) -- (A);

\end{tikzpicture}


}
\caption{The straight line.}
\label{fig:ch2_line}
\end{figure}
\proof Let $P = (x,y)$ be any point on the line. From the figure, is is obvious that
%
\begin{equation}
\frac{PR}{CR} = \frac{y-c}{x} = \tan \theta = m
\end{equation}
%
Rearranging terms,
\begin{equation}
y = mx + c
\end{equation}
%
\begin{problem}
Show that the equation of the line in Fig. \ref{fig:ch2_line} can also be expressed as
%
\begin{equation}
\label{eq:ch2_line_two_pt}
\frac{y-b_2}{x-b_1} = \frac{a_2-b_2}{a_1-b_1}
\end{equation}
%
\end{problem}
\proof Looking at the line segments $AB$ and $PB$ and following the approach in Problem \ref{prob:ch2_line_slope}, 
\begin{equation}
\tan \theta = \frac{a_2-b_2}{a_1-b_1} = \frac{y-b_2}{x - b_1}
\end{equation}
\begin{problem}
Show that the equation
\begin{equation}
px+qy + r = 0
\label{eq:ch2_line_ab}
\end{equation}
is that of a straight line.
\end{problem}
\proof The above equation can be rearranged as
%
\begin{equation}
y = -\frac{p}{q}x - \frac{r}{q}
\end{equation}
%
which is the same as \eqref{eq:ch2_line_slope} with $m = -\frac{p}{q}$ and $c = -\frac{r}{q}$.
\begin{problem}
The point of intersection of the lines
\begin{align}
p_1x+q_1y+r_1 &= 0
\\
p_2x+q_2y+r_2 &= 0
\end{align}
is
\begin{align}
x &= -\frac{
\begin{vmatrix}
r_1 & q_1
\\
r_2 & q_2
\end{vmatrix}
}{
\begin{vmatrix}
p_1 & q_1
\\
p_2 & q_2
\end{vmatrix}
}
\\
y &= -\frac{
\begin{vmatrix}
p_1 & r_1
\\
p_2 & r_2
\end{vmatrix}
}{
\begin{vmatrix}
p_1 & q_1
\\
p_2 & q_2
\end{vmatrix}
}
\end{align}
\end{problem}
\begin{problem}
Show that angle between the lines
%
\begin{align}
y &= m_1 x + c_1
\\
y &= m_2 x = c_2
\end{align}
%
is
%
\begin{equation}
\label{eq:ch1_line_ang}
\theta = \tan ^{-1} \frac{m_1 - m_2}{1 + m_1m_2}
\end{equation}
%
\end{problem}
\proof If $\theta_1$ and $\theta_2$ are the angles made by the two lines with the $X$-axis respectively, the angle between the two is $\theta = \theta_1 - \theta_2$.  Thus, using the trigonometric identity,
\begin{align}
\tan \theta &= \tan \brak{\theta_1 - \theta_2} = \frac{\tan \theta_1 - \tan \theta_2}{1 + \tan \theta_1 \tan \theta_2}
\\
&=
\frac{m_1 - m_2}{1 + m_1m_2}
\end{align}
\begin{problem}
\label{prob:ch2_perp}
Show that the angle between two lines is $90^{\degree}$ if $m_1m_2 = -1$.
\end{problem}
%
\subsection{Altitude}
\begin{definition}
In $\Delta ABC$ Fig. \ref{fig:ch2_alt_def}, $AD \perp BC$ is known as the altitude.
\end{definition}
%
\begin{figure}[!h]
\centering
\resizebox {\columnwidth} {!} {
\begin{tikzpicture}
  [
    scale=2,
    >=stealth,
    point/.style = {draw, circle,  fill = black, inner sep = 0.5pt},
    dot/.style   = {draw, circle,  fill = black, inner sep = .2pt},
  ]
  \coordinate [point, label={below left:$B$}] (B) at (0, 0);
    \node (A) at +(60:{2*sqrt(3)}) [point, label = above:$A$] {};
  \coordinate [point, label={below right:$C$}] (C) at ($ (3,0) + sqrt(3)*(1,0) $);
  \node (D) at ({sqrt(3)},0) [point, label = below:$D$] {};
  \draw  (A) -- (C) -- (B) -- (A);
%  \node (D) at ($(B)!0.5!(C)$) [point, label = {below:$D$}]{};
  \draw (A) -- (D);  
  \tkzMarkRightAngle[fill=blue!20,size=.2](A,D,C)

\end{tikzpicture}


}
\caption{Altitude $AD$.}
\label{fig:ch2_alt_def}
\end{figure}
\begin{problem}
In Fig. \ref{fig:ch2_alt_ortho}, the  altitudes $AD$ and $BE$ meet at $O$.  $CF$ is the line obtained by extending $CO$ to meet $AB$ at $F$.  Show that $CF$ is also an altitude of $\Delta ABC$.
\end{problem}
%
\begin{figure}[!h]
\centering
\resizebox {\columnwidth} {!} {
\begin{tikzpicture}
  [
    scale=2,
    >=stealth,
    point/.style = {draw, circle,  fill = black, inner sep = 0.5pt},
    dot/.style   = {draw, circle,  fill = black, inner sep = .2pt},
  ]
  \coordinate [point, label={below left:$B$ $\brak{0,0}$}] (B) at (0, 0);
    \node (A) at +(60:{2*sqrt(3)}) [point, label = above:$A$ $\brak{a,b}$] {};
  \coordinate [point, label={below right:$C$ $\brak{c,0}$}] (C) at ($ (3,0) + sqrt(3)*(1,0) $);
  \node (D) at ({sqrt(3)},0) [point, label = below:$D$ $\brak{a,0}$] {};
    \node (E) at +(45:{(3+sqrt(3))/sqrt(2)}) [point, label = above right:$E$] {};
    \node (O) at +(45:{sqrt(6)}) [point, label = right:$O$] {};    
    \node (F) at +(60:{(3+sqrt(3))/2}) [point, label = left:$F$] {};        
  \draw  (A) -- (C) -- (B) -- (A);
%  \node (D) at ($(B)!0.5!(C)$) [point, label = {below:$D$}]{};
  \draw (A) -- (D);  
  \draw (B) -- (E);    
  \draw[dashed] (C) -- (F);      
  \tkzMarkRightAngle[fill=blue!20,size=.2](A,D,C)
  \tkzMarkRightAngle[fill=blue!20,size=.2](B,E,A)  

\end{tikzpicture}


}
\caption{Altitudes $AD$ and $BE$ meet at $O$.}
\label{fig:ch2_alt_ortho}
\end{figure}
\proof The equation for $AD$ is
%
\begin{equation}
\label{ch2_line_AD}
x = a
\end{equation}
%
The slope of $AC$ is
%
\begin{equation}
\label{ch2_slope_AC}
\frac{b}{a-c}.
\end{equation}
%
Thus, using Problem \ref{prob:ch2_perp}, the slope of $CE$ is 
\begin{equation}
\label{ch2_slope_CE}
m_{CE} = \frac{c-a}{b}
\end{equation}
%
Using \eqref{eq:ch2_line_slope} and \eqref{ch2_slope_CE}, the equation of $CE$ is
\begin{equation}
\label{eq:ch2_line_CE}
y = m_{CE}\brak{x-0} = \frac{c-a}{b}x
\end{equation}
since $O$ is the intersection of $AD$ and $CE$, from \eqref{eq:ch2_line_CE} and  \eqref{ch2_line_AD}, its coordinates are
\begin{equation}
\label{eq:ch2_O_coord}
\brak{a,\brak{\frac{c-a}{b}}a}
\end{equation}
%
From \eqref{eq:ch2_O_coord}, the slope of $CO$ (and $CF$) is
%
\begin{equation}
\label{eq:ch2_slope_CF}
m_{CF} = \frac{\brak{\frac{c-a}{b}}a}{a-c} = -\frac{a}{b}
\end{equation}
%
From Fig. \ref{fig:ch2_alt_ortho}, the slope of $AB$ is
%
\begin{equation}
m_{AB} = \frac{b}{a}
\end{equation}
%
Since
%
\begin{equation}
m_{AB}m_{CF} = -1,
\end{equation}
%
using Problem \ref{prob:ch2_perp}, $AB \perp CF$ and $CF$ is an altitude of $\Delta ABC$.