
	\item Give five examples of data that you can collect from your day-to-day life.\\
	\item  Classify the data in Q.1 above as primary or secondary data.\\
	\item The blood groups of 30 students of Class VIII are recorded as follows:\\
A, B, O, O, AB, O, A, O, B, A, O, B, A, O, O,\\
A, AB, O, A, A, O, O, AB, B, A, O, B, A, B, O.\\
Represent this data in the form of a frequency distribution table. Which is the most common, and which is the rarest, blood group among these students?\\
\item The distance (in km) of 40 engineers from their residence to their place of work were
found as follows:
\begin{tabular}{ |c|c|c|c|c|c|c|c|c|c| } 
 5 &3 &10 &20 &25 &11 &13 &7 &12 &31  \\ 
 19 &10 &12 &17 &18 &11 &32 &17 &16 &2\\ 
 7 &9 &7 &8 &3 &5 &12 &15 &18 &3  \\ 
 12 &14 &2 &9 &6 &15 &15 &7 &6 &12\\ 
\end{tabular}\\
Construct a grouped frequency distribution table with class size 5 for the data given above taking the first interval as 0-5 (5 not included). What main features do you observe from this tabular representation?\\
\item The relative humidity (in $\%$) of a certain city for a month of 30 days was as follows:\\
\resizebox{\columnwidth}{12pt}{%
\begin{tabular}{ |c|c|c|c|c|c|c|c|c|c| } 
 98.1 &98.6 &99.2 &90.3 &86.5 &95.3 &92.9 &96.3 &94.2 &95.1  \\ 
 89.2 &92.3 &97.1 &93.5 &92.7 &95.1 &97.2 &93.3 &95.2 &97.3\\ 
 96.2 &92.1 &84.9 &90.2 &95.7 &98.3 &97.3 &96.1 &92.1 &89.0  \\ 
\end{tabular}\\%
}
(i) Construct a grouped frequency distribution table with classes 84 - 86, 86 - 88, etc.\\
(ii) Which month or season do you think this data is about?\\
(iii) What is the range of this data?\\
\item The heights of 50 students, measured to the nearest centimetres, have been found to be as follows:\\
\resizebox{\columnwidth}{12pt}{%
\begin{tabular}{ |c|c|c|c|c|c|c|c|c|c| } 
 161 &150 &154 &165 &168 &161 &154 &162 &150 &151  \\ 
 162 &164 &171 &165 &158 &154 &156 &172 &160 &170\\ 
 153 &159 &161 &170 &162 &165 &166 &168 &165 &164  \\ 
 154 &152 &153 &156 &158 &162 &160 &161 &173 &166\\ 
 161 &159 &162 &167 &168 &159 &158 &153 &154 &159  \\ 
\end{tabular}\\%
}\\

(i) Represent the data given above by a grouped frequency distribution table, taking the class intervals as 160 - 165, 165 - 170, etc.\\
(ii) What can you conclude about their heights from the table?\\
\item A study was conducted to find out the concentration of sulphur dioxide in the air in parts per million (ppm) of a certain city. The data obtained for 30 days is as follows:\\
\begin{tabular}{ |c|c|c|c|c|c| } 
 0.03 &0.08 &0.08 &0.09 &0.04 &0.17 \\ 
 0.16 &0.05 &0.02 &0.06 &0.18 &0.20 \\ 
 0.11 &0.08 &0.12 &0.13 &0.22 &0.07 \\ 
 0.08 &0.01 &0.10 &0.06 &0.09 &0.18 \\ 
 0.11 &0.07 &0.05 &0.07 &0.01 &0.04 \\ 
\end{tabular}\\
(i)Make a grouped frequency distribution table for this data with class intervals as 0.00-0.04, 0.04-0.08, and so on.\\
(ii) For how many days, was the concentration of sulphur dioxide more than 0.11 parts per million?\\
\item Three coins were tossed 30 times simultaneously. Each time the number of heads occurring was noted down as follows:\\
\begin{tabular}{ cccccccccc } 
 0 &1 &2 &2 &1 &2 &3 &1 &3 &0  \\ 
 1 &3 &1 &1 &2 &2 &0 &1 &2 &1\\ 
 3 &0 &0 &1 &1 &2 &3 &2 &2 &0\\ 
\end{tabular}\\
Prepare a frequency distribution table for the data given above.\\
\item The value of $\pi$ upto 50 decimal places is given below:
3.141592653589793238462643383279502884197\\
16939937510\\
(i)Make a frequency distribution of the digits from 0 to 9 after the decimal point.\\
(ii) What are the most and the least frequently occurring digits?\\
\item Thirty children were asked about the number of hours they watched TV programmes in the previous week. The results were found as follows:\\
\begin{tabular}{ cccccccccc } 
 1 &6 &2 &3 &5 &12 &5 &8 &4 &8  \\ 
 10 &3 &4 &12 &2 &8 &15 &1 &17 &6\\ 
 3 &2 &8 &5 &9 &6 &8 &7 &14 &12\\ 
\end{tabular}\\
(i) Make a grouped frequency distribution table for this data,taking class width 5 and one of the class intervals as 5-10.\\
(ii)  How many children watched television for 15 or more hours a week?\\
\item A company manufactures car batteries of a particular type. The lives (in years) of 40 such batteries were recorded as follows:\\
\begin{tabular}{ cccccccccc } 
 2.6 &3.0 &3.7 &3.2 &2.2 &4.1 &3.5 &4.5 \\ 
 3.5 &2.3 &3.2 &3.4 &3.8 &3.2 &4.6 &3.7\\ 
 2.5 &4.4 &3.4 &3.3 &2.9 &3.0 &4.3 &2.8\\ 
 3.5 &3.2 &3.9 &3.2 &3.2 &3.1 &3.7 &3.4\\ 
 4.6 &3.8 &3.2 &2.6 &3.5 &4.2 &2.9 &3.6\\ 
\end{tabular}\\
Construct a grouped frequency distribution table for this data, using class intervals of size 0.5 starting from the interval 2 - 2.5.\\
\item A survey conducted by an organisation for the cause of illness and death among the women between the ages 15 - 44 (in years) worldwide, found the following figures (in $\%$):\\

\begin{tabular}{|c|c|c|}
\hline
\textbf{S.No} &\textbf{Causes} &\textbf{female fatality rate($\%$)}\\
\hline
1 &Reproductive health conditionds &31.8\\
2 &Neuropsychiatric conditions &25.4\\
3 &Injuries &12.4\\
4 &Cardiovascular conditions &4.3\\
5 &Respiratory conditions &4.1\\
6 &Other causes &22.0\\
\hline
\end{tabular}\\
(i) Represent the information given above graphically.\\
(ii) Which condition is the major cause of women’s ill health and death worldwide?\\
(iii) Try to find out, with the help of your teacher, any two factors which play a major role in the cause in (ii) above being the major cause.\\
  \item The following data on the number of girls (to the nearest ten) per thousand boys in different sections of Indian society is given below.\\
  \begin{tabular}{|c|c|}
\hline
\textbf{Section} &\textbf{No. of girls per thousand boys}\\
\hline
 Scheduled Cast(SC) &940\\
 Scheduled Tribe(ST) &970\\
 Non SC/ST &920\\
 Backward districts &950\\
 Non backward districts &920\\
 Rural &930\\
 Urban &910\\
\hline
\end{tabular}\\

(i) Represent the information above by a bar graph.
(ii) In the classroom discuss what conclusions can be arrived at from the graph.\\
\item Given below are the seats won by different political parties in the polling outcome of a state assembly elections:\\
 \begin{tabular}{|c|c|c|c|c|c|c|}
\hline
\textbf{Political Party} &A &B &C &D &E &F\\
\hline
\textbf{Seats Won} &75 &55 &37 &29 &10 &37\\
\hline
\end{tabular}\\

(i) Draw a bar graph to represent the polling results.
(ii) Which political party won the maximum number of seats?\\
 \item The length of 40 leaves of a plant are measured correct to one millimetre, and the obtained data is represented in the following table:\\
\begin{tabular}{|c|c|}
\hline
\textbf{Length(in mm)} &\textbf{Number of leaves} \\
\hline
118-126 &3\\
127-135 &5\\
136-144 &9\\
145-153 &12\\
154-162 &5\\
163-171 &4\\
172-180 &2\\
\hline
\end{tabular}\\
 
(i) Draw a histogram to represent the given data. [Hint: First make the class intervals continuous]\\
(ii) Is there any other suitable graphical representation for the same data?\\
(iii) Is it correct to conclude that the maximum number of leaves are 153 mm long?\\
why?\\
\item The following table gives the life times of 400 neon lamps:\\

\begin{tabular}{|c|c|}
\hline
\textbf{Life time(in hours)} &\textbf{Number of lamps} \\
\hline
300-400 &14\\
400-500 &56\\
500-600 &60\\
600-700 &86\\
700-800 &74\\
800-900 &62\\
900-1000 &48\\
\hline
\end{tabular}\\
 

(i) Represent the given information with the help of a histogram.\\
(ii) How many lamps have a life time of more than 700 hours?\\
\item The following table gives the distribution of students of two sections according to
the marks obtained by them:\\

\begin{tabular}{|c|c|c|c|c|}
\hline
 \multicolumn{2}{c|}{Section A}  
    &\multicolumn{2}{c|}{Section B} \\
\cline{2-4}

 \textbf{Marks} &\textbf{Frequency} &\textbf{Marks} &\textbf{Frequency} \\
\hline
0-10 &3 &0-10 &5\\
10-20 &9 &10-20 &19\\
20-30 &17 &20-30 &15\\
30-40 &12 &30-40 &10\\
40-50 &9 &40-50 &1\\
\hline

\end{tabular}\\

Represent the marks of the students of both the sections on the same graph by two frequency polygons. From the two polygons compare the performance of the two sections.\\
\item The runs scored by two teams A and B on the first 60 balls in a cricket match are given below:\\

\begin{tabular}{|c|c|c|}
\hline
\textbf{Number of balls} &\textbf{Team A}  &textbf{Team B}\\
\hline
1-6 &2 &5\\
7-12 &1 &6\\
13-18 &8 &2\\
19-24 &9 &10\\
25-30 &4 &5\\
31-36 &5 &6\\
37-42 &6 &3\\
43-48 &10 &4\\
49-54 &6 &8\\
55-60 &2 &10\\
\hline
\end{tabular}\\

Represent the data of both the teams on the same graph by frequency polygons.\\
(Hint : First make the class intervals continuous.)\\
\item A random survey of the number of children of various age groups playing in a park was found as follows:\\

\begin{tabular}{|c|c|}
\hline
\textbf{Age (in years)} &\textbf{No.of Children}\\
\hline
1-2 &5\\
2-3 &3\\
3-5 &6\\
5-7 &12\\
7-10 &9\\
10-15 &10\\
15-17 &4\\
\hline
\end{tabular}\\

Draw a histogram to represent the data above.\\
\item 100 surnames were randomly picked up from a local telephone directory and a frequency distribution of the number of letters in the English alphabet in the surnames was found as follows:\\

\begin{tabular}{|c|c|}
\hline
\textbf{No.of letters} &\textbf{No.of Surnames}\\
\hline
1-4 &6\\
4-6 &30\\
6-8 &44\\
8-12 &16\\
12-20 &4\\
\hline
\end{tabular}\\

(i) Draw a histogram to depict the given information.\\
(ii) Write the class interval in which the maximum number of surnames lie.\\
\item The following number of goals were scored by a team in a series of 10 matches:\\
2, 3, 4, 5, 0, 1, 3, 3, 4, 3\\
Find the mean, median and mode of these scores.\\
\item In a mathematics test given to 15 students, the following marks (out of 100) are recorded:\\
41, 39, 48, 52, 46, 62, 54, 40, 96, 52, 98, 40, 42, 52, 60\\               
Find the mean, median and mode of this data.\\
\item The following observations have been arranged in ascending order. If the median of the data is 63, find the value of x.\\
29, 32, 48, 50, x, x + 2, 72, 78, 84, 95\\
\item Find the mode of \\
14, 25, 14, 28, 18, 17, 18, 14, 23, 22, 14, 18.\\
\item Find the mean salary of 60 workers of a factory from the following table:\\
\begin{tabular}{|c|c|}
\hline
\textbf{Salary (in \rupee)} &\textbf{No.of Workers}\\
\hline
3000 &16\\
4000 &12\\
5000 &10\\
6000 &8\\
7000 &6\\
8000 &4\\
9000 &3\\
10000 &1\\
\hline
\textbf{Total} &60\\
\hline
\end{tabular}\\

\item Give one example of a situation in which\\
(i) the mean is an appropriate measure of central tendency.\\
(ii) the mean is not an appropriate measure of central tendency but the median is an
appropriate measure of central tendency.\\



