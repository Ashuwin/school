\renewcommand{\theequation}{\theenumi}
\begin{enumerate}[label=\arabic*.,ref=\thesubsection.\theenumi]
\item Find the equations of the tangents to the following curves at the points specified:
\begin{multicols}{2}
\begin{enumerate}
{\small
\item
$
y=x\brak{x^2-1}, x=2
$
\item
$
y=x^2+\frac{1}{x^2}, x=1
$
\item
$
y=x^3+2x, x=0
$
\item
$
y=\brak{x+\frac{1}{x}}^3, x=2
$
\item
$
y=\brak{x^2-1}^2, x=1
$
\item
$
y=x^3-x+1, x=3
$
\item
$
y=\brak{x-a}^3, x=2a
$
\item
$
y=ax^2+2bx + c, \brak{x_1,y_1}
$
\item
$
y=\frac{x^3}{a^3}+\frac{a^3}{x^3}, x = a
$
\item
$
y = \frac{x^2}{a}+\frac{a^2}{x}, x = a
$
}
\end{enumerate}
\end{multicols}
\numberwithin{equation}{enumi}
\item Find the tangents to the curve 
\begin{align}
\vec{x}^T\myvec{1 & 0\\0 & 0}\vec{x} + \myvec{1 & -1}\vec{x}=0
\end{align}
%
 at the points where it is cut by
the line 
\begin{align}
\myvec{1 & -1}\vec{x} + 4 = 0
\end{align}
and find the point of intersection of the tangents.
\item Prove that the line 
\begin{align}
\myvec{3 & -4}\vec{x}+4=0
\end{align}
touches the curve 
\begin{align}
\vec{x}^T\myvec{1 & -\frac{1}{2}\\ -\frac{1}{2} & 0}\vec{x} + 1 = 0.
\end{align}
\item Find the points on the curve 
\begin{align}
3y=x^3+3x
\end{align}
at which the tangent is
parallel to the line 
\begin{align}
\myvec{5 & -1}\vec{x} = 0
\end{align}
\renewcommand{\theequation}{\theenumi}
\item Find at what points on the curve 
\begin{align}
\vec{x}^T\myvec{1 & 0 \\ 0 & 0}\vec{x}+\myvec{0 & -1}\vec{x}+9 = 0
\end{align}
 the tangents pass through the origin.
\numberwithin{equation}{enumi}
\item Show that there are three points on the curve
\begin{align}
3y=3x^4+8x^3-6x^2
\end{align}
at which the tangents are parallel to the line 
\begin{align}
\myvec{8 & -1}\vec{x} = 0
\end{align}
\item Show that the  line 
\begin{align}
\myvec{0 & 4}\vec{x}=17
\end{align}
meets the curve 
\begin{align}
y = x^2+\frac{1}{x^2}
\end{align}
in four points and that two of the points of intersection of the tangents at these
four points are on the line  
\begin{align}
\myvec{0 & 4}\vec{x}+1=0, 
\end{align}
and two are on the line 
\begin{align}
\myvec{1 & 0}\vec{x}=0.
\end{align}
\end{enumerate}
