\renewcommand{\theequation}{\theenumi}
\begin{enumerate}[label=\arabic*.,ref=\thesubsection.\theenumi]
\numberwithin{equation}{enumi}
\item $a_n = n(n+2)$
\item $a_n = \frac{n}{n+1}$
\item $a_n = 2^n$
\item $a_n = \frac{2n-3}{6}$
\item $a_n = (-1)^{n-1}5^{n+1}$
\item $a_n = n\frac{n^2+5}{4}$ \\
Find the indicated terms in each of the sequences whose $n^{th}$ terms are:
\item $a_n = 4n-3; a_{17}, a_{24}$
\item $a_n = \frac{n^2}{2^n};a_7$
\item $a_n = (-1)^{n-1}n^3;a_9$
\item $a_n = \frac{n(n-2)}{n+3};a_{20}.$\\
Write the first five terms of each of the sequences and obtain the
corresponding series:
\item $a_1 = 3, a_n = 3a_{n-1}+2$ for all n $>$ 1
\item $a_1 = -1, a_n = \frac{a_{n-1}}{n}, n \geq 2$
\item $a_1 = a_2 = 2, a_n = a_{n-1}-1,n > 2$ 
\item The fibonacci sequence is defined by \\$1 = a_1 = a_2$ and $a_n = a_{n-1}+a_{n-2},n>2 $\\
Find $\frac{a_{n+1}}{a_n},$ for n = 1, 2, 3, 4,5.
\item Find the sum of odd integers from 1 to 2001.
\item Find the sum of all natural numbers lying between 100 and 1000, which are multiples of 5.
\item In an A.P., the first term is 2 and the sum of the first five terms is one-fourth of the next five terms. Show that $20^{th}$ term is -112.
\item How many terms of the A.P. -6,$-\frac{11}{2}, -5$....are needed to give the sum -25?
\item In an A.P., If $p^{th}$ term is $\frac{1}{q}$ $q^{th}$ term is $\frac{1}{p}$, prove that the sum of first pq $\frac{1}{2}$(pq+1), where $p \neq q.$
\item  If the sum of a certain number of terms of the A.P. 25, 22, 19, ... is 116. Find the last term.
\item Find the sum to n terms of the A.P., whose $k^{th}$ term is 5k + 1.
\item If the sum of n terms of an A.P. is (pn + $qn^2$ ), where p and q are constants, find the common difference.
\item The sums of n terms of two arithmetic progressions are in the ratio 5n + 4 : 9n + 6. Find the ratio of their $18^{th}$ terms.
\item If the sum of first p terms of an A.P. is equal to the sum of the first q terms, then find the sum of the first (p + q) terms.
\item Sum of the first p, q and r terms of an A.P. are a, b and c, respectively. Prove that
 $\frac{a}{p}(q-r)+\frac{b}{q}(r-p)+\frac{c}{r}(p-q) = 0$
\item The ratio of the sums of m and n terms of an A.P. is $m^2 : n^2$ . Show that the ratio of 
$m^{th}$ and $n^{th}$ term is (2m - 1) : (2n - 1).
\item If the sum of n terms of an A.P. is $3n^2 + 5n$ and its $m^{th}$ term is 164, find the value
of m.
\item Insert five numbers between 8 and 26 such that the resulting sequence is an A.P.
\item If $\frac{a^n+b^n}{a^{n-1}+b^{n-1}}$is the A.M. between a and b, then find the value of n.
\item Between 1 and 31, m numbers have been inserted in such a way that the resulting sequence is an A. P. and the ratio of $7^{th}$ and $(m - 1)^{th}$ numbers is 5 : 9. Find the value of m.
\item A man starts repaying a loan as first instalment of Rs. 100. If he increases the
instalment by Rs 5 every month, what amount he will pay in the $30^{th}$ instalment?
\item The difference between any two consecutive interior angles of a polygon is 5$\degree$. If the smallest angle is 120$\degree$, find the number of the sides of the polygon. 
\item Find the $20^{th}$ and $n^{th}$ terms of the G.P.$\frac{5}{2}, \frac{5}{4}, \frac{5}{8},...$
\item Find the $12^{th}$ term of a G.P. whose $8^{th}$ term is 192 and the common ratio is 2.
\item The $5^{th}$, $8^{th}$ and $11^{th}$ terms of a G.P. are p, q and s, respectively. Show that 
$q^2$ = ps.
\item The $4^{th}$ term of a G.P. is square of its second term, and the first term is -3. Determine its $7^{th}$ term.
\item Which term of the following sequences:\\
(a) 2, 2, $\sqrt{2}$, 4,... is 128 ?\\
(b) $\sqrt{3}, 3, 3\sqrt{3},...is 729$\\
(c) $\frac{1}{3}, \frac{1}{9}, \frac{1}{27}...is \frac{1}{19683}$
\item For what values of x, the numbers $-\frac{2}{7}, x, -\frac{7}{2}$ are in G.P.?  Find the sum to indicated number of terms in each of the geometric progressions.
\item 0.15, 0.015, 0.0015,... 20 terms.
\item $\sqrt{7}, \sqrt{21}, \sqrt[3]{7}, ...n$ terms.
\item $1, -a, a^2, -a^2, a^3,.... n$ terms (if $a \neq -1$).
\item $x^3, x^5, x^7,.... n$ terms (if x $\neq \pm$ 1).
\item Evaluate $\sum _{k=1}^{11} (2 + 3^k).$
\item The sum of first three terms of a G.P. is $\frac{39}{10}$ and their product is 1. Find the 10
common ratio and the terms.
\item How many terms of G.P. 3, $3^2, 3^3$, ... are needed to give the sum 120?
\item The sum of first three terms of a G.P. is 16 and the sum of the next three terms is 128. Determine the first term, the common ratio and the sum to n terms of the G.P.
\item Given a G.P. with a = 729 and $7^{th}$ term 64, determine $S_7$.
\item Find a G.P. for which sum of the first two terms is -4 and the fifth term is 4 times the third term.
\item If the $4^{th}$, $10^{th}$ and $16^{th}$ terms of a G.P. are x, y and z, respectively. Prove that x, y, z are in G.P.
\item Find the sum to n terms of the sequence, 8, 88, 888, 8888... 
\item Find the sum of the products of the corresponding terms of the sequences 2, 4, 8, 16, 32, and 128, 32, 8, 2, $\frac{1}{2}$
\item Show that the products of the corresponding terms of the sequences a, ar, $ar^2,...ar^{n-1}$ and $A, AR, AR^2,... AR^{n -1}$ form a G.P, and find the common ratio.
\item Find four numbers forming a geometric progression in which the third term is greater than the first term by 9, and the second term is greater than the $4^{th}$ by 18.
\item If the $p^{th}, q^{th}$ and $r^{th}$ terms of a G.P. are a, b and c, respectively. Prove that 
$a^{q-r} b^{r-p} c^{p-q} = 1.$
\item If the first and the $n^{th}$ term of a G.P. are a and b, respectively, and if P is the product of n terms, prove that $P^2 = (ab)^n.$
\item Show that the ratio of the sum of first n terms of a G.P. to the sum of terms from $(n+1)^{th} to (2n)^{th}$term is $\frac{1}{r^n}$
\item If a, b, c and d are in G.P. show that $(a^2 + b^2 + c^2)(b^2 + c^2 + d^2) = (ab + bc + cd)^2.$
\item Insert two numbers between 3 and 81 so that the resulting sequence is G.P.
\item Find the value of n so that $\frac{a^{n + 1} + b^{n + 1}}{a^n + b^n}$ may be the geometric mean between a and b.
\item The sum of two numbers is 6 times their geometric mean, show that numbers are in the ratio $(3+2\sqrt{2}):(3-2\sqrt{2}).$
\item If A and G be A.M. and G.M., respectively between two positive numbers, prove that the numbers are $A\pm \sqrt{(A+G)(A-G)}.$
\item The number of bacteria in a certain culture doubles every hour. If there were 30 bacteria present in the culture originally, how many bacteria will be present at the end of $2^{nd}$ hour, 
$4^{th}$ hour and $n^{th}$ hour ?
\item What will Rs 500 amounts to in 10 years after its deposit in a bank which pays annual interest rate of 10\% compounded annually?
\item If A.M. and G.M. of roots of a quadratic equation are 8 and 5, respectively, then obtain the quadratic equation.\\
Find the sum to n terms of each of the series
\item $1 \times 2 + 2 \times 3 + 3 \times 4 + 4 \times 5 +...$
\item $1 \times 2 \times 3 + 2 \times 3 \times 4 + 3 \times 4 \times 5 + ...$
\item $3 \times 1^2 + 5 \times 2^2 + 7 \times 3^2 + ...$
\item $\frac{1}{1 \times 2}+\frac{1}{2\times3}+\frac{1}{3\times4}+....$
\item $5^2 + 6^2 + 7^2 + ... + 20^2$
\item $3 \times 8 + 6 \times 11 + 9 \times 14 + ...$
\item $1^2 + (1^2 + 2^2 ) + (1^2 + 2^2 + 3^2 ) + ...$\\
Find the sum to n terms of the series
\item n (n+1) (n+4).
\item $n^2 + 2^n$
\item $(2 n - 1)^2$
\item Show that the sum of $(m + n)^{th}$ and $(m - n)^{th}$ terms of an A.P. is equal to twice the $m^{th}$ term.
\item If the sum of three numbers in A.P., is 24 and their product is 440, find the numbers.
\item Let the sum of n, 2n, 3n terms of an A.P. be $S_1, S_2$ and $S_3$ , respectively, show that 
$S_3 = 3(S_2 - S_1)$
\item Find the sum of all numbers between 200 and 400 which are divisible by 7.
\item Find the sum of integers from 1 to 100 that are divisible by 2 or 5.
\item Find the sum of all two digit numbers which when divided by 4, yields 1 as remainder.
\item If f is a function satisfying f(x +y) = f(x) f(y) for all x, $y \in N$ such that \\
f(1) = 3 and $\sum_{x = 1}^n$ f(x) = 120 , find the value of n.
\item The sum of some terms of G.P. is 315 whose first term and the common ratio are 5 and 2, respectively. Find the last term and the number of terms.
\item  The first term of a G.P. is 1. The sum of the third term and fifth term is 90. Find the common ratio of G.P.
\item The sum of three numbers in G.P. is 56. If we subtract 1, 7, 21 from these numbers in that order, we obtain an arithmetic progression. Find the numbers.
\item A G.P. consists of an even number of terms. If the sum of all the terms is 5 times the sum of terms occupying odd places, then find its common ratio.
\item  The sum of the first four terms of an A.P. is 56. The sum of the last four terms is 112. If its first term is 11, then find the number of terms.
\item If $\frac{a+bx}{a-bx} = \frac{b+cx}{b-cx} = \frac{c+dx}{c-dx}(x \neq 0),$ then show that a, b, c and d are in G.P. 
\item Let S be the sum, P the product and R the sum of reciprocals of n terms in a G.P. Prove that $P^2R^n = S^n.$ 
\item The $p^{th}, q^{th}$ and $r^{th}$ terms of an A.P. are a, b, c, respectively. Show that 
(q - r )a + (r - p )b + (p - q )c = 0
\item If $a(\frac{1}{b}+\frac{1}{c}), b(\frac{1}{c}+\frac{1}{a}), c(\frac{1}{a}+\frac{1}{b})$ are in A.P., prove that a, b, c are in A.P.
\item If a, b, c, d are in G.P, prove that $(a^n + b^n), (b^n + c^n), (c^n + d^n)$ are in G.P.
\item If a and b are the roots of $x^2 - 3x + p = 0$ and c, d are roots of $x^2 - 12x + q = 0,$ where a, b, c, d form a G.P. Prove that (q + p) : (q - p) = 17:15.
\item The ratio of the A.M. and G.M. of two positive numbers a and b, is m : n. Show that 
a:b = $(m+\sqrt{m^2 - n^2}):(m - \sqrt{m^2 - n^2}).$
\item If a, b, c are in A.P.; b, c, d are in G.P. and $\frac{1}{c}, \frac{1}{d}, \frac{1}{e}$ are in A.P. prove that a, c, e are in G.P.
\item Find the sum of the following series up to n terms:\\
(i) 5 + 55 +555 + ...\\
(ii) .6 + .66 + .666+...
\item Find the $20^{th}$ term of the series $2 \times 4 + 4 \times 6 + 6 \times 8 + ... + n$ terms.
\item Find the sum of the first n terms of the series: 3 + 7 + 13 + 21 + 31 + ...
\item If $S_1, S_2, S_3$ are the sum of first n natural numbers, their squares and their cubes, respectively, show that $9 S_2^2 = S_3 (1 + 8S_1).$
\item Find the sum of the following series up to n terms:
$\frac{1^3}{1}+\frac{1^3+2^2}{1+3}+\frac{1^3+2^3+3^3}{1+3+5}+.....$
\item Show that $\frac{1 \times 2^2+ 2 \times 3^2+.....+n \times (n+1)^2}{1^2 \times 2 + 2^2 \times 3 +.....+n^2 \times (n+1)} = \frac{3n+5}{3n+1}.$
\item A farmer buys a used tractor for Rs 12000. He pays Rs 6000 cash and agrees to pay the balance in annual instalments of Rs 500 plus 12\% interest on the unpaid amount. How much will the tractor cost him?
\item Shamshad Ali buys a scooter for Rs 22000. He pays Rs 4000 cash and agrees to
pay the balance in annual instalment of Rs 1000 plus 10\% interest on the unpaid
amount. How much will the scooter cost him?
\item A person writes a letter to four of his friends. He asks each one of them to copy
the letter and mail to four different persons with instruction that they move the
chain similarly. Assuming that the chain is not broken and that it costs 50 paise to
mail one letter. Find the amount spent on the postage when $8^{th}$ set of letter is
mailed. 
\item A man deposited Rs 10000 in a bank at the rate of 5\% simple interest annually.Find the amount in $15^{th}$ year since he deposited the amount and also calculate the total amount after 20 years.
\item A manufacturer reckons that the value of a machine, which costs him Rs. 15625, will depreciate each year by 20\%. Find the estimated value at the end of 5 years. 
\item 150 workers were engaged to finish a job in a certain number of days. 4 workers dropped out on second day, 4 more workers dropped out on third day and so on.It took 8 more days to finish the work. Find the number of days in which the work was completed.
\end{enumerate}
%\end{document}
    
