%
\renewcommand{\theequation}{\theenumi}
\begin{enumerate}[label=\arabic*.,ref=\thesubsection.\theenumi]
\numberwithin{equation}{enumi}

%
\item
	Show that the median $AD$ in Fig. \ref{fig:tri_median_def} divides  $\triangle ABC$ into triangles $ADB$ and $ADC$ that have equal area.

\solution We have
%
\begin{align}
ar \brak{\triangle ADB} &= \frac{1}{2}\frac{a}{2}c \sin B =  \frac{1}{4}ac \sin B \\
ar \brak{\triangle ADC} &= \frac{1}{2}\frac{a}{2}b \sin C =  \frac{1}{4}ab \sin C 
\end{align}
%
Using the sine formula, $b \sin C = c \sin B$,
\begin{equation}
ar \brak{\triangle ADB} = ar \brak{\triangle ADC}
\end{equation}
\item
	$BE$ and $CF$ are the medians in Fig. \ref{ch2_median_ratio}.  Show that
	\begin{equation}
	ar \brak{\triangle BFC} = ar \brak{\triangle BEC}
	\end{equation} 
	\label{ch2_median_eq_tri}

\solution Since $BE$ and $CF$ are the medians, 

\begin{align}
ar \brak{\triangle BFC} &= \frac{1}{2} ar \brak{\triangle ABC} \\
ar \brak{\triangle BEC} &= \frac{1}{2} ar \brak{\triangle ABC} 
\end{align}
From the above, we infer that
%
\begin{equation}
ar \brak{\triangle BFC} = ar \brak{\triangle BEC}
\end{equation}


\begin{figure}[!ht]
	\begin{center}
		\resizebox{\columnwidth}{!}{\begin{tikzpicture}
[scale=2,>=stealth,point/.style={draw,circle,fill = black,inner sep=0.5pt},]

\node (E) at (1.5, 1.5)[point,label=above :$E$] {};
\node (F) at (-1.5, 1.5)[point,label=above :$F$] {};
\node (A) at (0, 3)[point,label=above :$A$]{};
\node (B) at (-3, 0)[point,label=below left:$B$]{};
\node (C) at (3, 0)[point,label=below right:$C$]{};
\node (a) at (0,0)[point,label=below :$a$] {};
\node (O) at (0,1)[point,label=above :$O$] {};

\draw (a)--(B);
\draw (B)--(A);
\draw (A)--(C);
\draw (C)--(a);
\draw (B)--(E);
\draw (C)--(F);
\node [above] at (-1.7,1.7) {$c$};
\node [above] at (1.7,1.7) {$b$};
\node [above] at (-2.5,.75) {$c/2$};
\node [above] at (-1,2.1) {$c/2$};
\node [above] at (2.5,.75) {$b/2$};
\node [above] at (1,2.1) {$b/2$};
%\node [above] at (1,1.3) {$p$};
%\node [above] at (-1,1.3) {$q$};

\tkzMarkAngle[size=.3](F,O,B);
\tkzMarkAngle[size=.3](C,O,E);
\draw (-0.5,1) node{$\theta$};
\draw (0.5,1) node{$\theta$};

\end{tikzpicture}}
	\end{center}
	\caption{$O$ is the Intersection of Two Medians}
	\label{ch2_median_ratio}	
\end{figure}
%
\end{enumerate}
