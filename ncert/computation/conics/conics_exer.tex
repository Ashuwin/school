\renewcommand{\theequation}{\theenumi}
\begin{enumerate}[label=\arabic*.,ref=\thesubsection.\theenumi]
\numberwithin{equation}{enumi}
\item Verify whether the following are zeroes of the polynomial, indicated against them. 
%\item p(x) = 3x + 1, x =
\begin{enumerate}

\item $ p(x) = x^2-1, x = 1, -1$
\item $ p(x) = \brak{x+1} \brak{x-2}, x = -1,2$
\item $ p(x) = x^2, x = 0$.
\item $ p(x) = 3x^2-1, x = -\frac{1}{\sqrt{3}}, \frac{2}{\sqrt{3}}$.
\end{enumerate}
\item Find the vaue of $k$, if $x-1$ is a factor of $p(x)$ in each of the following cases:
\begin{enumerate}
\item $p(x) = 2x^3+x^2-2x-1, g(x) = x+1$
\item $p(x) = x^3+3x^2+3x+1, g(x) = x+2$
\item $x^4-4x^2+x+6, g(x) = x-3$
\end{enumerate}
%
\item  Factorise : 
\begin{enumerate}
\item $12x^2 – 7x + 1 $
\item $6x^2+ 5x – 6$
\item $2x^2+ 7x + 3 $
\item $3x^2– x – 4$
\end{enumerate}
\item Find the zeroes of the following quadratic polynomials and verify the relationship between the zeroes and the coefficients.
\begin{enumerate}
\item $x^2 – 2x – 8$
\item  $4u^2 + 8u$
\item $4s^2 – 4s + 1$
\item $t^2 – 15$
\item $6x^2– 3 – 7x $
\item $3x^2 – x – 4$
\end{enumerate}
\item  Find a quadratic polynomial each with the given numbers as the sum and product of its zeroes respectively.
\begin{enumerate}
\item-1 , $\frac{1}{ 4}$
\item 1, 1
\item $0, \sqrt{5}$ 
\item 4, 1
 \item $\frac{1}{4}, \frac{1}{4}$
\item  $\sqrt{2}, \frac{1}{ 3}$
\end{enumerate}
\item Find the roots of the following quadratic equations:
\begin{enumerate}
\item $x^2 – 3x – 10=0$
\item $2x^2+x-6=0$
\item $\sqrt{2}x^2 +7x+5\sqrt{2}  = 0$
\item $2x^2– x +\frac{1}{8} = 0 $
\item $100x^2 – 20x +1 = 0$
\end{enumerate}
\item Find the roots of the following quadratic equations
\begin{enumerate}
\item 	$2x^2-7x+3 = 0$
\item 	2$x^2+x-4 = 0$
\item 	$4x^2+4\sqrt{3}x+3 = 0$
\item 	2$x^2+x+4 = 0$
\end{enumerate}
\item Find the nature of the roots of the following quadratic equations. If the real roots exist, find them:
\begin{enumerate}
\item 	$2x^2-3x+5 = 0$
\item 	$2x^2-6x+3 = 0$
\item 	$3x^2-4\sqrt{3}x+4 = 0$
\end{enumerate}
\item Solve each of the following equations
%
\begin{enumerate}
\item 	$x^2+3 = 0$
\item 	$2x^2+x+1 = 0$
\item 	$x^2+3x+9 = 0$
\item 	$-x^2+x-2 = 0$
\item 	$x^2+3x+5 = 0$
\item 	$x^2-3x+2 = 0$
\item 	$\sqrt{2}x^2+x+\sqrt{2} = 0$
\item 	$\sqrt{3}x^2-\sqrt{2}x+3\sqrt{3} = 0$
\item 	$x^2+x+\frac{1}{\sqrt{2}} = 0$
\item 	$x^2+\frac{x}{\sqrt{2}}+1 = 0$
\end{enumerate}
%
\item Solve each of the following equations
\begin{enumerate}
\item 	$3x^2-4x+\frac{20}{3} = 0$
\item 	$x^2-2x+\frac{3}{2} = 0$
\item 	$27x^2-10x+1 = 0$
\item 	$21x^2-28x+10 = 0$
\end{enumerate}
%
\item In each of the following exercises, find the coordinates of the focus, axis of the parabola, the equation of the directrix and the length of the latus rectum
\begin{enumerate}
\item $y^2 = 12x$
\item $x^2 = 6y$
\item $y^2 = -8x$
\item $x^2 = -16y$
\item $y^2 = 10x$
\item $x^2 = -9y$
\end{enumerate}
%
\item In each of the following exercises, find the equation of the parabola that satisfies the following conditions:
\begin{enumerate}
\item Focus \myvec{6\\0}, directrix $\myvec{1 & 0} = -6$.
\item Focus \myvec{0\\-3}, directrix $\myvec{0 & 1} = 3$.
\item Focus \myvec{3\\0}, vertex \myvec{0 & 0}.
\item Focus \myvec{-2\\0}, vertex \myvec{0 & 0}.
\item vertex \myvec{0 & 0} passing through \myvec{2\\2} and axis is along the x-axis
\item vertex \myvec{0 & 0} passing through \myvec{5\\2} and symmetric with respect to the y-axis.
\end{enumerate}
%
\item In each of the exercises, find the coordinates of the foci, the vertices, the length of major axis, the minor axis, the eccentricity and the length of the latus rectum of the ellipse.
%
\begin{enumerate}
\item 
$
\vec{x}^T\myvec{\frac{1}{36} & 0 \\ 0 & \frac{1}{16}}\vec{x} = 1
$
\item 
$
\vec{x}^T\myvec{\frac{1}{4} & 0 \\ 0 & \frac{1}{25}}\vec{x} = 1
$
\item 
$
\vec{x}^T\myvec{\frac{1}{16} & 0 \\ 0 & \frac{1}{9}}\vec{x} = 1
$
\item 
$
\vec{x}^T\myvec{\frac{1}{25} & 0 \\ 0 & \frac{1}{100}}\vec{x} = 1
$
\item 
$
\vec{x}^T\myvec{\frac{1}{49} & 0 \\ 0 & \frac{1}{36}}\vec{x} = 1
$
\item 
$
\vec{x}^T\myvec{\frac{1}{100} & 0 \\ 0 & \frac{1}{16}}\vec{x} = 1
$
%
\item 
$
\vec{x}^T\myvec{36 & 0 \\ 0 & 4}\vec{x} = 144
$
%
\item 
$
\vec{x}^T\myvec{16 & 0 \\ 0 & 1}\vec{x} = 16
$
%
\item 
$
\vec{x}^T\myvec{4 & 0 \\ 0 & 9}\vec{x} = 36
$
%
\end{enumerate}
%
\item In each of the following, find the equation for the ellipse that satisfies the given conditions:
%
\begin{enumerate}
\item Vertices $\myvec{\pm 5\\ 0}$, foci $\myvec{\pm 4\\ 0}$ \item  Vertices $\myvec{0\\ \pm 13}$, foci $\myvec{0\\ \pm 5}$ \item  Vertices $\myvec{\pm 6\\ 0}$, foci $\myvec{\pm 4\\ 0}$ \item  Ends of major axis $\myvec{\pm 3\\ 0}$, ends of minor axis $\myvec{0\\ \pm 2}$
\item  Ends of major axis $\myvec{0\\ \pm 5 }$, ends of minor axis $\myvec{\pm 1\\ 0}$ \item  Length of major axis 26, foci $\myvec{\pm 5\\ 0}$ \item  Length of minor axis 16, foci $\myvec{0\\ \pm 6}$. \item  Foci $\myvec{\pm 3\\ 0}$, a = 4 \item  b = 3, c = 4, centre at the origin; foci on the x axis. \item  Centre at $\myvec{0\\0}$, major axis on the y-axis and passes through the points $\myvec{3\\ 2}$ and $\myvec{1\\6}$.
\item  Major axis on the x-axis and passes through the points $\myvec{4\\3}$ and $\myvec{6\\2}$.
\end{enumerate}
%
%
\item In each of the exercises, find the coordinates of the foci, the vertices, the length of major axis, the minor axis, the eccentricity and the length of the latus rectum of the ellipse.
%
\begin{enumerate}
\item 
$
\vec{x}^T\myvec{\frac{1}{16} & 0 \\ 0 & -\frac{1}{9}}\vec{x} = 1
$
\item 
$
\vec{x}^T\myvec{\frac{1}{9} & 0 \\ 0 & -\frac{1}{27}}\vec{x} = 1
$
%
\item 
$
\vec{x}^T\myvec{9 & 0 \\ 0 & -4}\vec{x} = 36
$
%
\item 
$
\vec{x}^T\myvec{16 & 0 \\ 0 & -9}\vec{x} = 576
$
%
\item 
$
\vec{x}^T\myvec{5 & 0 \\ 0 & -9}\vec{x} = 36
$
\item 
$
\vec{x}^T\myvec{49 & 0 \\ 0 & -16}\vec{x} = 784
$
%
%
\end{enumerate}
\item In each of the following, find the equation for the ellipse that satisfies the given conditions:
%
\begin{enumerate}
\item Vertices $\myvec{\pm 2\\ 0}$, foci $\myvec{\pm 3\\ 0}$ 
\item  Vertices $\myvec{0\\ \pm 5}$, foci $\myvec{0\\ \pm 8}$ 
\item  Vertices $\myvec{ 0\\ \pm 3}$, foci $\myvec{0 \\\pm 5}$ 
\item  Transverse axis length 8, foci $\myvec{\pm 5\\ 0}$.
\item  Conjugate axis length 24, foci $\myvec{0 \\ \pm 13}$.
\item  Latus rectum  length 8, foci $\myvec{ \pm 3\sqrt{5} \\ 0}$.
\item  Latus rectum  lenght 12, foci $\myvec{ \pm 4 \\ 0}$.
\item  Ends of major axis $\myvec{0\\ \pm 5 }$, ends of minor axis $\myvec{\pm 1\\ 0}$ 
\item  Vertices $\myvec{  \pm 7 \\ 0}$, $e = \frac{4}{3}$
\item  Foci $\myvec{ 0\\  \pm \sqrt{10}}$, passing through $\myvec{2\\3}$.

\end{enumerate}
%
\item Find the slope of the tangent to the curve $y = \frac{x-1}{x-2}, x\ne 2$ at $x = 10$.
\item Find a point on the curve $y = (x – 2)^2$ at which the tangent is parallel to the chord joining the points \myvec{2\\ 0} and \myvec{4\\ 4}.
\item Find the equation of all lines having slope – 1 that are tangents to the curve $\frac{1}
{x -1}, x \ne 1$
\item Find the equation of all lines having slope 2 which are tangents to the curve $\frac{1}
{x - 3} , x \ne 3$.
%
\item Find points on the curve 
$
\vec{x}^T\myvec{\frac{1}{9} & 0 \\ 0 & \frac{1}{16}}\vec{x} = 1
$
%
at which tangents are
\begin{enumerate}
\item  parallel to x-axis
\item  parallel to y-axis.
\end{enumerate}
\item Find the equations of the tangent and normal to the given curves at the indicated points:
$
y = x^2
$
at \myvec{0\\0}.
\item Find the equation of the tangent line to the curve $y = x^2-2x+7$
\begin{enumerate}
%
\item  parallel to the line $\myvec{2 & -1}\vec{x}= -9$ 
\item  perpendicular to the line $\myvec{-15 & 5}\vec{x} = 13$. 
\end{enumerate}
\item Find the equation of the tangent to the curve $y = \sqrt{3x - 2}$ which is parallel to the line $\myvec{4 & 2}\vec{x}+ 5 =0$ .
\item Find the point at which the line $\myvec{-1 & 1}\vec{x} =  1$ is a tangent to the curve $y^2 = 4x$.
%
\item The line $\myvec{-m & 1}\vec{x} = 1$ is a tangent to the curve $y^2 = 4x$.  Find the value of $m$.
\item  Find the normal at the point \myvec{1\\1} on the curve $2y + x^2 = 3$ 
\item  Find the normal to the curve $x^2=4y$ passing through $\myvec{1\\2}$.
%
\item Find the area of the region bounded by the curve $y^2= x$ and the lines $x = 1, x = 4$ and the x-axis in the first quadrant.
\item  Find the area of the region bounded by $y^2=9x, x=2, x=4$ and the x-axis in the  first quadrant.
%
\item Find the area of the region bounded by $x^2 = 4y, y = 2, y = 4$ and the y-axis in the first quadrant.
\item Find the area of the region bounded by the ellipse 
$
\vec{x}^T\myvec{\frac{1}{16} & 0 \\ 0 & \frac{1}{9}}\vec{x} = 1
$

\item  Find the area of the region bounded by the ellipse 
$
\vec{x}^T\myvec{\frac{1}{4} & 0 \\ 0 & \frac{1}{9}}\vec{x} = 1
$
\item The area between $x=y^2$ and $x=4$ is divided into two equal parts by the line $x=a$, find the value of $a$.
\item  Find the area of the region bounded by the parabola $y = x^2$ and $y = \abs{x}$.
\item  Find the area bounded by the curve $x^2 = 4y$ and the line $\myvec{1 & -1}\vec{x} = -2$.
\item  Find the area of the region bounded by the curve $y^2 = 4x$ and the line $x = 3$.
%
\item Find the area of the region bounded by the curve $y^2 = x$, y-axis and the line $y = 3$.
%
\item Find the area of the region bounded by the two parabolas $y = x^2, y^2=x$.
\item Find the area lying above x-axis and included between the circle $\vec{x}^T\vec{x} -8\myvec{1 & 0}= 0$  and inside of the parabola $y^2 = 4x$.
%
\item AOBA is the part of the ellipse 
$
\vec{x}^T\myvec{9 & 0 \\ 0 & 1}\vec{x} = 36
$
in the first quadrant such that $OA = 2$ and $OB = 6$. Find the area between the arc $AB$ and the chord $AB$.
\item Find the area lying between the curves $y^2 = 4x$ and $y = 2x$.
\item  Find the area of the region bounded by the curves $y = x^2+2, y = x, x = 0$ and $ x = 3.$
%
\item Find the area under $y = x^2, x = 1, x = 2$ and x-axis.
\item Find the area between  $y = x^2$ and $y = x$.
\item Find the area of the region lying in the first quadrant and bounded by $y = 4x^2, x = 0, y = 1$ and $y = 4$.
\item Find the area enclosed by the parabola $4y = 3x^2$ and the line $\myvec{-3 & 2}\vec{x} = 12$.
%
\item Find the area of the smaller region bounded by the ellipse
$
\vec{x}^T\myvec{\frac{1}{9} & 0 \\ 0 & \frac{1}{4}}\vec{x} = 1
$
and the line 
$
\myvec{\frac{1}{a} & \frac{1}{b}}\vec{x} = 1
$
\item Find the area of the region enclosed by the parabola $x^2=y$, the line $\myvec{-1 & 1}\vec{x} = 2$ and the x-axis.
%
\item Find the area bounded by the curves
\begin{align}
\cbrak{\brak{x,y} : y > x^2, y = \abs{x}}
\end{align}
%
\item Find the area of the region
\begin{align}
\cbrak{\brak{x,y} : y^2 \le 4x, 4\vec{x}^T\vec{x} = 9}
\end{align}
%
\item Find the area of the circle $\vec{x}^T\vec{x} = 16$ exterior to the parabola $y^2 = 6$.
\end{enumerate}
