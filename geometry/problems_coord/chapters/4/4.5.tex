\begin{enumerate}[1.]
\item Find the locus of a point which moves so that the sum of the squares of its distances from the sides of an equilateral triangle
is constant.
\item Find the locus of a point which moves so that the sum of the squares of its distances from $n$ fixed points is constant.
\item Find the locus of a point at which two given circles
subtend equal angles.
\item A circle passes through the four points $\brak{a,0}$, $\brak{b,0}$, $\brak{0,c}$, $\brak{0,d}$.  By what relation 
are $a$, $b$, $c$, $d$ connected?  Find the equation of the 
circle and show that the tangent at the point $\brak{a,c+d}$ is
\begin{align*}
\brak{a-b}\brak{x-a}+\brak{c+d}\brak{y-c-d}=0
\end{align*}
\item Write down the equations of the tangents to the circles
\begin{align*}
x^2 + y^2 =2ax, x^2+ y^2 =2by
\end{align*}
at their points of intersection and verify that they cut at right angles.
\item Find the equation of the tangent to the circle $x^2 + y^2 =a^2$ at the point $\brak{a\cos\theta,a\sin\theta}$ and show that the length of the
tangent intercepted by the lines $x^2-y^2 = 0$ is $\pm 2a\sec\theta$.
\item $A$ and $B$ are two fixed points $\brak{c,0}$, $\brak{-c,0}$, and $P$ moves so that $PA=k.PB$.  Find the locus of $P$ and prove that it is 
cut orthogonally by any circle through $A$ and $B$.
\item Show that the common chord of the circles
\begin{align*}
x^2 +y^2 -6x -4y+9 =0,
\\
x^2+ y^2 -8x -6y+23 =0,
\end{align*}
is a diameter of the latter circle and find the angle at which the circles cut.
\item Prove analytically that the tangents to a circle at the ends of a chord are equally inclined to the chord.
\item Prove that for different values of $a$ the equation
\begin{align*}
x^2 +y^2 -2ax\text{cosec}\alpha +a^2\cot^2\alpha =0
\end{align*}
represents a family of circles touching the lines $y=\pm x\tan\alpha$.

Prove also that the locus of the poles of the line $lx+my=0$ with regard to the circles is the line
\begin{align*}
mx\sin^2\alpha+ly\cos^2\alpha = 0
\end{align*}
\item Find the coordinates of the middle point of the chord $lx+my=1$ of the circle
\begin{align*}
x^2 +y^2 +2gx+2f y+c =0,
\end{align*}
\item Prove that the points of intersection of the line $lx+my=1$ and the circle
\begin{align*}
x^2 +y^2 +2gx+2f y+c =0,
\end{align*}
subtend a right angle at the origin if
\begin{align*}
c\brak{l^2+m^2}+2gl+2fm+2=0
\end{align*}
\item Prove that the equation of the circle having for diameter the portion of the line $x\cos\alpha+y\sin\alpha = p$ intercepted by the circle $x^2 + y^2 =a^2$ is
\begin{align*}
x^2 +y^2 -2p\brak{x\cos\alpha+ y\sin\alpha-p} -a^2 =0,
\end{align*}
\item Prove that if a chord of the circle $x^2 + y^2 =a^2$ subtends a right angle at a fixed point $\brak{x_1,y_1}$, the locus of the middle point
of the chord is
\begin{align*}
2\brak{x^2+ y^2-xx_1-yy_1}+ x_1^2+ y_1^2 - a^2 =0,
\end{align*}
\item Prove that the equation of any tangent to the circle
\begin{align*}
\brak{x-a}^2+\brak{ y-b}^2 =r^2,
\end{align*}
may be written in the form
\begin{align*}
\brak{x-a}\cos\theta + \brak{y-b}\sin\theta =r.
\end{align*}
Deduce that the equation of the tangents from $\brak{x_1,y_1}$ to the circle is
\begin{multline*}
r^2\cbrak{\brak{x-x_1}^2+\brak{ y-y_1}^2} 
\\
 =\cbrak{\brak{x-a}\brak{y_1-b}-\brak{y-b}\brak{x_1-a} }^2
\end{multline*}
\item Prove that the distances of two points from the centre of a circle are proportional to the distance of each point from the polar of the
other.
\item Prove that the tangents to the circles of a coaxal system drawn from a limiting point are bisected by
the radical axis.
\item Show that a common tangent to the two circles is bisected by their radical axis and subtends a right angle at either limiting point.
\item Prove that if a point moves so that the difference of the squares of the tangents from it to two given circles is constant its locus
 is a straight line parallel to the radical axis of the circles.
 \item Prove that the polars of a fixed point with regard to a family of coaxal circles all pass through another fixed point.
 \item The circles
 \begin{align*}
 x^2 +y^2-2a x\sec\alpha- a^2 =0,
 \\
 x^2+ y^2 -2ay\text{cosec}\alpha -a^2 =0,
 \end{align*}
 where $\alpha$ is a given angle, both cut orthogonally every member of a coaxal family of circles.  Find the radical axis and the limiting
 points of the family.
 \item Prove that, if two points $P$, $Q$ are conjugate with regard to a circle, the circle on $PQ$ as diameter cuts the first circle orthogonally. 
 \item Prove that if $P$, $Q$ are conjugate points with regard to a circle, the circles
 with $P$, $Q$ as centres which cut the given circle orthogonally are orthogonal to one another.
 \item Prove that, if $PQ$ is a diameter of a circle, then $P$, $Q$ are conjugate points with regard to
 any circle which cuts the given circle orthogonally.
 \item Prove that if $P$, $Q$ are conjugate points with regard to a circle, the square on $PQ$ is equal to the
 sum of the squares on the tangents from $P$, $Q$ to the circle.
 \item The equation $ x^2 +y^2-2g\brak{x-1}=5$, where $g$ is a variable parameter, represents a family of coaxial circles.  
 Show that the radius of the smallest circle of the family is 2.
 \item Prove that, if perpendiculars are drawn from a fixed point $P$ to the polars of $P$ with regard to a
 family of coaxial circles, then the locus of the feet of these perpendiculars is a circle whose centre
 lies on the radical axis of the family.
 \item Prove that, if the points in which the line $lx+my+n$ meets the circle, $ x^2 +y^2+2gx+2fy+c=0$, and those in which the line $l_1x+m_1y+n=0$ 
 meets $x^2 +y^2+2g_1x+2f_1y+c_1=0$ lie on a circle,
 then 
 \begin{multline*}
 2\brak{g-g_1}\brak{mn_1-m_1n}+2\brak{f-f_1}
 \\
 \brak{nl_1-n_1l}+\brak{c-c_1}\brak{lm_1-l_1m}=0
 \end{multline*}
 \item Show that, if a diameter of a circle is the portion of the line $lx+my=1$ intercepted by the lines $ax^2+2hxy+by^2=0$, then the 
 equation of the circle is
 \begin{multline*}
\brak{am^2-2hlm+bl^2} \brak{x^2+ y^2}
\\
+ 2x\brak{hm-bl} + 2y\brak{hl-am} + a+b =0.
 \end{multline*}
\item Prove that, as $k$ varies, the equation
 \begin{align*}
 x^2 + y^2 +2ax +2by + c + 2k\brak{ax-by+1} =0,
 \end{align*}
 represents a system of coaxial circles.  Also prove that the orthogonal system is given by
 \begin{align*}
 x^2 +y^2 +\frac{c+2}{2a} x + \frac{c-2}{2b} y +h\brak{\frac{x}{2a}+\frac{y}{2b} + 1}=0,
 \end{align*}
 where $h$ is a variable parameter.
\end{enumerate}