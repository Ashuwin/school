\begin{enumerate}[1.]
\item An ellipse has its foci at the points $\brak{2,0}$, $\brak{-2,0}$
and passes through the point $\brak{2,3}$.  Find its equation.
\item Prove that the line $x+2y=8$ touches the ellipse $3x^2+4y^2=48$,
and find the coordinates of the point of contact.
\item Find the equation of the ellipse which toches the line $2x+3y=9$
and has the points $\brak{-2,0}$, $\brak{2,0}$ as its foci.  Find also
the coordinates of the point of contact of the line and the ellipse.
\item Show that the distance of the point $\brak{a\cos \theta,b\sin \theta}$
from the focus $\brak{ae,0}$ of the ellipse $\frac{x^2}{a^2}+\frac{y^2}{b^2}=1$
is $a\brak{1-e\cos\theta}$.  
\item Prove that if the normals at $P \brak{6,4}$ and $Q \brak{-8,3}$ on
the  ellipse $\frac{x^2}{100}+\frac{y^2}{25} = 1$ meet at $G$, then the 
diameter through $G$ is perpendicular to $PA$.  
\item Find the equation of an ellipse of eccentricity $\frac{1}{2}$ which has
a focus at $\brak{3,0}$ and $x=1$ for corresponding directrix.
\item Find the equations of the tangents to the ellipse $\frac{x^2}{9}+\frac{y^2}{4}=1$
which are parallel to the line $x=2y$.  
\item Find the equation of the ellipse of eccentricity $\frac{1}{2}$ which has
its foci at the points $\brak{-1,0}$, $\brak{1,0}$.  Find also the length
of the latus rectum and verify that the tangent at either end of the latus rectum
cuts the major axis on the directrix.
\item Find the equation of an ellipse which has the point $\brak{2,3}$ as an
end of a latus rectum and its axes along the coordinate axes.  At what point
does the line $x-2y=8$ touch the ellipse?
\item find the equation of an ellipse of eccentricity 0.8 which has its
centre at the origin and the lines $x= \pm 25$ as directrices.  Verify that
the ellipse touches the line $9x+20y=300$.
\item The axes of an ellipse are the coordinate axes, its directrices pass through
the points $\brak{\pm,5\frac{1}{3},0}$ and it touches the line $3x+4y=16$.  
Find its equation.
\item The major and minor axes of an ellipse lie along the lines $3x-4y+6=0$
and $4x+3y-17=0$ and the lengths of the semi-axes are 5 and 4.  Find the 
eccentricity and the coordinates of the centre and foci.
\item Find the equation of an ellipse which has its axes along the coordinate
axes and the line $3x-2y=5$ as the normal at the point $\brak{15,20}$.  
\item Prove that, if the tangent at an end of the minor axis of an
ellipse cuts the latus rectum produced in $D$, and $C$ is the centre,
then a perpendicular to $CD$ through $D$ cuts the major axis on the directrix.
\item Prove that the tangent at the ends of the latera recta of the ellipse $\frac{x^2}{a^2}+\frac{y^2}{b^2}=1$
form a quadrilateal of area $\frac{2a^2}{e}$, where $e$ is the eccentricity.  
\item Prove that, if a series of ellipses have the same major axis, the tangents
at the ends of the latera recta pass through one or other of the two fixed points
on the minor axis.
\item Find the equations of the tangents to the ellipse $4x^2+9y^2 = 180$ at the
points $P \brak{6,2}$ and $P_1 \brak{-6,-2}$.  Find also the equations of the
tangents that are parallel to the line $PP_1$, and the coordinates of their points
of contact.
\item Prove that the line $x+3y=9$ touches the ellipse $\frac{x^2}{9} + \frac{y^2}{8} =1$,
and find the coordinates of the point of contact.  Find the coordinates of the foci of the ellipse,
and verify that the product of the distances of the foci from the above tangent is equal to the
square on the minor axis.
\item An ellipse has its foci at the point $\brak{\pm 3,0}$ and passes through
the point $P \brak{2,2\sqrt{6}}$.  Prove that its eccentricity is $\frac{1}{2}$ and
that the normal at $P$ passes through the point $\brak{\frac{1}{2},0}$.
\item An ellipse has a focus at the point $\brak{3,0}$, the $y$-axis is the corresponding
directrix and the point $\brak{6,4}$ lies on the curve.  Prove that the axes are in the
ratio $6:\sqrt{11}$.  
\item Prove that, if the line $lx+my+n=0$ is a normal to the ellipse $\frac{x^2}{a^2}+\frac{y^2}{b^2}=1$,
then $\frac{a^2}{l^2}+\frac{b^2}{m^2} = \frac{\brak{a^2-b^2}}{n^2}$.
\item Find the coordinates of the points on the ellipse $8x^2+25y^2 = 200$ at which the normals
make angles of $60\degree$ with the major axis.
\item Find the values of $c$ for which the line $5x-2y=c$ is normal to the ellipse $x^2=5y^2=9$.
\item Find the coodrinates of the four points on the ellipse
\begin{equation}
9x^2+16y^2 = 1
\end{equation}
the tangents at which are equally inclined to the coordinate axes; and prove that the normals
at these points form a square of area $\frac{49}{1800}$.
\item Find the equation of the normal at the point $\brak{2,3}$ on the ellipse
$3x^2+4y^2 = 48$ and the coordinates of the point in which the normal again cuts the curve.
Show that the middle point of this normal chord is at a distance $\frac{\sqrt{73}}{19}$ from
the centre of the ellipse.
\item Find the equation of an ellipse of eccentricity $\frac{1}{3}$ which 
touches the line $2x+3y=5$ and has its axes along the coordinate axes.  find the coordinates
of the point of contact of the ellipse with the given line.
\item Find the equations of the two ellipses which have their axes along the coordinate axes,
pass through the point $\brak{2,1}$ and touch the line $6x+12y= 25$.  
\item Prove that, if an ordinate $NP$ to an ellipse is produced to meet the tangent at the end
of the latus rectum in $Q$, then $QN = SP$, where $S$ is the corresponding focus.
\item The tangent is drawn at the point $P \brak{2,1}$ on the ellipse $4x^2+9y^2 = 25$ whose
centre is $O$, and the diameter $DOD_1$ of the ellipse is parallel to the tangent at $P$.  
Find the coordinates of $D$ and $D_1$ and prove that the tangents at these points are parallel
to the radius $OP$.  
\item The tangent at $P$ to an ellipse meets a directrix in $T$ and $S$ is the corresponding
focus.  Prove that $PST$ is a right angle.
\item The foci  of an ellipse are the points $\brak{0,0}$ and $\brak{8,6}$ and the
eccentricity is $\frac{4}{5}$.  Find the coordinates of the centre and teh equations and
lengths of the major and minor axes.  Find also the equations of the directrices.
\item Show that, if the normal at a point $\P brak{x_1,y_1}$ on an ellipse of focus $S$
and eccentricity $e$ meets the major axis in $G$ and $GL$ is the perpendicular to $SP$,
then $GL=ey_1$ and $PL = $ the semi-latus rectum.
\item $P$ denotes any point on an ellipse of which the major axis is $AA_1$.  Prove that, if $AP$,
$A_1P$  cut the minor axis in $M$, $M_1$, then the tangent at $P$ bisects $MM_1$.
\end{enumerate}
