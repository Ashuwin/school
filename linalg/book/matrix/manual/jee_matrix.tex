\documentclass[journal,12pt,twocolumn]{IEEEtran}
\usepackage{setspace}
\usepackage{gensymb}
\usepackage{caption}
%\usepackage{multirow}
%\usepackage{multicolumn}
%\usepackage{subcaption}
%\doublespacing
\singlespacing
\usepackage{csvsimple}
\usepackage{amsmath}
\usepackage{multicol}
%\usepackage{enumerate}
\usepackage{amssymb}
%\usepackage{graphicx}
\usepackage{newfloat}
%\usepackage{syntax}
\usepackage{listings}
\usepackage{iithtlc}
\usepackage{color}
\usepackage{tikz}
\usetikzlibrary{shapes,arrows}



%\usepackage{graphicx}
%\usepackage{amssymb}
%\usepackage{relsize}
%\usepackage[cmex10]{amsmath}
%\usepackage{mathtools}
%\usepackage{amsthm}
%\interdisplaylinepenalty=2500
%\savesymbol{iint}
%\usepackage{txfonts}
%\restoresymbol{TXF}{iint}
%\usepackage{wasysym}
\usepackage{amsthm}
\usepackage{mathrsfs}
\usepackage{txfonts}
\usepackage{stfloats}
\usepackage{cite}
\usepackage{cases}
\usepackage{mathtools}
\usepackage{caption}
\usepackage{enumerate}	
\usepackage{enumitem}
\usepackage{amsmath}
%\usepackage{xtab}
\usepackage{longtable}
\usepackage{multirow}
%\usepackage{algorithm}
%\usepackage{algpseudocode}
\usepackage{enumitem}
\usepackage{mathtools}
\usepackage{hyperref}
%\usepackage[framemethod=tikz]{mdframed}
\usepackage{listings}
    %\usepackage[latin1]{inputenc}                                 %%
    \usepackage{color}                                            %%
    \usepackage{array}                                            %%
    \usepackage{longtable}                                        %%
    \usepackage{calc}                                             %%
    \usepackage{multirow}                                         %%
    \usepackage{hhline}                                           %%
    \usepackage{ifthen}                                           %%
  %optionally (for landscape tables embedded in another document): %%
    \usepackage{lscape}     


\usepackage{url}
\def\UrlBreaks{\do\/\do-}


%\usepackage{stmaryrd}


%\usepackage{wasysym}
%\newcounter{MYtempeqncnt}
\DeclareMathOperator*{\Res}{Res}
%\renewcommand{\baselinestretch}{2}
\renewcommand\thesection{\arabic{section}}
\renewcommand\thesubsection{\thesection.\arabic{subsection}}
\renewcommand\thesubsubsection{\thesubsection.\arabic{subsubsection}}

\renewcommand\thesectiondis{\arabic{section}}
\renewcommand\thesubsectiondis{\thesectiondis.\arabic{subsection}}
\renewcommand\thesubsubsectiondis{\thesubsectiondis.\arabic{subsubsection}}

% correct bad hyphenation here
\hyphenation{op-tical net-works semi-conduc-tor}

%\lstset{
%language=C,
%frame=single, 
%breaklines=true
%}

%\lstset{
	%%basicstyle=\small\ttfamily\bfseries,
	%%numberstyle=\small\ttfamily,
	%language=Octave,
	%backgroundcolor=\color{white},
	%%frame=single,
	%%keywordstyle=\bfseries,
	%%breaklines=true,
	%%showstringspaces=false,
	%%xleftmargin=-10mm,
	%%aboveskip=-1mm,
	%%belowskip=0mm
%}

%\surroundwithmdframed[width=\columnwidth]{lstlisting}
\def\inputGnumericTable{}                                 %%
\lstset{
%language=C,
frame=single, 
breaklines=true,
columns=fullflexible
}
 

\begin{document}
%
\tikzstyle{block} = [rectangle, draw,
    text width=3em, text centered, minimum height=3em]
\tikzstyle{sum} = [draw, circle, node distance=3cm]
\tikzstyle{input} = [coordinate]
\tikzstyle{output} = [coordinate]
\tikzstyle{pinstyle} = [pin edge={to-,thin,black}]

\theoremstyle{definition}
\newtheorem{theorem}{Theorem}[section]
\newtheorem{problem}{Problem}
\newtheorem{proposition}{Proposition}[section]
\newtheorem{lemma}{Lemma}[section]
\newtheorem{corollary}[theorem]{Corollary}
\newtheorem{example}{Example}[section]
\newtheorem{definition}{Definition}[section]
%\newtheorem{algorithm}{Algorithm}[section]
%\newtheorem{cor}{Corollary}
\newcommand{\BEQA}{\begin{eqnarray}}
\newcommand{\EEQA}{\end{eqnarray}}
\newcommand{\define}{\stackrel{\triangle}{=}}

\bibliographystyle{IEEEtran}
%\bibliographystyle{ieeetr}

\providecommand{\nCr}[2]{\,^{#1}C_{#2}} % nCr
\providecommand{\nPr}[2]{\,^{#1}P_{#2}} % nPr
\providecommand{\mbf}{\mathbf}
\providecommand{\pr}[1]{\ensuremath{\Pr\left(#1\right)}}
\providecommand{\qfunc}[1]{\ensuremath{Q\left(#1\right)}}
\providecommand{\sbrak}[1]{\ensuremath{{}\left[#1\right]}}
\providecommand{\lsbrak}[1]{\ensuremath{{}\left[#1\right.}}
\providecommand{\rsbrak}[1]{\ensuremath{{}\left.#1\right]}}
\providecommand{\brak}[1]{\ensuremath{\left(#1\right)}}
\providecommand{\lbrak}[1]{\ensuremath{\left(#1\right.}}
\providecommand{\rbrak}[1]{\ensuremath{\left.#1\right)}}
\providecommand{\cbrak}[1]{\ensuremath{\left\{#1\right\}}}
\providecommand{\lcbrak}[1]{\ensuremath{\left\{#1\right.}}
\providecommand{\rcbrak}[1]{\ensuremath{\left.#1\right\}}}
\theoremstyle{remark}
\newtheorem{rem}{Remark}
\newcommand{\sgn}{\mathop{\mathrm{sgn}}}
\providecommand{\abs}[1]{\left\vert#1\right\vert}
\providecommand{\res}[1]{\Res\displaylimits_{#1}} 
\providecommand{\norm}[1]{\left\Vert#1\right\Vert}
\providecommand{\mtx}[1]{\mathbf{#1}}
\providecommand{\mean}[1]{E\left[ #1 \right]}
\providecommand{\fourier}{\overset{\mathcal{F}}{ \rightleftharpoons}}
%\providecommand{\hilbert}{\overset{\mathcal{H}}{ \rightleftharpoons}}
\providecommand{\system}{\overset{\mathcal{H}}{ \longleftrightarrow}}
	%\newcommand{\solution}[2]{\textbf{Solution:}{#1}}
\newcommand{\solution}{\noindent \textbf{Solution: }}
\newcommand{\myvec}[1]{\ensuremath{\begin{pmatrix}#1\end{pmatrix}}}
\providecommand{\dec}[2]{\ensuremath{\overset{#1}{\underset{#2}{\gtrless}}}}
\DeclarePairedDelimiter{\ceil}{\lceil}{\rceil}
%\numberwithin{equation}{subsection}
%\numberwithin{equation}{section}
%\numberwithin{problem}{subsection}
%\numberwithin{definition}{subsection}
\makeatletter
\@addtoreset{figure}{section}
\makeatother

\let\StandardTheFigure\thefigure
%\renewcommand{\thefigure}{\theproblem.\arabic{figure}}
\renewcommand{\thefigure}{\thesection}


%\numberwithin{figure}{subsection}

%\numberwithin{equation}{subsection}
%\numberwithin{equation}{section}
%\numberwithin{equation}{problem}
%\numberwithin{problem}{subsection}
\numberwithin{problem}{section}
%%\numberwithin{definition}{subsection}
%\makeatletter
%\@addtoreset{figure}{problem}
%\makeatother
\makeatletter
\@addtoreset{table}{section}
\makeatother

\let\StandardTheFigure\thefigure
\let\StandardTheTable\thetable
\let\vec\mathbf
%%\renewcommand{\thefigure}{\theproblem.\arabic{figure}}
%\renewcommand{\thefigure}{\theproblem}

%%\numberwithin{figure}{section}

%%\numberwithin{figure}{subsection}



\def\putbox#1#2#3{\makebox[0in][l]{\makebox[#1][l]{}\raisebox{\baselineskip}[0in][0in]{\raisebox{#2}[0in][0in]{#3}}}}
     \def\rightbox#1{\makebox[0in][r]{#1}}
     \def\centbox#1{\makebox[0in]{#1}}
     \def\topbox#1{\raisebox{-\baselineskip}[0in][0in]{#1}}
     \def\midbox#1{\raisebox{-0.5\baselineskip}[0in][0in]{#1}}

\vspace{3cm}

\title{ 
	\logo{
JEE Problems in Matrices
	}
}

%\author{ G V V Sharma$^{*}$% <-this % stops a space
%	\thanks{*The author is with the Department
%		of Electrical Engineering, Indian Institute of Technology, Hyderabad
%		502285 India e-mail:  gadepall@iith.ac.in. All content in this manual is released under GNU GPL.  Free and open source.}
	
%}	

\maketitle

%\tableofcontents

\bigskip

\renewcommand{\thefigure}{\theenumi}
\renewcommand{\thetable}{\theenumi}


\begin{abstract}
	A  collection of problems from JEE papers related to matrices are available in this document.  Verify your soluions using  Python.
\end{abstract}
\section{Quadratic Form}
\begin{enumerate}[label=\thesection.\arabic*
,ref=\thesection.\theenumi]
\item Show that 
\begin{align}
\min_{a,b,c} \abs{a + b\omega + c\omega^2}^2
\end{align}
%
where $\omega^3 = 1, \omega \ne 1$ and $a,b,c$ are distinct nonzero integers
can be expressed as
\begin{align}
\label{eq:13_opt}
\min_{\vec{x}} \frac{1}{2}\vec{x}^T\vec{A}\vec{x}
\end{align}
%
where 
\begin{align}
\vec{x}&= \myvec{a \\ b \\c}, 
\vec{A}= 2\vec{P}^T\vec{P},
\\
\vec{P}&= \myvec{1 & \cos \theta & -\cos \theta \\ 0 & \sin \theta & \sin \theta},
\theta = \frac{\pi}{3} 
\end{align}
\item Show that 
\begin{align}
\label{eq:13_A}
\vec{A}  = \myvec{
2 & 1 & -1 \\ 
1 & 2 & 1
\\
-1 & 1 & 2
}
\end{align}
%
\solution 
\begin{align}
\vec{A}&= \myvec{1 &  0 \\ \cos \theta & \sin \theta  \\ -\cos \theta  & \sin \theta} 
\myvec{1 & \cos \theta & -\cos \theta \\ 0 & \sin \theta & \sin \theta} 
\nonumber \\
&=\myvec{
1 & \cos \theta & -\cos \theta \\ 
\cos \theta & 1 & -\cos 2\theta
\\
-\cos \theta & -\cos 2\theta & 1
},
\end{align}
resulting in \eqref{eq:13_A}.
\begin{align}
\because \cos 2 \theta = -\cos \theta = -\frac{1}{2}
\end{align}
\item Show that the characteristic equation of $\vec{A}$ is 
\begin{align}
f(\lambda) = \lambda^3 -6\lambda^2 + 9\lambda
\end{align}
\item Show that the eigenvalues of $\vec{A}$ are 0 and 3.
\item Verify that $tr\brak{\vec{A}}$ is the sum of its eigenvalues.
\item Verify that $\det\brak{\vec{A}}$ is the product of  its eigenvalues.
\item Show that $\vec{A}$ is positive definite.
\item Show that $\vec{x}^{T}\vec{A}\vec{x}$ is convex.
\item Show that the unconstrained $\vec{x}$ that minimizes $\vec{x}^{T}\vec{A}\vec{x}$ is given by the line
\begin{align}
\vec{x} = k \myvec{1 \\ -1 \\ 1}
\end{align}
\item Find $\vec{y}$ such that 
\begin{align}
\vec{A}\vec{y} = \lambda\vec{y}
\end{align}
where $\lambda$ is an eigenvalue of $\vec{A}$.
\item Show that 
\begin{align}
\vec{A} = \vec{P}^{-1} \vec{D}\vec{P}
\end{align}
%
where $\vec{D}$  is a diagonal matrix comprising of the eigenvalues of $\vec{A}$
and the columns of $\vec{P}$ are the corresponding eigenvectors.
\item Find $\vec{U}$ such that 
\begin{align}
\vec{A} = \vec{U}^{T} \vec{D}\vec{U}, \vec{U}^{T} \vec{U} = \vec{I}
\end{align}
\item Show that 
\begin{align}
\vec{x}^T \vec{A}\vec{x}  = 3 \vec{v}^{T} \vec{v},
\end{align}
where
\begin{align}
\vec{v} = \vec{U}\vec{x}
\end{align}
%
\item Show that when the entries of $\vec{x}$ are unequal and integers, the solution of \eqref{eq:13_opt} can be expressed as
\begin{align}
\vec{x} = \myvec{1 \\ -1 \\ 0} + c\myvec{1 \\ -1 \\ 1}
\end{align}

%\nonumber \\
%\\
%\solution The desired $\vec{x}$ is obtained by 
%\begin{align}
%\frac{d}{d\vec{x}}\vec{x}^{T}\vec{A}\vec{x} &= \vec{0}
%\nonumber \\
%\implies \vec{A}\vec{x} &= \vec{0}
%\end{align}
\end{enumerate}
\section{Matrices: Cayley-Hamilton Theorem}
\begin{enumerate}[label=\thesection.\arabic*
,ref=\thesection.\theenumi]

\item Let
\begin{align}
\label{eq:mat_def}
\vec{M} = \myvec{\sin^4\theta & -1-\sin^2\theta \\ 1+ \cos^2\theta & \cos^4\theta} = \alpha\vec{I} + \beta \vec{M}^{-1}
\end{align}
where $\alpha, \beta$ are real functions of $\theta$ and $\vec{I}$ is the identity matrix. Find the characteristic equation of $\vec{M}$.
\\
\solution \eqref{eq:mat_def} can be expressed as
\begin{align}
\vec{M}^2 -\alpha\vec{M} - \beta \vec{I} = 0
\end{align}
%
which yields   the characteristic equation of $\vec{M}$ as
\begin{align}
\lambda^2 -\alpha\lambda - \beta  = 0
\end{align}
\item Find $\alpha$ and $\beta$.
\\
\solution
Since the sum of the eigenvalues is equal to the trace and the determinant is the product of eigenvalues,
\begin{align}
 \alpha &= \sin^4\theta + \cos^4\theta 
\\
  \beta & = -\sin^4\theta \cos^4\theta + \brak{1+\sin^2\theta}\brak{ 1+ \cos^2\theta}
\end{align}
\item If 
\begin{align}
\alpha^{*} &= \min_{\theta}\alpha\brak{\theta}
\\
\beta^{*} &= \min_{\theta}\beta\brak{\theta}, 
\end{align}
find $\alpha^{*} + \beta^{*}$.
\\
\solution 
\begin{align}
\because  \alpha &= \sin^4\theta + \cos^4\theta = 1 - \frac{\sin^2 2\theta}{2},
\\
\alpha^{*} &= \frac{1}{2}, 
\end{align}
Similarly,
\begin{align}
 - \beta & = \sin^4\theta \cos^4\theta + \brak{1+\sin^2\theta}\brak{ 1+ \cos^2\theta}
\\
&=2 + \frac{\sin^2 2\theta}{4} + \frac{\sin^4 2\theta}{16}
\\
&= \brak{\frac{\sin^2 2\theta}{4}+\frac{1}{2}}^2+ \frac{7}{4} 
\end{align}
Thus,
\begin{align}
 \beta^{*} &= -\frac{37}{16}
\\
\implies \alpha^{*}+\beta^{*} &= -\frac{29}{16}
\end{align}
\end{enumerate}
\section{Matrices: Adjugate}
Let 
\begin{align}
\vec{M} = \myvec{0 & 1 & a \\ 1 & 2 & 3 \\ 3 & b & 1}, \quad 
\text{ adj}\brak{\vec{M}} = \myvec{-1 & 1 & -1 \\ 8 & -6 & 2 \\ -5 & 3 & -1}
\end{align}
\begin{enumerate}[label=\thesection.\arabic*
,ref=\thesection.\theenumi]
\item Show that $a+b = 3$
\\
\solution 
\begin{align}
\label{eq:adj_Minv}
\because \vec{M} \text{ adj}\brak{\vec{M}} &= \det \brak{\vec{M}}\vec{I},
%\\
%\myvec{0 & 1 & a }
% \myvec{-1 \\ 8 \\ -5 } &= \det \brak{\vec{M}}
\\
\myvec{0 & 1 & a } 
 \myvec{1 \\ -6 \\ 3} &= 0
\\
\myvec{3 & b & 1 } 
 \myvec{-1 \\ 8 \\ -5} &= 0
\end{align}
%
resulting in 
\begin{align}
a = 2, b = 1
\end{align}
Hence, $a+b = 3$.
\item Verify if 
\begin{align}
\brak{\text{ adj}\brak{\vec{M}}}^{-1} + \text{ adj}\brak{\vec{M}^{-1}} = -\vec{M}
\end{align}
\\
\solution From \eqref{eq:adj_Minv}
\begin{align}
\brak{\text{ adj}\brak{\vec{M}}}^{-1} = \frac{\vec{M}}{\det\brak{\vec{M}}}
\end{align}
%
and 
\begin{align}
\brak{\text{ adj}\brak{\vec{M}^{-1}}} &= \frac{\vec{M}^{-1}}{\det\brak{\vec{M}^{-1}}}
\\
&= {\vec{M}^{-1}}{\det\brak{\vec{M}}}
\end{align}
%
Thus, 
\begin{multline}
\brak{\text{ adj}\brak{\vec{M}^{-1}}} + \text{ adj}\brak{\vec{M}^{-1}}
\\
= {\vec{M}^{-1}}{\det\brak{\vec{M}}}+\frac{\vec{M}}{\det\brak{\vec{M}}}
\\
= \text{ adj}\brak{\vec{M}}+\frac{\vec{M}}{\det\brak{\vec{M}}}
\end{multline}
From \eqref{eq:adj_Minv}
\begin{align}
\myvec{0 & 1 & a } 
 \myvec{-1 \\ 8 \\ -5} &= \det\brak{\vec{M}}
\\
\implies \det\brak{\vec{M}} = 8-5a &= -2
\end{align}
If
\begin{align*}
\brak{\text{ adj}\brak{\vec{M}^{-1}}} + \text{ adj}\brak{\vec{M}^{-1}}
&= -\vec{M},
\\
\text{ adj}\brak{\vec{M}}-\frac{\vec{M}}{2} &=  -\vec{M}
\\
\implies \vec{M} &= - \text{ adj}\brak{\vec{M}}
\end{align*}
%
which is incorrect.
\item Verify if
\begin{align}
\det\brak{\text{ adj}\brak{\vec{M}^2}}  = 81
\end{align}
\solution 
\begin{align}
\text{ adj}\brak{\vec{M}^2}  &= \vec{M}^{-2}\det\brak{\vec{M}}^{2}
\\
&= 4{\vec{M}^{-2}}
\\
\implies \det\brak{\text{ adj}\brak{\vec{M}^2}}  &= 4^3\det\brak{\vec{M}}^{-2}
\\
&= 16 \ne 81
\end{align}
%
\item If 
\begin{align}
\vec{M}\myvec{\alpha \\ \beta \\ \gamma}  = \myvec{1 \\ 2 \\ 3}, 
\end{align}
show that 
\begin{align}
\alpha - \beta + \gamma = 3
\end{align}
\solution 
\begin{align}
\vec{M}\myvec{\alpha \\ \beta \\ \gamma}  &= \myvec{1 \\ 2 \\ 3}, 
\\
\implies \text{ adj}\brak{\vec{M}}\vec{M}\myvec{\alpha \\ \beta \\ \gamma}  &= \text{ adj}\brak{\vec{M}}\myvec{1 \\ 2 \\ 3}, 
\end{align}
%
which can be expressed as
\begin{align}
 \det\brak{\vec{M}}\myvec{\alpha \\ \beta \\ \gamma}  &= \text{ adj}\brak{\vec{M}}\myvec{1 \\ 2 \\ 3}, 
\\
\text{or, } \myvec{\alpha \\ \beta \\ \gamma} &= -\frac{1}{2}\text{ adj}\brak{\vec{M}}\myvec{1 \\ 2 \\ 3}, 
\end{align}
Thus, 
\begin{align}
\alpha - \beta + \gamma &= \myvec{1 & -1& 1} \myvec{\alpha \\ \beta \\ \gamma} 
\\
&=-\frac{1}{2}\myvec{1 & -1& 1}\text{ adj}\brak{\vec{M}}\myvec{1 \\ 2 \\ 3}
\\
&=\myvec{7 & -5& 2}\myvec{1 \\ 2 \\ 3}
=3
\end{align}
\end{enumerate}
\section{Linear Algebra: Binary Matrices}
 Let $S$ be the set of all $3 \times 3$ matrices whose entries are from $\cbrak{0,1}$ and
%Find $\abs{S}$.
%\\
%\solution The number of elements in a $3\times 3$ matrix is 9. Each element can have 2 possible values.  Hence, the total number of such matrices is
%\begin{align}
%\abs{S}= 2^9
%\end{align}
%\item Let
%\begin{align}
%E_0 = \cbrak{\vec{A}\in S: \det(\vec{A}) \ne 0}
%\end{align}
%Find $\abs{E_0}$.
%\\
%\solution $E_0$ is the set of all permutations of the identity matrix.  Hence,
%\begin{align}
%\abs{E_0}= 3! = 6
%\end{align}
%\item Let
\begin{align}
E_1 = \cbrak{\vec{A}\in S: \det(\vec{A}) = 0}
\end{align}
%Find $\abs{E_1}$.
%\solution 
%\begin{align}
%\abs{E_1}= \abs{S}-\abs{E_0} = 506
%\end{align}
%\item Let
and 
\begin{align}
E_2 = \cbrak{\vec{A}\in S: \text{ sum of entries of $A$ is 7}}
\end{align}
\begin{enumerate}[label=\thesection.\arabic*
,ref=\thesection.\theenumi]
%Find $\abs{E_2}$.
%\solution 
%\item 
\item Find $\abs{E_2}$.
\\
\solution 
\begin{align}
\label{eq:2019_15_e2}
\abs{E_2}= \frac{9!}{7!2!} = 72
\end{align}

\item Find $\abs{\brak{E_1|E_2}}$.
\\
\solution $E_2$ is the set of matrices with rows $\vec{v}_1,\vec{v}_2,\vec{v}_3$ and the following combinations in Table \ref{table:2019_15}. $\vec{e}_i, i = 1,2,3$ are the standard basis vectors. The equation 
\begin{align}
\vec{v}_1 =\lambda_2\vec{v}_2+\lambda_3\vec{v}_3
\end{align}
has a solution only for the first combination in Table \ref{table:2019_15}. Thus, $\det(A) = 0$ only for this combination.  Thus
\begin{align}
\label{eq:2019_15_e1e2}
\abs{\brak{E_1|E_2}} = 3\times 3 = 9
\end{align}
\begin{table}[!h]
\centering
%\resizebox {0.5\columnwidth} {!} {
%%%%%%%%%%%%%%%%%%%%%%%%%%%%%%%%%%%%%%%%%%%%%%%%%%%%%%%%%%%%%%%%%%%%%%%
%%                                                                  %%
%%  This is the header of a LaTeX2e file exported from Gnumeric.    %%
%%                                                                  %%
%%  This file can be compiled as it stands or included in another   %%
%%  LaTeX document. The table is based on the longtable package so  %%
%%  the longtable options (headers, footers...) can be set in the   %%
%%  preamble section below (see PRAMBLE).                           %%
%%                                                                  %%
%%  To include the file in another, the following two lines must be %%
%%  in the including file:                                          %%
%%        \def\inputGnumericTable{}                                 %%
%%  at the beginning of the file and:                               %%
%%        \input{name-of-this-file.tex}                             %%
%%  where the table is to be placed. Note also that the including   %%
%%  file must use the following packages for the table to be        %%
%%  rendered correctly:                                             %%
%%    \usepackage[latin1]{inputenc}                                 %%
%%    \usepackage{color}                                            %%
%%    \usepackage{array}                                            %%
%%    \usepackage{longtable}                                        %%
%%    \usepackage{calc}                                             %%
%%    \usepackage{multirow}                                         %%
%%    \usepackage{hhline}                                           %%
%%    \usepackage{ifthen}                                           %%
%%  optionally (for landscape tables embedded in another document): %%
%%    \usepackage{lscape}                                           %%
%%                                                                  %%
%%%%%%%%%%%%%%%%%%%%%%%%%%%%%%%%%%%%%%%%%%%%%%%%%%%%%%%%%%%%%%%%%%%%%%



%%  This section checks if we are begin input into another file or  %%
%%  the file will be compiled alone. First use a macro taken from   %%
%%  the TeXbook ex 7.7 (suggestion of Han-Wen Nienhuys).            %%
\def\ifundefined#1{\expandafter\ifx\csname#1\endcsname\relax}


%%  Check for the \def token for inputed files. If it is not        %%
%%  defined, the file will be processed as a standalone and the     %%
%%  preamble will be used.                                          %%
\ifundefined{inputGnumericTable}

%%  We must be able to close or not the document at the end.        %%
	\def\gnumericTableEnd{\end{document}}


%%%%%%%%%%%%%%%%%%%%%%%%%%%%%%%%%%%%%%%%%%%%%%%%%%%%%%%%%%%%%%%%%%%%%%
%%                                                                  %%
%%  This is the PREAMBLE. Change these values to get the right      %%
%%  paper size and other niceties.                                  %%
%%                                                                  %%
%%%%%%%%%%%%%%%%%%%%%%%%%%%%%%%%%%%%%%%%%%%%%%%%%%%%%%%%%%%%%%%%%%%%%%

	\documentclass[12pt%
			  %,landscape%
                    ]{report}
       \usepackage[latin1]{inputenc}
       \usepackage{fullpage}
       \usepackage{color}
       \usepackage{array}
       \usepackage{longtable}
       \usepackage{calc}
       \usepackage{multirow}
       \usepackage{hhline}
       \usepackage{ifthen}

	\begin{document}


%%  End of the preamble for the standalone. The next section is for %%
%%  documents which are included into other LaTeX2e files.          %%
\else

%%  We are not a stand alone document. For a regular table, we will %%
%%  have no preamble and only define the closing to mean nothing.   %%
    \def\gnumericTableEnd{}

%%  If we want landscape mode in an embedded document, comment out  %%
%%  the line above and uncomment the two below. The table will      %%
%%  begin on a new page and run in landscape mode.                  %%
%       \def\gnumericTableEnd{\end{landscape}}
%       \begin{landscape}


%%  End of the else clause for this file being \input.              %%
\fi

%%%%%%%%%%%%%%%%%%%%%%%%%%%%%%%%%%%%%%%%%%%%%%%%%%%%%%%%%%%%%%%%%%%%%%
%%                                                                  %%
%%  The rest is the gnumeric table, except for the closing          %%
%%  statement. Changes below will alter the table's appearance.     %%
%%                                                                  %%
%%%%%%%%%%%%%%%%%%%%%%%%%%%%%%%%%%%%%%%%%%%%%%%%%%%%%%%%%%%%%%%%%%%%%%

\providecommand{\gnumericmathit}[1]{#1} 
%%  Uncomment the next line if you would like your numbers to be in %%
%%  italics if they are italizised in the gnumeric table.           %%
%\renewcommand{\gnumericmathit}[1]{\mathit{#1}}
\providecommand{\gnumericPB}[1]%
{\let\gnumericTemp=\\#1\let\\=\gnumericTemp\hspace{0pt}}
 \ifundefined{gnumericTableWidthDefined}
        \newlength{\gnumericTableWidth}
        \newlength{\gnumericTableWidthComplete}
        \newlength{\gnumericMultiRowLength}
        \global\def\gnumericTableWidthDefined{}
 \fi
%% The following setting protects this code from babel shorthands.  %%
 \ifthenelse{\isundefined{\languageshorthands}}{}{\languageshorthands{english}}
%%  The default table format retains the relative column widths of  %%
%%  gnumeric. They can easily be changed to c, r or l. In that case %%
%%  you may want to comment out the next line and uncomment the one %%
%%  thereafter                                                      %%
\providecommand\gnumbox{\makebox[0pt]}
%%\providecommand\gnumbox[1][]{\makebox}

%% to adjust positions in multirow situations                       %%
\setlength{\bigstrutjot}{\jot}
\setlength{\extrarowheight}{\doublerulesep}

%%  The \setlongtables command keeps column widths the same across  %%
%%  pages. Simply comment out next line for varying column widths.  %%
\setlongtables

\setlength\gnumericTableWidth{%
	133pt+%
	53pt+%
	57pt+%
0pt}
\def\gumericNumCols{3}
\setlength\gnumericTableWidthComplete{\gnumericTableWidth+%
         \tabcolsep*\gumericNumCols*2+\arrayrulewidth*\gumericNumCols}
\ifthenelse{\lengthtest{\gnumericTableWidthComplete > \linewidth}}%
         {\def\gnumericScale{\ratio{\linewidth-%
                        \tabcolsep*\gumericNumCols*2-%
                        \arrayrulewidth*\gumericNumCols}%
{\gnumericTableWidth}}}%
{\def\gnumericScale{1}}

%%%%%%%%%%%%%%%%%%%%%%%%%%%%%%%%%%%%%%%%%%%%%%%%%%%%%%%%%%%%%%%%%%%%%%
%%                                                                  %%
%% The following are the widths of the various columns. We are      %%
%% defining them here because then they are easier to change.       %%
%% Depending on the cell formats we may use them more than once.    %%
%%                                                                  %%
%%%%%%%%%%%%%%%%%%%%%%%%%%%%%%%%%%%%%%%%%%%%%%%%%%%%%%%%%%%%%%%%%%%%%%

\ifthenelse{\isundefined{\gnumericColA}}{\newlength{\gnumericColA}}{}\settowidth{\gnumericColA}{\begin{tabular}{@{}p{100pt*\gnumericScale}@{}}x\end{tabular}}
\ifthenelse{\isundefined{\gnumericColB}}{\newlength{\gnumericColB}}{}\settowidth{\gnumericColB}{\begin{tabular}{@{}p{50pt*\gnumericScale}@{}}x\end{tabular}}
\ifthenelse{\isundefined{\gnumericColC}}{\newlength{\gnumericColC}}{}\settowidth{\gnumericColC}{\begin{tabular}{@{}p{70pt*\gnumericScale}@{}}x\end{tabular}}

%\begin{longtable}[c]{%
\begin{table}[!h]


\begin{tabular}[c]{
	b{\gnumericColA}%
	b{\gnumericColB}%
	b{\gnumericColC}%
	}

%%%%%%%%%%%%%%%%%%%%%%%%%%%%%%%%%%%%%%%%%%%%%%%%%%%%%%%%%%%%%%%%%%%%%%
%%  The longtable options. (Caption, headers... see Goosens, p.124) %%
%	\caption{The Table Caption.}             \\	%
% \hline	% Across the top of the table.
%%  The rest of these options are table rows which are placed on    %%
%%  the first, last or every page. Use \multicolumn if you want.    %%

%%  Header for the first page.                                      %%
%	\multicolumn{3}{c}{The First Header} \\ \hline 
%	\multicolumn{1}{c}{colTag}	%Column 1
%	&\multicolumn{1}{c}{colTag}	%Column 2
%	&\multicolumn{1}{c}{colTag}	\\ \hline %Last column
%	\endfirsthead

%%  The running header definition.                                  %%
%	\hline
%	\multicolumn{3}{l}{\ldots\small\slshape continued} \\ \hline
%	\multicolumn{1}{c}{colTag}	%Column 1
%	&\multicolumn{1}{c}{colTag}	%Column 2
%	&\multicolumn{1}{c}{colTag}	\\ \hline %Last column
%	\endhead

%%  The running footer definition.                                  %%
%	\hline
%	\multicolumn{3}{r}{\small\slshape continued\ldots} \\
%	\endfoot

%%  The ending footer definition.                                   %%
%	\multicolumn{3}{c}{That's all folks} \\ \hline 
%	\endlastfoot
%%%%%%%%%%%%%%%%%%%%%%%%%%%%%%%%%%%%%%%%%%%%%%%%%%%%%%%%%%%%%%%%%%%%%%

\hhline{|-|-|-}
	 \multicolumn{1}{|p{\gnumericColA}|}%
	{\gnumericPB{\centering}\textbf{Component}}
	&\multicolumn{1}{p{\gnumericColB}|}%
	{\gnumericPB{\raggedright}\textbf{Value}}
	&\multicolumn{1}{p{\gnumericColC}|}%
	{\gnumericPB{\centering}\textbf{Quantity}}
\\
\hhline{|---|}
	 \multicolumn{1}{|p{\gnumericColA}|}%
	{\gnumericPB{\centering}Breadboard}
	&\multicolumn{1}{p{\gnumericColB}|}%
	{\gnumericPB{\raggedright} }
	&\multicolumn{1}{p{\gnumericColC}|}%
	{\gnumericPB{\centering}1}
\\
\hhline{|---|}
	 \multicolumn{1}{|p{\gnumericColA}|}%
	{\gnumericPB{\centering}Resistor}
	&\multicolumn{1}{p{\gnumericColB}|}%
	{\gnumericPB{\raggedright} $\ge 220 \Omega$}
	&\multicolumn{1}{p{\gnumericColC}|}%
	{\gnumericPB{\centering}1}
\\
\hhline{|---|}
	 \multicolumn{1}{|p{\gnumericColA}|}%
	{\gnumericPB{\centering}Pi}
	&\multicolumn{1}{p{\gnumericColB}|}%
	{Model B, Rev 3}
	&\multicolumn{1}{p{\gnumericColC}|}%
	{\gnumericPB{\centering}1}
\\
\hhline{|---|}
	 \multicolumn{1}{|p{\gnumericColA}|}%
	{\gnumericPB{\centering}Seven Segment Display}
	&\multicolumn{1}{p{\gnumericColB}|}%
	{Common Anode}
	&\multicolumn{1}{p{\gnumericColC}|}%
	{\gnumericPB{\centering}1}
\\
\hhline{|---|}
	 \multicolumn{1}{|p{\gnumericColA}|}%
	{\gnumericPB{\centering}\gnumbox{Jumper Wires}}
	&\multicolumn{1}{p{\gnumericColB}|}%
	{Female-Male}
	&\multicolumn{1}{p{\gnumericColC}|}%
	{\gnumericPB{\centering}\gnumbox{20}}
\\
\hhline{|-|-|-|}
%\end{longtable}
\end{tabular}
\caption{}
\label{table:components}
\end{table}
\ifthenelse{\isundefined{\languageshorthands}}{}{\languageshorthands{\languagename}}
\gnumericTableEnd

%%%%%%%%%%%%%%%%%%%%%%%%%%%%%%%%%%%%%%%%%%%%%%%%%%%%%%%%%%%%%%%%%%%%%%
%%                                                                  %%
%%  This is the header of a LaTeX2e file exported from Gnumeric.    %%
%%                                                                  %%
%%  This file can be compiled as it stands or included in another   %%
%%  LaTeX document. The table is based on the longtable package so  %%
%%  the longtable options (headers, footers...) can be set in the   %%
%%  preamble section below (see PRAMBLE).                           %%
%%                                                                  %%
%%  To include the file in another, the following two lines must be %%
%%  in the including file:                                          %%
%%        \def\inputGnumericTable{}                                 %%
%%  at the beginning of the file and:                               %%
%%        \input{name-of-this-file.tex}                             %%
%%  where the table is to be placed. Note also that the including   %%
%%  file must use the following packages for the table to be        %%
%%  rendered correctly:                                             %%
%%    \usepackage[latin1]{inputenc}                                 %%
%%    \usepackage{color}                                            %%
%%    \usepackage{array}                                            %%
%%    \usepackage{longtable}                                        %%
%%    \usepackage{calc}                                             %%
%%    \usepackage{multirow}                                         %%
%%    \usepackage{hhline}                                           %%
%%    \usepackage{ifthen}                                           %%
%%  optionally (for landscape tables embedded in another document): %%
%%    \usepackage{lscape}                                           %%
%%                                                                  %%
%%%%%%%%%%%%%%%%%%%%%%%%%%%%%%%%%%%%%%%%%%%%%%%%%%%%%%%%%%%%%%%%%%%%%%



%%  This section checks if we are begin input into another file or  %%
%%  the file will be compiled alone. First use a macro taken from   %%
%%  the TeXbook ex 7.7 (suggestion of Han-Wen Nienhuys).            %%
\def\ifundefined#1{\expandafter\ifx\csname#1\endcsname\relax}


%%  Check for the \def token for inputed files. If it is not        %%
%%  defined, the file will be processed as a standalone and the     %%
%%  preamble will be used.                                          %%
\ifundefined{inputGnumericTable}

%%  We must be able to close or not the document at the end.        %%
	\def\gnumericTableEnd{\end{document}}


%%%%%%%%%%%%%%%%%%%%%%%%%%%%%%%%%%%%%%%%%%%%%%%%%%%%%%%%%%%%%%%%%%%%%%
%%                                                                  %%
%%  This is the PREAMBLE. Change these values to get the right      %%
%%  paper size and other niceties.                                  %%
%%                                                                  %%
%%%%%%%%%%%%%%%%%%%%%%%%%%%%%%%%%%%%%%%%%%%%%%%%%%%%%%%%%%%%%%%%%%%%%%

	\documentclass[12pt%
			  %,landscape%
                    ]{report}
       \usepackage[latin1]{inputenc}
       \usepackage{fullpage}
       \usepackage{color}
       \usepackage{array}
       \usepackage{longtable}
       \usepackage{calc}
       \usepackage{multirow}
       \usepackage{hhline}
       \usepackage{ifthen}

	\begin{document}


%%  End of the preamble for the standalone. The next section is for %%
%%  documents which are included into other LaTeX2e files.          %%
\else

%%  We are not a stand alone document. For a regular table, we will %%
%%  have no preamble and only define the closing to mean nothing.   %%
    \def\gnumericTableEnd{}

%%  If we want landscape mode in an embedded document, comment out  %%
%%  the line above and uncomment the two below. The table will      %%
%%  begin on a new page and run in landscape mode.                  %%
%       \def\gnumericTableEnd{\end{landscape}}
%       \begin{landscape}


%%  End of the else clause for this file being \input.              %%
\fi

%%%%%%%%%%%%%%%%%%%%%%%%%%%%%%%%%%%%%%%%%%%%%%%%%%%%%%%%%%%%%%%%%%%%%%
%%                                                                  %%
%%  The rest is the gnumeric table, except for the closing          %%
%%  statement. Changes below will alter the table's appearance.     %%
%%                                                                  %%
%%%%%%%%%%%%%%%%%%%%%%%%%%%%%%%%%%%%%%%%%%%%%%%%%%%%%%%%%%%%%%%%%%%%%%

\providecommand{\gnumericmathit}[1]{#1} 
%%  Uncomment the next line if you would like your numbers to be in %%
%%  italics if they are italizised in the gnumeric table.           %%
%\renewcommand{\gnumericmathit}[1]{\mathit{#1}}
\providecommand{\gnumericPB}[1]%
{\let\gnumericTemp=\\#1\let\\=\gnumericTemp\hspace{0pt}}
 \ifundefined{gnumericTableWidthDefined}
        \newlength{\gnumericTableWidth}
        \newlength{\gnumericTableWidthComplete}
        \newlength{\gnumericMultiRowLength}
        \global\def\gnumericTableWidthDefined{}
 \fi
%% The following setting protects this code from babel shorthands.  %%
 \ifthenelse{\isundefined{\languageshorthands}}{}{\languageshorthands{english}}
%%  The default table format retains the relative column widths of  %%
%%  gnumeric. They can easily be changed to c, r or l. In that case %%
%%  you may want to comment out the next line and uncomment the one %%
%%  thereafter                                                      %%
\providecommand\gnumbox{\makebox[0pt]}
%%\providecommand\gnumbox[1][]{\makebox}

%% to adjust positions in multirow situations                       %%
\setlength{\bigstrutjot}{\jot}
\setlength{\extrarowheight}{\doublerulesep}

%%  The \setlongtables command keeps column widths the same across  %%
%%  pages. Simply comment out next line for varying column widths.  %%
\setlongtables

\setlength\gnumericTableWidth{%
	53pt+%
	53pt+%
	53pt+%
0pt}
\def\gumericNumCols{3}
\setlength\gnumericTableWidthComplete{\gnumericTableWidth+%
         \tabcolsep*\gumericNumCols*2+\arrayrulewidth*\gumericNumCols}
\ifthenelse{\lengthtest{\gnumericTableWidthComplete > \linewidth}}%
         {\def\gnumericScale{\ratio{\linewidth-%
                        \tabcolsep*\gumericNumCols*2-%
                        \arrayrulewidth*\gumericNumCols}%
{\gnumericTableWidth}}}%
{\def\gnumericScale{1}}

%%%%%%%%%%%%%%%%%%%%%%%%%%%%%%%%%%%%%%%%%%%%%%%%%%%%%%%%%%%%%%%%%%%%%%
%%                                                                  %%
%% The following are the widths of the various columns. We are      %%
%% defining them here because then they are easier to change.       %%
%% Depending on the cell formats we may use them more than once.    %%
%%                                                                  %%
%%%%%%%%%%%%%%%%%%%%%%%%%%%%%%%%%%%%%%%%%%%%%%%%%%%%%%%%%%%%%%%%%%%%%%

\ifthenelse{\isundefined{\gnumericColA}}{\newlength{\gnumericColA}}{}\settowidth{\gnumericColA}{\begin{tabular}{@{}p{53pt*\gnumericScale}@{}}x\end{tabular}}
\ifthenelse{\isundefined{\gnumericColB}}{\newlength{\gnumericColB}}{}\settowidth{\gnumericColB}{\begin{tabular}{@{}p{53pt*\gnumericScale}@{}}x\end{tabular}}
\ifthenelse{\isundefined{\gnumericColC}}{\newlength{\gnumericColC}}{}\settowidth{\gnumericColC}{\begin{tabular}{@{}p{53pt*\gnumericScale}@{}}x\end{tabular}}

\begin{tabular}[c]{%
	b{\gnumericColA}%
	b{\gnumericColB}%
	b{\gnumericColC}%
	}

%%%%%%%%%%%%%%%%%%%%%%%%%%%%%%%%%%%%%%%%%%%%%%%%%%%%%%%%%%%%%%%%%%%%%%
%%  The longtable options. (Caption, headers... see Goosens, p.124) %%
%	\caption{The Table Caption.}             \\	%
% \hline	% Across the top of the table.
%%  The rest of these options are table rows which are placed on    %%
%%  the first, last or every page. Use \multicolumn if you want.    %%

%%  Header for the first page.                                      %%
%	\multicolumn{3}{c}{The First Header} \\ \hline 
%	\multicolumn{1}{c}{colTag}	%Column 1
%	&\multicolumn{1}{c}{colTag}	%Column 2
%	&\multicolumn{1}{c}{colTag}	\\ \hline %Last column
%	\endfirsthead

%%  The running header definition.                                  %%
%	\hline
%	\multicolumn{3}{l}{\ldots\small\slshape continued} \\ \hline
%	\multicolumn{1}{c}{colTag}	%Column 1
%	&\multicolumn{1}{c}{colTag}	%Column 2
%	&\multicolumn{1}{c}{colTag}	\\ \hline %Last column
%	\endhead

%%  The running footer definition.                                  %%
%	\hline
%	\multicolumn{3}{r}{\small\slshape continued\ldots} \\
%	\endfoot

%%  The ending footer definition.                                   %%
%	\multicolumn{3}{c}{That's all folks} \\ \hline 
%	\endlastfoot
%%%%%%%%%%%%%%%%%%%%%%%%%%%%%%%%%%%%%%%%%%%%%%%%%%%%%%%%%%%%%%%%%%%%%%

\hhline{|-|-|-}
	 \multicolumn{1}{|p{\gnumericColA}|}%
	{\gnumericPB{\centering}\gnumbox{$\vec{v}_1$}}
	&\multicolumn{1}{p{\gnumericColB}|}%
	{\gnumericPB{\centering}\gnumbox{$\vec{v}_2$}}
	&\multicolumn{1}{p{\gnumericColC}|}%
	{\gnumericPB{\centering}\gnumbox{$\vec{v}_3$}}
\\
\hhline{|---|}
	 \multicolumn{1}{|p{\gnumericColA}|}%
	{\gnumericPB{\centering}\gnumbox{$\myvec{1 \\ 1 \\1}$}}
	&\multicolumn{1}{p{\gnumericColB}|}%
	{\gnumericPB{\centering}\gnumbox{$\myvec{1 \\ 1 \\1}$}}
	&\multicolumn{1}{p{\gnumericColC}|}%
	{\gnumericPB{\centering}\gnumbox{$\vec{e}_i$}}
\\
\hhline{|---|}
	 \multicolumn{1}{|p{\gnumericColA}|}%
	{\gnumericPB{\centering}\gnumbox{$\myvec{1 \\ 1 \\1}$}}
	&\multicolumn{1}{p{\gnumericColB}|}%
	{\gnumericPB{\centering}\gnumbox{$\myvec{1 \\ 1 \\0}$}}
	&\multicolumn{1}{p{\gnumericColC}|}%
	{\gnumericPB{\centering}\gnumbox{$\myvec{1 \\ 0 \\1}$}}
\\
\hhline{|---|}
	 \multicolumn{1}{|p{\gnumericColA}|}%
	{\gnumericPB{\centering}\gnumbox{$\myvec{1 \\ 1 \\1}$}}
	&\multicolumn{1}{p{\gnumericColB}|}%
	{\gnumericPB{\centering}\gnumbox{$\myvec{1 \\ 1 \\0}$}}
	&\multicolumn{1}{p{\gnumericColC}|}%
	{\gnumericPB{\centering}\gnumbox{$\myvec{0 \\ 1 \\1}$}}
\\
\hhline{|---|}
	 \multicolumn{1}{|p{\gnumericColA}|}%
	{\gnumericPB{\centering}\gnumbox{$\myvec{1 \\ 1 \\1}$}}
	&\multicolumn{1}{p{\gnumericColB}|}%
	{\gnumericPB{\centering}\gnumbox{$\myvec{1 \\ 0 \\1}$}}
	&\multicolumn{1}{p{\gnumericColC}|}%
	{\gnumericPB{\centering}\gnumbox{$\myvec{0 \\ 1 \\1}$}}
\\
\hhline{|-|-|-|}
\end{tabular}

\ifthenelse{\isundefined{\languageshorthands}}{}{\languageshorthands{\languagename}}
\gnumericTableEnd

%}
\caption{}
\label{table:2019_15}
\end{table}
\item Find $\pr{E_1|E_2}$.
\\
\solution From \eqref{eq:2019_15_e2} and \eqref{eq:2019_15_e1e2}, 
\begin{align}
\label{eq:2019_15_sol}
\pr{E_1|E_2} = \frac{\abs{\brak{E_1|E_2}}}{\abs{E_2}} = \frac{9}{72} = \frac{1}{8}
\end{align}
\item Verify using a python script.
\end{enumerate}
\section{Matrices: Trace}
\begin{enumerate}[label=\thesection.\arabic*
,ref=\thesection.\theenumi]
\item Obtain the  $3 \times 3$ matrices $\cbrak{\vec{P}_k}_{k=1}^{6}$ from permutations of the  vectors
\begin{align}
\vec{v}_1 = \myvec{0 \\ 0 \\ 1},
\vec{v}_2 = \myvec{0 \\ 1 \\ 0},
\vec{v}_3 = \myvec{1 \\ 0 \\ 0}
\end{align}
%
%

\item Let 
\begin{align}
\vec{X} = \sum_{k=1}^{6}\vec{P}_k\myvec{2 & 1 & 3 \\ 1 & 0 & 2 \\ 3 & 2 & 1}\vec{P}_k^T.
\end{align}
Given 
\begin{align}
\vec{X}\myvec{1 \\1 \\1} = \alpha\myvec{1 \\1 \\1},
\label{eq:2019_qp2_1_alpha}
\end{align}
is
$\alpha= 30$?
\\
\solution 
\begin{align}
\because \vec{P}_k^T\myvec{1 \\1 \\1} &=  \myvec{1 \\1 \\1},
\nonumber \\
\vec{X}\myvec{1 \\1 \\1} &= \sum_{k=1}^{6}\vec{P}_k\myvec{2 & 1 & 3 \\ 1 & 0 & 2 \\ 3 & 2 & 1}\vec{P}_k^T\myvec{1 \\1 \\1} 
\nonumber \\
&= \sum_{k=1}^{6}\vec{P}_k\myvec{2 & 1 & 3 \\ 1 & 0 & 2 \\ 3 & 2 & 1}\myvec{1 \\1 \\1} 
\nonumber \\
&= \sum_{k=1}^{6}\vec{P}_k\myvec{5 \\3 \\5} =2\myvec{1 & 1 & 1 \\ 1 & 1 & 1 \\ 1 & 1 & 1}\myvec{6 \\3 \\6}
\nonumber \\
&= 30\myvec{1 \\1 \\1} 
\end{align}
%
Thus, $\alpha=30$.
\item Is $\vec{X}$  symmetric?
\\
\solution Yes. Trivial.
\item Show that 
\begin{align}
\label{eq:2019_qp2_1_I}
\vec{P}_k\vec{P}_k^T = \vec{I}
\end{align}
\solution
\begin{align}
\vec{P}_k = \myvec{\vec{v}_{k1}^T\\\vec{v}_{k2}^T\\\vec{v}_{k3}^T}
\end{align}
%
where $\vec{v}_{ki}, i = 1,2,3$ are from the standard basis.  Then,
\begin{align}
\vec{P}_k\vec{P}_k^T &= \myvec{\vec{v}_{k1}^T\\\vec{v}_{k2}^T\\\vec{v}_{k3}^T}\myvec{\vec{v}_{k1}&\vec{v}_{k2}&\vec{v}_{k3}}
 \vec{I}
\nonumber
\\
\because 
\vec{v}_{ji}^T\vec{v}_{kj} &= \delta_{jk}
\end{align}
%
\item For $2\times 2$ matrices $\vec{A},\vec{B}$, verify that 
\begin{align}
tr\brak{AB}=tr\brak{BA}
\end{align}
%
Show that this is true for any square matrix.
\item Verify if the sum of the diagonal entries of $X$ is 18.
\\
\solution 
\begin{align}
\text{tr}\brak{\vec{X}} &= \sum_{k=1}^{6}\text{tr}\cbrak{\vec{P}_k\myvec{2 & 1 & 3 \\ 1 & 0 & 2 \\ 3 & 2 & 1}\vec{P}_k^T}
\nonumber \\
&= \sum_{k=1}^{6}\text{tr}\cbrak{\myvec{2 & 1 & 3 \\ 1 & 0 & 2 \\ 3 & 2 & 1}\vec{P}_k\vec{P}_k^T}
\nonumber \\
&= \sum_{k=1}^{6}\text{tr}\myvec{2 & 1 & 3 \\ 1 & 0 & 2 \\ 3 & 2 & 1} = 6\times 3 = 18
\end{align}
after substituting from \eqref{eq:2019_qp2_1_I}.
%
\item Is $\vec{X}-30\vec{I}$ invertible?
\\
\solution From \eqref{eq:2019_qp2_1_alpha},
\begin{align}
\vec{X}\myvec{1 \\1 \\1} &=30\vec{I}\myvec{1 \\1 \\1} 
\nonumber \\
\implies \brak{\vec{X}-30\vec{I}}\myvec{1 \\1 \\1}  &=0
\end{align}
If $\brak{\vec{X}-30\vec{I}}^{-1}$ exists,
\begin{align}
\brak{\vec{X}-30\vec{I}}^{-1}\brak{\vec{X}-30\vec{I}}\myvec{1 \\1 \\1}  &=0
\nonumber \\
\implies \myvec{1 \\1 \\1}  &=\vec{0}
\end{align}
%
which is a contradiction.  Hence, $\vec{X}-30\vec{I}$ is not invertible.
\end{enumerate}
\section{Linear Algebra: Eigenvector and Null Space}
Let 
\begin{align}
\vec{P} = \myvec{1 & 1 & 1 \\ 0 & 2 & 2 \\ 0 & 0 & 3}, 
\vec{Q} = \myvec{2 & x & x \\ 0 & 4 & 0 \\ x & x & 6}
\label{eq:2019_qp2_2_q}
\end{align}
%
\begin{enumerate}[label=\thesection.\arabic*
,ref=\thesection.\theenumi]

\item Find $x$ such that $PQ=QP$.
%
\\
\solution 
\begin{align}
\because \vec{Q} &= \myvec{2 & 0 & 0 \\ 0 & 4 & 0 \\ 0 & 0 & 6} +x\myvec{0 & 1 & 1 \\ 0 & 0 & 0 \\ 1 & 1 & 0}, 
\end{align}
\begin{align}
\vec{P}\vec{Q} = 
 \myvec{2 & 4 & 6 \\ 0 & 8 & 12 \\ 0 & 0 & 18}+x\myvec{1 & 2 & 1 \\ 2 & 2 & 0 \\ 3 & 3 & 0}
\end{align}
and 
\begin{multline}
\vec{Q}\vec{P} = 
\myvec{2 & 2 & 2 \\ 0 & 8 & 8 \\ 0 & 0 & 18}+x\myvec{0 & 2 & 5 \\ 0 & 0 & 0 \\ 1 & 3 & 3} 
\end{multline}
%
Thus, 
\begin{align}
\vec{P}\vec{Q} = \vec{Q}\vec{P} 
\implies 
\myvec{0 & 2 & 4 \\ 0 & 0 & 4 \\ 0 & 0 & 0} &= x\myvec{-1 & 0 & 4 \\ -2 & -2 & 0 \\ -2 & 0 & 3}
\end{align}
which has no solution.
\item If 
\begin{align}
\vec{R} = \vec{P}\vec{Q}\vec{P}^{-1},
\end{align}
verify whether 
\begin{align}
\det{\vec{R}}= 
\det\myvec{2 & x & x \\ 0 & 4 & 0 \\ x & x & 5} + 8
\end{align}
for all $x$.
\\
\solution 
\begin{align}
\det(\vec{R}) &= \det(\vec{P})\det(\vec{Q})\det(\vec{P})^{-1}=\det(\vec{Q})
\nonumber \\
&= 4\brak{12-x^2}
\end{align}
%
Thus, 
\begin{multline}
\det(\vec{R}) - \det\myvec{2 & x & x \\ 0 & 4 & 0 \\ x & x & 5}
\\
= 4\cbrak{\brak{12-x^2}-\brak{10-x^2}}
\\
= 8
\end{multline}
%
which is true.
\item For $x= 0$, if 
\begin{align}
\label{eq:2019_qp2_2_ab}
\vec{R}\myvec{1 \\ a \\ b} = 6\myvec{1 \\ a \\ b}, 
\end{align}
%
then show that 
\begin{align}
a+b = 5.
\end{align}
\solution For $x=0$, 
\begin{align}
\vec{R} = \vec{P}\vec{Q}\vec{P}^{-1},
\end{align}
%
where $\vec{Q}$ is a diagonal matrix.  This is the eigenvalue decomposition of $\vec{R}$.  Thus, 
\begin{align}
\label{eq:2019_qp2_2_6eig}
\vec{R}\myvec{1 \\ 2 \\ 3} = 6\myvec{1 \\ 2 \\ 3}, 
\end{align}
%
where 
\begin{align}
\myvec{1 \\ 2 \\ 3}
\end{align}
%
is the eigenvector corresponding to the eigenvalue $6$. Comparing with \eqref{eq:2019_qp2_2_6eig},
\begin{align}
a=2,b=3 \implies a+b = 5.
\end{align}
%
\item For $x = 1$, verify if  there exists a  vector $\vec{y}$ for which $\vec{R}\vec{y} = \vec{0}$. 
\\
\solution 
\begin{align}
\vec{R}\vec{y} &= \vec{0} \implies \vec{P}\vec{Q}\vec{P}^{-1}\vec{y} = \vec{0}
\nonumber \\
\implies \vec{Q} \vec{z}&= \vec{0},
\label{eq:2019_qp2_2_null}
\end{align}
%
where 
\begin{align}
\label{eq:2019_qp2_2_yz}
\vec{z} = \vec{P}^{-1}\vec{y} 
\end{align}
For $x=1$, \eqref{eq:2019_qp2_2_q} and \eqref{eq:2019_qp2_2_null} yield
\begin{align}
\label{eq:2019_qp2_2_x1}
\myvec{2 & 1 & 1 \\ 0 & 4 & 0 \\ 1 & 1 & 6}\vec{z} &= \vec{0} 
\end{align}
Using row reduction,
\begin{align}
%\label{eq:2019_qp2_2_x1}
\myvec{2 & 1 & 1 \\ 0 & 4 & 0 \\ 1 & 1 & 6} &\leftrightarrow
\myvec{2 & 1 & 1 \\ 0 & 1 & 0 \\ 0 & 1 & 11} \leftrightarrow
\myvec{1 & 0 & -5 \\ 0 & 1 & 0 \\ 0 & 1 & 11} \leftrightarrow
\nonumber \\
\myvec{1 & 0 & -5 \\ 0 & 1 & 0 \\ 0 & 0 & 11} &
\end{align}
%
Thus, $\vec{Q}^{-1}$ exists and 
\begin{align}
\vec{z} = \vec{0} \implies \vec{y}= \vec{0}
\end{align}
upon substituting from \eqref{eq:2019_qp2_2_yz}.
This implies that the null space of $\vec{R}$ is empty. 

\end{enumerate}
\section{Definite Integral: Limit of a Sum}
\begin{enumerate}[label=\thesection.\arabic*
,ref=\thesection.\theenumi]
\item Show that 
\begin{align}
\label{eq:qp2_6_croot}
\lim_{n \to \infty}
\frac{1 + 2^{\frac{1}{3}}+ \dots +n^{\frac{1}{3}}}{n^{\frac{4}{3}}} 
= \int_{0}^{1}x^{\frac{1}{3}}\, dx = \frac{3}{4}
\end{align}
\item Show that 
\begin{multline}
\label{eq:qp2_6_recip}
\lim_{n \to \infty}
\frac{1}{n}\sbrak{\frac{1}{\brak{a+\frac{1}{n}}^2}+\frac{1}{\brak{a+\frac{1}{2}}^2}+\dots+\frac{1}{\brak{a+1}^2}}
\\
= \int_{0}^{1}\frac{1}{\brak{a+x}^2}\, dx = \frac{1}{a\brak{a+1}}
\end{multline}
\item If 
\begin{multline}
\label{eq:qp2_6_prob}
\lim_{n \to \infty}\brak{
\frac{1 + 2^{\frac{1}{3}}+ \dots +n^{\frac{1}{3}}}{n^{\frac{7}{3}}\cbrak{\sbrak{\frac{1}{\brak{an+1}^2}+\frac{1}{\brak{an+2}^2}+\dots+\frac{1}{\brak{an+n}^2}}
}}} 
\\
= 54, \quad \abs{a} > 1,
\end{multline}
%
find $a$.
\\
\solution Substituting from \eqref{eq:qp2_6_croot} and \eqref{eq:qp2_6_recip} in \eqref{eq:qp2_6_prob},
\begin{align}
\frac{3}{4}a\brak{a+1} &= 54
\nonumber \\
a\brak{a+1} &= 72
\nonumber \\
\implies a = 8,-9.
\end{align}
%
\end{enumerate}
\section{Exercises}
\begin{enumerate}[label=\arabic*.]
\item For any two $3 \times 3$ matrices $A$ and $B$, let $A+B = 2B^T$ and $3A+2B=I_3$.  Which of the following 
is true?
\begin{enumerate}
\item $5A+10B=2I_3$.
\item $10A+5B=3I_3$.
\item $2A+B=3I_3$.
\item $3A+6B=2I_3$.
\end{enumerate}
\end{enumerate}
\end{document}
