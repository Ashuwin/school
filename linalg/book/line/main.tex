\documentclass{article}
\usepackage[utf8]{inputenc}
\usepackage{graphicx}
\title{Straight lines}
\begin{document}

\maketitle

\section{\LARGE Section-A}\\*\\*
{\Large\textbf {A Fill in the Blanks}}\\*\\* 
{\large 1. The area enclosed within the curve \vert x \vert + \vert y \vert = 1 \enspace is..........\\*\\*
2.\enspace $y=10^x$ is the reflection of $y=log_1_0x$ in the line whose equation is...........\\*\\*
3.\enspace The set of lines ax+by+c =0, where 3a+2b+4c=0 is a concurrent at 
24
 the point......\\*\\*
4.\enspace Given the points A(0,4) and B(0,-4), the equation of the locus of the point P(x,y) such that \vert AP-BP \vert=6 \enspace is......\\*\\*
5.\enspace $If a,b and c are in A.P,then the straight line ax+by+c=0 will always pass through a fixed point whose coordinates are$..........\\*\\*
6.\enspace $The orthocentre of the triangle formed by the lines x+y=1,\enspace 2x+3y=6 and 4x-y+4=0 lies in quadrant number .........$\\*\\*
7.\enspace $Let the algebraic sum of the perpendicular distances from the points (2,0),((0,2) and (1,1) to a variable straight line be zero; then the line passes through a fixed points whose coordinates are .........$\\*\\*
8.\enspace $The vertices of a triangle are A(-1,-7),(B(5,1) and C(1,4). The equation of the bisector of the \angle ABC is .............$\\*\\*}
{\Large \textbf{B\enspace True/False}}\\*\\* 
{\large 1. $The Straight line 5x+4y =0 passes through the point of intersection of the straight lines x+2y-10=0 and 2x+y+5=0$\\*\\*
2.\enspace $The lines 2x+3y+19=0 and 9x+6y-17=0 cut the coordinate axes in concyclic points$\\*\\*
}
{\Large \textbf{C}\enspace \textbf{MCQs with one Correct Answer}}\\*\\*
{\large $1.The points (-a,b),(0,0),(a,b) and (a^2,ab) are:$\\*
(a) Collinear\\*
(b) Vertices of a parallelogram\\*
(c) Vertices of a rectangle\\*
(d) None of these\\*\\*
2.\enspace The point(4,1) undergoes the following three transformations successively.\\*
(i) Reflection about the line y=x\\*
(ii)Translation through a distance 2 units along the positive direction of x-axis\\*
(iii) Rotation through an angle p/4 about the origin in counter clockwise direction.\\*
Then the final position of the point is given by the coordinates$\\*\\*
(a)\enspace $(1/\sqrt{2},7/\sqrt{2})$\\*
(b)\enspace $(\sqrt{-2},7/\sqrt{2})$\\*
(c)\enspace $(-1/\sqrt{2},7/\sqrt{2})$\\*
(b)\enspace $(\sqrt{2},7/\sqrt{2})$\\*\\*
3.\enspace The straight line x+y=0, 3x+y-4=0, x+3y-4=0 form a triangle which is \\* 
(a)isosceles\\
(b)Equilateral\\*
(c)right angled\\*
(d)none of these\\*\\*
4.\enspace If P=(1,0), Q=(-1,0)and R=(2,0) are three given points, then locus of the point S satisfying the relation SQ^2+SR^2=2SP^2, is}\\*
(a)$\enspace a straight line parallel to X-axis$\\*
(b)$\enspace a circle passing through the origin$\\*
(c)$\enspace a circle with centre at the origin$\\*
(d)$\enspace a straight line parallel to Y-axis.$\\*\\*
5.\enspace $Line L has intercepts a and b on the coordinate axes.When the axes are rotated through a given angle, keeping the origin fixed, the same line L has intercepts p and q, then$\\*
(a) $a^2+b^2=p^2+q^2$\\*
(b)$1/a^2+1/b^2=1/p^2+1/q^2$\\*
(c)$a^2+p^2=b^2+q^2$\\*
(d)$1/a^2+1/p^2=1/b^2+1/q^2$\\*\\*
6.\enspace If the sum of the distances of a point from two perpendicular lines in a plane is 1, then its locus is\\*
(a)\enspace Square\\*
(b)\enspace Circle\\*
(c)\enspace Straight line\\*
(d)\enspace Two intersecting lines\\*\\*
7.\enspace The locus of a variable point whose distances from (-2,0) is 2/3 times its distance from the line x=-9/2 is\\*
a)\enspace Ellipse\\*
(b)\enspace Parabola\\*
(c)\enspace Hyperbola\\*
(d)\enspace None of these\\*\\*
8.\enspace The equation to a pair of opposite sides of a parallelogram are $x^2-5x+c=0$ and $y^2-6y+5=0$ the equation to its diagonals are\\*
(a)x+4y=13,y=4x-7\\*
(b)4x+y=13,4y=x-7\\*
(c)4x+y=13,y=4x-7\\*
(d)y-4x=13,y+4x=7\\*\\*
9.\enspace The orthocentre of the triangle formed by the lines xy=0 and x+y=1 is \\*
(a)(1/2,1/2)\\*
(b)(1/3,1/3)\\*
(c)(0,0)\\*
(d)(1/4,1/4)\\*\\*
10.\enspace Let PQR be a right angled isosceles triangle, right angled at  P(2,1). If the equation of the line QR is 2x+y=3,then the equation representing the pair of lines PQ and PR is\\*
(a)$3x^2-3y^2+8xy+20x+10y+25=0$\\*
(b)$3x^2-3y^2+8xy-20x-10y+25=0$\\*
(c)$3x^2-3y^2+8xy+10x+15y+20=0$\\*
(d)$3x^2-3y^2-8xy-10x-15y-20=0$\\*\\*
11.\enspace If $x_1,x_2,x_3$ as well as $y_1,y_2,y_3$ are in G.P with the same common ratio, then the points$(x_1,y_1),(x_2,y_2) and (x_3,y_3)$\\*
a)\enspace lie on a straight line\\*
(b)\enspace lie on a ellipse\\*
(c)\enspace lie on a circle\\*
(d)\enspace vertices of a triangle\\*\\*
12.\enspace Let PS be the median of the triangle with vertices P(2,2), Q(6,-1) and R(7,3). The equation of the line passing through(1,-1) and Parallel to PS is\\*
(a)2x-9y-7=0\\*
(b)2x-9y-11=0\\*
(c)2x+9y-11=0\\*
(d)2x+9y+7=0\\*\\*
13.\enspace The incentre of the triangle with vertices (1,\sqrt{3}),(0,0) and(2,0) is \\*
(a)\enspace $(1,\enspace\sqrt3/2)$\\*
(b)\enspace $(2/3,\enspace1/\sqrt3)$\\*
(c)\enspace $(2/3,\enspace\sqrt3/2)$\\*
(b)\enspace $(1,\enspace1/\sqrt3})$\\*\\*
14.\enspace The number of integer values of m, for which the x coordinate of the point of intersection of the lines 3x+4y=9 and y=mx+1 is also an integer,is\\*
(a)\enspace 2\\*
(b)\enspace 0\\*
(c)\enspace 4\\*
(d)\enspace 1\\*\\*
15.\enspace The area of the parallelogram formed by the lines y=mx, y=mx+1 , y=nx and y=nx+1 equals\\*
(a)\vert m+n\vert /$(m-n)^2$\\*
(b)2/\vert m+n\vert$\\*
(c)1/(\vert m+n\vert)$\\*
(d)1/(\vert m-n\vert)$\\*\\*
16.\enspace Let $0\textless\alpha<\pi/$ be fixed angle. If\\*
$P=(cos\theta,sin\theta)$ and $Q=(cos(\alpha-\theta),sin(\alpha-\theta))$\\*
then the Q is obtained from P by\\*
(a)\enspace clockwise rotation around origin through an angle \alpha\\*
(b)\enspace $anticlockwise rotation around origin through an angle \alpha$\\*
(c)\enspace reflection in the line through origin with slope tan\alpha\\*
(d)\enspace $reflection in the line through origin with slope tan\alpha/2$\\*\\*
17.\enspace Let P=(-1,0) Q=(0,0) and R=(3,$\sqrt3$) be three points.Then the equation of the bisector of the angle PQR is\\*
(a)\enspace $\sqrt3/2x+y=0$\\*
(b)\enspace $x+\sqrt3y=0$\\*
(c)\enspace $\sqrt3+y=0$\\*
(d)\enspace $x+\sqrt3/2y=0$\\*\\*
18.\enspace A straight line through the origin O meets the parallel lines 4x+2y=9 and 2x+y+6=0 ar points P and Q respectively. Then the point O divides the segment PQ in the ratio//*
(a)\enspace1:2\\*
(b)\enspace3:4\\*
(c)\enspace 2:1\\*
(d)\enspace4:3\\*\\*
19.\enspace The number of integral points (integral points means both the coordinate should be integer)exactly in the interior of the triangle with the vertices (0,0) ,(0,21) and (21,0) is \\*
(a)\enspace133\\*
(b)\enspace190\\*
(c)\enspace233\\*
(d)\enspace105\\*\\*
20.\enspace Orthocentre of a triangle with vertices (0,0),(3,4),(4,0) is\\*
(a)\enspace(3,\enspace5/4)\\*
(b)\enspace(3,\enspace12)\\*
(c)\enspace(3,\enspace3/4)\\*
(d)\enspace(3,\enspace9)\\*\\*
21.\enspace Area of the triangle formed by the line x+y=3 and angle bisectors of the pair of straight lines $x^2-y^2+2y=1$ is \\*
(a)\enspace 2 sq.units\\*
(b)\enspace 4 sq.units\\*
(c)\enspace 6 sq.units\\*
(d)\enspace 8 sq.units\\*\\*
22.\enspace Let O(0,0),P(3,4),Q(6,0) be the vertices of the triangle OPQ. The point R inside the triangle OPQ is such that the triangles OPR,PQR, OQR are of equal area. The coordinates of R are\\*
(a)\enspace (4/3,\enspace3)\\*
(b)\enspace (3,\enspace2/3)\\*
(c)\enspace (3,\enspace4/3)\\*
(d)\enspace (4/3,\enspace2/3)\\*\\*
23.\enspace A straight line L through the point (3,-2) is inclined at an angle $60^0$ to the line $\sqrt3x+y=1$. If L also intersects the x-axis, then the equation of L is\\*
(a)\enspace y+$\sqrt3x$+2-3$\sqrt3=0\\*
(b)\enspace y-$\sqrt3x$+2+ 3$\sqrt3=0\\*
(c)\enspace $\sqrt3 y$-x+3+2$\sqrt3$=0\\*
(d)\enspace $\sqrt3 y$+x-3+2$\sqrt3$=0\\*\\*\\*
}
{\Large \textbf{C\enspace MCQs with one or More than one correct}}\\*\\*
{\large 1.\enspace Three lines px+qy+r=0, qx+ry+p=0 and rx+py+q=0 are concurrent if \\*
(a)\enspace p+q+r=0\\*
(b)\enspace $p^2+q^2+r^2=qr+rp+pq$\\*
(c)\enspace $p^3+q^3+r^3=3pqr$\\*
(d)\enspace none of these\\*\\*
2.\enspace The points (0, 8/3),(1,3) and (82,30) are vertices of \\*
(a)\enspace an obtuse angled triangle\\*
(b)\enspace an acute angled triangle\\*
(c)\enspace a right angled triangle\\*
(d)\enspace none of these\\*\\*
3.\enspace All points lying inside the triangle formed by the points (1,3),(5,0) and (-1,2) satisfy\\*
(a)\enspace3x+2y$\geq 0$\\*
(b)\enspace2x+y-13$\geq 0$\\*
(c)\enspace2x-3y-12$\leq 0$\\*
(d)\enspace-2x+y$\geq 0$\\*
(e)\enspace none of these\\*\\*
4.\enspace A vector $a\bar$ has components 2p and 1 with respect to a rectangluar cartesian system.The system is rotated through a certain angle about the origin in the counter clockwise sense. If, with respect to the new system,  $a\bar$ has components p+1 and 1, then\\*
(a)\enspace p=0\\*
(b)\enspace p=1 or p=-1/3\\*
(c)\enspace p=-1 or p=1/3\\*
(d)\enspace p=1 or p=-1\\*
(e)\enspace none of these\\*\\*
5.\ensopace If P(1,2), Q(4,6), R(5,7) and S(a,b) are the vertices of a parallelogram PQRS,then\\*
(a)\enspace a=2,\enspace b=4\\*
(b)\enspace a=3,\enspace b=4\\*
(c)\enspace a=2,\enspace b=3\\*
(d)\enspace a=3,\enspace b=5\\*\\*
6.\enspace The diagonals of a parallelogram PQRS are along the lines x+3y=4 and 6x-2y=7. Then PQRS must be a.\\*
(a)\enspace rectangle\\*
(b)\enspace square\\*
(c)\enspace cyclic quadrilateral\\*
(d)\enspace rhombus\\*\\*
7.\enspace If the vertices P, Q , R of a triangle PQR are rational points, which of the following points of the triangle PQR is(are) always rational point(s)?\\*
(a)\enspace centroid\\*
(b)\enspace incentre\\*
(c)\enspace circumcentre\\*
(d)\enspace orthocentre\\*\\*
8. Let $L_1$ be a straight line passing through the origin and $L_2$ be the straight line x+y=1.If the intercepts made by the circle $x^2+y^2 -x+3y= 0$ on $L_1$ and $L_2$ are equal, then which of the following equations can represent $L_1$?\\*
(a)\enspace x+y=0\\*
(b)\enspace x-y=0\\*
(c)\enspace x+7y=0\\*
(d)\enspace x-7y=0\\*\\*
9.\enspace For a\textgreater b\textgreater c\textgreater 0, the distance between(1,1) and the point of intersection of the lines ax+by+c=0 and bx+ay+c=0 is less than 2$\sqrt2$. Then \\*
(a)\enspace a+b-c\textgreater 0\\*
(b)\enspace a-b+c\textless 0\\*
(c)\enspace a-b+c\textgreater 0\\*
(d)\enspace a+b-c\textless 0\\*\\*
}
{\Large \textbf{E\enspace Subjective Problems}}\\*\\*
{\large 1.\enspace A straight line segment of length l, moves with its ends on two mutually perpendicular lines. Find the locus of the points which divides the line segment in the ration 1:2.\\*\\*
2.\enspace The area of triangle is 5. Two of its vertices are A(2,1), B(3,-2). The third vertex C lies on y=x+3. Find C.\\*\\*
3.\enspace one side of a rectangle lies along the line 4x+7y+5=0. Two of its vertices are (-3,1) and (1,1). Find the equations of the other three sides.\\*\\*
4.\enspace (a) Two vertices of a triangle are (5,-1) and(-2,3). If the orthocentre of the triangle is the origin, find the coordinates of the third point.\\*\\*
\enspace\enspace (b) Find the equation of the line which bisects the obtuse angle between the lines x-2y+4=0 and 4x-3y=2=0.\\*\\*
5.\enspace A straight line L is perpendicular to the line 5x-y=1. The area of the triangle formed by the line L and the coordinate axes is 5. Find the equation of the line L.\\*\\*
6.\enspace The end A,B of a straight line segment of constant length c slide upon the fixed rectangular axis OX,OY respectively. If the rectangle OAPB be completed, then the show that the locus of the foot of the perpendicular drawn from P to AB is $x^2^/^3+y^2^/^3=c^2^/^3$\\*\\*
7.\enspace The vertices of a triangle are [a $t_1t_2$, a($t_1+t_2$)],[a $t_2t_3$, a($t_2+t_3$)],[a $t_3t_1$, a($t_3+t_1$)]. Find the orthocentre of the triangle.\\*\\*
8.\enspace The coordinates of A,B,C are (6,3),(-3,5), (4,-2) respectively, and P is any point (x,y). Show that the ratio of the area of the triangles $\triangle PBC$ and $\triangle ABC$ is $\vert {\frac{x+y-2}{7}}\vert$\\*\\*
9.\enspace Two equal sides of an isosceles triangles are given by the equations 7x-y+3=0 and x+y-3=0 and its third side passes through the point (1,-10). Determine the equation of third side.\\*\\*
10.\enspace One of the diameters of the circle circumscribing the rectangle ABCD is 4y=x+7. If A and B are the points (-3,4) and (5,4) respectively, then find the area of the rectangle.\\*\\*
11.\enspace Two sides of a rhombus ABCD are parallel to the lines y=x+2 and y=7x+3. If the diagonals of the rhombus intersects at the point(1,2) under vertex A is on the y axis. Find possible coordinates of A.\\*\\*
12.\enspace Lines $L_1\equiv$ax+by+c=0 $L_2\equiv$lx+my+n=0 intersect at the point P and make an angle $\theta$ with each other. Find the equation of a line L different from $L_2$ which passes through P and makes the same angle $\theta$ with $L_1$\\*\\*
13.\enspace Let ABC be a traingle with AB=AC. If D is the mid point of BC,E is the foot of the perpendicular drawn from D to AC and F the mid-point of DE, Prove that AF perpendicular to BE.\\*\\*
14.\enspace Straight lines 3x+4y=5 and 4x-3y=15 intersect at the point A. Points B and C are chosen on these two lines such that AB=AC. Determine the possible equations of the lines BC passing through the point (1,2)\\*\\*
15.\enspace A line cuts the x-axis at A(7,0) and the y-axis at B(0,-5). A variable line PQ is draw perpendicular to AB cutting the x-axis in P and the y-axis in Q. If AQ and BP intersect at R, find the locus of R.\\*\\*
16.\enspace Find the equation of the line passing through the point (2,3) and making intercept of a length 2 units between the lines y+2x=3 and y+2x=5.\\*\\*\\*
\includegraphics[scale=1.5]{sample}

17.\enspace Show that all chords of the curve $3x^2-y^2-2x+4y=0$ which subtend a right angle at the origin, Pass through a fixed point. Find the coordinates of the point.\\*\\*
18.\enspace Determine all values of $\alpha$ for which the point $(\alpha, \alpha^2)$ lies inside the triangle formed by the lines\\* 
\enspace2x+3y-1=0\\* x+2y-3=0 \\* 5x-6y-1=0\\*\\*
19.\enspace Tangent at a point $P_1$ [other than (0,0)] on the curve y=$x^3$ meets the curve again at $P_2$. The tangent at $P_2$ meets the curve at $P_3$ and so on. Show that the abscissae of $P_1,P_2,P_3........P_n$,form a G.P. Also find the ratio.\enspace [area($\triangle(P_1,P_2,P_3)$/area($\triangle(P_2,P_3,P_4)$]\\*\\*
20.\enspace A line through A(-5,-4) meets the line x+3y+2=0, 2x+y+4=0 and x-y-5=0 at the points B,C and D respectively. If$(15/AB)^2+(10/AC)^2=(6/AD)^2$,find the equation of the line.\\*\\*
21.\enspace A rectangle PQRS has its side PQ parallel to the line y=mx and vertices P,Q and S on the lines y=a, x=b and x=-b respectively. Find the locus of vertex R.\\*\\*
22.\enspace Using coordinate geometry, prove that the three altitudes of any triangle are concurrent.\\*\\*
23.\enspace For points P=($x_1,y_1$) and Q=($x_2,y_2$) of the coordinate plane, a new distance d(P,Q) is defined by d(P,Q)=$\vert x_1-x_2\vert+ \vert y_1-y_2\vert$. Let O=(0,0) and A=(3,2). Prove that the set of points in the first quadrant which are equidistant (with respect to the new distance) from O and A consists of the union of a line segment of finite length and infinite ray. Sketch this set in a labelled diagram.\\*\\*
24.\enspace Let ABC and PQR be any two triangles in the same plane. Assume that the perpendiculars from the points A,B,C to the sides QR,RP,PQ respectively are concurrent. Using vector methods or otherwise, prove that the perpendiculars from P,Q,R to BC,CA,AB respectively are also concurrent.\\*\\*
25.\enspace Let a,b,c be real numbers with $a^2+b^2+c^2=1$. show that the equation\\* $\vert
ax-by-c\enspace\enspace bx+ay \enspace\enspace\enspace cx+a \vert\\* 
\vert bx+ay\enspace\enspace -ax+by-c\enspace cy+b\vert\enspace\enspace=0\\* 
\vert cx+a\enspace\enspace cy+b\enspace\enspace -ax-by+c\vert\\*$ represents a straight line.\\*\\*
26.\enspace A straight line L through the origin meets the line and x+y=3 at P and Q respectively. Through P is straight lines $L_1 and L_2$ are drawn parallel to 2x, 3x+y=5 respectively. Lines $L_1$ and $L_2$ intersect at that the locus of R, as L varies, is a straight line.\\*\\*
27.\enspace A straight line L with negative slope passes through Point(8,2) and cuts the positive coordinates are P and Q. Find the absolute minimum value of OP varies, where O is origin.\\*\\*
28.\enspace The area of the triangle formed by the intersection of parallel to x-axis and passing through P(h,k) with y=x and x+y=2 is 4$h^2$.Find the locus of the point.\\*\\*
}
{\Large \textbf{H\enspace Assertion & Reason Type Questions}}\\*\\*
{\large 1. Lines $L_1$:y-x=0 and $L-2$:2x+y=0 intersect the line $L_3$:y+2 at P and Q respectively. The bisector of the acute angle between $L_1$ and $L_2$ intersects $L_3$ at R.}\\*\\*
{\large\textbf{STATEMENT-1:}}{\large The ratio PR:RQ equals 2$\sqrt2:\sqrt5$ because}\\*\\*
{\large\textbf{STATEMENT-2:}}{\large If any triangle, bisector of an angle divides the triangle into two similar triangles}\\*\\*
{\large (a)\enspace Statement-1 is true, Statement-2 is true; Statement-2 is not a correct explanation for Statement-1\\*\\*
(b)\enspace Statement-1 is true, Statement-2 is true; Statement-2 is not a correct explanation for Statement-1\\*\\*
(c)\enspace Statement-1 is True,Statement False\\*\\*
(d)\enspace Statement-1 is False,Statement True\\*\\*}
{\Large \textbf{I\enspace Integer Value Correct Type}}\\*\\*
{\large 1. For a point P in the plane, let $d_1(P)$ and $d_2(P)$ be the distance of the point P from the lines x-y=0 and x+y=0 respectively. The area of the region R consisting of all points P lying in the first quadrant of the plane and satisfying 2$\leq d_1(P)+d_2(P) \leq 4$, is\\*\\*}
\section{\LARGE Section-B}\\*\\*
{\large 1. A triangle with vertices (4,0),(-1,-1) and (3,5) is\\*
(a)\enspace isosceles and right angled\\*
(b)\enspace isosceles but not right angled\\*
(c)\enspace right angled but not isosceles\\*
(d)\enspace neither right angled nor isosceles\\*\\*
2.\enspace Locus of mid points of the portion between the axis of x $cos\alpha$+ y$sin \alpha$=p where p is constant\\*\\*
(a)\enspace $x^2+y^2=4/p^2$\\*
(b)\enspace $x^2+y^2=4p^2$\\*
(c)\enspace $1/x^2+1/y^2=2/p^2$\\*
(d)\enspace $1/x^2+1/y^2=4/p^2$\\*\\*
3.\enspace If the pair of the lines $ax^2+2hxy+by^2+2gx+2fy+c=0$ intersects the y-axis then\\*
(a)\enspace 2fgh=b$g^2$+c$h^2$\\*
(a)\enspace b$g^2\neq$c$h^2$\\*
(a)\enspace abc=2fgh\\*
(a)\enspace none of these\\*\\*
4.\enspace A pair of lines represented by 3a$x^2$+5xy+($a^2-2$)$y^2$\\*\\*\\*\\*\\*\\*\\*
5.\enspace A square of side of a lies above the x-axis and has one vertex at the origin. The side passing through the origin makes an angle $\alpha(0\textless\alpha\textless\pi$/4) with the positive direction of the x-axis. The equation of its diagonal not passing through the origin is\\*\\*
(a)\enspace y($cos\alpha+sin\alpha$)+x($cos\apha-sin\alpha$)=a\\*
(b)\enspace y($cos\alpha-sin\alpha$)-x($sin\apha-cos\alpha$)=a\\*
(c)\enspace y($cos\alpha+sin\alpha$)+x($sin\apha-cos\alpha$)=a\\*
(d)\enspace y($cos\alpha+sin\alpha$)+x($cos\apha+sin\alpha$)=a\\*\\*
6.\enspace If the pair of straight lines $x^2-2pxy-y^2$=0 and $x^2-2qxy-y^2$=0 be such that each pair bisects the angle between the other pair, then\\*\\*
(a)\enspace pq=\enspace-1\\*
(b)\enspace p=q\\*
(c)\enspace p=\enspace-q\\*
(d)\enspace pq=1\\*\\*
7.\enspace Locus of centroid of the triangle whose vertices are (a cos t,a sin t) , (b sin t, -b cos t) and (1,0), where t is the parameter is\\*\\*
(a)\enspace (3x+1)$^2$+(3y)$^2$=\enspace a$^2$ - b$^2$\\*
(b)\enspace (3x-1)$^2$+(3y)$^2$=\enspace a$^2$ - b$^2$\\*
(c)\enspace (3x-1)$^2$+(3y)$^2$=\enspace a$^2$ + b$^2$\\*
(d)\enspace (3x+1)$^2$+(3y)$^2$=\enspace a$^2$ + b$^2$\\*\\*
8.\enspace If $x_1$,$x_2$,$x_3$ and $y_1$,$y_2$,$y_3$ are both in G.P. with the common ratio, then the common points ($x_1$,$y_1$),($x_2$,$y_2$) and ($x_3$,$y_3$)\\*
(a)\enspace are verices of a triangle\\*
(b)\enspace lie on a straight line\\*
(c)\enspace lie on a ellipse\\*
(d)\enspace lie on a circle\\*\\*
9.\enspace If the equation of the locus of a point equidistant from the point($a_1$,$b_1$) and ($a_1$,$b_2$) is ($a_1-b_1$)x+($a_1-b_2$)y+c=0 then the value of c is\\*\\*
(a)\enspace$\sqrt {a_1^2+b_1^2-a_2^2-b_2^2}$\\*
(b)\enspace 1/2($a_2^2+b_2^2-a_1^2-b_1^2$)\\*
(c)\enspace $a_1^2-a_2^2+b_1^2-b_2^2$\\*
(d)\enspace 1/2($a_1^2+a_2^2+b_1^2+b_2^2$)\\*\\*
10.\enspace Let A(2,-3) and B(-2,3) be verices of a triangle ABC. If the centroid of the triangle moves on the line 2x+3y=1 then the locus of the vertex C is the line\\*
(a)\enspace 3x-2y=3 \\*
(b)\enspace 2x-3y=7 \\*
(c)\enspace 3x+2y=5\\*
(d)\enspace 2x+3y=0\\*\\*
11.The equation of the straight line passing through the point (4,3) and making intercepts on the coordinate axis whose sum is -1 is\\*
(a)\enspace x/2 \enspace -\enspace y/3=\enspace1 and x/-2\enspace+\enspace y/1=\enspace1\\*
(b)\enspace x/2 \enspace -\enspace y/3=\enspace -1 and x/-2\enspace+\enspace y/1=\enspace1\\*
(c)\enspace x/2 \enspace +\enspace y/3=\enspace 1 and x/2\enspace+\enspace y/1=\enspace1\\*
(b)\enspace x/2 \enspace +\enspace y/3=\enspace -1 and x/-2\enspace+\enspace y/1=\enspace-1\\*\\*
12.\enspace If the sum of the slopes of the lines given by $x^2-2cxy-7y^2=0$ is 4 times their product c has the value\\*
(a)\enspace -2 \\*
(b)\enspace -1 \\*
(c)\enspace 2\\*
(d)\enspace 1\\*\\*
13.\enspace If one of the lines given by $6x^2-xy+4cy^2=0$ is 3x+4y=0, then c equals \\*
(a)\enspace -3 \\*
(b)\enspace -1 \\*
(c)\enspace 3\\*
(d)\enspace 1\\*\\*
14.\enspace The line parallel to x-axis and passing through the intersection of the lines ax+2by+3b=0 and bx-2ay-3a=0 , where (a,b)$\neq$(0,0)\\*
(a)\enspace below the x-axis at a distance of 3/2 from it\\*
(b)\enspace below the x-axis at a distance of 2/3 from it\\*
(c)\enspace above the x-axis at a distance of 3/2 from it\\*
(d)\enspace above the x-axis at a distance of 2/3 from it\\*\\*
15.\enspace If a vertex of a triangle is (1,1) and the mid points of two sides through this vertex are (-1,2) and (3,2) then the centroid of the triangle is\\*\
(a)\enspace (-1,\enspace 7/3) \\*
(b)\enspace (-1/3,\enspace 7/3) \\*
(c)\enspace (1, \enspace 7/3)\\*
(d)\enspace (1/3, \enspace 7/3)\\*\\*
16.\enspace A straight line through the point A(3,4) is such that its intercepts between the axes is bisected at A. Its equation is\\*
(a)\enspace x+y=7\\*
(b)\enspace 3x-4y+7=0\\*
(c)\enspace 4x+3y=24\\*
(d)\enspace 3x+4y=25\\*\\*
17.\enspace If (a,$a^2$) falls inside the angle made by the lines y=x/2, x>0 and y=3x, x>0,then a belong to\\*
(a)\enspace (0,\enspace1/2)\\*
(b)\enspace (3,\enspace$\infty$)\\*
(c)\enspace (1/2,\enspace3)\\*
(d)\enspace (-3,\enspace-1/2)\\*\\*
18.\enspace Let A(h,k), B(1,1) and  C(2,1) be the vertices of a right angled triangle with AC has its hypotenuse. If the area of the triangle is 1 sq.unit, then the set of values which 'k' can take is given by\\*
(a)\enspace \{-1,3\}\\*
(b)\enspace \{-3,-2\}\\*
(c)\enspace \{1,3\}\\*
(d)\enspace \{0,2\}\\*\\*
19.\enspace Let P=(-1,0), Q=(0,0) and R=(3,$sqrt3$) be three point. The equation of the bisector of the angle PQR is \\*
(a)\enspace  $\sqrt3$/2x+y=0\\*
(b)\enspace x+$\sqrt3$y=0\\*
(c)\enspace $\sqrt3$x+y=0\\*
(d)\enspace x+$\sqrt3$/2y=0\\*\\*
20.\enspace If one of the lines of $my^2+(1-m^2)xy-mx^2=0$ is a bisector of the angle between the lines xy=0, then m is \\*
(a)\enspace 1 \\*
(b)\enspace 2\\*
(c)\enspace -1/2\\*
(d)\enspace -2\\*\\*
21.\enspace The perpendicular bisector of the line segment joining P(1,4) and Q(k,3) has y-intercept -4. Then a possible value of k is\\*
(a)\enspace 1 \\*
(b)\enspace 2\\*
(c)\enspace -2\\*
(d)\enspace -4\\*\\*
22.\enspace The shortest distance between the line y-x=1 and the curve x=$y^2$ is:\\*
(a)\enspace 2$\sqrt3$/8 \\*
(b)\enspace 3$\sqrt2$/5 \\*
(c)\enspace $\sqrt3$/4 \\*
(d)\enspace 3$\sqrt2$/8 \\*\\*
23.\enspace The lines p($p^2+1$)x-y+q=0 and $(p^2+1)^2x+(p^2+1)y+2q=0$ are perpendicular to a common line for:\\*
(a)\enspace exactly one values of p\\*
(b)\enspace exactly two values of p\\*
(c)\enspace more than two values of p\\*
(d)\enspace no value of p\\*\\*
24.\enspace Three distinct points A,B and C are given in the two dimensional coordinates plane such that the ratio of the distance of any one of them from the point(1,0) to the distance from the point(-1,0) is equal to 1/3. Then the circumcentre of the triangle ABC is at the point:\\*
(a)\enspace (5/4,\enspace 0)\\*
(b)\enspace (5/2,\enspace 0)\\*
(c)\enspace (5/3,\enspace 0)\\*
(d)\enspace (0,\enspace 0)\\*\\*
25.\enspace The line L given by (x/5)+(y/b)=1 \enspace passes through the point(13,32). The line K is parallel to L and has the equation x/c+y/3=1. Then the distance between L and K is:\\*
(a)\enspace $\sqrt17$\\*
(b)\enspace 17/$\sqrt15$\\*
(c)\enspace 23/$\sqrt17$\\*
(d)\enspace 23/$\sqrt15$\\*\\*
26.\enspace The lines $L_1$:\enspace y-x=0 and $L_2$:\enspace 2x+y=0 intersect the line $L_3$:y+2=0 at P and Q respectively. The bisector of the acute angle between $L_1$ and $L_2$ intersects $L_3$ at R.\\*
Statement-1: The ratio PR:RQ equals 2$\sqrt2$:$\sqrt5$
Statement-2: In any triangle, bisector of an angle divides the triangle into two similar triangles\\*
(a)\enspace Statement-1 is True, Statement-2 is True; Statement-2 is not a correct explanation for Statement-1\\*
(b)\enspace Statement-1 is True, Statement-2 is False\\*
(c)\enspace Statement-1 is False, Statement-2 is True\\*
(d)\enspace Statement-1 is True, Statement-2 is True; Statement-2 is a correct explanation for Statement-1\\*\\*
27.\enspace If the line 2x+y=k passes through the point which divides the line segment joining the points(1,1) and (2,4) in the ration 3:2, then k equals:\\*
(a)\enspace 29/5 \\*
(b)\enspace 5\\*
(c)\enspace 6\\*
(d)\enspace 11/5\\*\\*
28.\enspace A ray of light along x+$\sqrt3$y=$\sqrt$3 gets reflected upon reaching x-axis, then the equation of the reflected ray is:\\*
(a)\enspace y=x+$\sqrt3$\\*
(b)\enspace $\sqrt3$y=x-$\sqrt3$\\*
(c)\enspace y=$\sqrt3$x-$\sqrt3$\\*
(d)\enspace $\sqrt3$y=x-1\\*\\*
29.\enspace The x coordinate of the incentre of the triangle that has the coordinates of mid points of its sides as (0,1),(1,1) and (1,0):\\*
(a)\enspace 2+$\sqrt2$\\*
(b)\enspace 2-$\sqrt2$\\*
(c)\enspace 1+$\sqrt2$\\*
(d)\enspace 1-$\sqrt2$\\*\\*
30.\enspace Let PS be the median of the triangle with vertices P(2,2),Q(6,-1) and R(7,3). The equation of the line passing through (1,-1) and parallel to PS is :\\*
(a)\enspace 4x+7y+3=0\\*
(b)\enspace 2x-9y-11=0\\*
(c)\enspace 4x-7y-11=0\\*
(d)\enspace 2x+9y+7=0\\*\\*
31.\enspace Let a,b,c and d be non zero numbers. If the point of the intersection of the lines 4ax+2ay+c=0 and 5bx+2by+d=0 lies in the fourth quadrant and is equidistant from the two axes, then:\\*
(a)\enspace 3bc-2ad=0\\*
(b)\enspace 3bc+2ad=0\\*
(c)\enspace 2bc-3ad=0\\*
(d)\enspace 2bc+3ad=0\\*\\*
32.\enspace The numbers of points, having both coordinates as integers, that lie in the interior of the triangle with vertices (0,0),(0,41) and (41,0) is:\\*
(a)\enspace 820 \\*
(b)\enspace 780\\*
(c)\enspace 901\\*
(d)\enspace 861\\*\\*
33.\enspace Two sides of rhombus are along the lines, xy-y+1=0 and 7x-y-5=0. If its diagonals intersect at (-1,-2) then which one of the following is a vertex of the rhombus?\\*
(a)\enspace (1/3,\enspace 8/3)\\*
(b)\enspace (10/3,\enspace 7/3)\\*
(c)\enspace (-3,\enspace -9)\\*
(d)\enspace (-3,\enspace -8)\\*\\*
34.\enspace A straight the through a fixed point(2,3) intersects the coordinate axes at distinct point P and Q. If O is the origin and the rectangle OPRQ is completed, then the locus of R is\\*
(a)\enspace 2x+3y=xy\\*
(b)\enspace 3x+2y=xy\\*
(c)\enspace 3x+2y=6xy\\*
(d)\enspace 3x+2y=6\\*\\*
35.\enspace Consider the set of all lines px+qy+r=0 such that 3p+2q+4r=0. Which one of the following statement is true?\\*
(a)\enspace The lines are concurrent at the point(3/4,\enspace 1/2)\\*
(b)\enspace Each line passes through the origin\\*
(c)\enspace The lines are all parallel\\*
(d)\enspace The lines are not concurrent\\*\\*
36.\enspace Slope of a line passing through P(2,3) and intersecting the line x+y=7 at a distance of 4 units from P, is:\\*
(a)\enspace{$\frac{1-\sqrt5}{1+\sqrt5}$}\\*
(b)\enspace{$\frac{1-\sqrt7}{1+\sqrt7}$}\\*
(c)\enspace{$\frac{\sqrt7-1}{\sqrt7+1}$}\\*
(d)\enspace{$\frac{\sqrt5-1}{\sqrt5+1}$}\\*\\*
}

\end{document}

