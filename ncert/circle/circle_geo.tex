\renewcommand{\theequation}{\theenumi}
\begin{enumerate}[label=\arabic*.,ref=\thesubsection.\theenumi]
\numberwithin{equation}{enumi}
\item Find the coordinates of a point $\vec{A}$, where $AB$ is the diameter of a circle whose centre is \myvec{2,-3} and $\vec{B} = \myvec{1\\4}$.
\item Find the centre of a circle passing through the points \myvec{6\\-6}, \myvec{3\\-7} and  \myvec{3\\3}.
\item Find the locus of all the unit vectors in the xy-plane.
\item Two circles of radii 5 cm and 3 cm intersect at two points and the distance between their centres is 4 cm. Find the length of the common chord.
\item  If two equal chords of a circle intersect within the circle, prove that the segments of one chord are equal to corresponding segments of the other chord.
\item If two equal chords of a circle intersect within the circle, prove that the line joining the point of intersection to the centre makes equal angles with the chords.
\item If a line intersects two concentric circles (circles with the same centre) with centre O at A, B, C and D, prove that AB = CD.
%
\item  A,B and C are three points on a circle with centre O such that $\angle BOC = 30\degree $ and $ \angle AOB = 60\degree$. If D is a point on the circle other than the arc ABC, find $\angle ADC$.
\item A chord of a circle is equal to the radius of the
circle. Find the angle subtended by the chord at
a point on the minor arc and also at a point on the
major arc.
\item $ \angle PQR = 100\degree$, where $P, Q$ and R are
points on a circle with centre O. Find $\angle OPR$.
Fig. 10.37
\item $A, B, C, D$ are points on a circle such that $ \angle ABC = 69\degree, \angle ACB = 31\degree$, find
$\angle BDC$.
\item $A, B, C$ and $D$ are four points on a
circle. $AC$ and $BD$ intersect at a point $E$ such
that $\angle BEC = 130\degree$ and $\angle ECD = 20\degree$. Find $\angle BAC$.
\item $ABCD$ is a cyclic quadrilateral whose diagonals intersect at a point $E$. If $\angle DBC = 70\degree,
\angle BAC is 30\degree$, find $\angle BCD$. Further, if $AB = BC$, find $\angle ECD$.
\item If diagonals of a cyclic quadrilateral are diameters of the circle through the vertices of
the quadrilateral, prove that it is a rectangle.
\item If the non-parallel sides of a trapezium are equal, prove that it is cyclic.
\item Two circles intersect at two points $B$ and $C$.
Through $B$, two line segments $ABD$ and $PBQ$
are drawn to intersect the circles at $A, D$ and $P$,
$Q$ respectively. Prove that
$\angle ACP = \angle QCD$.
\item If circles are drawn taking two sides of a triangle as diameters, prove that the point of
intersection of these circles lie on the third side.
\item $ABC$ and $ADC$ are two right triangles with common hypotenuse $AC$. Prove that
$\angle CAD = \angle CBD$.
\item Prove that a cyclic parallelogram is a rectangle.
\item Prove that the line of centres of two intersecting circles subtends equal angles at the
two points of intersection.
\item Two chords $AB$ and $CD$ of lengths 5 cm and 11 cm respectively of a circle are parallel
to each other and are on opposite sides of its centre. If the distance between $AB$ and
$CD$ is 6 cm, find the radius of the circle.
\item The lengths of two parallel chords of a circle are 6 cm and 8 cm. If the smaller chord is
at distance 4 cm from the centre, what is the distance of the other chord from the
centre?
\item Let the vertex of an angle $ABC$ be located outside a circle and let the sides of the angle
intersect equal chords $AD$ and $CE$ with the circle. Prove that $\angle ABC$ is equal to half the
difference of the angles subtended by the chords $AC$ and $DE$ at the centre.
\item Prove that the circle drawn with any side of a rhombus as diameter, passes through
the point of intersection of its diagonals.
\item $ABCD$ is a parallelogram. The circle through $A, B$ and $C$ intersect $CD$ (produced if
necessary) at $E$. Prove that $AE = AD$.
\item $AC$ and $BD$ are chords of a circle which bisect each other. Prove that (i) $AC$ and $BD$ are
diameters, (ii) $ABCD$ is a rectangle.
\item Bisectors of angles $A, B$ and $C$ of a $\triangle ABC$ intersect its circumcircle at $D, E$ and
$F$ respectively. Prove that the angles of the $\triangle DEF$ are $90\degree – \frac{A}{2}, 90\degree – \frac{B}{2}$ and $90\degree – \frac{C}{2}$.
\item Two congruent circles intersect each other at points A and B. Through A any line segment PAQ is drawn so that $P, Q$ lie on the two circles. Prove that $BP = BQ$.
\item In any $\triangle ABC$, if the angle bisector of $\angle A$ and perpendicular bisector of $BC$ intersect, prove that they intersect on the circumcircle of the $\triangle ABC$.
\end{enumerate}
