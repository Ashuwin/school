\begin{enumerate}[1.]
\item Find the value of $c$ if the parabola $y^2=4ax$ intercepts
a length $4a$ on the line $y = x+c$.
\item What is the equation of a parabola which is symmetrical
about the $x$-axis, touches the $y$-axis at the origin and has a latus
rectum of length 8?  What are the coordinates of the focus?   What are the coordinates of a
point on the curve at a distance 20 from the focus?
\item The parabola $y^2=4ax$ passes through the point $P \brak{6,6}$.  If $S$
is the focus find the coordinates of the other point in which $PS$ meets the curve.
\item Find the value of $a$ if the parabola $y^2=4ax$ touches the line $y = 3x+4$.  
What are the coordinates of the point of contact?
\item Prove that the line $ty=x+at^2$ touches the parabola $y^2=4ax$, and find
the coordinates of the point of contact.  Also find the coordinates of the foot of the
perpendicular to this tangent from the focus of the parabola.
\item  Prove that, if $P$ is a point on a parabola $y^2=4ax$ whose vertex
is $A$, and $PL$ at right angles to $AP$ meets the $x$-axis in $L$,  and $PN$
is the ordinate of $P$, then $NL=4a$.
\item $PN$ is the ordinate at a point $P$ on a parabola and the normal at $P$ meets
the axis in $G$.  Prove that $PG$ is equal to the ordinate that passes through the middle
point of $NG$.  
\item Find the locus of a point $P$ which moves so that its distance from the point
$\brak{1,0}$ is equal to its distance from the line $x+1=0$.  Also find the coordinates
of the middle point of the chord of this locus which lies along the line $3y=2x+4$.  
\item The normal at $P$ to a parabola meets the axis in $G$ and $S$ is the focus.  Prove that,
if the triangle $SPG$ is equilateral, then $SP$ is equal to the latus rectum.
\item $A$ is the vertex of a parabola $y^2=4ax$ and $LL_1$ is the latus rectum.
Prove that the diameter of the circle $LAL_1$ is $5a$.
\item The chord joining the points $\brak{x_1,y_1}$, $\brak{x_2,y_2}$ on the parabola
$y^2=4ax$ cuts the axis at $C$.  Prove that, if $A$ is the vertex, $x_1x_2 = AC^2$.  
\item $C$ is a fixed point $\brak{0,c}$ on the axis $Oy$ and $Q$ is a variable point
on the line through $C$ parallel to $Ox$.  A point $P$ is taken on $OQ$ so that the ordinate
of $P$ is equal to $CQ$.  Prove that the locus of $P$ is the parabola $y^2=cx$.  
\item the chord $PQ$ of a parabola passes through the focus.  If $P$ is the point 
$\brak{at^2,2at}$, what are the coordinates of $Q$?
\item The chord $PQ$ of a parabola is the normal at $P$.  If $P$   is the point $\brak{at^2,2at}$,
what are the coordinates of $Q$?
\item Find the value of $a$ if the parabola $y^2=4ax$ touches the line $3y=x+8$. Find the
equation of the normal at the point of contact, and the coordinates of the second
point of intersection of the normal and the curve.
\item Find the equations of the tangents to the parabola $y^2=16x$ at the points $\brak{36,24}$,
$\brak{\frac{4}{9},-\frac{8}{3}}$ and verify that they intersect on the directrix.
\item Prove that the line $x-6y+36=0$ touches the parabola $y^2=4x$ and find the coordinates
of the point of contact.  Find also the coordinates of the foot of the perpendicular drawn
to this tangent from the focus of the parabola.
\item Prove that the lines $wy=6x+1$ and $8y=32x+3$  both touch the parabola $y^2=6x$,
and find the equation of the line joining the points of contact.
\item Find the value of $c$ if the parabola $y^2=8x$ intercepts a length 8  on the line
$y=x+c$.  
\item Points on a parabola are represented parametrically by the relations $x=at^2$, $y=2at$,
where $t$ is a variable parameter.  Normals are drawn at the points $t=2$ and $t=1$.  Prove that
they intersect on the curve.
\item Prove that, if the points $\brak{x_2,y_1}$, $x_2,y_2$ are the ends of a chord passing through
the focus of a parabola $y^2=4ax$, then $x_1x_2=a^2$ and $y_1y_2=-4a^2$.  
\item From the focus $S$ of a parabola a line is drawn parallel to the tangent at $P \brak{at^2,2at}$
meeting the line $y=2at$ in $Q$.  Prove that the locus of $Q$ is the parabola $y^2=2a\brak{x-a}$.
\item Prove that the tangents and normal to a parabola at the points $\brak{at^2,2at}$, $\brak{\frac{a}{t^2},\frac{-2a}{t}}$
enclose a rectangle of area $a^2\brak{\frac{t+1}{t^2}}$.  
\end{enumerate}
