\renewcommand{\theequation}{\theenumi}
The following problems refer to the ellipse whose 
equation is
\begin{align}
\vec{x}^T\myvec{b^2 & 0\\0 & a^2}\vec{x} =a^2b^2
\end{align}
\begin{enumerate}[label=\arabic*.,ref=\thesubsection.\theenumi]
\numberwithin{equation}{enumi}
\item Show that the eccentric angle of one end of a latus rectum is $\cos^{-1}e$.
\item If $\alpha$ is the eccentric angle of a point $\vec{P}$ on an ellipse, what are the coordinates of the
corresponding point $\vec{Q}$ on the auxilary circle?
Write down the equations of the tangents at $\vec{P}$ and $\vec{Q}$ and show that they intersect on the major axis.
\item Prove that, if $CP$, $CD$ are conjugate radii of an ellipse and $\omega$ the angle between them, then
$\sin^{2}\omega$ varies as $CP^{-2}+CD^{-2}$.
\item Prove that if a parallelogram is inscribed in an ellipse, its sides are parallel to 
the conjugate diameters.
\item Prove that if $QQ_1$ is a chord of an ellipse parallel to the tangent at $\vec{P}$ the eccentric angles
of $\vec{Q}$ and $\vec{Q}_1$ differ from the eccentric angle at $\vec{P}$ by equal amounts.
\item Prove that the equation of the perpendicular bisector of the chord joining the points $\vec{Q}$, $\vec{Q}_1$ whose
eccentric angles are $\alpha+\beta$, $\alpha-\beta$ is 
\begin{align}
\myvec{a\sec \alpha & -b\text{ cosec } \alpha}\vec{x} = \brak{a^2-b^2}\cos\beta
\end{align}
and deduce the equation of the normal at the point whose eccentric angle is $\alpha$.
\item Prove that, if the same chord passes through a focus, then
\begin{align}
\cos\beta = \pm e \cos\alpha
\end{align}
\item Prove that the equation of a focal chord parallel to the tangent at the point whose
eccentric angle is $\alpha$ is
\begin{align}
\myvec{\frac{\cos\alpha}{a}  & \frac{\sin\alpha}{b}}\vec{x} = \pm e \cos\alpha
\end{align}
\item Prove that, if $\alpha$ is variable and $\beta$ constant, the chord joining the points whose eccentric angles are $\alpha+\beta$
and $\alpha-\beta$ touches the
ellipse
\begin{align}
\vec{x}^T\myvec{b^2 & 0\\0 & a^2}\vec{x} =a^2b^2\cos^2\beta
\end{align}
and that the locus of the poles of the chord is
\begin{align}
\vec{x}^T\myvec{b^2 & 0\\0 & a^2}\vec{x} =a^2b^2\sec^2\beta
\end{align}
\item Prove that the tangents at the points whose eccentric angles are $\alpha$, $\alpha+\frac{\pi}{2}$ intersect on the ellipse
\begin{align}
\vec{x}^T\myvec{b^2 & 0\\0 & a^2}\vec{x} =2a^2b^2
\end{align}
\item Prove that, if $\vec{P}$, $\vec{Q}$ are corresponding points on an ellipse and its auxiliary circle and the normals
at $\vec{P}$, $\vec{Q}$ intersect in $R$, then
\begin{align}
CR = a+b
\end{align}
\item Prove that, if the line joining the ends of two equal conjugate radii of an ellipse passes through
a focus the eccentricity is $\frac{1}{\sqrt{2}}$.
\item Prove that the chords that join the ends of conjugate radii all touch the ellipse
\begin{align}
2\vec{x}^T\myvec{b^2 & 0\\0 & a^2}\vec{x} =a^2b^2
\end{align}

\end{enumerate}
