\item Consider the cube ABCD$A^{'}B^{'}C^{'}D^{'}$ (ABCD and $A^{'}B^{'}C^{'}D^{'}$ are the upper and
lower bases, respectively, and edges $AA^{'}, BB^{'}, CC^{'}, DD^{'}$are parallel). The
point X moves at constant speed along the perimeter of the square ABCD
in the direction ABCDA, and the point Y moves at the same rate along
the perimeter of the square $B^{'}C^{'}$CB in the direction $B^{'}C^{'}C BB^{'}$. Points X
and Y begin their motion at the same instant from the starting positions A
and $B^{'}$, respectively. Determine and draw the locus of the midpoints of the
segments XY.

\item On the circle K there are given three distinct points A, B, C. Construct (using only straightedge and compasses) a fourth point D on K such that a circle
can be inscribed in the quadrilateral thus obtained.

\item Consider an isosceles triangle. Let r be the radius of its circumscribed circle and $\rho$ the radius of its inscribed circle. Prove that the distance d between
the centers of these two circles is
\begin{align*}
d=\sqrt{r(r-2\rho)}
\end{align*}

\item The tetrahedron SABC has the following property: there exist five spheres,
each tangent to the edges SA, SB, SC, BCCA, AB, or to their extensions.
\begin{enumerate}
\item Prove that the tetrahedron SABC is regular.
\item Prove conversely that for every regular tetrahedron five such spheres
exist.
\end{enumerate} 
 

