%\documentclass[journal,10pt,twocolumn]{IEEEtran}
%\usepackage{setspace}
%\usepackage{gensymb}
%\usepackage{caption}
%%\usepackage{multirow}
%%\usepackage{multicolumn}
%%\usepackage{subcaption}
%%\doublespacing
%\singlespacing
%\usepackage{csvsimple}
%\usepackage{amsmath}
%\usepackage{multicol}
%\usepackage{tfrupee}
%%\usepackage{enumerate}
%\usepackage{amssymb}
%%\usepackage{graphicx}
%\usepackage{newfloat}
%%\usepackage{syntax}
%\usepackage{listings}
%%\usepackage{iithtlc}
%\usepackage{color}
%\usepackage{tikz}
%\usetikzlibrary{shapes,arrows}
%
%
%
%%\usepackage{graphicx}
%%\usepackage{amssymb}
%%\usepackage{relsize}
%%\usepackage[cmex10]{amsmath}
%%\usepackage{mathtools}
%%\usepackage{amsthm}
%%\interdisplaylinepenalty=2500
%%\savesymbol{iint}
%%\usepackage{txfonts}
%%\restoresymbol{TXF}{iint}
%%\usepackage{wasysym}
%\usepackage{amsthm}
%\usepackage{mathrsfs}
%\usepackage{txfonts}
%\usepackage{stfloats}
%\usepackage{cite}
%\usepackage{cases}
%\usepackage{mathtools}
%\usepackage{caption}
%\usepackage{enumerate}	
%\usepackage{enumitem}
%\usepackage{amsmath}
%%\usepackage{xtab}
%\usepackage{longtable}
%\usepackage{multirow}
%%\usepackage{algorithm}
%%\usepackage{algpseudocode}
%\usepackage{enumitem}
%\usepackage{mathtools}
%\usepackage{hyperref}
%%\usepackage[framemethod=tikz]{mdframed}
%\usepackage{listings}
%    %\usepackage[latin1]{inputenc}                                 %%
%    \usepackage{color}                                            %%
%    \usepackage{array}                                            %%
%    \usepackage{longtable}                                        %%
%    \usepackage{calc}                                             %%
%    \usepackage{multirow}                                         %%
%    \usepackage{hhline}                                           %%
%    \usepackage{ifthen}                                           %%
%  %optionally (for landscape tables embedded in another document): %%
%    \usepackage{lscape}     
%
%
%\usepackage{url}
%\def\UrlBreaks{\do\/\do-}
%
%
%%\usepackage{stmaryrd}
%
%
%%\usepackage{wasysym}
%%\newcounter{MYtempeqncnt}
%\DeclareMathOperator*{\Res}{Res}
%%\renewcommand{\baselinestretch}{2}
%\renewcommand\thesection{\arabic{section}}
%\renewcommand\thesubsection{\thesection.\arabic{subsection}}
%\renewcommand\thesubsubsection{\thesubsection.\arabic{subsubsection}}
%
%\renewcommand\thesectiondis{\arabic{section}}
%\renewcommand\thesubsectiondis{\thesectiondis.\arabic{subsection}}
%\renewcommand\thesubsubsectiondis{\thesubsectiondis.\arabic{subsubsection}}
%
%% correct bad hyphenation here
%\hyphenation{op-tical net-works semi-conduc-tor}
%
%%\lstset{
%%language=C,
%%frame=single, 
%%breaklines=true
%%}
%
%%\lstset{
%	%%basicstyle=\small\ttfamily\bfseries,
%	%%numberstyle=\small\ttfamily,
%	%language=Octave,
%	%backgroundcolor=\color{white},
%	%%frame=single,
%	%%keywordstyle=\bfseries,
%	%%breaklines=true,
%	%%showstringspaces=false,
%	%%xleftmargin=-10mm,
%	%%aboveskip=-1mm,
%	%%belowskip=0mm
%%}
%
%%\surroundwithmdframed[width=\columnwidth]{lstlisting}
%\def\inputGnumericTable{}                                 %%
%\lstset{
%%language=C,
%frame=single, 
%breaklines=true,
%columns=fullflexible
%}
% 
%
%\begin{document}
%%
%\tikzstyle{block} = [rectangle, draw,
%    text width=3em, text centered, minimum height=3em]
%\tikzstyle{sum} = [draw, circle, node distance=3cm]
%\tikzstyle{input} = [coordinate]
%\tikzstyle{output} = [coordinate]
%\tikzstyle{pinstyle} = [pin edge={to-,thin,black}]
%
%\theoremstyle{definition} 
%\newtheorem{theorem}{Theorem}[section]
%\newtheorem{problem}{Problem}
%\newtheorem{proposition}{Proposition}[section]
%\newtheorem{lemma}{Lemma}[section]
%\newtheorem{corollary}[theorem]{Corollary}
%\newtheorem{example}{Example}[section]
%\newtheorem{definition}{Definition}[section]
%%\newtheorem{algorithm}{Algorithm}[section]
%%\newtheorem{cor}{Corollary}
%\newcommand{\BEQA}{\begin{eqnarray}}
%\newcommand{\EEQA}{\end{eqnarray}}
%\newcommand{\define}{\stackrel{\triangle}{=}}
%
%\bibliographystyle{IEEEtran}
%%\bibliographystyle{ieeetr}
%
%\providecommand{\nCr}[2]{\,^{#1}C_{#2}} % nCr
%\providecommand{\nPr}[2]{\,^{#1}P_{#2}} % nPr
%\providecommand{\mbf}{\mathbf}
%\providecommand{\pr}[1]{\ensuremath{\Pr\left(#1\right)}}
%\providecommand{\qfunc}[1]{\ensuremath{Q\left(#1\right)}}
%\providecommand{\sbrak}[1]{\ensuremath{{}\left[#1\right]}}
%\providecommand{\lsbrak}[1]{\ensuremath{{}\left[#1\right.}}
%\providecommand{\rsbrak}[1]{\ensuremath{{}\left.#1\right]}}
%\providecommand{\brak}[1]{\ensuremath{\left(#1\right)}}
%\providecommand{\lbrak}[1]{\ensuremath{\left(#1\right.}}
%\providecommand{\rbrak}[1]{\ensuremath{\left.#1\right)}}
%\providecommand{\cbrak}[1]{\ensuremath{\left\{#1\right\}}}
%\providecommand{\lcbrak}[1]{\ensuremath{\left\{#1\right.}}
%\providecommand{\rcbrak}[1]{\ensuremath{\left.#1\right\}}}
%\theoremstyle{remark}
%\newtheorem{rem}{Remark}
%\newcommand{\sgn}{\mathop{\mathrm{sgn}}}
%\providecommand{\abs}[1]{\left\vert#1\right\vert}
%\providecommand{\res}[1]{\Res\displaylimits_{#1}} 
%\providecommand{\norm}[1]{\left\Vert#1\right\Vert}
%\providecommand{\mtx}[1]{\mathbf{#1}}
%\providecommand{\mean}[1]{E\left[ #1 \right]}
%\providecommand{\fourier}{\overset{\mathcal{F}}{ \rightleftharpoons}}
%%\providecommand{\hilbert}{\overset{\mathcal{H}}{ \rightleftharpoons}}
%\providecommand{\system}{\overset{\mathcal{H}}{ \longleftrightarrow}}
%	%\newcommand{\solution}[2]{\textbf{Solution:}{#1}}
%\newcommand{\solution}{\noindent \textbf{Solution: }}
%\newcommand{\myvec}[1]{\ensuremath{\begin{pmatrix}#1\end{pmatrix}}}
%\providecommand{\dec}[2]{\ensuremath{\overset{#1}{\underset{#2}{\gtrless}}}}
%\DeclarePairedDelimiter{\ceil}{\lceil}{\rceil}
%%\numberwithin{equation}{section}
%%\numberwithin{problem}{subsection}
%%\numberwithin{definition}{subsection}
%\makeatletter
%\@addtoreset{figure}{section}
%\makeatother
%
%\let\StandardTheFigure\thefigure
%%\renewcommand{\thefigure}{\theproblem.\arabic{figure}}
%\renewcommand{\thefigure}{\thesection}
%
%
%%\numberwithin{figure}{subsection}
%
%%\numberwithin{equation}{subsection}
%%\numberwithin{equation}{section}
%%\numberwithin{equation}{problem}
%%\numberwithin{problem}{subsection}
%\numberwithin{problem}{section}
%%%\numberwithin{definition}{subsection}
%%\makeatletter
%%\@addtoreset{figure}{problem}
%%\makeatother
%\makeatletter
%\@addtoreset{table}{section}
%\makeatother
%
%\let\StandardTheFigure\thefigure
%\let\StandardTheTable\thetable
%\let\vec\mathbf
%%%\renewcommand{\thefigure}{\theproblem.\arabic{figure}}
%%\renewcommand{\thefigure}{\theproblem}
%
%%%\numberwithin{figure}{section}
%
%%%\numberwithin{figure}{subsection}
%
%
%
%\def\putbox#1#2#3{\makebox[0in][l]{\makebox[#1][l]{}\raisebox{\baselineskip}[0in][0in]{\raisebox{#2}[0in][0in]{#3}}}}
%     \def\rightbox#1{\makebox[0in][r]{#1}}
%     \def\centbox#1{\makebox[0in]{#1}}
%     \def\topbox#1{\raisebox{-\baselineskip}[0in][0in]{#1}}
%     \def\midbox#1{\raisebox{-0.5\baselineskip}[0in][0in]{#1}}
%
%\vspace{3cm}
%
%\title{ 
%%	\logo{
%Matrices
%%	}
%}
%
%\author{ G V V Sharma$^{*}$% <-this % stops a space
%	\thanks{*The author is with the Department
%		of Electrical Engineering, Indian Institute of Technology, Hyderabad
%		502285 India e-mail:  gadepall@iith.ac.in. All content in this manual is released under GNU GPL.  Free and open source.}
%	
%}	
%
%\maketitle
%
%%\tableofcontents
%
%\bigskip
%
%\renewcommand{\thefigure}{\theenumi}
%\renewcommand{\thetable}{\theenumi}
%
%
%
%\begin{enumerate}[label=\arabic*]
%\numberwithin{equation}{enumi}

\renewcommand{\theequation}{\theenumi}
\begin{enumerate}[label=\arabic*.,ref=\thesubsection.\theenumi]
\numberwithin{equation}{enumi}
\item In the matrix A=\myvec{2 &5 &19 &-7\\ 35 &-2 &\frac{5}{2} &12 \\ \sqrt{3} &1 &-5 &17},write
\begin{enumerate}
\item The order of the matrix
\item The number of elements
\item Write the elements $a_{31},a_{21},a_{33},a_{24},a_{23}.$
\end{enumerate}
\item If a matrix has 24 elements,what are the possible orders it can have? What,if it has 13 elements?\\
\item If a matrix has 18 elements,what are the possible orders it can have? What,if it has 5 elements?\\
\item Construct a $2 \times 2$ matrix,A=[$a_{ij}$],whose elements are given by:\\
(i) $a_{ij}$=$\frac{(i+j)^2}{2}$\ (ii) $a_{ij}$=$\frac{i}{j}$\ (iii) $a_{ij}$=$\frac{(i+2j)^2}{2}$\\
\item Construct a $3\times 4$ matrix,whose elements are given by:\\
(i) $a_{ij}$=$\frac{1}{2}\abs{-3i+j}$ (ii) $a_{ij}$=2i-j\\
\item Find the values of x,y and z from the following equations:\\
(i) \myvec{4 &3\\x &5} = \myvec{y &z\\1 &5} (ii) \myvec{x+y &2\\5+z &xy} = \myvec{6 &2\\5 &8} (iii) \myvec{x+y+z\\x+y\\y+z}=\myvec{9\\5\\7}\\
\item Find the values of a,b,c and d from the equations: \myvec{a-b &2a+c\\2a-b &3c+d} = \myvec{-1 &5\\0 &13}\\
\item A=$[a_{ij}]_{mxn}$ is a square matrix,if\\
(A) m$<$n (B)m$>$n (C) m=n (D) None of these\\
\item Which of the given values of x and y make the following pair of matrices equal \myvec{3x+7 &5\\y+1 &2-3x},\myvec{0 &y-2\\8 &4}\\
(A)x=$\frac{-1}{3}$,y=7 \\
 (B) Not possible to find\\
(C) y=7, x=$\frac{-2}{3}$\\
 (D) x=$\frac{-1}{3}$,y=$\frac{-2}{3}$\\
\item The number of all possible matrices of order 3X3 with each entry 0 or 1 is:\\
(A) 27 (B)18 (C)81 (D)512\\
\item Let A=\myvec{2 &4\\3 &2},B=\myvec{1 &3\\-2 &5},C=\myvec{-2 &5\\3 &4}
Find each of the following:\\
(i) A+B  (ii)A-B  (iii)3A-C  (iv)AB  (v)BA\\
\item Compute the following:\\
(i)\myvec{a &b\\-b &a}+\myvec{a &b\\b&a} (ii)\myvec{a^2+b^2 &b^2+c^2\\a^2+c^2 &a^2+b^2}+\myvec{2ab &2bc\\-2ac &-2ab} \\
  (iii) \myvec{-1 &4 &-6\\8 &5 &16\\2 &8 &5}+\myvec{12 &7 &6\\8 &0 &5\\3 &2 &4}\\ 
(iv) \myvec{cos^2x &sin^2x\\sin^2x &cos^2x}+\myvec{sin^2x &cos^2x\\cos^2x &sin^2x}\\
\item Compute the indicated products.\\
(i)\myvec{a &b\\-b &a}\myvec{a &-b\\b &a} \\
(ii)\myvec{1\\2\\3}\myvec{2 &3 &4} (iii)\myvec{1 &-2\\2 &3}\myvec{1 &2 &3\\2 &3 &1} \\
(iv)\myvec{2 &3 &4\\3 &4 &5\\4 &5 &6}\myvec{1 &-3 &5\\0 &2 &4\\3 &0 &5} (v)\myvec{2 &1\\3 &2\\-1 &1}\myvec{1 &0 &1\\-1 &2 &1} \\
(vi) \myvec{3 &-1 &3\\-1 &0 &2}\myvec{2 &-3\\1 &0\\3 &1}\\
\item If,A=\myvec{1 &2 &-3\\5 &0 &2\\1 &-1 &1},B=\myvec{3 &-1 &2\\4 &2 &5\\2 &0 &3}and C=\myvec{4 &1 &2\\0 &3 &2\\1 &-2 &3},then compute (A+B) and (B-C).Also ,verify that A+(B-C)=(A+B)-C.\\
\item If A=$\myvec{\frac{2}{3} & 1 & \frac{5}{3}\\ \frac{1}{3} & \frac{2}{3} & \frac{4}{3} \\ \frac{7}{3} &2  & \frac{2}{3}}$ and B=$\myvec{\frac{2}{5} & \frac{3}{5} &1 \\ \frac{1}{5} & \frac{2}{5} & \frac{4}{5}\\ \frac{7}{5} & \frac{6}{5} & \frac{2}{5}}$,then compute 3A-5B.\\
\item Simplify $\cos\theta$$\myvec{\cos\theta &\sin\theta\\ -\sin\theta &\cos\theta}$+$\sin\theta$$\myvec{\sin\theta &-\cos\theta\\ \cos\theta &\sin\theta}$\\
\item Find X and Y,if\\
(i)X+Y=\myvec{7 &0\\2 &5} and X-Y=\myvec{3 &0\\0 &3}\\
(ii)2X+3Y=\myvec{2 &3\\4 &0} and 3X+2Y=\myvec{2 &-2\\-1 &5}\\ 
\item Find X if Y=\myvec{3 &2\\1 &4} and 2X+Y=\myvec{1 &0\\-3 &2}\\
\item Find x and y,if 2 \myvec{1 &3\\0 &x}+\myvec{y &0\\1 &2}=\myvec{5 &6\\1 &8}\\
\item Solve the equation for x,y,z and t,if \\
2\myvec{x &z\\y &t}+3\myvec{1 &-1\\0 &2}=3\myvec{3 &5\\4 &6}\\
\item If x=\myvec{2 \\3}+y\myvec{-1 \\1}=\myvec{10 \\5},find the values of x and y.\\
\item Given 3\myvec{x &y\\z &w}=\myvec{x &6\\-1 &2w}+\myvec{4 &x+y\\z+w &3},find the values of x,y,z and w.\\
\item If F(x)=$\myvec{\cos x &-\sin x &0\\ \sin x &\cos x &0\\0 &0 &1}$\\,show that F(x)F(y)=F(x+y)\\
\item Show that\\
(i)$\myvec{5 &-1\\6 &7}\myvec{2 &1\\3 &4}\neq\myvec{2 &1\\3 &4}\myvec{5 &-1\\6 &7}$
(ii)$\myvec{1 &2 &3\\0 &1 &0\\1 &1 &0}\myvec{-1 &1 &0\\0 &-1 &1\\2 &3 &4}\neq \myvec{-1 &1 &0\\0 &-1 &1\\2 &3 &4}\myvec{1 &2 &3\\0 &1 &0\\1 &1 &0}$\\
\item Find $A^{2}-5A+6I$,if A = \myvec{2 &0 &1\\2 &1 &3\\1 &-1 &0}\\
\item If A=\myvec{1 &0 &2\\0 &2 &1\\2 &0 &3},prove that $A^3-6A^2+7A+2I=0$\\
\item If A=\myvec{3 &-2\\4 &-2} and I=\myvec{1 &0\\0 &1},find k\\
 so that $A^2=kA-2I$\\
\item If A=$\myvec{0 &-\tan\frac{\alpha}{2}\\  \tan\frac{\alpha}{2} &0}$ and I is the identity matrix of order 2,show that \\I+A=(I-A)$\myvec{\cos\alpha &-\sin\alpha\\\sin\alpha &\cos\alpha}$\\
\item A trust fund has \rupee{30,000} that must be invested in two different types of bonds.
The first bond pays 5\% interest per year, and the second bond pays 7\% interest
per year. Using matrix multiplication, determine how to divide \rupee{ 30,000} among
the two types of bonds. If the trust fund must obtain an annual total interest of:\\
(a) \rupee{1800} (b)\rupee{2000}\\
\item The bookshop of a particular school has 10 dozen chemistry books, 8 dozen
physics books, 10 dozen economics books. Their selling prices are \rupee{80}, \rupee{60} and
 \rupee{40} each respectively. Find the total amount the bookshop will receive from
selling all the books using matrix algebra.\\
Assume X,Y,Z,W and P are matrices of orders $2\times n$,$3 \times k$,$2\times p$,$n\times 3$ and $p\times k$,respectively.\\
Choose the correct answer in Exercise 31 and 32.\\
\item The restriction on n,k and p so that PY+WY will be defined are:\\
(A)k=3,p=n\\
 (B)k is arbitrary,p=2 \\
 (C)p is arbitrary,k=3 \\
 (D)k=2,p=3\\
\item If n=p,then the order of the matrix 7X-5Z is:\\
(A)$p \times 2$ (B)$2 \times n$ (C)$n \times 3$ (D)$p \times n$\\
\item Find the transpose of each of the following matrices:\\
(i)\myvec{5\\ \frac{1}{2} \\-1}\\ (ii)\myvec{1 &-1\\2 &3}\\ (iii)\myvec{-1 &5 &6\\\sqrt{3} &5 &6\\2 &3 &-1}\\
\item If A=\myvec{-1 &2 &3\\5 &7 &9\\-1 &1 &1} and B=\myvec{-4 &1 &-5\\1 &2 &0\\1 &3 &1},then verify that\\
(i)$(A+B)^{'}=A^{'}+B^{'}$ \\(ii) $(A-B)^{'}=A^{'}-B^{'}$\\
\item If $A^{'}$=\myvec{3 &4\\-1 &2\\0 &1} and B=\myvec{-1 &2 &1\\1 &2 &3},then verify that\\
(i) $(A+B)^{'}=A^{'}+B^{'}$ (ii)$(A-B)^{'}=A^{'}-B^{'}$
\item If$ A^{'}$=\myvec{-2 &3\\1 &2} and B=\myvec{-1 &0\\1 &2},then find that $(A+2B)^{'}$\\
\item For the matrices A and B,verify that $(AB)^{'}$=$B^{'}A^{'}$,where\\
(i)A=\myvec{1\\-4\\3},B=\myvec{-1 &2 &1} (ii)A=\myvec{0\\1\\2},B=\myvec{1 &5 &7}
\item If (i)  A=$\myvec{\cos\alpha &\sin\alpha\\-\sin\alpha &\cos\alpha}$, then verify that $A^{'}A=I$\\
        (ii) If A=$\myvec{\sin\alpha &\cos\alpha\\-\cos\alpha &\sin\alpha}$,then verify that $A^{'}A=I$\\
  \item (i) Show that the matrix A=\myvec{1 &-1&5\\-1 &2 &1\\5 &1 &3} is a symmetric matrix.\\
  (ii) Show that the matrix A=\myvec{0 & 1 &-1\\-1 &0 &1\\1&-1 &0} is a skew symmetric matrix.\\
  \item For the matrix A=\myvec{1 &5\\6 &7},verify that\\
  (i)$(A+A^{'})$ is a symmetric matrix\\
  (ii)$(A-A^{'})$ is a skew symmetric matrix\\
  
  \item Find $\frac{1}{2}(A+A^{'}) $and $\frac{1}{2}(A-A^{'})$,when A=\myvec{0 &a &b\\-a &0 &c\\-b &-c &0}\\
  \item Express the following matrices as the sum of a symmetric and a skew symmetric matrix:\\
  (i) \myvec{3 &5\\1 &1} \\(ii) \myvec{6 &-2 &2\\-2 &3 &-1\\2 &-1 &3} \\
  (iii) \myvec{3 &3 &-1\\-2 &-2 &1\\-4 &-5 &2}\\ (iv) \myvec{1 &5\\-1 &2}\\
  
  Choose the correct answer in question number 43 and 44\\
  \item If A,B are symmetric matrices of same order,then AB-BA is a\\
  (A)Skew symmetric matrix \\(B)Symmetric matrix\\
  (C)Zero matrix \\ (D)Identity matrix\\
  \item If A=$\myvec{\cos\alpha &-\sin\alpha\\ \sin\alpha &\cos\alpha}$,and $A+A^{'}=I$,then the value of $\alpha$ is\\
  (A) $\frac{\pi}{6}$\\ (B)$\frac{\pi}{3}$ \\
  (C) $\pi$ \\ (D)$\frac{3\pi}{2}$\\
  
  Using elementary transforamtions,find the inverse of each of the matrices,if it exists questions 45-61\\
  
  \item\myvec{1 &-1\\2 &3}\\
  \item \myvec{2 &1\\1 &1}\\
  \item\myvec{1 &3\\2 &7}\\
  \item\myvec{2 &3\\5 &7}\\
  \item\myvec{2 &1\\7 &4}\\
  \item \myvec{2 &5\\1 &3}\\
  \item \myvec{3 &1\\5 &2}\\
  \item \myvec{4 &5\\3 &4}\\
  \item \myvec{3 &10\\2 &7}\\
  \item \myvec{3 &-1\\-4 &2}\\
  \item \myvec{2 &-6\\1 &-2}\\
  \item \myvec{6 &-3\\-2 &1}\\
  \item \myvec{2 &-3\\-1 &2}\\
  \item \myvec{2 &1\\4 &2}\\
  \item \myvec{2 &-3 &3\\2 &2 &3\\3 &-2 &2}\\
  \item \myvec{1 &3 &-2\\-3 &0 &-5\\2 &5 &0}\\
  \item \myvec{2 &0 &-1\\5 &1 &0\\0 &1 &3}\\
  
  \item Matrices Aand B will be inverse of each other only if\\
  (A)AB=BA (B)AB=BA=0\\
  (C)AB=0,BA=I (D)AB=BA=I\\
  
  \item If A=\myvec{1 &1 &1\\1 &1 &1\\1 &1 &1},\\prove that $A^{n}$=$\myvec{3^{n-1} &3^{n-1}&3^{n-1}\\3^{n-1}&3^{n-1}&3^{n-1}\\3^{n-1}&3^{n-1} &3^{n-1}}$,$n \epsilon N$\\
  \item Let A=\myvec{0 &1\\0 &0},show that \\$(aI+bA)^{n}=a^{n}I+na^{n-1}bA$,where I is the identity matrix of order 2 and $n \epsilon N$\\
  \item If A=\myvec{3 &-4\\1 &-1},\\then prove that $A^{n}$=\myvec{1+2n &-4n\\n &1-2n},where n is any positive integer\\
  \item If A and B are symmetric matrices,prove that AB-BA is a skew symmetric matrix.\\
  \item Show that the matrix $ B^{'}AB$ is symmetric or skew symmetric according as A is symmetric or skew symmetric\\
  
  \item Find the values of x,y,z if the matrix A=\myvec{0 &2y &z\\x &y &-z\\x &-y &z} satisfy the equation $A^{'}A$=I\\
  \item For what values of x: \\\myvec{1 &2 &1}\myvec{1 &2 &0\\2 &0 &1\\1 &0 &2}\myvec{0 &2 &x}=0?\\
  \item If A=\myvec{3 &1\\-1 &2},show that $A^{2}$-5A+7I=0\\
  \item Find x, if \myvec{x &-5 &-1}\myvec{1 &0 &2\\0 &2 &1\\2 &0 &3}\myvec{x\\4\\1}=0\\
  \item A manufactrer produces three products x,y,z which he sells in two markets. Annual sales are indicated below:\\
 
  \begin{tabular}{cccc}
  \hline
  Market & Products\\
  \hline
  I &10,000 &2,000 &18,000\\
  \hline
  II &6,000 &20,000 &8,000\\
  \hline
  \end{tabular}\\
  (a) If unit sale prices of x,y and z are \rupee{2.50},\rupee{1.50} and \rupee{1.00} respectively,find the total revenue in each market with the help of matrix algebra.\\
  (b) If the unit cost of the above three commodities are \rupee{2.00},\rupee{1.00} and 50 paise respectively.Find the gross profit.\\
  \item Find the matrix X so that\\ X\myvec{1 &2 &3\\4 &5 &6}=\myvec{-7 &-8 &-9\\2 &4 &6}\\
  \item If A and B are square matrices of the same order such that AB=BA,then prove by indication that $AB^{n}=B^{n}A$.Further prove that $(AB)^{n}=A^{n}B^{n}$ for all $n \epsilon N$.\\
  Choose the correct answer in the following questions:\\
  \item If A=$\myvec{\alpha &\beta\\ \gamma &-\alpha}$ is such that $A^{2}=I$,then\\
  (A)$1+\alpha^{2}+\beta\gamma=0$ (B)$1-\alpha^{}2+\beta\gamma=0$\\
  (C)$1-\alpha^{2}-\beta\gamma=0$ (D)$1+\alpha^{2}-\beta\gamma=0$\\
  \item If the matrix A is both symmetric and skew symmetric,then\\
  (A) A is a diagonal matrix \\
  (B) A is a zero matriz\\
  (C)A is a square matrix \\
  (D)None of these\\
  \item If A is square matrix such that $A^{2}=A$,then $(I+A)^{3}-7A$ is equal to\\
  (A)A \\(B)I-A\\ (C)I\\ (D)3A
  \end{enumerate}

%    \end{document}    
