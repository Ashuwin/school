\renewcommand{\theequation}{\theenumi}
\begin{enumerate}[label=\arabic*.,ref=\thesubsection.\theenumi]
\numberwithin{equation}{enumi}
	%
\item A parallelogram is a quadrilateral whose opposite sides are parallel. A parallelogram is shown in Fig. \eqref{fig:pgm_sss}.	

%
\begin{figure}[!ht]
	\begin{center}
		\resizebox{\columnwidth}{!}{%Code by GVV Sharma
%December 10, 2019
%released under GNU GPL
%Drawing a parallelogram given 2 sides and a diagonal

\begin{tikzpicture}
[scale=2,>=stealth,point/.style={draw,circle,fill = black,inner sep=0.5pt},]

%Triangle sides
\def\a{5}
\def\b{6}
\def\c{7.467975323683154}
%Coordinates of A
%\def\p{{\a^2+\c^2-\b^2}/{(2*\a)}}
\def\p{4.477065543514051}
\def\q{{sqrt(\c^2-\p^2)}}

%Labeling points
\node (D) at (\p,\q)[point,label=above right:$D$] {};
\node (B) at (0, 0)[point,label=below left:$B$] {};
\node (C) at (\a, 0)[point,label=below right:$C$] {};
\node (O) at ($(B)!0.5!(D)$)[point,label=below right:$O$] {};

%Foot of perpendicular

\node (A) at ($(O)!-1!(C)$)[point,label=above right:$A$] {};


%Drawing parallelogram ABCD
\draw (A) -- (B) --  (C) --(D)--(A);
\draw (A) -- (C);
\draw (B) --(D);


\tkzMarkAngle[fill=green!20](C,A,D)
\tkzMarkAngle[fill=green!20](A,C,B)
%
%
\tkzMarkAngle[fill=red!30](A,D,B)
\tkzMarkAngle[fill=red!30](C,B,D)


\tkzMarkAngle[fill=orange!40](D,C,O)
\tkzMarkAngle[fill=orange!40](B,A,C)

\tkzMarkAngle[fill=blue!50](B,D,C)
\tkzMarkAngle[fill=blue!50](D,B,A)

\end{tikzpicture}
}
	\end{center}
	\caption{Parallelogram}
	\label{fig:pgm_sas}	
\end{figure}

\item Sum of the angles of a quadrilateral is 360$\degree$. 
\item  In a parallelogram, 
\label{them:pgm_basic}
\begin{enumerate}
\item opposite sides are equal 
\item  opposite angles are equal
\item  diagonals bisect each other
\end{enumerate}
\solution $\because AD \parallel BC, AB \parallel CD$, alternate angles are equal.  Hence, 
%
\begin{align}
\triangle ADB &\cong \triangle  DBC \quad (ASA)
\\
\because \angle ADB &= \angle DBC
\\
\angle ABD &= \angle BDC
\\
DB \text{ common}
\end{align}
%
So $AB = CD$ and $\angle A = \angle C$.  Similarly, it can be shown that $AD = BC$ and $\angle B = \angle D$
Also, 
%
\begin{align}
\triangle OAD &\cong \triangle  OBC \quad (ASA)
\\
\because AD = BC
\end{align}
%
Hence $OD = OB, OA=OC$
\item Construct parallelogram $ABCD$ 	in Fig. \ref{fig:pgm_sss}	
given that  $BC = 5, AB = 6, \angle C = 85 \degree$.
\\
\solution $BD$ is found using the cosine formula and $\triangle BDC$ is drawn using the approach in Problem \ref{const:tri_sss} with 
%
\begin{align}
\vec{B} = \myvec{0\\0},
\vec{C} = \myvec{5\\0},
\end{align}
%
Since the diagonals bisect each other, 
%
\begin{align}
\vec{O} &= \frac{\vec{B}+\vec{D}}{2}
\\
\vec{A} &= 2\vec{O} - \vec{C}.
\end{align}
%
$AB$ and $AD$ are then joined to complete the $\parallel$gm.
The python code for  Fig. \ref{fig:pgm_sas} is
\begin{lstlisting}
codes/quad/pgm_sas.py
\end{lstlisting}
%
and the equivalent latex-tikz code is
%
\begin{lstlisting}
figs/quad/pgm_sas.tex
\end{lstlisting}
%
\item Draw the $\parallel$gm $ABCD$ in 	Fig. \label{fig:pgm_sss}	
with $BC = 6, CD = 4.5$ and $BD=7.5$.  Show that it is a rectangle.
%
\begin{figure}[!ht]
	\begin{center}
		\resizebox{\columnwidth}{!}{%Code by GVV Sharma
%December 10, 2019
%released under GNU GPL
%Drawing a parallelogram given 2 sides and a diagonal

\begin{tikzpicture}
[scale=2,>=stealth,point/.style={draw,circle,fill = black,inner sep=0.5pt},]

%Triangle sides
\def\a{6}
\def\b{4.5}
\def\c{7.5}
%Coordinates of A
%\def\p{{\a^2+\c^2-\b^2}/{(2*\a)}}
\def\p{6}
\def\q{{sqrt(\c^2-\p^2)}}

%Labeling points
\node (D) at (\p,\q)[point,label=above right:$D$] {};
\node (B) at (0, 0)[point,label=below left:$B$] {};
\node (C) at (\a, 0)[point,label=below right:$C$] {};
\node (O) at ($(B)!0.5!(D)$)[point,label=below right:$O$] {};

%Foot of perpendicular

\node (A) at ($(O)!-1!(C)$)[point,label=above right:$A$] {};


%Drawing parallelogram ABCD
\draw (A) -- (B) --  (C) --(D)--(A);
\draw (A) -- (C);
\draw (B) --(D);


%\tkzMarkAngle[fill=green!20](C,A,D)
%\tkzMarkAngle[fill=green!20](A,C,B)
%%
%%
%\tkzMarkAngle[fill=red!30](A,D,B)
%\tkzMarkAngle[fill=red!30](C,B,D)
%
%
%\tkzMarkAngle[fill=orange!40](D,C,O)
%\tkzMarkAngle[fill=orange!40](B,A,C)
%
%\tkzMarkAngle[fill=blue!50](B,D,C)
%\tkzMarkAngle[fill=blue!50](D,B,A)
%
\end{tikzpicture}
}
	\end{center}
	\caption{Parallelogram}
	\label{fig:pgm_sss}	
\end{figure}
\\
\solution Using the approach in Problem \ref{const:tri_sss}, $\triangle BCD$ is drawn with
%
\begin{align}
%\vec{A} = \myvec{p\\q}
\vec{B} = \myvec{0\\0}
\vec{C} = \myvec{6\\0}
\end{align}
%
The python code for  Fig. \ref{fig:pgm_sss} is
\begin{lstlisting}
codes/quad/pgm_sss.py
\end{lstlisting}
%
and the equivalent latex-tikz code is
%
\begin{lstlisting}
figs/quad/pgm_sss.tex
\end{lstlisting}
%
$\because $
%
\begin{align}
BD^2 = BC^2+CD^2,
\end{align}
using Baudhyana's theorem, $\angle BCD =90\degree$. Since opposite angles  of parallelogram are equal, all angles are $90\degree$.  Hence, $ABCD$ is a rectangle. 
\item  Diagonals of a rectangle bisect each other and are equal and vice-versa. 
\\
\solution Proof follows from Baudhayan's theorem in $\triangle BCD$ and $\triangle ABC$ and noting that $AB = CD, BC = AD$.
\item A rhombus is a parallelogram with equal sides.  Diagonals of a rhombus bisect each other at right angles.
%
\\
\solution 
\item A rectangle is a parallelogram with one angle that is 90$\degree$.  Show that all angles of the rectangle are 90$\degree$.
\solution Follows from Theorem \ref{them:pgm_basic}.

\item Draw the rhombus $BEST$ with $BE = 4.5$ and $ET = 6$. 
\\
\solution 
%%
%
%
\item  Diagonals of a rhombus bisect each other at right angles and vice-versa. 
%
%
\item  Diagonals of a square bisect each other at right angles and are equal, and vice-versa. 
%
\\
\solution A square has the properties of a rectangle as well as a rhombus.
%
%
\item Area of a parallelogram is the product of its base and the corresponding altitude. 
%
\end{enumerate}
