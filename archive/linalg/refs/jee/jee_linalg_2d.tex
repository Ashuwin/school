\documentclass[journal,12pt,twocolumn]{IEEEtran}
\usepackage{setspace}
\usepackage{gensymb}
\usepackage{caption}
%\usepackage{multirow}
%\usepackage{multicolumn}
%\usepackage{subcaption}
%\doublespacing
\singlespacing
\usepackage{csvsimple}
\usepackage{amsmath}
\usepackage{multicol}
%\usepackage{enumerate}
\usepackage{amssymb}
%\usepackage{graphicx}
\usepackage{newfloat}
%\usepackage{syntax}
\usepackage{listings}
\usepackage{iithtlc}
\usepackage{color}
\usepackage{tikz}
\usetikzlibrary{shapes,arrows}



%\usepackage{graphicx}
%\usepackage{amssymb}
%\usepackage{relsize}
%\usepackage[cmex10]{amsmath}
%\usepackage{mathtools}
%\usepackage{amsthm}
%\interdisplaylinepenalty=2500
%\savesymbol{iint}
%\usepackage{txfonts}
%\restoresymbol{TXF}{iint}
%\usepackage{wasysym}
\usepackage{amsthm}
\usepackage{mathrsfs}
\usepackage{txfonts}
\usepackage{stfloats}
\usepackage{cite}
\usepackage{cases}
\usepackage{mathtools}
\usepackage{caption}
\usepackage{enumerate}	
\usepackage{enumitem}
\usepackage{amsmath}
%\usepackage{xtab}
\usepackage{longtable}
\usepackage{multirow}
%\usepackage{algorithm}
%\usepackage{algpseudocode}
\usepackage{enumitem}
\usepackage{mathtools}
\usepackage{hyperref}
%\usepackage[framemethod=tikz]{mdframed}
\usepackage{listings}
    %\usepackage[latin1]{inputenc}                                 %%
    \usepackage{color}                                            %%
    \usepackage{array}                                            %%
    \usepackage{longtable}                                        %%
    \usepackage{calc}                                             %%
    \usepackage{multirow}                                         %%
    \usepackage{hhline}                                           %%
    \usepackage{ifthen}                                           %%
  %optionally (for landscape tables embedded in another document): %%
    \usepackage{lscape}     


\usepackage{url}
\def\UrlBreaks{\do\/\do-}


%\usepackage{stmaryrd}


%\usepackage{wasysym}
%\newcounter{MYtempeqncnt}
\DeclareMathOperator*{\Res}{Res}
%\renewcommand{\baselinestretch}{2}
\renewcommand\thesection{\arabic{section}}
\renewcommand\thesubsection{\thesection.\arabic{subsection}}
\renewcommand\thesubsubsection{\thesubsection.\arabic{subsubsection}}

\renewcommand\thesectiondis{\arabic{section}}
\renewcommand\thesubsectiondis{\thesectiondis.\arabic{subsection}}
\renewcommand\thesubsubsectiondis{\thesubsectiondis.\arabic{subsubsection}}

% correct bad hyphenation here
\hyphenation{op-tical net-works semi-conduc-tor}

%\lstset{
%language=C,
%frame=single, 
%breaklines=true
%}

%\lstset{
	%%basicstyle=\small\ttfamily\bfseries,
	%%numberstyle=\small\ttfamily,
	%language=Octave,
	%backgroundcolor=\color{white},
	%%frame=single,
	%%keywordstyle=\bfseries,
	%%breaklines=true,
	%%showstringspaces=false,
	%%xleftmargin=-10mm,
	%%aboveskip=-1mm,
	%%belowskip=0mm
%}

%\surroundwithmdframed[width=\columnwidth]{lstlisting}
\def\inputGnumericTable{}                                 %%
\lstset{
%language=C,
frame=single, 
breaklines=true,
columns=fullflexible
}
 

\begin{document}
%
\tikzstyle{block} = [rectangle, draw,
    text width=3em, text centered, minimum height=3em]
\tikzstyle{sum} = [draw, circle, node distance=3cm]
\tikzstyle{input} = [coordinate]
\tikzstyle{output} = [coordinate]
\tikzstyle{pinstyle} = [pin edge={to-,thin,black}]

\theoremstyle{definition}
\newtheorem{theorem}{Theorem}[section]
\newtheorem{problem}{Problem}
\newtheorem{proposition}{Proposition}[section]
\newtheorem{lemma}{Lemma}[section]
\newtheorem{corollary}[theorem]{Corollary}
\newtheorem{example}{Example}[section]
\newtheorem{definition}{Definition}[section]
%\newtheorem{algorithm}{Algorithm}[section]
%\newtheorem{cor}{Corollary}
\newcommand{\BEQA}{\begin{eqnarray}}
\newcommand{\EEQA}{\end{eqnarray}}
\newcommand{\define}{\stackrel{\triangle}{=}}

\bibliographystyle{IEEEtran}
%\bibliographystyle{ieeetr}

\providecommand{\nCr}[2]{\,^{#1}C_{#2}} % nCr
\providecommand{\nPr}[2]{\,^{#1}P_{#2}} % nPr
\providecommand{\mbf}{\mathbf}
\providecommand{\pr}[1]{\ensuremath{\Pr\left(#1\right)}}
\providecommand{\qfunc}[1]{\ensuremath{Q\left(#1\right)}}
\providecommand{\sbrak}[1]{\ensuremath{{}\left[#1\right]}}
\providecommand{\lsbrak}[1]{\ensuremath{{}\left[#1\right.}}
\providecommand{\rsbrak}[1]{\ensuremath{{}\left.#1\right]}}
\providecommand{\brak}[1]{\ensuremath{\left(#1\right)}}
\providecommand{\lbrak}[1]{\ensuremath{\left(#1\right.}}
\providecommand{\rbrak}[1]{\ensuremath{\left.#1\right)}}
\providecommand{\cbrak}[1]{\ensuremath{\left\{#1\right\}}}
\providecommand{\lcbrak}[1]{\ensuremath{\left\{#1\right.}}
\providecommand{\rcbrak}[1]{\ensuremath{\left.#1\right\}}}
\theoremstyle{remark}
\newtheorem{rem}{Remark}
\newcommand{\sgn}{\mathop{\mathrm{sgn}}}
\providecommand{\abs}[1]{\left\vert#1\right\vert}
\providecommand{\res}[1]{\Res\displaylimits_{#1}} 
\providecommand{\norm}[1]{\left\Vert#1\right\Vert}
\providecommand{\mtx}[1]{\mathbf{#1}}
\providecommand{\mean}[1]{E\left[ #1 \right]}
\providecommand{\fourier}{\overset{\mathcal{F}}{ \rightleftharpoons}}
%\providecommand{\hilbert}{\overset{\mathcal{H}}{ \rightleftharpoons}}
\providecommand{\system}{\overset{\mathcal{H}}{ \longleftrightarrow}}
	%\newcommand{\solution}[2]{\textbf{Solution:}{#1}}
\newcommand{\solution}{\noindent \textbf{Solution: }}
\newcommand{\myvec}[1]{\ensuremath{\begin{pmatrix}#1\end{pmatrix}}}
\providecommand{\dec}[2]{\ensuremath{\overset{#1}{\underset{#2}{\gtrless}}}}
\DeclarePairedDelimiter{\ceil}{\lceil}{\rceil}
%\numberwithin{equation}{subsection}
%\numberwithin{equation}{section}
%\numberwithin{problem}{subsection}
%\numberwithin{definition}{subsection}
\makeatletter
\@addtoreset{figure}{section}
\makeatother

\let\StandardTheFigure\thefigure
%\renewcommand{\thefigure}{\theproblem.\arabic{figure}}
\renewcommand{\thefigure}{\thesection}


%\numberwithin{figure}{subsection}

%\numberwithin{equation}{subsection}
%\numberwithin{equation}{section}
%\numberwithin{equation}{problem}
%\numberwithin{problem}{subsection}
\numberwithin{problem}{section}
%%\numberwithin{definition}{subsection}
%\makeatletter
%\@addtoreset{figure}{problem}
%\makeatother
\makeatletter
\@addtoreset{table}{section}
\makeatother

\let\StandardTheFigure\thefigure
\let\StandardTheTable\thetable
\let\vec\mathbf
%%\renewcommand{\thefigure}{\theproblem.\arabic{figure}}
%\renewcommand{\thefigure}{\theproblem}

%%\numberwithin{figure}{section}

%%\numberwithin{figure}{subsection}



\def\putbox#1#2#3{\makebox[0in][l]{\makebox[#1][l]{}\raisebox{\baselineskip}[0in][0in]{\raisebox{#2}[0in][0in]{#3}}}}
     \def\rightbox#1{\makebox[0in][r]{#1}}
     \def\centbox#1{\makebox[0in]{#1}}
     \def\topbox#1{\raisebox{-\baselineskip}[0in][0in]{#1}}
     \def\midbox#1{\raisebox{-0.5\baselineskip}[0in][0in]{#1}}

\vspace{3cm}

\title{ 
	\logo{
JEE Problems in Linear Algebra: 2D 
	}
}

%\author{ G V V Sharma$^{*}$% <-this % stops a space
%	\thanks{*The author is with the Department
%		of Electrical Engineering, Indian Institute of Technology, Hyderabad
%		502285 India e-mail:  gadepall@iith.ac.in. All content in this manual is released under GNU GPL.  Free and open source.}
	
%}	

\maketitle

%\tableofcontents

\bigskip

\renewcommand{\thefigure}{\theenumi}
\renewcommand{\thetable}{\theenumi}


\begin{abstract}
	A  collection of problems from JEE mains papers related to 2D coordinate geometry are 
available in this document.  These problems should be solved using linear algebra.
\end{abstract}
\begin{enumerate}[label=\arabic*.]
\item Draw the tangents through the point 
\begin{equation}
\label{eq:circle_3_p}
\vec{P} = \myvec{4\\7} 
\end{equation}
to the circle
\begin{equation}
\label{eq:circle_3}
C: \vec{x}^T\vec{x}  = 9.
\end{equation}
\item The sides of a rhombus $ABC$ are parallel to the lines
\begin{align}
\myvec{1 & -1}\vec{x} + 2 &=0
\\
\myvec{7 & -1}\vec{x} + 3 &=0.
\end{align}
If the diagonals of the rhombus intersect at
\begin{align}
\vec{P} = \myvec{1 \\ 2}
\end{align}
and the vertex $\vec{A}$ (different) from the origin is on the $y$-axis, then find the ordinate of $A$.
\item The points 
%
\begin{align}
\myvec{h \\ k},\myvec{1 \\ 2},\myvec{-3 \\ 4}
\end{align}
%
lie on the line $L_1$. Given that $L_2 \perp L_1$ and passes through 
\begin{align}
\myvec{h \\ k},\myvec{4 \\ 3},
\end{align}
find $\frac{k}{h}$.
%
\item Tangent and normal are drawn at 
\begin{equation}
\vec{P}= \myvec{16 \\ 16}
\end{equation}
on the parabola 
\begin{equation}
\vec{x}^T\myvec{0 & 0 \\ 0 & 1}\vec{x} + \myvec{16 & 0}\vec{x} =0
\end{equation}
%
which intersect the axis of the parabola at $\vec{A}$ and $\vec{B}$ respectively.  If $\vec{C}$ is the centre 
of the circle through the ponts $\vec{P}$, $\vec{A}$ and $\vec{B}$, find $\tan  CPB$.

%
%\item If the tangents drawn to the hyperbola 
%\begin{equation}
%\vec{x}^TV\vec{x} +1=0
%\end{equation}
%%
%where
%\begin{equation}
%V = \myvec{1 & 0 \\ 0 & -4}
%\end{equation}
%%
%intersect the coordinate axes at the distinct points $\vec{A}$ and $\vec{B}$, find the locus of the mid point 
%of $AB$.
\item $\beta$ is one of the angles between the normals to the ellipse
\begin{equation}
\vec{x}^TV\vec{x} =9
\end{equation}
%
where
\begin{equation}
V = \myvec{1 & 0 \\ 0 & 3}
\end{equation}
%
at the points
\begin{equation}
\myvec{3\cos\theta\\ \sqrt{3}\sin \theta},
\myvec{-3\sin\theta\\ \sqrt{3}\cos \theta}, \quad \theta \in \brak{0,\frac{\pi}{2}},
\end{equation}
then find $\frac{2\cot \beta}{\sin 2\theta}$.
\item Tangents drawn from the point $\myvec{-8\\0}$ to the parabola
\begin{equation}
\vec{x}^T\myvec{0 & 0 \\ 0 & 1}\vec{x} + \myvec{-8 & 0}\vec{x} 
 = 0
\end{equation}
%
touch the parabola at $\vec{P}$ and  $\vec{Q}$. If $\vec{F}$ is the focus of the parabola, then find the area 
of $\triangle PFQ$.
\item A normal to the hyperbola 
\begin{equation}
\vec{x}^T\myvec{4 & 0 \\ 0 & -9}\vec{x} 
 = 36
\end{equation}
%
meets the coordinate axes $x$ and $y$ at $\vec{A}$ and $\vec{B}$ respectively.  If the parallelogram $OABP$ is 
formed, find the locus of $\vec{P}$.
%\item Let $\vec{P}$ 
%be a point on the parabola
%\begin{equation}
%\vec{x}^T\myvec{1 & 0 \\ 0 & 0}\vec{x} +\myvec{0 & 4 }\vec{x} 
% = 0
%\end{equation}
%%%
%Given that the distance of $\vec{P}$ from the centre of the circle
%\begin{equation}
%\vec{x}^T\vec{x} +\myvec{6 \\ 0 }\vec{x} + 8 = 0
%\end{equation}
%%
%is minimum.  Find the equation of the tangent to the parabola at $\vec{P}$.
\item The length of the latus rectum of an ellipse is 4 ad the distance between a focus and its nearest
vertex on the major axis is $\frac{3}{2}$.  Find its eccentricity.
\item Find the eccentricity of an ellipse having centre at the origin, axes along the coordinate axes and 
passing through the points 
\begin{equation}
\vec{P} = \myvec{4\\-1}, 
\vec{Q} = \myvec{-2\\2}. 
\end{equation}
\item $\myvec{m & -1}\vec{x} + c =0$ is the normal at a point on the parabola
\begin{equation}
\vec{x}^T\myvec{0 & 0 \\ 0 & 1}\vec{x} -\myvec{8 & 0 }\vec{x} = 0
 = 0
\end{equation}
%
whose focal distance is 8.  Find $\abs{c}$.
\item The common tangents to the parabola
\begin{equation}
\vec{x}^T\myvec{1 & 0 \\ 0 & 0}\vec{x} -\myvec{0\\ 4 }\vec{x} = 0
 = 0
\end{equation}
%
intersect at the point $\vec{P}$.  Find the distance of $\vec{P}$ from the origin.
\item Find the equation of the normal to the hyperbola
\begin{equation}
\vec{x}^T\myvec{9 & 0 \\ 0 & -16}\vec{x}  = 144
\end{equation}
drawn at the point
\begin{equation}
\myvec{8\\ 3 \sqrt{3}}
\end{equation}
\item Find the eccentricity of the hyperbola whose length of the latus rectum is equal to 8 and the length of 
its conjugate axis is equal to half the distance between its foci. 
%\item $\vec{P}$ and $\vec{Q}$ are two distinct points on the parabola
%\begin{equation}
%\vec{x}^T\myvec{0 & 0 \\ 0 & 1}\vec{x} -\myvec{4 & 0 }\vec{x} 
% = 0
%\end{equation}
%with parameters $t$ and $t_1$ respectively.  If the normal at $\vec{P} $ passes through $\vec{Q}$, then find 
%the minimum value of $t_1^2$.
\item A hyperbola whose transverse axis is along the major axis of the conic
\begin{equation}
\vec{x}^TV\vec{x} =51
\end{equation}
%
where
\begin{equation}
V = \myvec{3 & 0 \\ 0 & 27}
\end{equation}
and has vertices at the foci of this conic.  If the eccentricity of the hyperbola is $\frac{3}{2}$, which of 
the following points doesnot lie on it?
\begin{enumerate}
\item $\myvec{ 0 \\ 2}$
\item $\myvec{\sqrt{5}  \\ 2\sqrt{2}}$
\item $\myvec{\sqrt{10}  \\ 2\sqrt{3}}$
\item $\myvec{5 \\ 2\sqrt{3}}$
\end{enumerate}
%\item A tangent at a point on the ellipse 
%\begin{equation}
%\vec{x}^TV\vec{x} =51
%\end{equation}
%%
%where
%\begin{equation}
%V = \myvec{3 & 0 \\ 0 & 27}
%\end{equation}
%%
%meets the coordinate axes at  $\vec{A}$  and  $\vec{B}$.  If   $\vec{O}$  be the origin, find the minimum area 
%of $\triangle OAB$.
\item The eccentricity of the standard hyperbola passing through the point 
\begin{equation}
\vec{P}= \myvec{4 \\ 6}
\end{equation}
is 2.  Find the equation of the tangent to the hyperbola at $\vec{P}$.
\item The tangent and normal lines at the point 
\begin{equation}
\vec{P}= \myvec{\sqrt{3} \\ 1}
\end{equation}
%
to the circle 
\begin{equation}
\norm{\vec{x}}= 2
\end{equation}
and the $x-$axis form a triangle. Find the area of this triangle.
\item An ellipse has centre at the origin and the difference of the lengths of its major and minor axis is 10. If one of its foci is at 
\begin{equation}
\vec{F}= \myvec{0 \\ 5\sqrt{3}},
\end{equation}
%
find the length of its latus rectum.
\item The tangent to the parabola 
\begin{equation}
\vec{x}^T\myvec{0 & 0 \\ 0 & 1}\vec{x} - \myvec{4 & 0}\vec{x} =0
\end{equation}
%
at the point where it intersects the circle 
\begin{equation}
\norm{\vec{x}}= \sqrt{5}
\end{equation}
%
in the first quadrant, passes through the point
\begin{enumerate}
\item $\myvec{-\frac{1}{3}\\\frac{4}{3}}$
\item $\myvec{\frac{1}{4}\\\frac{3}{4}}$
\item $\myvec{\frac{3}{4}\\\frac{7}{4}}$
\item $\myvec{-\frac{1}{4}\\\frac{1}{2}}$

\end{enumerate}
%\item Find the shortest distance between the line 
%\begin{equation}
%\myvec{1 & -1 }\vec{x}  =0
%\end{equation}
%%
%and the curve
%\begin{equation}
%\vec{x}^T\myvec{0 & 0 \\ 0 & 1}\vec{x} - \myvec{1 & 0}\vec{x} + 2 =0
%\end{equation}
%%
\item Find the sum of the squares of the lengths of the chords intercepted on the circle 
\begin{equation}
\norm{\vec{x}}= 4
\end{equation}
by the lines
\begin{equation}
\myvec{1 & 1 }\vec{x}  =n
\end{equation}
\item The tangents on the ellipse
\begin{equation}
\vec{x}^T\vec{V}\vec{x} =8
\end{equation}
%
where
\begin{equation}
\vec{V} = \myvec{2 & 0 \\ 0 & 1}
\end{equation}
at the points 
\begin{equation}
 \myvec{2 \\ 2}, \myvec{a \\ b}
\end{equation}
%
are perpendicular to each other. Find $a^2$.
\item Let 
\begin{equation}
\vec{O} = \myvec{ 0 \\ 0 },
\vec{A} = \myvec{ 0 \\ 1 }
\end{equation}
Find the locus of $\vec{P}$ such that the perimeter of $\triangle AOP = 4$.
\item Find the area of the smaller of the two circles that touch the parabola 
\begin{equation}
\vec{x}^T\myvec{0 & 0 \\ 0 & 1}\vec{x} - \myvec{4 & 0}\vec{x}  =0
\end{equation}
at the point 
\begin{equation}
\myvec{ 1 \\ 2 }
\end{equation}
and the $x$-axis.
\item The lines 
\begin{align}
\myvec{1 & a-1 }\vec{x}  &=1
\\
\myvec{2 & a^2 }\vec{x}  &=1
\end{align}
%
are perpendicular. Find the distance of their point of intersection from the origin.
%
\item The common tangent to the circles 
\begin{align}
\norm{\vec{x}}&= 2
\\
\norm{\vec{x}-\myvec{ 3 \\ 4 }}&= 7
\end{align}
passes through the point  
\begin{enumerate}
\item $\myvec{4\\-2}$
\item $\myvec{-6\\4}$
\item $\myvec{6\\-2}$
\item $\myvec{-4\\6}$

\end{enumerate}
\item The tangent to the parabola 
\begin{equation}
\vec{x}^T\myvec{0 & 0 \\ 0 & 1}\vec{x} - \myvec{1 & 0}\vec{x}  =0
\end{equation}
%
at a point 
\begin{equation}
\myvec{ \alpha \\ \beta }, \beta > 0
\end{equation}
%
is also a tangent to the ellipse 
\begin{equation}
\vec{x}^T\vec{V}\vec{x} =1
\end{equation}
%
where
\begin{equation}
\vec{V} = \myvec{1 & 0 \\ 0 & 2}.
\end{equation}
Find $\alpha$.
\item A rectangle is inscribed in a circle with a diameter lying along the line 
\begin{equation}
\myvec{1 & -3 }\vec{x}  =-7
\end{equation}
If two adjacent vertices of the rectangle are 
\begin{equation}
\myvec{ -8 \\ 5 },
\myvec{ 6\\ 5 }
\end{equation}
%
Find the area of the rectangle.
\item Find the direction vector of a line passing through 
\begin{equation}
\vec{P}= \myvec{ 2 \\ 3 },
\end{equation}
%
and intersecting the line 
\begin{equation}
\myvec{1 & 1 }\vec{x}  =7
\end{equation}
%
at a distance of 4 units from $\vec{P}$.
\item The line 
\begin{equation}
\myvec{-m & 1 }\vec{x}  =7\sqrt{3}
\end{equation}
%
is normal to the hyperbola
\begin{equation}
\vec{x}^T\vec{V}\vec{x} = 72
\end{equation}
%
where
\begin{equation}
V = \myvec{3 & 0 \\ 0 & -4}
\end{equation}
Find $m$.
\item One end of the focal chord of the parabola
\begin{equation}
\vec{x}^T\myvec{0 & 0 \\ 0 & 1}\vec{x} - \myvec{16 & 0}\vec{x}  =0
\end{equation}
%
is at 
\begin{equation}
\myvec{ 1 \\ 4 }.
\end{equation}
%
Find the length of this focal chord.
\item Let $\vec{P}$ be the point on the parabola 
\begin{equation}
\vec{x}^T\myvec{0 & 0 \\ 0 & 1}\vec{x} - \myvec{3 & 0}\vec{x}  =0
\end{equation}
%
such that $OP$ makes an angle $\frac{\pi}{6}$ with the $x-$axis, where $\vec{O}$ is the origin. A normal is drawn to the parabola intersecting the axis of the parabolat $\vec{Q}$.  If $\vec{S}$ is the focus of the parabola, then find $SQ$.
\item Let ellipse 
\begin{equation}
\vec{x}^T\vec{V}\vec{x} =16
\end{equation}
%
where
\begin{equation}
\vec{V} = \myvec{1 & 0 \\ 0 & 16}
\end{equation}
%
be inscribed in a rectangle whose sides are parallel to the coordinate axes.  If the rectangle is inscribed in another ellipse that passes through the point 
\begin{equation}
\myvec{ 16 \\ 0 },
\end{equation}
%
find the equation of the outer ellipse.
\item Let 
\begin{equation}
\vec{A}=\myvec{1 \\ 3 },
\vec{C}=\myvec{5 \\ 1 },
\end{equation}
be the opposite vertices of a rectangle.  The other two vertices
\begin{equation}
\vec{B}=\myvec{a \\ b },
\vec{D}=\myvec{c \\ d },
\end{equation}
%
lie on the line 
\begin{equation}
\myvec{-2 & 1 }\vec{x}  =k
\end{equation}
%
for some $k$.  Find the value of 
\begin{equation}
\brak{a+b}\brak{c+d}
\end{equation}
\item The abscissae of $\vec{A},\vec{B}$ are roots of
\begin{equation}
x^2+2x-4 = 0
\end{equation}
and their ordinates are roots of 
\begin{equation}
y^2+4y-16 = 0
\end{equation}
If $AB$ is a diameter of a circle, find its radius.
\item A line is drawn from a point 
\begin{equation}
\vec{P}=\myvec{-4 \\ 3 },
\end{equation}
%
to cut the circle 
\begin{equation}
\norm{\vec{x}}= 2
\end{equation}
%
at the points $\vec{A}$ and $\vec{B}$.  Find $PA.PB$.
\item The normal to the ellipse
\begin{equation}
\vec{x}^T\vec{V}\vec{x} =144
\end{equation}
%
where
\begin{equation}
\vec{V} = \myvec{ 4 & 0 \\ 0 & 9}
\end{equation}
%
at a point $\vec{P}$ has direction vector 
\begin{equation}
\vec{m}=\myvec{3 \\ 2 }.
\end{equation}
%
If this normal intersects the major axis of the ellipse at at a point $\vec{A}$, find $PA^2$.
\end{enumerate}
\end{document}
