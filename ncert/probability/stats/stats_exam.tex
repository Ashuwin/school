\item The marks obtained by 30 students of Class X of a certain school in a Mathematics paper consisting of 100 marks are presented in table below. Find the mean of the marks obtained by the students.\\
\begin{tabular}{|c|c|c|c|c|c|c||c|c|c|c|c|c|c|c|}
\hline
Marks obtained $(x_i)$&10&20&30&40&50&60&70&72&80&88&92&95\\
\hline
Number of students$(f_i)$ &1&1&3&4&3&2&4&4&1&1&2&3&1\\
\hline
\end{tabular}
\item The table below gives the percentage distribution of female teachers in the primary schools of rural areas of various states and union territories (U.T.) of India. Find the mean percentage of female teachers by all the three methods discussed
in this section.\\
\begin{tabular}{|c|c|c|c|c|c|c|c|}
\hline
Percentage of female teachers $(x_i)$&15-25&25-35&35-45&45-55&55-65&65-75&75-85\\
\hline
Number of states/U.T.$(f_i)$&6&11&7&4&4&2&1\\
\hline
\end{tabular}\\
Source : Seventh All India School Education Survey conducted by NCERT
\item The distribution below shows the number of wickets taken by bowlers in one-day cricket matches. Find the mean number of wickets by choosing a suitable
method. What does the mean signify?
\begin{tabular}{|c|c|c|c|c|c|c|c|c|}
\hline
Number of wickets &20-60&60-100&100-150&150-250&250-350&250-450\\
\hline
Number of bowlers &7&5&6&12&2&3\\
\hline
\end{tabular}\\*
{\Large \textbf{Mode of Grouped Data}}
\item The wickets taken by a bowler in 10 cricket matches are as follows:\\
2 6 4 5 0 2 1 3 2 3\\
Find the mode of the data.\\
\item A survey conducted on 20 households in a locality by a group of students
resulted in the following frequency table for the number of family members in a
household:
\begin{tabular}{|c|c|c|c|c|c|c|}
\hline
Family size &1-3&3-5&5-7&7-9&9-11&\\
\hline
Number of families &7&8&2&2&1\\
\hline
\end{tabular}\\\\
Find the mode of this data
\item The marks distribution of 30 students in a mathematics examination are
given in Table 14.3 of Example 1. Find the mode of this data. Also compare and
interpret the mode and the mean.\\
{\Large \textbf{Median of Grouped Data}}
\item A survey regarding the heights (in cm) of 51 girls of Class X of a school
was conducted and the following data was obtained:
\begin{tabular}{|c|c|c|c|c|c|c|c|c|}
\hline
height (in cm) &lessthan 140&lessthan 145&lessthan 150&lessthan 155&lessthan160&lessthan 165\\
\hline
Number of girls &41&29&40&46&51\\
\hline
\end{tabular}\\\\
Find the median height.
\item The median of the following data is 525. Find the values of x and y, if the
total frequency is 100.
\begin{tabular}{|c|c|c|c|c|c|c|c|c|c|c|}
\hline
Class interval &0-100&100-200&200-300&300-400&400-500&500-600&600-700&700-800&800-900&900-1000\\
\hline
Frequency &2&5&x&12&517&20&y&9&7&4\\
\hline
\end{tabular}
\item The annual profits earned by 30 shops of a shopping complex in a locality give rise to  the following distribution:
%\begin{tabular}{|c|c|c|c|c|c|c|c|}
%\hline
%Profit (Rs in lakhs) & more than or equal to 5 & morethan or equal to 10 & morethan or equal to 5&morethan orequal to 15 & morethan or equal to 20 & morethan or equal to 25 & morethan or equal to 30 & morethan or equal to 35\\
%\hline
%Number of shops (frequency)&30&28&16&14&10&7&3\\
%\hline
%\end{tabular}
        
