\item To each vertex of a regular pentagon an integer is assigned in such a way that the sum of all five numbers is positive. If three consecutive vertices are assigned the numbers x, y, z respectively and $y < 0$ then the following operation is allowed: the numbers x, y, z are replaced by x + y, -y, z + y respectively. Such an operation is performed repeatedly as long as at least one of the five numbers is negative. Determine whether this procedure necessarily comes to and end after a finite number of steps.

\item Find all functions f, defined on the non-negative real numbers and taking non-negative real values, such that:
\begin{enumerate}
\item f(xf(y))f(y) = f(x + y) for all x, y $\geq$ 0,
\item f(2) = 0,
\item f(x) $\neq$ 0 for 0 $\leq x < 2$.
\end{enumerate}

\item One is given a finite set of points in the plane, each point having integer coordinates. Is it always possible to color some of the points in the set red and the remaining points white in such a way that for any straight line L parallel to either one of the coordinate axes the difference (in absolute value) between the numbers of white point and red points on L is not greater than 1?






