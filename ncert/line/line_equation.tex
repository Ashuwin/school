\renewcommand{\theequation}{\theenumi}
\begin{enumerate}[label=\arabic*.,ref=\thesubsection.\theenumi]
\numberwithin{equation}{enumi}
\item Determine the ratio in which the line 
\begin{align}
\myvec{2 & 1} - 4 = 0
\end{align}
%
divides the line segment joining the points $\vec{A}=\myvec{2\\-2}, \vec{B}=\myvec{3\\7}$
\item Find the equation of a line through the point \myvec{5\\2\\-4} and parallel to the vector \myvec{3\\2\\-8}.
\item Find the equation of a line passing through the points \myvec{-1\\0\\2} and \myvec{3\\4\\6}.
\item If
\begin{align}
\frac{x+3}{2} = \frac{y-5}{4} = \frac{z+6}{2}, 
\end{align}
%
find the equation of the line.
\item Find the angle between the pair of lines given by 
\begin{align}
\bm{x} &= \myvec{3\\2\\-4} + \lambda_1\myvec{1 \\ 2 \\2}
\\
\bm{x} &= \myvec{5\\-2\\0} + \lambda_2\myvec{3 \\ 2 \\6}
\end{align}
\item Find the angle between the pair of lines
\begin{align}
\frac{x+3}{3} = \frac{y-1}{5} = \frac{z+3}{4}, 
\frac{x+1}{1} = \frac{y-4}{1} = \frac{z-5}{2} 
\end{align}
\item Find the shortest distance between the lines 
\begin{align}
L_1: \quad \bm{x} &= \myvec{1\\1\\0} + \lambda_1\myvec{2 \\ -1 \\1}
\\
L_2: \quad \bm{x} &= \myvec{2\\1\\-1} + \lambda_2\myvec{3 \\ -5 \\2}
\end{align}
\item Find the 
distance between the lines 
\begin{align}
L_1: \quad \bm{x} &= \myvec{1\\2\\-4} + \lambda_1\myvec{2 \\ 3 \\6}
\\
L_2: \quad \bm{x} &= \myvec{3\\3\\-5} + \lambda_2\myvec{2 \\ 3 \\6}
\end{align}
%
\item Find the equation of a line which passes through the point \myvec{1\\2\\3} and is parallel to the vector \myvec{3\\2\\-2}.
\item Find the equaion off the line that passes through \myvec{2\\-1\\4} and is in the direction \myvec{1\\2\\-1}.
\item Find the equation of the line which passes through  the point \myvec{-2\\4\\-5} and parallel to the line given by 
\begin{align}
\frac{x+3}{3} = \frac{y-4}{5} = \frac{z+8}{6}. 
\end{align}
\item Find the equation of the line given by 
\begin{align}
\frac{x-5}{3} = \frac{y+4}{7} = \frac{z-6}{2}. 
\end{align}
\item Find the equation of the line passing through the origin and the point \myvec{5\\-2\\3}.
\item Find the equation of the line passing through the points \myvec{3\\-2\\-5}, \myvec{3\\-2\\6}.
\item Find the angle between the following pair of lines:
\begin{enumerate}
\item
\begin{align}
L_1: \quad \bm{x} &= \myvec{2\\-5\\1} + \lambda_1\myvec{3 \\ 2 \\6}
\\
L_2: \quad \bm{x} &= \myvec{7\\-6\\0} + \lambda_2\myvec{1 \\ 2 \\2}
\end{align}
\item
\begin{align}
L_1: \quad \bm{x} &= \myvec{3\\1\\-2} + \lambda_1\myvec{1 \\ -1 \\-2}
\\
L_2: \quad \bm{x} &= \myvec{2\\-1\\-56} + \lambda_2\myvec{3 \\ -5 \\-4}
\end{align}
\end{enumerate}
\item Find the angle between the following pair of lines
\begin{enumerate}
\item 
\begin{align}
\frac{x-2}{2} = \frac{y-1}{5} = \frac{z+3}{-3}, 
\frac{x+2}{-1} = \frac{y-4}{8} = \frac{z-5}{4} 
\end{align}
\item 
\begin{align}
\frac{x}{2} = \frac{y}{2} = \frac{z}{1}, 
\frac{x-5}{4} = \frac{y-2}{1} = \frac{z-3}{8} 
\end{align}
\end{enumerate}
\item Find the values of $p$ so that the lines 
\begin{align}
\frac{1-x}{3} = \frac{7y-14}{2p} = \frac{z-3}{2}, 
\frac{7-7x}{3p} = \frac{y-5}{1} = \frac{6-z}{5} 
\end{align}
are at right angles.
\item Show that the lines 
\begin{align}
\frac{x-5}{7} = \frac{y+2}{-5} = \frac{z}{1}, 
\frac{x}{1} = \frac{y}{2} = \frac{z}{3} 
\end{align}
%
are perpendicular to each other.
\item Find the shortest distance between the lines 
\begin{align}
L_1: \quad \bm{x} &= \myvec{1\\2\\1} + \lambda_1\myvec{1 \\ -1 \\1}
\\
L_2: \quad \bm{x} &= \myvec{2\\-1\\-1} + \lambda_2\myvec{2 \\ 1 \\2}
\end{align}
\item Find the shortest distance between the lines 
\begin{align}
\frac{x+1}{7} = \frac{y+1}{-6} = \frac{z+1}{1}, 
\frac{x-3}{1} = \frac{y-5}{-2} = \frac{z-7}{1} 
\end{align}
%
\item Find the shortest distance between the lines 
\begin{align}
L_1: \quad \bm{x} &= \myvec{1\\2\\3} + \lambda_1\myvec{1 \\ -3 \\2}
\\
L_2: \quad \bm{x} &= \myvec{4\\5\\6} + \lambda_2\myvec{2 \\ 3 \\1}
\end{align}
%
\item Find the shortest distance between the lines 
\begin{align}
L_1: \quad \bm{x} &= \myvec{1-t\\t-2\\3-2t} 
\\
L_2: \quad \bm{x} &= \myvec{s+1\\2s-1\\-2s-1}
\end{align}

\end{enumerate}
