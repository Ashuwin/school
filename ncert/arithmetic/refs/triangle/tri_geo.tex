\renewcommand{\theequation}{\theenumi}
\begin{enumerate}[label=\arabic*.,ref=\thesubsection.\theenumi]
\numberwithin{equation}{enumi}
%
\item Draw the graphs of the equations 
\begin{align}
\myvec{1 & -1}\vec{x} + 1 &= 0 
\\
\myvec{ 3 & 2} - 12 &= 0
\end{align}
%
 Determine the coordinates of the vertices of the triangle formed by these lines and the x-axis, and shade the triangular region.
%
\item In a $\triangle ABC, \angle C = 3 \angle B = 2 (\angle A + \angle B)$. Find the three angles. 
\item Draw the graphs of the equations $5x – y = 5$ and $3x – y = 3$. Determine the co-ordinates of the vertices of the triangle formed by these lines and the y axis.

\item The vertices of $\triangle PQR$ are 

$
\vec{P} = \myvec{2 \\1},
\vec{Q} = \myvec{-2\\3},
\vec{R} = \myvec{4\\5}.
$
Find the equation of the median through the vertex $\vec{R}$.
\item In the $\triangle ABC$ with vertices
$
\vec{A}=\myvec{2\\3}, 
\vec{B}=\myvec{4\\-1},
 \vec{C}=\myvec{1\\2}
$,
find the equation and length of the altitude from the vertex $\vec{A}$.
\item Find the area of the triangle whose vertices are
\begin{enumerate}
\item \myvec{2\\3}, \myvec{-1\\0},  \myvec{2\\-4}
\item  \myvec{-5\\-1},  \myvec{3\\-5},  \myvec{5\\2}
\end{enumerate}
\item Find the area of the triangle formed by joining the mid points o the sides of a triangle whose vertices are  \myvec{0\\-1},  \myvec{2\\1},  \myvec{0\\3}.
\item Verify that the median of $\triangle ABC$ with vertices $\vec{A}=\myvec{4\\-6},  \vec{B}=\myvec{3\\-2}$ and  $\vec{C} =  \myvec{5\\2}$ divides it into two triangles of equal areas.
\item The vertices of $\triangle ABC$ are $\vec{A}=\myvec{4\\6},  \vec{B}=\myvec{1\\5}$ and  $\vec{C} =  \myvec{7\\2}$.  A line is drawn to intersect sides $AB$ and $AC$ at $D$ and $E$ respectively, such that
\begin{align}
\frac{AD}{AB}=\frac{AE}{AC}= \frac{1}{4}
\end{align}
%
Find 
\begin{align}
\frac{\text{area of }\triangle ADE}{\text{area of }\triangle ABC}.
\end{align}
\item Let $\vec{A}=\myvec{4\\2},  \vec{B}=\myvec{6\\5}$ and  $\vec{C} =  \myvec{1\\4}$ be the vertices of $\triangle ABC$.
\begin{enumerate}
\item The median from $\vec{A}$ meets $BC$ at $\vec{D}$.  Find the coordinates of the point $\vec{D}$.
\item Find the coordinates of the point $\vec{P}$ on $AD$ such that $AP:PD = 2:1$.
\item Find the coordinates of the points $\vec{Q}$ and $\vec{R}$ on medians $BE$ and $CF$ respectively such that $BQ:QE = 2:1$ and $CR:RF = 2:1$.
\end{enumerate}
\item In $\triangle ABC$, Show that the centroid 
\begin{align}
\vec{O} = \frac{\vec{A}+\vec{B}+\vec{C}}{3}
\end{align}
\item Show that the points 
\begin{align}
\vec{A} = \myvec{2\\-1 \\1},
\vec{B} = \myvec{1\\-3 \\-5},
\vec{C} = \myvec{3\\ -4\\-4}
\end{align}
%
are the vertices of a right angled triangle.
\item In $\triangle ABC$, 
$
\vec{A} = \myvec{1\\2 \\3},
\vec{B} = \myvec{-1\\0 \\0},
\vec{C} = \myvec{0\\ 1\\2}.
$
Find $\angle B$.
\item Show that the vectors 
$
\myvec{2\\-1 \\1},
\myvec{1\\-3 \\-5},
\myvec{3\\ -4\\-4}
$
form the vertices of a right angled triangle.
\item Find the area of a triangle having the points 
$
\vec{A} = \myvec{1\\1 \\1},
\vec{B} = \myvec{1\\2 \\3}, \text{ and }
\vec{C} = \myvec{2\\ 3\\1}
$
as its vertices.
\item Find the area of a triangle with vertices
$
\vec{A} = \myvec{1\\1 \\2},
\vec{B} = \myvec{2\\3 \\5}, \text{ and }
\vec{C} = \myvec{1\\ 5\\5}
$
\item Find the direction vectors of the sides of a triangle with vertices
$
\vec{A} = \myvec{3\\5 \\-4},
\vec{B} = \myvec{-1\\1 \\2}, \text{ and }
\vec{C} = \myvec{-5\\ -5\\-2}
$
\item Without using the Pythagoras theorem, show that the points \myvec{4\\ 4}, \myvec{3\\ 5} and \myvec{–1\\ –1} are the vertices of a right angled triangle.
\item Check whether 
\begin{align}
\myvec{5\\-2}, \myvec{6\\4}, \myvec{7\\-2}
\end{align}
are the vertices of an isosceles triangle.
%

\end{enumerate}
%

