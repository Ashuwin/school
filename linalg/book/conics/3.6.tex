\renewcommand{\theequation}{\theenumi}
\begin{enumerate}[label=\arabic*.,ref=\thesubsection.\theenumi]
\numberwithin{equation}{enumi}
\item What does the equation 
\begin{align}
\vec{x}^T\myvec{1 & 0\\0 & -1}\vec{x}-\myvec{4 & 6}\vec{x}-6=0
\end{align}
become when the origin is moved to the point $\myvec{2\\-3}$?
\item To what point must the origin be moved in order that the equation
\begin{align}
\vec{x}^T\myvec{2 & -\frac{3}{2}\\ -\frac{3}{2} & 4}\vec{x}+\myvec{10 & -19}\vec{x}+23=0
\end{align}
may become
\begin{align}
\vec{x}^T\myvec{2 & -\frac{3}{2}\\ -\frac{3}{2} & 4}\vec{x} = 1
\end{align}
\item Show that the equation
\begin{align}
\vec{x}^T\vec{x}= a^2
\end{align}
remains unaltered by any rotation of the axes.
\item What does the equation
\begin{align}
\vec{x}^T\myvec{1 & \sqrt{3}\\ \sqrt{3} & -1}\vec{x} = 2a^2
\end{align}
become when the axes are turned through $30\degree$?
\item What does the equation
\begin{align}
\vec{x}^T\myvec{1 & -1\\-1 & 1}\vec{x}-4\sqrt{2}a\myvec{1 & 1}\vec{x}=0
\end{align}
become when the axes are turned through $45\degree$?
\item To what point must the origin be moved in order that the equation
\begin{align}
\vec{x}^T\myvec{1 & 2\\2 & -2}\vec{x}+\myvec{10 & -4}\vec{x}=0
\end{align}
may become
\begin{align}
\vec{x}^T\myvec{1 & 2\\2 & -2}\vec{x}= 1
\end{align}
and through what angle must the axes be turned in order to obtain
\begin{align}
\vec{x}^T\myvec{p & 0\\0 & q}\vec{x}= 1
\end{align}
\item Through what angle must the axes be turned to reduce the equation
\begin{align}
\vec{x}^T\myvec{1 & -1\\-1 & -1}\vec{x}=1
\end{align}
to the form
\begin{align}
\vec{x}^T\myvec{0 & \frac{1}{2}\\ \frac{1}{2} & 0}\vec{x} = c
\end{align}
where $c$ is a constant.
\item Show that, by changing the origin, the equation
\begin{align}
2\vec{x}^T\vec{x}+\myvec{7 & 5}\vec{x} - 13 = 0
\end{align}
can be transformed to 
\begin{align}
8\vec{x}^T\vec{x} = 89
\end{align}
\item Show that, by rotating the axes, the equation
\begin{align}
\vec{x}^T\myvec{3 & \frac{7}{2}\\ \frac{7}{2} & -3}\vec{x}= 1
\end{align}
can be reduced to 
\begin{align}
\sqrt{85}\vec{x}^T\myvec{1 & 0\\ 0 & -1}\vec{x}= 2
\end{align}
\item Show that, by rotating the axes, the equation
\begin{align}
\vec{x}^T\myvec{41 & 12\\ 12 & 34}\vec{x}= 75
\end{align}
can be reduced to 
\begin{align}
\vec{x}^T\myvec{2 & 0\\ 0 & 1}\vec{x}= 3
\end{align}
\item Show that, by a change of origin and the directions of the coordinate axes, the equation
\begin{align}
\vec{x}^T\myvec{5 & 1\\ 1 & 5}\vec{x}-\myvec{14 & 22}\vec{x}+27= 0
\end{align}
can be transformed to
\begin{align}
\vec{x}^T\myvec{3 & 0\\ 0 & 2}\vec{x}= 1
\end{align}
or
\begin{align}
\vec{x}^T\myvec{2 & 0\\ 0 & 3}\vec{x}= 1
\end{align}
\end{enumerate}

