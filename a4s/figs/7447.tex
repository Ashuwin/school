\documentclass{standalone}
\usepackage{tikz}
\usepackage{amsmath,amssymb}
\makeatletter
\newsavebox\myboxA
\newsavebox\myboxB
\newlength\mylenA
\newcommand*\xoverline[2][0.75]{%
    \sbox{\myboxA}{$\m@th#2$}%
    \setbox\myboxB\null% Phantom box
    \ht\myboxB=\ht\myboxA%
    \dp\myboxB=\dp\myboxA%
    \wd\myboxB=#1\wd\myboxA% Scale phantom
    \sbox\myboxB{$\m@th\overline{\copy\myboxB}$}%  Overlined phantom
    \setlength\mylenA{\the\wd\myboxA}%   calc width diff
    \addtolength\mylenA{-\the\wd\myboxB}%
    \ifdim\wd\myboxB<\wd\myboxA%
       \rlap{\hskip 0.5\mylenA\usebox\myboxB}{\usebox\myboxA}%
    \else
        \hskip -0.5\mylenA\rlap{\usebox\myboxA}{\hskip 0.5\mylenA\usebox\myboxB}%
    \fi}
\makeatother


\begin{document}
\begin{tikzpicture}[scale=1,
     pin/.style={draw,rectangle,minimum width=2em,font=\small}
     ]
   % Main trick: loop over the label numbers and then adjust their position
   % in the tikzpicture using evaluate to calculate \y=y-coordinate of pin

   \foreach \i/\desc [evaluate=\i  as \x using (\i+4.8 )]
      in {1/\tiny{B},
          2/\tiny{C},
          3/\tiny{\xoverline{LT}},
          4/\tiny{\xoverline{BI/RBO}},
          5/\tiny{\xoverline{RBI}},
          6/\tiny{D},
          7/\tiny{A},
          8/\tiny{GND}}
   {
     \draw node[pin,anchor=east,rotate=360] at (\x,-0.233){\small$\i$};
     \node[align=right,anchor=east,rotate=360] at (\x,-.8){\desc};
   }
  
   \foreach \i/\desc [evaluate=\i as \x using (21.1-\i)]
      in { 9/\xoverline{e},
          10/\xoverline{d},
          11/\xoverline{c},
          12/\xoverline{b},
          13/\xoverline{a},
          14/\xoverline{g},
          15/\xoverline{f},
          16/{V$_{\text{CC}}$}}
   {
     \draw node[pin,anchor=west,rotate=360] at (\x,3.22){\small$\i$};
     \node[align=right,anchor=west,rotate=360] at (\x,3.8){\desc};
   }
   \draw
      (5,0.8)--(5,3)--(5,3)--(13,3)--(13,3)
             --(13,0.8)--(13,0)--(5,0)--cycle;
    \begin{scope}         
    \clip (18,.5) rectangle (5,2);         
   \draw(5.2,1.5)circle[radius=0.4];
   \end{scope}
   \begin{scope}
   \clip (0,-1.5) rectangle (1.5,1.5);
    \draw (2,) circle(6.5);
\end{scope}

   \node at (9,1.5){\textbf{\LARGE{7447}}};
 \end{tikzpicture}
\end{document}