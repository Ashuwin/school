The point at which the reflected line touches the surface is the solution of the equation
%
\begin{equation}
A = 
\begin{pmatrix}
2 &1 \\
 7 & -1
 \end{pmatrix}
  \begin{pmatrix}
x \\
  y
 \end{pmatrix}
=
 \begin{pmatrix}
1 \\
  -1
 \end{pmatrix}
\end{equation}
%  

%
\begin{equation}
\end{equation}
%
yielding the point $\brak{0,1}$.
Angle   between   the   given   line   is   given   
\begin{equation}
\theta =\tan ^{ -1 }{ \frac { (m_1-m_2) }{ (1+m_1*m_2) }  } 
\end{equation} 
 where   $m_1, m_2$   are   slopes   of   surface   and   reflected   line   respectively.
\begin{equation}\theta =\tan ^{ -1 }{ \frac { (7-(-2)) }{ (1+7*(-2))) }  } 
 \end{equation} 
 \begin{equation}
 \theta =\tan ^{ -1 }{ \frac { -9 }{ 13 }  } 
  \end{equation}  
  The     slope     of     the     incident          line     can     be     found     by     reversing     the     direction     of     the     angle     along     the     surface. Letting the angle that the incident line makes along the $x$-axis to be $\phi$, 
  \begin{equation}
   \phi =\tan ^{ -1 }{ \frac { (m_1-\tan\theta ) }{ (1+m_1*\tan\theta ) }  } 
    \end{equation}
    \begin{equation}
        m=\frac { (7-\frac { 9 }{ 13 } ) }{ (1+7*\frac { 9 }{ 13 } ) } 
        \end{equation}
         \begin{equation}
         m=\frac { (91-9) }{ (63+13) }  \end{equation}
          \begin{equation} 
          m=\frac{41}{38 }
          \end{equation}
           Since     m     is     the     slope     and     1     is     the     intercept     and     thus     in     slope     form     equation     of     line     is     $y=mx+1$. Thus     the     equation     of     the     incident     line     is                                                                              
\begin{equation}
 38y=41x+38
\end{equation}           


