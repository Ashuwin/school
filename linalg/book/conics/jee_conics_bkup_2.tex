\begin{enumerate}[label=\arabic*]
\numberwithin{equation}{enumi}
\item The point of intersection of the tangents at the ends of the latus rectum of the Parabola \begin{align} \vec x^T \myvec{1 & 0 \\ 0 & 0} \vec x=(4,0)\vec x \end{align} is .....
    
\item An ellipse has accentricity $\frac{1}{2}$ and one focus at the point P($\frac{1}{2}$,1) .Its one directrix is the common tangent,nearer to the point P,to the circle \begin{align}\vec x^T \vec x =1\end{align} and the hyperbola \begin{align}\vec x^T \myvec{1&0 \\0 &-1}\vec x=1\end{align}.The equation of the ellipse ,in the standard form is .......
\end{enumerate}

{\Large \textbf{MCQs with One Correct Answer }}
\begin{enumerate}

\item The equation \begin{align}\vec x^T \myvec  {\frac{1}{1-r}&0 \\0 &  \frac{-1}{1+r}} \vec x=1\end{align},r$>$1 represents

\choice (a) an ellipse

\choice (b) a hyperbola

\choice (c) a circle

\choice (d) none of these\\

\item Each of the four inequalities given below defines a region in the xy plane .One of these four regions does not have the following property .For any two points ($x_1,y_1$) and ($x_2,y_2$) in the region, the points
($\frac{x_1+x_2}{2}$,$\frac{y_1+y_2}{2}$) is also in the region.The inequality defining this region is\\


    \choice (a) $\vec x^T \myvec{1&0 \\0 &2}\vec x \leq 1$
    
    \choice (b) Max \{${\mid x \mid,\mid y \mid }\}$ \leq1$
    
    \choice (c) $\vec x^T \myvec{1&0 \\0 &-1}\vec x \leq1$
    
    \choice (d) $\vec x^T \myvec{0 &0 \\ 0& 1 } + (-1, 0)\vec x\leq0$
    
    \item The equation \begin{align}\vec x^T \myvec {2&0 \\ 0& 3} \vec {x} + (-8, -18)\vec x+35=k\end {align} represents\\ 
    
    \choice (a) no locus if $k>0$
    
    \choice (b) an ellipse if$ k< 0$
    
    \choice (c) a point if k=0
    
    \choice (d) a hyperbola if $k>0$\\
    
    \item Let E be the ellipse \begin{align}\vec x^T \myvec {\frac{1}{9}&0 \\ 0&\frac{1}{4}}\vec x=1\end{align} and C be the circle \begin{align}\vec x^T \vec x=9\end{align}.let P and Q be the points (1,2) and (2,1) respectively. Then\\
    
    \choice (a) Q lies inside C but outside E
    
    \choice (b) Q lies outside both C and E 
    
    \choice (c) P lies inside both C and E 
    
    \choice (d) P lies inside C but outside E\\
    
    \item Consider a circle with its center lying on the focus of the parabola \begin{align}\vec x^T \myvec {0&0 \\ 0&1}\vec x=(2p,0)\vec x\end{align}  such that it touches the directrix of the parabola.Then a point of inter section of the circle and parabola is\\
    
    \choice (a) ($\frac{p}{2}$,p) or ($\frac{p}{2}$,-p)
    
    \choice (b) ($\frac{p}{2}$,$\frac{p}{2}$)
    
    \choice (c) ($\frac{-p}{2},p$)
    
    \choice (d) ($\frac{-p}{2}$,$\frac{-p}{2}$)\\
    
    \item The radius of the circle passing through the foci of the ellipse \begin{align}\vec x^T \myvec {\frac{1}{16}&0 \\ 0&\frac{1}{9}}\vec x=1\end{align}, and having its centre at (0,3) is \\
    
    \choice (a) 4
    
    \choice (b) 3
    
    \choice (c) $\sqrt{\frac{1}{2}}$
    
    \choice (d) $\frac{7}{2}$\\
    
    \item Let P(a sec$\theta$,b tan$\theta$) and Q(a sec$\phi$,b tan$\phi$),where $\theta$+$\phi$=$\frac{\pi}{2}$,be two points on the hyperbola \begin{align}\vec x^T \myvec{ \frac{1}{a^2}&0 \\ 0&\frac{-1}{b^2}}\vec x=1\end{align}.If (h,k) is the points of intersection of the normals at P and Q ,then k is equal to\\
    
    \choice (a) $\frac{a^2+b^2}{a}$
    
     \choice (b) -($\frac{a^2+b^2}{a}$)
    
    \choice (c) $\frac{a^2+b^2}{b}$
    
    \choice (d) $-(\frac{a^2+b^2}{b}$)\\
    
    \item If \begin{align}(1,0)\vec x=9\end{align} is the chord of contact of the hyperbola \begin{align}\vec x^T \myvec {1&0 \\ 0&-1}\vec x=9\end{align} then the equation of the corresponding pair of tangents is\\
    
    \choice (a) $\vec x^T \myvec{9&0 \\ 0&-8}\vec x+(18,0)\vec x-9=0$
    
    \choice (b) $\vec x^T \myvec {9&0 \\ 0&-8}\vec x+(-18,0)\vec x+9=0$

    \choice (c) $\vec x^T \myvec{9&0 \\ 0&-8}\vec x+(-18,0)\vec x-9=0$

   \choice (d) $\vec x^T \myvec{ 9&0 \\ 0&-8}\vec x+(18,0)\vec x+9=0$\\

\item The curve describes para metrically by \begin{align}(1,0)\vec x=t^2+t+1\end{align},\begin{align}(0,1)\vec x=t^2-t+1\end{align} represents\\

\choice (a) a pair of straight lines 

\choice (b) an ellipse

\choice (c) a parabola

\choice (d) a hyperbola\\

\item If \begin{align}(1,1)\vec x=k\end{align} is normal to \begin{align}\vec x^T\myvec{0 & 0 \\0 & 1} \vec x=(12,0)\vec x \end{align},then k is \\
    
    \choice (a) 3
    
    \choice (b) 9
    
    \choice (c) -9
    
    \choice (d) -3\\
    
    \item If the line \begin{align}(1,0)\vec x-1=0\end{align}is the directrix of the parabola \begin{align}\vec x^T \myvec{0 & 0 \\0 & 1} \vec x-(k,0)\vec x +8=0\end{align},then one of the values of k is \\
    
    \choice (a) $\frac{1}{8}$
    
    \choice (b) 8
    
    \choice (c) 4
    
    \choice (d) $\frac{1}{4}$\\
    
    \item The equation of the common tangent touches the circle \begin{align}x^T\myvec{1 & 0 \\0 & 1}\vec x-(6,0)\vec x =0\end{align} and the parabola \begin{align}\vec x^T \myvec{0 & 0 \\0 & 1}\vec x=(4,0)\vec x\end{align} above the x-axis is \\
    
    \choice (a) $(0,\sqrt{3})\vec x=(3,0)\vec x+1$
    
    \choice (b) $(0,\sqrt{3})\vec x=(-1,0)\vec x-3$
    
    \choice (c) $(0,\sqrt{3})\vec x=(1,0)\vec x+3$
    
    \choice (d) $(0,\sqrt{3})\vec x=(-3,0)\vec -1$\\
    
    \item The equation of the directrix of the parabola \begin{align}\vec x^T \myvec {0 & 0 \\0 & 1}\vec x+(4,4)\vec x +2=0\end{align} is\\
    
    \choice (a) $(1,0)\vec x=-1$
    
    \choice (b) $(1,0)\vec x=1$
    
    \choice (c) $(1,0)\vec x=-\frac{3}{2}$
    
    \choice (d) $(1,0)\vec x=\frac{3}{2}$\\
    
    \item If $a>2b>0$ then the positive value of m for which \begin{align}(0,1)\vec x=(m,0)\vec x-b\sqrt{1+m^2}\end{align} is the common tangent to \begin{align}\vec x^T\myvec{1 & 0 \\0 & 1}\vec x=b^2\end {align}and \begin{align}\vec x^T\myvec {1 & 0 \\0 &1}\vec x+(-2a,0)\vec x=a^2-b^2\end{align} is \\
    
    \choice (a) $\frac{2b}{\sqrt{a^2-4b^2}}$
    
    \choice (b) $\frac{\sqrt{a^2-4b^2}}{2b}$
    
    \choice (c) $\frac{2b}{a-2b}$
    
    \choice (d) $\frac{b}{a-2b}$\\
    
    \item the locus of the mid-point of the line segment joining the focus to a moving point on the parabola  \begin{align} \vec x^T\myvec{0 & 0 \\0 & 1}\vec x=(4a,0)\vec x\end{align} is another parabola with directrix \\
    
    \choice (a) $(1,0)\vec x=-a$
    
    \choice (b) $(1,0)\vec x=\frac{-a}{2}$
    
    \choice (c) $(1,0)\vec x=0$
    
    \choice (d) $(1,0)\vec x=\frac{a}{2}$\\
    
    \item The equation of the common tangent to the line curves \begin{align} x^T \myvec {0 & 0 \\0 & 1}\vec x=(8,0)\vec x\end {align} and \begin{align} x^T \myvec
    {0 & 0 \\1& 0}\vec x=-1\end{align}is\\
    
    \choice (a) $(0,3)\vec x=(9,0)\vec x+2$
    
    \choice (b) $(0,1)\vec x=(2,0)\vec x+1$
    
    \choice (c) $(0,2)\vec x=(1,0)\vec x+8$
    
    \choice (d) $(0,1)\vec x=(1,0)\vec x+2$\\
    
    \item The area of the quadrilateral formed by the tangents at the end points of the latus rectum to the ellipse \begin{align} x^T \myvec{\frac{1}{9} & 0 \\0 & \frac{1}{5}} \vec x=1\end{align},is\\
    
    \choice (a) $\frac{27}{4}$sq.units
    
    \choice (b) $9$sq.units
    
    \choice (c) $\frac{27}{2}$sq.units
    
    \choice (b) $27$sq.units\\
    
    \item The focal chord to \begin{align} x^T \myvec
    {0 & 0 \\0 & 1}\vec x=(16,0)\vec x\end{align} is tangent to \begin{align} x^T\myvec{0 & 0 \\0 & 1}\vec x-(12,0)\vec x+34=0\end {align} then the possible values of the slop of the chord,are \\
    
    \choice (a) \{${-1,1}$\}
    
    \choice (b) \{${-2,2}$\}
    
    \choice (c) \{${-2,-\frac{1}{2}}$\}
    
    \choice (d) \{${2,-\frac{1}{2}}$\}\\
    
    \item For hyperbola \begin{align} x^T\myvec {\frac{1}{\cos^2\alpha} & 0 \\0 & -\frac{1}{\sin^2\alpha}} \vec x=1\end{align} which of the following remains constant with change in '$\alpha$'\\
    
    \choice (a) abscissae of vertices
    
    \choice (b) abscissae of foci
    
    \choice (c) eccentricity
    
    \choice (d) directrix\\
    
    \item if tangents are drawn to the elipse \begin{align} x^T\myvec{1&0\\0&2}\vec x=2\end{align} then the locus of the mid point of the intercept made by the tangents between the coordinate axes is\\
    \choice (a)$\frac{1}{2x^2}+\frac{1}{4y^2}=1$
    
    \choice (b) $\frac{1}{4x^2}+\frac{1}{2y^2}=1$
    
    \choice (c) $ x^T\myvec{\frac{1}{2}&0\\0&\frac{1}{4}}\vec x=1$
    
    \choice (d) $ x^T\myvec{\frac{1}{4}&0\\0&\frac{1}{2}}\vec x=1$\\
    
    \item The angle between the tangents drawn from the points(1,4) to the parabola \begin{align} x^T\myvec{
    1 & 0 \\0 & 0}\vec x=(4,0)\vec x \end{align} is\\ 
    
    \choice (a) $\frac{\pi}{6}$
    
    \choice (b) $\frac{\pi}{4}$
    
    \choice (c) $\frac{\pi}{3}$
    
    \choice (d) $\frac{\pi}{2}$\\
    
    \item If the line \begin{align}(2,\sqrt{6})\vec x=2\end{align} touches the hyperbola\begin{align}x^T\myvec{1 & 0 \\0 & -2} \vec x=4\end{align}then the point of contact is\\ 
    
    \choice (a) $(-2,\sqrt{6})$
    
    \choice (b) $(-5,2\sqrt{6})$
    
    \choice (c) $(\frac{1}{2},\frac{1}{\sqrt{6}})$
    
    \choice (d) $(4,\sqrt{6})$\\
    
\item The minimum area of the triangle is formed by the tangent to the \begin{align}x^T \myvec{1& 0 \\0 & 2} \vec x=1 \end{align} the coordinate axes is \\
    
    \choice (a) ab sq.units
    
    \choice (b) ${\frac{a^2+b^2}{2}}$sq.units 
    
    \choice (c) ${\frac{(a+b)^2}{2}}$sq.units 
    
    \choice (d) $\frac{a^2+ab+b^2}{3}$sq.units\\
    
    \item Tangent to the curve \begin{align}(0,1)\vec x= x^T\myvec {1 & 0 \\0 & 0}\vec x+ 6\end{align} at the points (1,7)touches the circle \begin{align} x^T \vec x+(16,12)\vec x+c=0\end{align}at a point Q.Then the coordinates of Q are \\
    
    \choice (a) (-6,-11)
    
    \choice (b) (-9,-13)
    
    \choice (c) (-10,-15)
    
    \choice (d) (-6,-7)\\
    
    \item The axis of a parabola is along the line \begin{align}(0,1)\vec x= (1,0)\vec x\end{align} and the distance of its vertex and focus from the origin are $\sqrt{2}$ and $2\sqrt{2}$ respectively.If vertex and focus both lies in the first quaderent ,then the equation of the parabola is \\
    
    \choice (a) $\vec x^T\vec x+2\vec x^T\begin{vmatrix} 0&1\\0&0 \end{vmatrix} \vec x+(-1,0)\vec x+(0,1)\vec x+2=0$
    
     \choice (b) $\vec x^T\vec x-2\vec x^T\begin{vmatrix} 0&1\\0&0 \end{vmatrix} \vec x+(-1,0)\vec x+(0,-1)\vec x+2=0$
     
      \choice (c) $\vec x^T\vec x+2\vec x^T\begin{vmatrix} 0&1\\0&0 \end{vmatrix} \vec x+(-4,0)\vec x+(0,-4)\vec x+8=0$
      
      \choice (d) $\vec x^T\vec x+2\vec x^T\begin{vmatrix} 0&1\\0&0 \end{vmatrix} \vec x+(-4,0)\vec x+(0,-4)\vec x+16=0$\\
      
      \item A hyperbola, having the transverse axis of length $2\sin\theta$,is confocal with the ellipse \begin{align} x^T\myvec {3&0\\0&4 }\vec x=12\end{align}.Then its equation is \\
      
      \choice (a) $\vec x^T\begin{vmatrix} cosec^2 \theta&0\\0&\sec^2\theta \end{vmatrix} \vec x =1$
      
      \choice (b) $\vec x^T\begin{vmatrix} \sec^2\theta&0\\0& cosec^2\theta \end{vmatrix} \vec x=1$
     
      
      \choice (c)$\vec x^T\begin{vmatrix} \sin^2\theta&0\\0&\cos^2\theta \end{vmatrix} \vec x=1$
      
      \choice (d) $\vec x^T\begin{vmatrix} \cos^2\theta&0\\0&\sin^2\theta \end{vmatrix} \vec x=1$\\
      
      \item Let a and b are non zero real numbers.Then the equation \begin{align}(\vec x^T\myvec{ a&0\\0&b }\vec x+c)( x^T\myvec {1&0\\0&6} \vec x-5x^T\myvec{0&0\\1&0} \vec x)=0\end{align}represents\\
      
      \choice (a) four straight lines, when c=0 and a,b are of the same sign 
      
      \choice (b) two straight lines and a circle ,when a=b,and c is of sign opposite to that of a
      
      \choice (c) two straight lines and a hyperbola ,when a and b are of the same sign and c is of sign opposite to that of a 
      
    \choice (d) a circle and an ellipse ,when a and b are of the same sign and c is of sign opposite to that of a\\
    
    \item Consider a branch  of the hyperbola \begin{align} x^T\myvec{1&0\\0&-2}\vec x+(-2\sqrt{2},-4\sqrt{2})\vec x-6=0\end{align} with the vertex at the point A.Let B be the one of the end points of its latus rectum.If C is the focus of the hyperbola nearer to the point A,then the area of the triangle ABC is\\ 
    
    \choice (a) $1-\sqrt{\frac{2}{3}}$
    
    \choice (b) $\sqrt{\frac{3}{2}}-1$

    \choice (c) $1+\sqrt{\frac{2}{3}}$

    \choice (d) $\sqrt{\frac{3}{2}}+1$\\

\item The line passing through the extremity A of the major axis and extremity B of the minor axis of the ellipse  \begin{align}x^T\myvec{1&0\\0&9 } \vec x =9\end{align}
meets its auxiliary circle at the point M Then the area of the triangle with vertices at A,M and the origin O is\\

\choice (a) $\frac{31}{10}$

\choice (b) $\frac{29}{10}$

\choice (c) $\frac{21}{10}$

\choice (d) $\frac{27}{10}$\\

\item The normal at a point P on the ellipse \begin{align}
 x^T\myvec{1&0\\0&4} \vec x =16\end{align} meets the x-axis at Q.If M is the mid point of the line segment PQ ,then the locus of M intersects the latus rectums of the given ellipse at the points\\  

\choice (a) $(\pm{\frac{3\sqrt{5}}{2}}, \pm{\frac{2}{7}})$

\choice (b) $(\pm{\frac{3\sqrt{5}}{2}}, \pm{\sqrt{\frac{19}{4}}})$

\choice (c) $(\pm 2\sqrt{3}, \pm{\frac{1}{7}})$

\choice (d) $(\pm 2\sqrt{3}, \pm{\frac{4\sqrt{3}}{7}})$\\

\item The locus of the orthocentre of the triangle formed by the lines \begin{align}((1+p),-p)\vec x+p(1+p)=0\end{align} \begin{align}((1+q),-q)\vec x+q(1+q)=0\end{align}and \begin{align}(0,1)\vec x=0\end{align},where p \neq q is \\

\choice (a) a hyperbola

\choice (b) a parabola

\choice (c) an ellipse

\choice (d) a straight lines\\

\item Let P(6,3) be a points on the hyperbola \begin{align} x^T\myvec {\frac{1}{a^2}&0\\0&\frac{1}{b^2}} \vec x =1\end{align}. If the normal at the points P intersects the x-axis at (9,0), then the eccentricity of the hyperbola is\\ 

\choice (a) $\sqrt{\frac{5}{2}}$

\choice (b) $\sqrt{\frac{3}{2}}$

\choice (c) $\sqrt{2}$

\choice (d)  $\sqrt{3}$\\

\item Let(x,y)be any point on the parabola \begin{align} x^T \myvec {0 & 0 \\0 & 1}\vec x=(4,0)\vec x \end{align}.Let P be the points that divides the lines segment from (0,0) to (x,y) in the ratio 1:3. Then the locus of P is \\
    
    \choice (a) $\vec x^T$ \begin{vmatrix}
    1 & 0 \\0 & 0\end{vmatrix}= (0,1)\vec x$
    
    \choice (b) $\vec x^T$ \begin{vmatrix}
    0 & 0 \\0 & 1 \end{vmatrix} $\vec x$=$(2,0)\vec x $
    
    \choice (c) $\vec x^T$ \begin{vmatrix}
    0 & 0 \\0 & 1 \end{vmatrix} $\vec x$=$(1,0)\vec x $
    
    \choice (d) $\vec x^T \begin{vmatrix}
    1 & 0 \\0 & 0\end{vmatrix}= (0,2)\vec x$ \\
    
    \item The ellipse $E_1$:\begin{align} x^T\myvec{\frac{1}{9} & 0 \\0 & \frac{1}{4}}\vec x=1\end{align}is inscribed in a rectangle R whose sides are parallel to the coordinate axes.Another ellipse $E_2$ passing through the points (0,4) circumscribes the rectangle R.The eccentricity of the ellipse E_2 is\\
    
    \choice (a) $\frac{\sqrt{2}}{2}$
    
    \choice (b) $\frac{\sqrt{3}}{2}$
    
    \choice (c) $\frac{1}{2}$
    
    \choice (d) $\frac{3}{4}$\\
    
    \item The common tangent to the circle \begin{align} x^T\myvec{1& 0 \\0 & 1 }\vec x=2\end{align} and the parabola\begin{align}x^T \myvec{0 & 0 \\0 & 1}\vec x=(8,0)\vec x\end{align} touch the circle at the points P,Q and the parabola at the points R,S.Then the area of the quadrilateral PQRS is \\
    
    \choice (a) 3
     
    \choice (b) 6
    
    \choice (c) 9
    
    \choice (d) 15\\

\end{enumerate}

{\Large\textbf{MCQs with One or More than One Correct Answer }}
\begin{enumerate}

\item The number of values of c such that the straight line \begin{align}(0,1)\vec x=(4,0)\vec x+c\end{align}ltgniouaches the curve \begin{align}x^T\myvec {\frac{1}{4}& 0 \\0 & 1} \vec x=1\end{align}is \\
    
    \choice (a) 0
    
    \choice (b) 1
    
    \choice (c) 2
    
    \choice (d) infinite\\
    
    \item If P=(x,y),$F_1=(3,0),F_2=(-3,0)$and \begin{align}x^T\myvec{16& 0 \\0 & 25}\vec x=400\end{align},then $PF_1+PF_2$equals \\
    
    \choice (a) 8
     
    \choice (b) 6
    
    \choice (c) 10
    
    \choice (d) 12\\
    
    \item On the ellipse \begin{align} x^T\myvec{4& 0 \\0 & 9} \vec x=1\end{align},the points at which the tangents are parallel to the line \begin{align}(8,0)\vec x=(0,9)\vec x\end{align} are \\
    
     \choice (a) $\frac{2}{5},\frac{1}{5}$
     
     \choice (b) $-\frac{2}{5},\frac{1}{5}$
     
     \choice (c) $-\frac{2}{5},-\frac{1}{5}$
     
     \choice (d) $\frac{2}{5},-\frac{1}{5}$\\
     
     \item The equation of the common tangents to the parabola \begin{align}(0,1)\vec x= x^T\myvec{1& 0 \\0 & 0}\vec x\end{align} and \begin{align}(0,1)\vec x+x^T\myvec{
    -1& 0 \\0 & 0 }\vec x=4\end{align} is/are \\
    
    \choice (a) $(0,1)\vec x+(-4,0)\vec x+4$=0
    
    \choice (b) $(0,1)\vec x=0$
    
    \choice (c) $(0,1)\vec x+(4,0)\vec x-4=0$
    
    \choice (d) $(0,1)\vec x+(30,0)\vec x+50=0$\\
    
    \item Let the hyperbola passes through the focus of the ellipse \begin{align} x^T\myvec{\frac{1}{25}& 0 \\0 & \frac{1}{16}}\vec x =1\end{align} The transverse and conjugate axes of the given ellipse, also the product of eccentricities of given ellipse,also the product of eccentricities of given ellipse and hyperbola is 1,then\\ 
    
     \choice (a) the equation of the hyperbola is $\vec x^T$ \begin{vmatrix}
    \frac{1}{9}& 0 \\0 & -\frac{1}{16} \end{vmatrix} $\vec x$ =1  
    
    \choice (b) the equation of the hyperbola is $\vec x^T$ \begin{vmatrix}
    \frac{1}{9}& 0 \\0 & -\frac{1}{25} \end{vmatrix} $\vec x$ =1
    
    \choice (c) focus of hyperbola is (5,0)
    
    \choice (d) vertex of hyperbola is (5\sqrt{3},0)\\
    
    \item Let $P(x_1,y_1)$and $Q(x_2,y_2),y_1<0,y_2<0$ ,be the end point of the latus rectum of the ellipse \begin{align}x^T\myvec{1& 0 \\0 & 4}\vec x=4\end{align}.The equation of parabola with latus rectum PQ are\\
    
    \choice (a) $\vec x^T$ \begin{vmatrix}
    1& 0 \\0 & 0 \end{vmatrix} $\vec x+(0, 2\sqrt{3})\vec x$=3+\sqrt{3}$
    
    \choice (b) $\vec x^T$ \begin{vmatrix}
    1& 0 \\0 & 0 \end{vmatrix} $\vec x-(0, -2\sqrt{3})\vec x$=3+\sqrt{3}$
    
    \choice (c) $\vec x^T$ \begin{vmatrix}
    1& 0 \\0 & 0 \end{vmatrix} $\vec x+(0, 2\sqrt{3})\vec x$=3-\sqrt{3}$
    
    \choice (d) $\vec x^T$ \begin{vmatrix}
    1& 0 \\0 & 0 \end{vmatrix} $\vec x-(0, -2\sqrt{3})\vec x$=3-\sqrt{3}$\\
    
    \item In a triangle ABC with fixed base BC, the vertex A moves such that $\cos B+\cos C=4\sin^2\frac{A}{2}$.If a,b and c denote the lengths of the triangle A,B and C,respectively,then \\
    
    \choice (a) b+c=4a
    
    \choice (b) b+c=2a
    
    \choice (c) locus of point A is an ellipse
    
    \choice (c) locus of point A is a pair of straight lines\\ 
    
    \item The tangent PT and the normal PN to the parabola\begin{align} x^T \myvec{0& 0 \\0 & 1}\vec x=(4a,0)\vec x\end{align}at a point P on it meet its axis at points T and N,respectively.The locus of the centroid of the triangle PTN is a parabola whose\\
    
    \choice (a) vertex is ($\frac{2a}{0},0$)
    
    \choice (b) directrix is x=0
    
    \choice (c) latus rectum is $\frac{2a}{3}$
    
    \choice (d) focus is (a,0)\\
    
    \item An ellipse intersects the hyperbola \begin{align} x^T \myvec{2& 0 \\0 & -2} \vec x=1\end{align} orthogonally.The eccentricity of the ellipse is reciprocal of that of the hyperbola.If the axes of the ellipse are along the coordinate axes,then\\ 
    
    \choice (a) equation of ellipse is $\vec x^T \begin{vmatrix}
    1& 0 \\0 & 2 \end{vmatrix} \vec x=2$
    
    \choice (b) the foci of ellipse are (\pm 1,0)
    
    \choice (c) equation of ellipse is $\vec x^T \begin{vmatrix}
    1& 0 \\0 & 2 \end{vmatrix} \vec x=4$
    
    \choice (d) the foci of ellipse are (\pm \sqrt{2},0)\\
    
    \item Let A and B two distinct points on the parabola \begin{align} x^T \myvec{0& 0 \\0 & 1} \vec x=(4,0)\vec x\end{align}.If the axis of a parabola touches a circle of radius r having AB as its diameter,then the slop of the line joining A and B can be \\
    
    \choice (a) $-\frac{1}{r}$
    
    \choice (b) $\frac{1}{r}$
    
    \choice (c) $\frac{2}{r}$
    
    \choice (d) $-\frac{2}{r}$\\ 
    
    \item Let the eccentricity of the hyperbola \begin{align}\vec x^T \myvec{\frac{1}{a^2}& 0 \\0 & -\frac{1}{b^2}}\vec x=1\end{align}.If the hyperbola passes to that of the ellipse \begin{align}x^T \myvec{1& 0 \\0 & 4 }\vec x=4\end{align}.If the hyperbola passing through a focus of the ellipse,then\\
    
    \choice (a) the equation of the hyperbola is $\vec x^T \begin{vmatrix}
    \frac{1}{3}& 0 \\0 & -\frac{1}{2} \end{vmatrix} \vec x=1$
    
    \choice (b) the focus of the hyperbola is (2,0)
    
    \choice (c) the eccentricity of the hyperbola is $\sqrt{\frac{5}{3}}$
    
    \choice (d) the equation of the hyperbola is $\vec x^T \begin{vmatrix}
    1& 0 \\0 & -3 \end{vmatrix} \vec x=3$\\
    
    \item Let L be a normal to the parabola \begin{align}\vec x^T \myvec{0& 0 \\0 & 1}\vec x=(4,0)\vec x\end{align}.If L passes though the point(9,6),then L is given by\\
    
     \choice (a) $(-1,1)\vec x+3=0$
     
     \choice (b) $(3,1)\vec x-33=0$
     
     \choice (c) $(1,1)\vec x-15=0$
     
     \choice (d) $(-2,1)\vec x+12=0$\\
     
     \item Tangents are drawn to the hyperbola \begin{align} x^T \myvec{\frac{1}{9}& 0 \\0 & -\frac{1}{4}}\vec x=1\end{align},parallel to the straight line \begin{align}(2,-1)\vec x=1\end{align}.The point of contact of the tangents on the hyperbola are \\
    
    \choice (a) $(\frac{9}{2\sqrt{2}},\frac{1}{\sqrt{2}})$
    
    \choice (b) $(-\frac{9}{2\sqrt{2}},-\frac{1}{\sqrt{2}})$
    
    \choice (c) $(3\sqrt{3},-2\sqrt{2})$
    
    \choice (d) $(-3\sqrt{3},2\sqrt{2})$\\
    
    \item Let P and Q be distinct points on the parabola \begin{align} x^T \myvec{0& 0 \\0 & 1}\vec x=(2,0)\vec x\end{align}such that a circle with PQ as diameter passes through the vertex O of the parabola.If P lies in the first quadrant and the area of the triangle$ \Delta OPQ $is $3\sqrt{2}$,then which of the following is (are)the coordinates of P?\\
    
    \choice (a) $(4,2\sqrt{2})$
    
    \choice (b) $(9,3\sqrt{2})$
    
    \choice (c) $(\frac{1}{4},\frac{1}{\sqrt{2}})$
    
    \choice (d) $(1,\sqrt{2})$\\
    
    \item Let $E_1$and $E_2$be two ellipses whose centers are at the origin.The major axes of $E_1 $and $E_2$ lie along the x-axis and the y-axis,respectively.Let S be the circle \begin{align} x^T \myvec{1& 0 \\0 & 1}\vec x+(0,-2)\vec x=1\end{align}.The straight line \begin{align}(1,1)\vec x=3\end{align} touches the curves S.$E_1$and $E_2$ at P,Q and R respectively.Suppose that PQ=PR=$\frac{2\sqrt{2}}{3}$.If $e_1$ and $e_2$ are the eccentricities of $E_1$ and$ E_2$,respectively ,Then the correct expression(s) is (are)\\
    
    \choice (a) $e_1^2+e_2^2=\frac{43}{40}$
    
    \choice (b) $e_1e_2=\frac{\sqrt{7}}{2\sqrt{10}}$
    
    \choice (c) $\mid e_1^2-e_2^2\mid=\frac{5}{8}$
    
    \choice (d) $e_1e_2=\frac{\sqrt{3}}{4}$\\
    
    \item Consider a hyperbola H:\begin{align} x^T \myvec{
    1& 0 \\0 & 1}\vec x=1\end{align} and a circle S with center N$(x_2,0)$.Suppose that H and S touches each other at a point P$(x_1,y_1)$ with $x_1>1 $and$ y_1>0$. The common tangent to H and S at P intersects the x-axis at point M.If(l,m) is the centroid of the triangle PMN,then the correct expression is(are)\\
    
    \choice (a) $\dv{l}{x_1}= 1-\frac{1}{3x_1^2}$for x_1>1
    
    \choice (b) $\dv{m}{x_1}=\frac{x_1}{3(\sqrt{x_1^2-1})}$for x_1>1

    
    \choice (c) $\dv{l}{x_1}=1+\frac{1}{3x_1^2}$for x_1>1$
    
    \choice (d) $\dv{m}{x_1}=\frac{1}{3}$for y_1>0\\
    
    \item The circle $C_1$:\begin{align}x^T \myvec{
    1& 0 \\0 & 1 }\vec x=3\end{align},with centre at O,intersects the parabola \begin{align}x^T \myvec{
    1& 0 \\0 & 0 \vec x=(0,2)\vec x\end{align}and centres $Q_2$and$ Q_3$,respectively.If $Q_2$and $Q_3 $lie on the y-axis,then\\
    
    \choice (a) $ Q_2$ $Q_3$=12$
    
    \choice (b) $R_2$ $R_3$=4$\sqrt{6}$
    
     \choice (c) area of the triangle O$R_2R_3$is 6\sqrt{2}
     
     \choice (d) area of the triangle P$Q_2Q_3$is 4\sqrt{2}\\
     
\end{enumerate}
\end{document}
