\renewcommand{\theequation}{\theenumi}
\begin{enumerate}[label=\arabic*.,ref=\thesubsection.\theenumi]
\numberwithin{equation}{enumi}
	%
\item In Fig. 	\ref{fig:tri_med_pgm}, 	$BDEF$ is defined as a {\em parallelogram}. Based on the properties of medians and similar triangles, we know that
\label{them:pgm_basic}
%
\begin{align}
BD &\parallel EF, BD = EF
\\
BF &\parallel DE, BF = DE
\\
OD &= OF, OB = OE
\end{align} 
Hence, 
\begin{enumerate}
\item opposite sides are equal 
\item  opposite angles are equal
\item  diagonals bisect each other
\item Adjacent angles of a paralleogram are supplementary.
\end{enumerate}
%
\begin{figure}[!ht]
	\begin{center}
		\resizebox{\columnwidth}{!}{%Code by GVV Sharma
%December 11, 2019
%released under GNU GPL
%Medians meet at a point

\begin{tikzpicture}
[scale=2,>=stealth,point/.style={draw,circle,fill = black,inner sep=0.5pt},]

%Triangle sides
\def\a{5}
\def\b{6}
\def\c{4}
 
%Coordinates of A
\def\p{2.25}
\def\q{{sqrt(\c^2-\p^2)}}

%Labeling points
\node (A) at (\p,\q)[point,label=above right:$A$] {};
\node (B) at (0, 0)[point,label=below left:$B$] {};
\node (C) at (\a, 0)[point,label=below right:$C$] {};

%Foot of median

\node (D) at ($(B)!0.5!(C)$)[point,label=below:$D$] {};
\node (E) at ($(A)!0.5!(C)$)[point,label=right:$E$] {};
\node (F) at ($(B)!0.5!(A)$)[point,label=left:$F$] {};

%Drawing triangle ABC
\draw (A) --  (B) -- (C) --  (A);

%Drawing medians AD, BE and CF
\draw (B) -- (E);
\draw (D) -- (F);
\node [above] at ($(F)!0.5!(D)$) {$O$};
%\draw (C) -- (F);
%\draw (A) -- (D);

%Drawing EF
\draw [dashed] (E) -- node[above] {$\frac{a}{2}$}(F);
\draw [dashed] (E) -- node[right] {$\frac{c}{2}$}(D);

%Centroid
%\node (G) at ($(B)!0.67!(E)$)[label={[shift={(0.8,-0.5)}]$G$}] {};

%Labeling sides
\node [right] at ($(A)!0.5!(E)$) {$\frac{b}{2}$};
\node [right] at ($(C)!0.5!(E)$) {$\frac{b}{2}$};
\node [left] at ($(B)!0.5!(F)$) {$\frac{c}{2}$};
\node [left] at ($(A)!0.5!(F)$) {$\frac{c}{2}$};
\node [below] at ($(B)!0.5!(D)$) {$\frac{a}{2}$};
\node [below] at ($(D)!0.5!(C)$) {$\frac{a}{2}$};
%\node [above] at ($(E)!0.5!(G)$) {$p$};
%\node [below] at ($(B)!0.5!(G)$) {$2p$};

%\node (G) at ($(B)!0.67!(E)$)[label={[shift={(-0.8,-0.5)}]$G_1$}] {};

%
\end{tikzpicture}

}
	\end{center}
	\caption{Parallelogram}
	\label{fig:tri_med_pgm}	
\end{figure}
%
\item In Fig. 	\ref{fig:tri_med_pgm}, $BFEC$ is a quadrilateral with $EF \parallel BC$ and is known as a {\em trapezium}.
\item Sum of the angles of a quadrilateral is 360$\degree$. 
\item Construct parallelogram $ABCD$ 	in Fig. \ref{fig:pgm_sas}	
given that  $BC = 5, AB = 6, \angle C = 85 \degree$.
\begin{figure}[!ht]
	\begin{center}
		\resizebox{\columnwidth}{!}{%Code by GVV Sharma
%December 10, 2019
%released under GNU GPL
%Drawing a parallelogram given 2 sides and a diagonal

\begin{tikzpicture}
[scale=2,>=stealth,point/.style={draw,circle,fill = black,inner sep=0.5pt},]

%Triangle sides
\def\a{5}
\def\b{6}
\def\c{7.467975323683154}
%Coordinates of A
%\def\p{{\a^2+\c^2-\b^2}/{(2*\a)}}
\def\p{4.477065543514051}
\def\q{{sqrt(\c^2-\p^2)}}

%Labeling points
\node (D) at (\p,\q)[point,label=above right:$D$] {};
\node (B) at (0, 0)[point,label=below left:$B$] {};
\node (C) at (\a, 0)[point,label=below right:$C$] {};
\node (O) at ($(B)!0.5!(D)$)[point,label=below right:$O$] {};

%Foot of perpendicular

\node (A) at ($(O)!-1!(C)$)[point,label=above right:$A$] {};


%Drawing parallelogram ABCD
\draw (A) -- (B) --  (C) --(D)--(A);
\draw (A) -- (C);
\draw (B) --(D);


\tkzMarkAngle[fill=green!20](C,A,D)
\tkzMarkAngle[fill=green!20](A,C,B)
%
%
\tkzMarkAngle[fill=red!30](A,D,B)
\tkzMarkAngle[fill=red!30](C,B,D)


\tkzMarkAngle[fill=orange!40](D,C,O)
\tkzMarkAngle[fill=orange!40](B,A,C)

\tkzMarkAngle[fill=blue!50](B,D,C)
\tkzMarkAngle[fill=blue!50](D,B,A)

\end{tikzpicture}
}
	\end{center}
	\caption{Parallelogram Properties}
	\label{fig:pgm_sas}	
\end{figure}
%
\\
\solution $BD$ is found using the cosine formula and $\triangle BDC$ is drawn using the approach in Construction \ref{const:tri_sss} with 
%
\begin{align}
\vec{B} = \myvec{0\\0},
\vec{C} = \myvec{5\\0},
\end{align}
%
Since the diagonals bisect each other, 
%
\begin{align}
\vec{O} &= \frac{\vec{B}+\vec{D}}{2}
\\
\vec{A} &= 2\vec{O} - \vec{C}.
\end{align}
%
$AB$ and $AD$ are then joined to complete the $\parallel$gm.
The python code for  Fig. \ref{fig:pgm_sas} is
\begin{lstlisting}
codes/quad/pgm_sas.py
\end{lstlisting}
%
and 
The equivalent latex-tikz code is
%
\begin{lstlisting}
figs/quad/pgm_sas.tex
\end{lstlisting}
%

\item A rectangle is a parallelogram with one right angle.
\item Draw the $\parallel$gm $ABCD$ in 	Fig. \ref{fig:pgm_sss}	
with $BC = 6, CD = 4.5$ and $BD=7.5$.  Show that it is a rectangle.
\label{const:pgm_sss}
%
\begin{figure}[!ht]
	\begin{center}
		\resizebox{\columnwidth}{!}{%Code by GVV Sharma
%December 10, 2019
%released under GNU GPL
%Drawing a parallelogram given 2 sides and a diagonal

\begin{tikzpicture}
[scale=2,>=stealth,point/.style={draw,circle,fill = black,inner sep=0.5pt},]

%Triangle sides
\def\a{6}
\def\b{4.5}
\def\c{7.5}
%Coordinates of A
%\def\p{{\a^2+\c^2-\b^2}/{(2*\a)}}
\def\p{6}
\def\q{{sqrt(\c^2-\p^2)}}

%Labeling points
\node (D) at (\p,\q)[point,label=above right:$D$] {};
\node (B) at (0, 0)[point,label=below left:$B$] {};
\node (C) at (\a, 0)[point,label=below right:$C$] {};
\node (O) at ($(B)!0.5!(D)$)[point,label=below right:$O$] {};

%Foot of perpendicular

\node (A) at ($(O)!-1!(C)$)[point,label=above right:$A$] {};


%Drawing parallelogram ABCD
\draw (A) -- (B) --  (C) --(D)--(A);
\draw (A) -- (C);
\draw (B) --(D);


%\tkzMarkAngle[fill=green!20](C,A,D)
%\tkzMarkAngle[fill=green!20](A,C,B)
%%
%%
%\tkzMarkAngle[fill=red!30](A,D,B)
%\tkzMarkAngle[fill=red!30](C,B,D)
%
%
%\tkzMarkAngle[fill=orange!40](D,C,O)
%\tkzMarkAngle[fill=orange!40](B,A,C)
%
%\tkzMarkAngle[fill=blue!50](B,D,C)
%\tkzMarkAngle[fill=blue!50](D,B,A)
%
\end{tikzpicture}
}
	\end{center}
	\caption{Rectangle}
	\label{fig:pgm_sss}	
\end{figure}
\\
\solution It is easy to verify that 
%Using the approach in Construction\ref{const:tri_sss}, $\triangle BCD$ is drawn with
%
\begin{align}
BD^2=BC^2+C^2
\end{align}
%
Hence, using Baudhayana theorem, 
%
\begin{align}
\angle BCD = 90\degree
\end{align}
%
and  $ABCD$ is a rectangle.
\begin{align}
\vec{A} = \myvec{0\\4.5}
\vec{B} = \myvec{0\\0}
\vec{C} = \myvec{6\\0}
\vec{D} = \myvec{6\\4}
\end{align}
%
The python code for  Fig. \ref{fig:pgm_sss} is
\begin{lstlisting}
codes/quad/pgm_sss.py
\end{lstlisting}
%
and the equivalent latex-tikz code is
%
\begin{lstlisting}
figs/quad/pgm_sss.tex
\end{lstlisting}
%
\item  Diagonals of a rectangle are equal and vice-versa. 
\\
\solution Follows from the fact that $\triangle BCD \cong \triangle ABC$. 
%
\item A rhombus, shown in Fig. 	\ref{fig:rhom_sss}	
 is a parallelogram with equal sides.  
%
\begin{figure}[!ht]
	\begin{center}
		\resizebox{\columnwidth}{!}{%Code by GVV Sharma
%December 10, 2019
%released under GNU GPL
%Drawing a rhombus given a side and a diagonal

\begin{tikzpicture}
[scale=2,>=stealth,point/.style={draw,circle,fill = black,inner sep=0.5pt},]

%Triangle sides
\def\a{4.5}%BE
\def\b{6}%ET
\def\p{\b/2}%OE
\def\q{{sqrt(\a^2-\p^2)}}%OB

%Labeling points
\node (B) at (0,-\q)[point,label=below :$B$] {};
\node (E) at (\p, 0)[point,label=right:$E$] {};
\node (S) at (0, \q)[point,label=above:$S$] {};
\node (T) at (-\p,0)[point,label=left:$T$] {};
\node (O) at (0, 0)[point,label=below left:$O$] {};


%Drawing parallelogram ABCD
\draw (B) -- (E) --  (S) --(T)--(B);
\draw (B) -- (S);
\draw (E) --(T);


\tkzMarkRightAngle[fill=blue!20,size=.2](B,O,E)

%
\end{tikzpicture}
}
	\end{center}
	\caption{Rhombus}
	\label{fig:rhom_sss}	
\end{figure}
%
\item Diagonals of a rhombus bisect each other at right angles.
\\
\solution 	In Fig. \ref{fig:rhom_sss}, 	from Theorem \ref{them:pgm_basic}, $OB = OS$.
From Theorem \ref{them:isos_pb}, $OE \perp BS$.
%
\item Draw the rhombus $BEST$ with $BE = 4.5$ and $ET = 6$. 
\\
\solution The coordinates of the various points in Fig. \ref{fig:rhom_sss} are obtained as
%
\begin{align}
\vec{O} = \myvec{0\\0},
\vec{B} = \myvec{0\\-4.5}
\\
\vec{E} = \myvec{3\\0},
\vec{S} = \myvec{4.5\\0},
\vec{T} = \myvec{0\\-3}
\end{align}
%
\item A square is a rectangle whose sides are equal.  Draw a square of side 4.5.
\\
\solution The coordinates of the various points in Fig. \ref{fig:square} are obtained as
%
\begin{align}
\vec{A} = \myvec{0\\4.5}
\\
\vec{B} = \myvec{0\\0},
\vec{C} = \myvec{4.5\\0},
\vec{D} = \myvec{4.5\\4.5}
\vec{O} = \frac{\vec{B}+\vec{C}}{2}
%
\end{align}
%
\begin{figure}[!ht]
	\begin{center}
		\resizebox{\columnwidth}{!}{%Code by GVV Sharma
%December 12, 2019
%released under GNU GPL
%Drawing a square given a side 

\begin{tikzpicture}
[scale=2,>=stealth,point/.style={draw,circle,fill = black,inner sep=0.5pt},]

%Square side
\def\a{4.5}

%Labeling points
\node (A) at (0,\a)[point,label=below :$A$] {};
\node (B) at (0,0)[point,label=below :$B$] {};
\node (C) at (\a, 0)[point,label=right:$C$] {};
\node (D) at (\a, \a)[point,label=above:$D$] {};
\node (O) at ($(B)!0.5!(D)$)[point,label=below left:$O$] {};


%Drawing square ABCD
\draw (A) -- (B) --  (C) --(D)--(A);
\draw (B) -- (D);
\draw (A) --(C);


\tkzMarkRightAngle[fill=blue!20,size=.2](C,O,D)
\tkzMarkRightAngle[fill=blue!20,size=.2](A,B,C)

%
\end{tikzpicture}
}
	\end{center}
	\caption{Square}
	\label{fig:square}	
\end{figure}

\item  Diagonals of a square bisect each other at right angles and are equal, and vice-versa. 
%
\\
\solution A square has the properties of a rectangle as well as a rhombus.
%
%
\item Area of a parallelogram is the product of its base and the corresponding altitude. 
%
\end{enumerate}
