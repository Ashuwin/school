\renewcommand{\theequation}{\theenumi}
\begin{enumerate}[label=\arabic*.,ref=\thesubsection.\theenumi]
\numberwithin{equation}{enumi}

 \item Suppose 
    \begin{align}
    \sin{^3x} \sin{3x} = \sum_{m=0}^{n} C_m\cos{mx}
    \end{align} is an identity in x, where $C_0 , C_1 ,......C_n$ are constants, and $C_n\neq0$ then find the value of n.
    \item Find the solution set of the system of equations 
    \begin{align}
    x + y =\frac{2\pi}{3}\\
    \cos{x} + \cos{y} = \frac{3}{2},
    \end{align} where x and y are real.
    \item Find the set of all x in the interval[0,$\pi$] for which \begin{align}
        2\sin{^2x}-3\sin{x} + 1\geq 0
    \end{align}
    \item The sides of a triangle inscribed in a given circle subtend angles $\alpha,\beta$ and $\gamma$ at the centre. Find the minimum value of the arithmetic mean of $\cos{(\alpha + \frac{\pi}{2})}, \cos{(\beta + \frac{\pi}{2})}$ and $\cos{(\gamma + \frac{\pi}{2})}.$
    \item Find the value of \\
    $\sin{\frac{\pi}{14}}\sin{\frac{3\pi}{14}}\sin{\frac{5\pi}{14}}\sin{\frac{7\pi}{14}}\sin{\frac{9\pi}{14}}\sin{\frac{11\pi}{14}}\sin{\frac{13\pi}{14}}.$
    \item If 
    \begin{align}
    K =\sin{(\frac{\pi}{18})}\sin{(\frac{5\pi}{18})}\sin{(\frac{7\pi}{18})},
    \end{align} 
    then find the numerical value of K ?
    \item If A$>$0,B$>$0 and 
    \begin{align}
    A + B = \frac{\pi}{3},
    \end{align}
    then find the maximum value of tan A tan B.
    \item Find the general value of $\theta$ satisfying the equation 
    \begin{align}
    \tan{^2\theta}  + \sec{2\theta} = 1. 
    \end{align}
    \item Find the real roots of the equation 
    \begin{align}
    \cos{^7x} + \sin{^4 x} = 1
    \end{align} 
    in the interval $(-\pi , \pi)$.
    \item If $\tan{\theta} = -\frac{4}{3},$ then find $\sin{\theta}.$ 
    \item If $\alpha+\beta+\gamma = 2\pi$ then 
    \begin{enumerate}
        \item $\tan{\frac{\alpha}{2}}+\tan{\frac{\beta}{2}}+\tan{\frac{\gamma}{2}} = \tan{\frac{\alpha}{2}}\tan{\frac{\beta}{2}}\tan{\frac{\gamma}{2}}$\\
        \item $\tan{\frac{\alpha}{2}} \tan{\frac{\beta}{2}}+\tan{\frac{\beta}{2}} \tan{\frac{\gamma}{2}}+\tan{\frac{\gamma}{2}} \tan{\frac{\alpha}{2}} = 1$\\
        \item $\tan{\frac{\alpha}{2}}+\tan{\frac{\beta}{2}}+\tan{\frac{\gamma}{2}} = -\tan{\frac{\alpha}{2}}\tan{\frac{\beta}{2}}\tan{\frac{\gamma}{2}}$\\
        \item None of these
    \end{enumerate}
    \item Given 
    \begin{align}
    A = \sin{^2 \theta} + \cos{^4 \theta}
    \end{align} then for all real values of $\theta$
    \begin{enumerate}
        \item $1\leq A \leq 2$
        \item $\frac{3}{4}\leq A \leq 1$
        \item $\frac{13}{16}\leq A \leq 1$
        \item $\frac{3}{4}\leq A \leq \frac{13}{16}$
    \end{enumerate}
   \item The equation 
   \begin{align}
       2\cos{^2\frac{x}{2}}\sin{^2 x} = x^2 + x^{-2};
       0<x<\frac{\pi}{2}
   \end{align} 
   has
   \begin{enumerate}
       \item no real solution
       \item One real solution
       \item more than the one solution
       \item none of these
   \end{enumerate}
   \item The general solution of the trigonometric equation 
   \begin{align}
      \sin{x} +\cos{x} = 1
   \end{align} is given by :
   \begin{enumerate}
       \item x = 2n$\pi; n= 0,\pm1,\pm2...$
       \item x = 2n$\pi + {\frac{\pi}{2}}; n= 0,\pm1,\pm2...$
       \item x = n$\pi + (-1)^n \frac{\pi}{4} - \frac{\pi}{4}; n= 0,\pm1,\pm2...$
       \item none of these
   \end{enumerate}
   \item The value of expression $\sqrt{3}cosec 20\degree-\sec{20\degree}$ is equal to 
   \begin{enumerate}
       \item 2
       \item $\frac{2\sin{20\degree}}{\sin{40\degree}}$
       \item 4
       \item $\frac{4\sin{20\degree}}{\sin{40\degree}}$
   \end{enumerate}
   \item The general solution of
   \begin{align}
       \sin{x}-3\sin{2x} +\sin{3x} =
       \cos{x} - 3\cos{2x} +\cos{3x}
   \end{align} is 
   \begin{enumerate}
       \item $n\pi + \frac{\pi}{8}$
       \item $\frac{n\pi}{2} + \frac{\pi}{8}$
       \item $(-1)^n \frac{n\pi}{2} + \frac{\pi}{8}$
       \item $2n\pi + \cos{^{-1}\frac{3}{2}}$
   \end{enumerate}
   \item The equation  \begin{align}
       (\cos{p} - 1) x^2 + (\cos{p}) x + \sin{p} = 0
   \end{align} In the variable x, has real roots. Then p can take any value in the interval
   \begin{enumerate}
       \item ( 0, $2\pi$)
       \item ($-\pi$ , 0) 
       \item ({$-\frac{\pi}{2}$},{$\frac{\pi}{2}$})
       \item (0 , $\pi$)
   \end{enumerate}
   \item Number of solutions of the equation 
   \begin{align}
       \tan{x} + \sec{x} = 2\cos{x}
   \end{align} 
   lying in the interval [0 , $2\pi$] is :
   \begin{enumerate}
       \item 0
       \item 1
       \item 2
       \item 3
   \end{enumerate}
   \item Let $0< x<\frac{\pi}{4}$ then ($\sec{2x} - \tan{2x}$) equals 
   \begin{enumerate}
       \item $\tan{(x - \frac{\pi}{4})}$
       \item $\tan {(\frac{\pi}{4} - x)}$
       \item $\tan{(x + \frac{\pi}{4})}$
       \item $\tan{^2 (x + \frac{\pi}{4})}$
   \end{enumerate}
   \item Let n be a positive integer such that $\sin{\frac{\pi}{2n}}+\cos{\frac{\pi}{2n}} = \frac{\sqrt n}{2}$.Then
   \begin{enumerate}
       \item $6\leq n\leq 8$
       \item $4< n\leq 8$
       \item $4\leq n\leq 8$
       \item $4< n< 8$
   \end{enumerate}
   \item If $\omega$ is an imaginary cube root of unity then the value of sin $\{(\omega^{10} + \omega^{23}) \pi -\frac{\pi}{4}\}$ is 
   \begin{enumerate}
       \item $-\frac{\sqrt3}{2}$
       \item $-\frac{1}{\sqrt2}$
       \item  $\frac{1}{\sqrt2}$
       \item  $\frac{\sqrt3}{2}$
   \end{enumerate}
   \item $3(\sin{x} -\cos{x})^4 + 6(\sin{x} +\cos{x})^2 + 4(\sin{^6x}+ \cos{^6x}) =$
   \begin{enumerate}
       \item 11
       \item 12
       \item 13
       \item 14
   \end{enumerate}
   \item The general values of $\theta$ satisfying equation 
   \begin{align}
       2\sin{^2\theta} - 3 \sin{\theta} -2 = 0
   \end{align} is 
   \begin{enumerate}
       \item $n\pi+(-1)^n\frac{\pi}{6}$
       \item $n\pi+(-1)^n\frac{\pi}{2}$
       \item $n\pi+(-1)^n\frac{5\pi}{6}$
       \item $n\pi+(-1)^n\frac{7\pi}{6}$
   \end{enumerate}
   \item $\sec{^2\theta} = \frac{4xy}{(x + y)^2}$ is true if and only if 
   \begin{enumerate}
       \item $x + y \neq 0$
       \item $x=y , x\neq 0$
       \item $x=y$
       \item $x\neq0, y\neq0$
   \end{enumerate}
   \item In a triangle PQR, $\angle{R} = \pi/2$.If $\tan({\frac{P}{2}})$ and $\tan{(\frac{Q}{2})}$ are the roots of the equation 
   \begin{align}
       ax^2 + bx + c = 0(a\neq 0)
   \end{align}
   then 
   \begin{enumerate}
       \item a+b=c
       \item b+c=a
       \item a+c=b
       \item b=c
   \end{enumerate}
   \item Let f($\theta$) =$\sin{\theta(\sin{\theta} +\sin{3\theta})}$. Then $f(\theta)$ is 
   \begin{enumerate}
       \item $\geq0$ only when $\theta\geq0$
       \item $\leq 0$ for all real $\theta$
       \item $\geq 0$ for all real $\theta$
       \item $\leq0$ only when $\theta\leq0$
   \end{enumerate}
   \item The number of distinct real roots of 
   $\begin{vmatrix}
   \sin{x} & \cos{x}  & \cos{x} \\ \cos{x} & \sin{x} & \cos{x} \\ \cos{x} & \cos{x} & \sin{x} 
   \end{vmatrix}$=0 \\
   in the interval $-\frac{\pi}{4}\leq x\leq \frac{\pi}{4}$ is 
   \begin{enumerate}
       \item 0
       \item 2
       \item 1
       \item 3
   \end{enumerate}
   \item The  maximum value of\\ $(\cos{\alpha_1}).(\cos{\alpha_2})...(\cos{\alpha_n})$, under the restrictions,
   $0\leq\alpha_1,\alpha_2,.....\alpha_n\leq{\frac{\pi}{2}}$ and $(\cot{\alpha_1}).(\cot{\alpha_2})...(\cot{\alpha_n}) = 1$
  is
   \begin{enumerate}
       \item $\frac{1}{2^{\frac{n}{2}}}$
       \item $\frac{1}{2^n}$
       \item $\frac{1}{2n}$
       \item 1
   \end{enumerate}
   \item If $\alpha+\beta =\frac{\pi}{2}$ and $\beta +\gamma =\alpha$,then $\tan{\alpha}$ equals
   \begin{enumerate}
       \item $2(\tan{\beta} + \tan{\gamma})$
       \item $\tan{\beta} + \tan{\gamma}$
       \item $\tan{\beta} +2\tan{\gamma}$
       \item 2$\tan{\beta} +\tan{\gamma}$
   \end{enumerate}
   \item The number of integral values of  k for which the equation 
   \begin{align}
       7 \cos{x}+5\sin{x}= 2k + 1
   \end{align} 
   has a solution is 
   \begin{enumerate}
       \item 4
       \item 8
       \item 10
       \item 12
   \end{enumerate}
   \item Given both $\theta$ and $\phi$ are acute angles and $\sin{\theta} =\frac{1}{2},\cos{\phi} =\frac{1}{3},$ then the value of $\theta + \phi$ belongs to 
   \begin{enumerate}
       \item $(\frac{\pi}{3},\frac{\pi}{2}]$
       \item $(\frac{\pi}{2},\frac{2\pi}{3})$
       \item $(\frac{2\pi}{3},\frac{5\pi}{6}]$
       \item $(\frac{5\pi}{6},\pi]$
   \end{enumerate}
   \item $\cos{(\alpha-\beta)} = 1$ and $\cos{(\alpha + \beta)} =\frac{1}{e}$ where $\alpha,\beta \epsilon [-\pi , \pi]$. Pairs of $\alpha,\beta$which satisfy both the equations is/are
   \begin{enumerate}
       \item 0
       \item 1
       \item 2
       \item 4
   \end{enumerate}
   \item The values of $\theta \epsilon (0 , 2\pi)$ for which $2\sin{^2\theta} - 5 \sin{\theta} + 2 > 0$, are
   \begin{enumerate}
       \item $(0 ,\frac{\pi}{6}) \cup (\frac{5\pi}{6},2\pi)$
       \item $(\frac{\pi}{8},\frac{5\pi}{6})$
       \item $(0 ,\frac{\pi}{8}) \cup (\frac{\pi}{6},\frac{5\pi}{6})$
       \item $(\frac{41\pi}{48},\pi)$
   \end{enumerate}
   \item Let $\theta \epsilon (0 ,\frac{\pi}{4})$ and $t_1 = (\tan{\theta})^{\tan{\theta}},t_2 = (\tan{\theta})^{\cot{\theta}},t_3 = (\cot{\theta})^{\tan{\theta}}$ and $t_4 = (\cot{\theta})^{\cot{\theta}}$, then
   \begin{enumerate}
       \item $t_1> t_2> t_3 > t_4$
       \item $t_4> t_3> t_1 > t_2$
       \item $t_3> t_1> t_2 > t_4$
       \item $t_2> t_3> t_1 > t_4$
   \end{enumerate}
   \item The number of solutions of the pair of equations 
   \begin{align}
       2\sin{^2\theta} - \cos{2\theta} = 0\\
       2\cos{^2\theta} - 3\sin{\theta} = 0
   \end{align}
   in the interval [0,$2\pi$] is
   \begin{enumerate}
       \item zero
       \item one
       \item two
       \item four
   \end{enumerate}
   \item For $x\epsilon (0 ,\pi)$, the equation 
   \begin{align}
       \sin{x} + 2\sin{2x}-\sin{3x} =3
   \end{align} has
   \begin{enumerate}
       \item infinitely many solutions
       \item three solutions
       \item one solution
       \item no solution
   \end{enumerate}
   \item Let $S = \{x\epsilon(-\pi , \pi) : x \neq 0, \pm{\frac{\pi}{2}}\}$. The sum of all distinct solutions of the  equation 
   \begin{align}
       \sqrt3\sec{x} + \mathrm{cosec}{x} +2(\tan{x}- \cot{x}) =0  
   \end{align}
   in the set S is equal to
   \begin{enumerate}
       \item $-\frac{7\pi}{9}$
       \item $-\frac{2\pi}{9}$
       \item 0
       \item $\frac{5\pi}{9}$
   \end{enumerate}
   \item The value of\\ $\sum_{k=1}^{13}\frac{1}{\sin{(\frac{\pi}{4}+\frac{(k-1)\pi}{6})}\sin{(\frac{\pi}{4}+\frac{k\pi}{6}})}$is equal to
   \begin{enumerate}
       \item $3 - \sqrt3$
       \item 2($3 - \sqrt3$)
       \item 2($\sqrt3 - 1$)
       \item 2($2 - \sqrt3$)
   \end{enumerate}
    \item $(1+\cos{\frac{\pi}{8}})(1+\cos{\frac{3\pi}{8}})(1+\cos{\frac{5\pi}{8}})(1+\cos{\frac{7\pi}{8}})$ is equal to 
    \begin{enumerate}
        \item $\frac{1}{2}$
        \item $\cos{(\frac{\pi}{8})}$
        \item $\frac{1}{8}$
        \item $\frac{1+\sqrt2}{2\sqrt2}$
    \end{enumerate}
    \item The expression $3[\sin{^4(\frac{3\pi}{2}-\alpha)}+\sin{^4(3\pi+\alpha]}-2[\sin{^6(\frac{\pi}{2}+\alpha)}+\sin{^6(5\pi-\alpha)]}$ is equal to 
    \begin{enumerate}
        \item 0
        \item 1
        \item 3
        \item $\sin{4\alpha} + \cos{6\alpha}$
        \item none of these
    \end{enumerate}
    \item The number of all possible triplets $(a_1,a_2,a_3)$ such that \begin{align}
        a_1 + a_2\cos{(2x)} +a_3\sin{^2(x)} =0
    \end{align}
    for all x is 
    \begin{enumerate}
        \item zero
        \item one
        \item three
        \item infinite
        \item none
    \end{enumerate}
    \item The values of $\theta$ lying between $\theta = 0$ and $\theta =\pi/2$ and satisfying the equation
    \begin{align}
        \begin{vmatrix}
        1+\sin{^2\theta} & \cos{^2\theta} & 4\sin{4\theta} \\ \sin{^2\theta }& 1+\cos{^2\theta} & 4\sin{4\theta} \\ \sin{^2\theta} & \cos{^2\theta} & 1+4\sin{4\theta}
        \end{vmatrix}=0 
    \end{align}are
    \begin{enumerate}
        \item $\frac{7\pi}{24}$
        \item $\frac{5\pi}{24}$
        \item $\frac{11\pi}{24}$
        \item $\frac{\pi}{24}$
    \end{enumerate}
    \item Let 
    \begin{align}
    2\sin{^2x}+3\sin{x}-2>0\\x^2-x-2 < 0
    \end{align}(x is measured in radians). Then x lies in the interval
    \begin{enumerate}
        \item$(\frac{\pi}{6},\frac{5\pi}{6})$
        \item$(-1,\frac{5\pi}{6})$
        \item (-1 , 2)
        \item$(\frac{\pi}{6},2)$
    \end{enumerate}
    \item The minimum value of the expression $\sin{\alpha} + \sin{\beta} + \sin{\gamma}$, where$\alpha,\beta,\gamma$are real numbers satisfying $\alpha+\beta+\gamma=\pi$ is
    \begin{enumerate}
        \item Positive
        \item zero
        \item negative
        \item -3
    \end{enumerate}
    \item The number of values of x in the interval $[0,\pi]$ satisfying the equation 
    \begin{align}
        3\sin{^2x} -7\sin{x}+2 =0
    \end{align}
    is 
    \begin{enumerate}
        \item 0
        \item 5
        \item 6
        \item 10
    \end{enumerate}
    \item Which of the following number(s) is/are/rational?
    \begin{enumerate}
        \item $\sin{15\degree}$
        \item $\cos{15\degree}$
        \item $\sin{15\degree}\cos{15\degree}$
        \item $\sin{15\degree}\cos{75\degree}$
    \end{enumerate}
    \item For a positive integer n, let $f_n(\theta) =\tan{ (\frac{\theta}{2})}(1 +\sec{\theta})(1+\sec{2\theta})(1+\sec{4\theta})......(1+\sec{2^n\theta}).$Then
    \begin{enumerate}
        \item $f_2(\frac{\pi}{16})= 1$
        \item $f_3(\frac{\pi}{32})= 1$
        \item $f_4(\frac{\pi}{64})= 1$
        \item $f_5(\frac{\pi}{128})= 1$
    \end{enumerate}
    \item If $\frac{\sin{^4x}}{2} + \frac{\cos{^4x}}{3} =\frac{1}{5},$ then
    \begin{enumerate}
        \item $\tan{^2x}=\frac{2}{3}$
        \item $\frac{\sin{^8x}}{8}+\frac{\cos{^8x}}{27}=\frac{1}{125}$
        \item $\tan{^2x}=\frac{1}{3}$
        \item $\frac{\sin{^8x}}{8}+\frac{\cos{^8x}}{27}=\frac{2}{125}$
    \end{enumerate}
    \item For $0<\theta<\frac{\pi}{2}.$ the solution(s) of 
    $\sum_{m=1}^{6}\mathrm{cosec}(\theta+\frac{(m-1)\pi}{4})\mathrm{cosec}(\theta+\frac{m\pi}{4}) = 4\sqrt{2}$ is(are)
    \begin{enumerate}
        \item $\frac{\pi}{4}$
        \item $\frac{\pi}{6}$
        \item $\frac{\pi}{12}$
        \item $\frac{5\pi}{12}$
    \end{enumerate}
    \item Let $\theta,\varphi \epsilon[0,2\pi]$ be such that $2 \cos{\theta}(1-\sin{\varphi}) = \sin{^2\theta}(\tan{\frac{\theta}{2}}+\cot{\frac{\theta}{2}})\cos{\varphi -1} , \tan{(2\pi-\theta)} > 0$ and $-1<\sin{\theta} < -\frac{\sqrt3}{2}$ , then $\varphi$  can not satisfy
    \begin{enumerate}
        \item $0< \varphi < \frac{\pi}{2}$
        \item $\frac{\pi}{2}< \varphi < \frac{4\pi}{3}$
        \item $\frac{4\pi}{3}< \varphi < \frac{3\pi}{2}$
        \item $\frac{3\pi}{2}< \varphi < 2\pi$
    \end{enumerate}
    \item The number of points in$(-\infty,\infty)$, for which \begin{align}
        x^2-x\sin{x} -\cos{x} =0
    \end{align}
    is
    \begin{enumerate}
        \item 6
        \item 4
        \item 2
        \item 0
    \end{enumerate}
    \item Let 
    \begin{align}
    f(x) = x \sin{\pi x}, x> 0
    \end{align}Then for all natural numbers n, $f'(x)$ vanishes at
    \begin{enumerate}
        \item A unique point in the interval (n,n+$\frac{1}{2}$)
        \item A unique point in the interval (n+$\frac{1}{2}$, n+1)
        \item A unique point in the interval (n,n+1)
        \item Two points in the interval (n,n+1)
    \end{enumerate}
    \item Let $\alpha$ and $\beta$ be non-zero real numbers such that $2(\cos{\beta}-\cos{\alpha}) + \cos{\alpha}\cos{\beta}= 1$.Then which of the following is/are true?
    \begin{enumerate}
        \item $\tan{(\frac{\alpha}{2})}+\sqrt3\tan{(\frac{\beta}{2})} = 0$
        \item $\sqrt3\tan{(\frac{\alpha}{2})}+\tan{(\frac{\beta}{2})} = 0$
        \item $\tan{(\frac{\alpha}{2})}-\sqrt3\tan{(\frac{\beta}{2})} = 0$
        \item $\sqrt3\tan{(\frac{\alpha}{2})}-\tan{(\frac{\beta}{2})} = 0$
    \end{enumerate}
    \item If $\tan{\alpha} = \frac{m}{m+1}$ and $\tan{\beta} =\frac{1}{2m+1}$, find the possible values of $(\alpha+\beta).$
    \item (a) Draw the graph of 
    \begin{align}
    y= \frac{1}{\sqrt2}(\sin{x} +\cos{x})
    \end{align} 
    from $x=-\frac{\pi}{2}$ to $x=\frac{\pi}{2}$\\
    (b)  If $\cos{(\alpha+\beta)} = \frac{4}{5}$, $\sin{(\alpha-\beta)} = \frac{5}{13}$ and $\alpha,\beta$ lies between 0 and $\frac{\pi}{4}$, find $\tan{2\alpha}$
    \item Given $\alpha+\beta-\gamma =\pi$, prove that 
    \begin{align}
    \sin{^2\alpha}+\sin{^2\beta}-\sin{^2\gamma} = 2\sin{\alpha}\sin{\beta} \cos\gamma
    \end{align}
    \item Given A= \{ x: $\frac{\pi}{6} \leq x\leq \frac{\pi}{3}$\} and \begin{align}
        f(x) = \cos{x} - x(1+x);
    \end{align}
    find f(A)
    \item  For all $\theta$ in $[0,\pi/2]$ show that, 
    \begin{align}
    \cos{(\sin{\theta})} \geq \sin{(\cos{\theta})}
    \end{align}.
    \item Without using tables, Prove that $(\sin{12\degree})(\sin{48\degree})(\sin{54\degree}) = \frac{1}{8}$
    \item Show that $16 \cos{(\frac{2\pi}{15})}\cos{(\frac{4\pi}{15})}\cos{(\frac{8\pi}{15})}\cos{(\frac{16\pi}{15})} = 1$
    \item Find all the solution of $4\cos{^2x}\sin{x} -2\sin{^2x} = 3\sin{x}$
    \item Find the values of x$\epsilon(-\pi,\pi)$ which satisfy the equation 
    \begin{align}
        8^{(1+\abs{\cos{x \vert}} + \vert cos^2x \vert +\vert cos^3x \vert+ .....)} = 4^3
    \end{align}
    \item Prove that $\tan{\alpha} + 2 \tan{2\alpha} + 4
    \tan{4\alpha} + 8\cot{8\alpha} = \cot{\alpha}$
    \item ABC is a triangle such that 
    $\sin{(2A+B)} = \sin{(C-A)} = -\sin{(B+2c)} = \frac{1}{2}$ If A,B and C are in arithmetic progression, determine the values of A, B and C.
    \item if exp\{$(\sin{^2x} + \sin{^4x}+ \sin{^6x}+ ......\infty$) In 2 \} satisifies the equation 
    \begin{align}
        x^2 - 9x + 8 =0
    \end{align}, find the value of $\frac{\cos{x}}{\cos{x}+\sin{x}}$, $0< x<\frac{\pi}{2}.$\\
    \item Show that the value of $\frac{\tan{x}}{\tan{3x}}$, wherever defined never lies between $\frac{1}{3}$ and 3.
    \item Determine the smallest positive value of x(in degrees) for which $\tan{(x+100\degree)} = \tan{(x+50\degree)}\tan{(x)}\tan{(x-50\degree)}.$
    \item Find the smallest positive number p for which the equation $\cos{(p\sin{x})} = \sin{(p\cos{x})}$ has a solution  $x\epsilon [0 , 2\pi]$
    \item Find all values of $\theta$ in the interval $(-\frac{\pi}{2},\frac{\pi}{2})$ satisfying the equation \begin{align}
        (1-\tan{\theta})(1+\tan{\theta})\sec{^2\theta} + 2^{\tan{^2\theta}} = 0
    \end{align}
    \item Prove that the values of the function $\frac{\sin{x} \cos{3x}}{\sin{3x}\cos{x}}$ do not lie between $\frac{1}{3}$ and 3 for any real x.
    \item Prove that $\sum_{k=1}^{n-1}(n-k) cos {\frac{2k\pi}{n}} = -\frac{n}{2}$, where $n\geq3$ is an integer
    \item If any triangle ABC, Prove that 
    $\cot{\frac{A}{2}}+\cot{\frac{B}{2}}+\cot{\frac{C}{2}}=\cot{\frac{A}{2}}\cot{\frac{B}{2}}\cot{\frac{c}{2}}$
    \item Find the range of values of t for which $2 \sin{t} = {\frac{1-2x+5x^2}{3x^2-2x-1}}, t \epsilon [{-\frac{\pi}{2}},{\frac{\pi}{2}}].$\\           

This section contains 1 paragraph, Based on each paragraph,there are 2 questions. Each question has four options (A),(B),(C) and (D) ONLY ONE of these four options is correct.
{\center\textbf{PARAGRAPH 1}}\\
Let O be the origin, and $\vec {OX},\vec {OY},\vec {OZ}$ be three unit vectors in the directions of the sides $\vec {QR},\vec {RP},\vec {PQ}$ respectively, of a triangle PQR

    \item $\abs{ \vec {OX} \times \vec {OY}}$ =
    \begin{enumerate}
        \item $\sin{(P+Q)}$
        \item $\sin{2R}$
        \item $\sin{(P+R)}$
        \item $\sin{(Q+R)}$
    \end{enumerate}
    \item If the triangle PQR varies, then the minimum value of $\cos{(P+Q)}+\cos{(Q+R)}+\cos{(R+P)}$ is \begin{enumerate}
        \item $-\frac{5}{3}$
        \item $-\frac{3}{2}$
        \item $ \frac{3}{2}$
        \item $ \frac{5}{3}$
    \end{enumerate}
    \item The number of all possible values of $\theta$ where $0<\theta<\pi$,for which the system of equations\\\\
    $(y+z) \cos 3\theta = (xyz)\sin 3 \theta$\\\\
    $x \sin 3\theta =  {\frac{2\cos 3 \theta}{y}}+{\frac{2 \sin 3\theta}{z}}$\\\\
    $(xyz)\sin 3\theta = (y+2z)\cos 3\theta + y\sin 3\theta$\\
    have a solution ($x_0 , y_0 , z_0$) with $y_0 z_0 \neq 0$,is
    \item The number of values of $\theta$ in the interval, $({-\frac{\pi}{2}},{\frac{\pi}{2}})$ such that $\theta \neq {\frac{n \pi}{5}}$ for n = 0 , $\pm1$,$\pm2$ and $\tan\theta = \cot 5\theta$ as well as $\sin 2\theta = \cos 4\theta$ is 
    \item The maximum value of the expression ${\frac{1}{\sin^2\theta + 3 \sin\theta \cos\theta + 5\cos^2 \theta}}$ is 
    \item Two parallel chords of a circle of radius 2 are at a distance $\sqrt3 + 1$ apart. If the chords subtend at the center, angles of ${\frac{\pi}{k}}$ and ${\frac{2\pi}{k}}$, where $k > 0$, then the value of [k] is\\
    \textbf{Note}: [k] denotes the largest integer less than or equal to k.
    \item The positive integer value of $n>3$ satisfying the equation \\
    ${\frac{1}{\sin({\frac{\pi}{n}})}}= {\frac{1}{\sin({\frac{2\pi}{n}})}} + {\frac{1}{\sin({\frac{3\pi}{n}})}}$ is 
    \item The number of distinct solutions of the equation ${\frac{5}{4}}\cos^2 2x + \cos^4 x + \sin^4 x + \cos^6 x + \sin^6 x = 2$ in the interval $[0 , 2\pi]$ is
    \item Let a,b,c be three non-zero real numbers such that the equation : $\sqrt 3 a \cos x+ 2b \sin x = c , x\epsilon[{-\frac{\pi}{2}},{\frac{\pi}{2}}]$ has two distinct real roots $\alpha$ and $\beta$ with $\alpha + \beta = {\frac{\pi}{3}}$. Then, the value of ${\frac{b}{a}}$ is 
    \item The period of $\sin^2 \theta $ is 
    \begin{enumerate}
        \item $\pi^2$
        \item $\pi$
        \item $2\pi$
        \item $\pi/2$
    \end{enumerate}
    \item The number of solution of $\tan x + \sec x = 2\cos x $ in $[0,2\pi)$ is 
    \begin{enumerate}
        \item 2
        \item 3
        \item 0
        \item 1
    \end{enumerate}
    \item Which one is not periodic 
    \begin{enumerate}
        \item $\abs{\sin3x} + \sin^2 x$
        \item $\cos\sqrt x + \cos^2 x$
        \item $\cos 4x + \tan^2 x$
        \item $\cos 2x + \sin x$
    \end{enumerate}
    \item Let $\alpha, \beta$ be such that $\pi<\alpha-\beta< 3\pi$. If $\sin\alpha + \sin \beta = {-\frac{21}{65}}$ and $\cos \alpha + \cos\beta = {-\frac{27}{65}}$, then the value of $\cos{\frac{\alpha-\beta}{2}}$
    \begin{enumerate}
        \item ${-\frac{6}{65}}$
        \item ${\frac{3}{\sqrt{130}}}$
        \item ${\frac{6}{65}}$
        \item ${-\frac{3}{\sqrt{130}}}$
    \end{enumerate}
    \item If u = $\sqrt{a^2\cos^2\theta +b^2\sin^2\theta}$+$\sqrt{a^2\sin^2\theta +b^2\cos^2\theta}$then the difference between the maximum and minimum values of $u^2$ is given by
    \begin{enumerate}
        \item $(a - b) ^2$
        \item $2\sqrt{a^2 + b^2}$
        \item $(a + b) ^2$
        \item $2 (a^2+b^2)$
    \end{enumerate}
    \item A line makes the same angle $\theta$, with each of the x and z axis. If the angle $\beta$, which it makes with y-axis, is such that $\sin^2\beta = 3\sin^2\theta$, then $\cos^2\theta$ equals
    \begin{enumerate}
        \item ${\frac{2}{5}}$
        \item ${\frac{1}{5}}$
        \item ${\frac{3}{5}}$
        \item ${\frac{2}{3}}$
    \end{enumerate}
    \item The number of values of x in the interval $[0, 3\pi]$ satisfying the equation 
    \begin{align}
    2\sin{^2x} + 5\sin{x} -3 =0
    \end{align} is
    \begin{enumerate}
        \item 4
        \item 6
        \item 1
        \item 2
    \end{enumerate}
    \item If $0< x<\pi$ and $\cos x+\sin x={\frac{1}{2}} $, then $\tan x$ is 
    \begin{enumerate}
        \item ${\frac{(1-\sqrt7)}{4}}$
        \item ${\frac{(4-\sqrt7)}{3}}$
        \item ${-\frac{(4+\sqrt7)}{3}}$
        \item ${\frac{(1+\sqrt7)}{4}}$
    \end{enumerate}
    \item Let A and B denote the statements\\
    A : $\cos\alpha+\cos\beta+\cos\gamma = 0$\\
    B : $\sin\alpha+\sin\beta+\sin\gamma = 0$\\
    If $\cos(\beta-\gamma)+\cos(\gamma-\alpha)+\cos(\alpha-\beta) = {-\frac{3}{2}}$,then :
    \begin{enumerate}
        \item A is false and B is true
        \item Both A and B are true
        \item both A and B are false
        \item A is true and B is false
    \end{enumerate}
    \item Let $\cos(\alpha+\beta) = {\frac{4}{5}}$ and $\sin(\alpha-\beta) = {\frac{5}{13}}$, where $0\leq\alpha, \beta\leq{\frac{\pi}{4}}$, Then $\tan 2\alpha = $
    \begin{enumerate}
        \item ${\frac{56}{33}}$
        \item ${\frac{19}{12}}$
        \item ${\frac{20}{7}}$
        \item ${\frac{25}{16}}$
    \end{enumerate}
    \item If A = $\sin^2x+\cos^4 x$ , then for all real x:
    \begin{enumerate}
        \item ${\frac{13}{16}\leq A\leq 1}$
        \item $1\leq A\leq 2$
        \item $\frac{3}{4}\leq A\leq \frac{13}{16} $
        \item ${\frac{3}{4}\leq A\leq 1}$
    \end{enumerate}
    \item In a $\triangle PQR$, If $3 \sin P + 4 \cos Q = 6$ and 4 $\sin Q + 3 \cos P = 1$, then the angle R is equal to :
    \begin{enumerate}
        \item ${\frac{5\pi}{6}}$
        \item ${\frac{\pi}{6}}$
        \item ${\frac{\pi}{4}}$
        \item ${\frac{3\pi}{4}}$
    \end{enumerate}
    \item ABCD is a trapezium such that AB and CD are parallel and $BC\perp CD$. If $\angle ADB= \theta$, BC=p and CD=q, then AB is equal to :
    \begin{enumerate}
        \item ${\frac{(p^2+q^2)\sin\theta}{p\cos\theta + q\sin \theta}}$\\
        \item ${\frac{p^2 + q^2\cos \theta}{p\cos \theta + q\sin \theta}}$\\
        \item ${\frac{p^2 + q^2}{p^2\cos \theta + q^2\sin \theta}}$\\
        \item ${\frac{(p^2 + q^2)\sin \theta}{(p \cos \theta + q \sin \theta)^2}}$\\
    \end{enumerate}
    \item The expression $\frac{\tan{A}}{1- \cot{A}}+\frac{\cot{A}}{1-\tan{A}}$ can be written as:
    \begin{enumerate}
        \item $\sin A \cos A + 1$
        \item $\sec{A} cosec A+ 1$
        \item $\tan{A} + \cot{A}$
        \item $\sec{A} + cosec A$
    \end{enumerate}
    \item Let $f_k(x) = {\frac{1}{k}}(\sin^k x+ \cos^k x)$ where $x\epsilon R$ and $k\geq 1$. Then $f_4(x) - f_6(x)$ equals
    \begin{enumerate}
        \item {$\frac{1}{4}$}
        \item {$\frac{1}{12}$}
        \item {$\frac{1}{6}$}
        \item {$\frac{1}{3}$}
    \end{enumerate}
    \item If $0\leq x < 2\pi$, then the number of real values of x, which satisfy the equation $\cos x+\cos 2x+ \cos 3x+\cos 4x =0$ is:
    \begin{enumerate}
    \item 7
    \item 9
    \item 3
    \item 5
    \end{enumerate}
    \item If $5(\tan^2x-\cos^2 x) = 2\cos2x + 9$,then the value of cos4x is :
    \begin{enumerate}
        \item {$-\frac{7}{9}$}
        \item {$-\frac{3}{5}$}
        \item {$\frac{1}{3}$}
        \item {$\frac{2}{9}$}
    \end{enumerate}
    \item If sum of all the solutions of the equation 8 $\cos x.(\cos({\frac{\pi}{6}}+ x)(\cos({\frac{\pi}{6}}- x)-{\frac{1}{2}}) -1 $ in $[0 , \pi]$ is $k\pi$. then k is equal to :
    \begin{enumerate}
        \item {$\frac{13}{9}$}
        \item {$\frac{8}{9}$}
        \item {$\frac{20}{9}$}
        \item {$\frac{2}{3}$}
    \end{enumerate}
    \item For any $\theta \epsilon ({\frac{\pi}{4}},{\frac{\pi}{2}})$ the expression $3(\sin\theta - \cos\theta)^4 + 6(\sin\theta + \cos \theta)^2 + 4 \sin^2\theta$ equals:
    \begin{enumerate}
        \item $13- 4\cos^2\theta+6\sin^2\theta\cos^2\theta$
        \item $13- 4\cos^6\theta$
        \item $13- 4\cos^2\theta+6\cos^4\theta$
        \item $13- 4\cos^4\theta+2\sin^2\theta\cos^2\theta$
    \end{enumerate}
    \item The value of $\cos^2 10\degree- \cos10\degree\cos 50\degree + \cos^2 50\degree$ is:
    \begin{enumerate}
        \item ${\frac{3}{4}}+ \cos 20\degree$
        \item ${\frac{3}{4}}$
        \item ${\frac{3}{2}}(1+ \cos 20\degree)$
        \item ${\frac{3}{2}}$
    \end{enumerate}
    \item Let S=\{$\theta \epsilon [-2\pi,2\pi]: 2\cos^\theta+3\sin\theta =0$\}. Then the sum of the elements of S is 
    \begin{enumerate}
        \item ${\frac{13\pi}{6}}$
        \item ${\frac{5\pi}{3}}$
        \item 2
        \item 1
    \end{enumerate}
    {\Large\textbf{Match the Following}}\\\\
{\textbf{DIRECTIONS (Q.1):}}
\begin{textit}
{Each question contains statements given in two columns, which have to be matched. The statements in Column-I are labelled A, B , C and D, while the statements in Column-II are labelled p,q,r,s and t. Any given statement in Column-I can have correct matching with ONE OR MORE statement(s) in Column-II. The appropriate bubbles corresponding to the answers to theses questions have to be darkened as illustrated in the following example:\\
If the correct matches are A-p, s and t; B-q and r; C-p and q; D -s then the correct darkening of bubbles will look like the given} \end{textit}
\line(1,0){250}
\begin{enumerate}
    \item In this question there are entries in columns 1 and 2. Each entry in column 1 is related to exactly one entry in column 2. Write the correct letter from column 2 against the entry number in column 1 in your answer book.\\
{\Large{$\frac{\sin3\alpha}{\cos2\alpha}$}} is\\\\
\begin{tabular}{llll}
\textbf{Column-I} &   \enspace   &   \textbf{Column-II}\\
(A) Positive &   \enspace   &   (p)$({\frac{13\pi}{48}},{\frac{14\pi}{48}})$\\
&&&\\
(B) Negative    &   \enspace   & (q)$({\frac{14\pi}{48}},{\frac{18\pi}{48}})$\\
&&&\\
    &\enspace   &   (r)$({\frac{18\pi}{48}},{\frac{23\pi}{48}})$\\
&&&\\
  &\enspace  &   (s)$(0,{\frac{\pi}{2}})$\\
&&&\\
\end{tabular}
\item Let\\
f(x)=$\sin{(\pi \cos{x})}$ and g(x)=$\cos{(2\pi \sin{x})}$be two functions defined for $x>0$. Define the following sets whose elements are written i n the increasing order.\\\\
X =\{$x : f(x) = 0$\},Y =\{$x : f'(x) = 0$\} \\\\
Z =\{$x : g(x) = 0$\},W =\{$x : g'(x) = 0$\}\\\\
List-I contains the sets X,Y,Z and W. List-II contains some information regarding these sets.\\
\begin{tabular}{llll}
\textbf{Column-I} &   \enspace   &   \textbf{Column-II}\\
(A)X &   \enspace   &   (p)$\supseteq\{{\frac{\pi}{2}},{\frac{3\pi}{2}},4\pi,7\pi\}$\\
&&&\\
(B)Y    &   \enspace   & (q)an arithmetic progression\\
&&&\\
(C)Z    &\enspace   &   (r)NOT an arithmetic progression\\
&&&\\
(D)W &\enspace   &   (s)$\supseteq\{{\frac{\pi}{6}},{\frac{7\pi}{6}},{\frac{13\pi}{6}}\}$\\
&&&\\
    &\enspace   &   (t)$\supseteq\{{\frac{\pi}{3}},{\frac{2\pi}{3}},\pi\}$\\&&&\\
    &\enspace   &   (u)$\supseteq\{{\frac{\pi}{6}},{\frac{3\pi}{4}}\}$\\
\end{tabular}
Which of the following is the only CORRECT combination?
\begin{enumerate}
    \item (IV),(P),(R),(S)
    \item (III),(P),(Q),(U)
    \item (III),(R),(U)
    \item (IV),(Q),(T)
\end{enumerate}
\item Let $f(x) = \sin(\pi\cos x)$ and $g(x) = \cos(2\pi\sin x)$ be two functions defined for $x>0$.Define the following sets whose elements are written in the increasing order\\\\
X =\{$x : f(x) = 0$\},Y =\{$x : f'(x) = 0$\} \\\\
Z =\{$x : g(x) = 0$\},W =\{$x : g'(x) = 0$\}\\\\
List-I contains the sets X,Y,Z and W. List-II contains some information regarding these sets.\\
\begin{tabular}{llll}
\textbf{Column-I} &   \enspace   &   \textbf{Column-II}\\
(A)X &   \enspace   &   (p)$\supseteq\{{\frac{\pi}{2}},{\frac{3\pi}{2}},4\pi,7\pi\}$\\
&&&\\
(B)Y    &   \enspace   & (q)an arithmetic progression\\
&&&\\
(C)Z    &\enspace   &   (r)NOT an arithmetic progression\\
&&&\\
(D)W &\enspace   &   (s)$\supseteq\{{\frac{\pi}{6}},{\frac{7\pi}{6}},{\frac{13\pi}{6}}\}$\\
&&&\\
    &\enspace   &   (t)$\supseteq\{{\frac{\pi}{3}},{\frac{2\pi}{3}},\pi\}$\\&&&\\
    &\enspace  &   (u)$\supseteq\{{\frac{\pi}{6}},{\frac{3\pi}{4}}\}$\\
\end{tabular}
Which of the following is the only CORRECT combination?\\
\begin{enumerate}
    \item (I),(Q),(U)
    \item (I),(P),(R)
    \item (II),(R),(S)
    \item (II),(Q),(T)
\end{enumerate}
\end{enumerate}
    \end{enumerate}
%\end{document}
    
