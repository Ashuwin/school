\item Given the tetrahedron ABCD whose edges AB and CD have lengths a and b respectively. The distance between the skew lines AB and CD is d, and the angle between them is $\omega$. Tetrahedron ABCD is divided into two solids by plane $\varepsilon$, parallel to lines AB and CD. The ratio of the distances of $\varepsilon$ from AB and CD is equal to k. Compute the ratio of the volumes of the two solids obtained.
\item Consider a triangle OAB with acute angle AOB. Through a point $M \neq O$ perpendiculars are drawn to OA and OB, the feet of which are P and Q respectively. The point of intersection of the altitudes of $\triangle{OPQ}$ is H. What is the locus of H if M is permitted to range over
\begin{enumerate}
\item the side AB, 
\item  the interior of $\triangle{OAB}$?
\end{enumerate}
\item In a plane a set of n points ($n \geq 3$) is given. Each pair of points is connected by a segment. Let d be the length of the longest of these segments. We define a diameter of the set to be any connecting segment of length d. Prove that the number of diameters of the given set is at most n.



